\documentclass[french, 12pt]{article}


\usepackage[applemac]{inputenc}
\usepackage[T1]{fontenc}
\usepackage{babel}
\usepackage{array,epsfig}
\usepackage{geometry}
\usepackage{amsmath}
\usepackage{amsfonts}
\usepackage{amssymb}
\usepackage{amsxtra}
\usepackage{amsthm}
\usepackage{latexsym}
\usepackage{dsfont}
\usepackage{calrsfs}
\usepackage[mathscr]{eucal}
\usepackage{color}
\usepackage[all]{xy}
\usepackage{hyperref}
%\usepackage[notref,notcite]{showkeys}
\usepackage{graphicx}






\renewcommand{\labelenumi}{\textbf{\arabic{enumi}.}}
\renewcommand{\labelenumii}{\textbf{\alph{enumii}.}}
\renewcommand{\labelenumiii}{(\roman{enumiii})}






\theoremstyle{definition}
\newtheorem{defn}{D\'efinition}
\newtheorem{thm}{Th\'eor\`eme}
\newtheorem{cor}{Corollaire}
\newtheorem*{rmk}{Remarque}
\newtheorem{lem}{Lemme}
\newtheorem{ex}{Exercice}
\newtheorem*{soln}{Solution}
\newtheorem{prop}{Proposition}




\newcommand{\set}[1]{\left\{#1\right\}}
\newcommand{\tuple}[1]{\left(#1\right)}
\newcommand{\oin}[1]{\left]#1\right[}
\newcommand{\cin}[1]{\left[#1\right]}
\newcommand{\olin}[1]{\left]#1\right]}
\newcommand{\orin}[1]{\left[#1\right[}
\newcommand{\abs}[1]{\left|#1\right|}
\newcommand{\norm}[1]{\left\|#1\right\|}
\newcommand{\sprod}[1]{\left<#1\right>}
\newcommand{\floor}[1]{\left\lfloor#1\right\rfloor}
\newcommand{\ceil}[1]{\left\lceil#1\right\rceil}
\newcommand{\ol}[1]{\overline{#1}}
\newcommand{\wh}[1]{\widehat{#1}}
\newcommand{\wt}[1]{\widetilde{#1}}
\newcommand{\triplenorm}[1]{{\left\vert\kern-0.25ex\left\vert\kern-0.25ex\left\vert #1 \right\vert\kern-0.25ex\right\vert\kern-0.25ex\right\vert}}







\newcommand{\indi}{\mathds{1}}
\newcommand{\emb}{\hookrightarrow}
\newcommand{\proj}{\twoheadrightarrow}
\newcommand{\funct}{\rightsquigarrow}
\newcommand{\del}{\partial}
\newcommand{\Ra}{\Rightarrow}
\newcommand{\Lra}{\Leftrightarrow}





\renewcommand{\H}{\mathrm{H}}
\newcommand{\N}{\mathrm{N}}






\renewcommand{\Bbb}{\mathbb{B}}
\newcommand{\Cbb}{\mathbf{C}}
\newcommand{\Dbb}{\mathbb{D}}
\newcommand{\Ebb}{\mathbb{E}}
\newcommand{\Fbb}{\mathbb{F}}
\newcommand{\Hbb}{\mathbb{H}}
\newcommand{\Ibb}{\mathbb{I}}
\newcommand{\Kbb}{\mathbb{K}}
\newcommand{\kbb}{\mathbb{K}}
\newcommand{\Nbb}{\mathbf{N}}
\newcommand{\Obb}{\mathbb{O}}
\newcommand{\Pbb}{\mathbb{P}}
\newcommand{\Qbb}{\mathbf{Q}}
\newcommand{\Rbb}{\mathbf{R}}
\newcommand{\Sbb}{\mathbb{S}}
\newcommand{\Tbb}{\mathbf{T}}
\newcommand{\Zbb}{\mathbf{Z}}




\newcommand{\Acal}{\mathcal{A}}
\newcommand{\Bcal}{\mathcal{B}}
\newcommand{\Ccal}{\mathcal{C}}
\newcommand{\Dcal}{\mathcal{D}}
\newcommand{\Ecal}{\mathcal{E}}
\newcommand{\Fcal}{\mathcal{F}}
\newcommand{\Gcal}{\mathcal{G}}
\newcommand{\Ical}{\mathcal{I}}
\newcommand{\Jcal}{\mathcal{J}}
\newcommand{\Lcal}{\mathcal{L}}
\newcommand{\Mcal}{\mathcal{M}}
\newcommand{\Pcal}{\mathcal{P}}
\newcommand{\Ocal}{\mathcal{O}}
\newcommand{\Ucal}{\mathcal{U}}
\newcommand{\Vcal}{\mathcal{V}}
\newcommand{\Ncal}{\mathcal{N}}

\newcommand{\Cscr}{\mathcal{C}}







\newcommand{\Fix}{\operatorname{Fix}}
\newcommand{\tr}{\operatorname{tr}}
\newcommand{\id}{\operatorname{id}}
\newcommand{\GL}{\operatorname{GL}}
\newcommand{\Hyp}{\operatorname{Hyp}}
\newcommand{\spec}{\operatorname{sp}}
\newcommand{\rang}{\operatorname{rang}}
\newcommand{\Img}{\operatorname{Im}}
\newcommand{\Homeo}{\operatorname{Homeo}}
\newcommand{\diag}{\operatorname{diag}}







\newcommand{\ds}{\displaystyle}	


\title{\textsc{Syst\`emes dynamiques} \\Corrig\'e DM 2}
\date{}
\author{\Large Bas\'e sur la copie de Manh-Linh Nguyen}

\begin{document}
{\noindent \'Ecole Normale Sup\'erieure  \hfill Pour toute question :}
\\
{2021/2022 \hfill \hfill \texttt{chaubet\at dma.ens.fr}}

{\let\newpage\relax\maketitle}
\maketitle
	
	
\vspace{0.2 cm}

\section*{Perturbation des valeurs propres}

\begin{enumerate}
    \item \label{Partie1} On a
        $$(A - \mu)(A - \lambda)[(A - \lambda)^{-1} - (A - \mu)^{-1}] = (A - \mu) - (A - \lambda) = (\lambda - \mu)\id_n,$$
    d'o\`u 
        \begin{equation} \label{eq1}
            (A - \lambda)^{-1}(A - \mu)^{-1} = \frac{(A - \lambda)^{-1} - (A - \mu)^{-1}}{\lambda - \mu}.
        \end{equation}
        
    \item \label{Partie2} Soient $\rho > \rho' > 0$ tels que $\ol{D}(\lambda,\rho) \cap \spec(A) = \ol{D}(\lambda,\rho') \cap \spec(A) = \{\lambda\}$. Alors $(z-A)^{-1}$ est bien d\'efini, donc tous ses entr\'ees sont holomorphes au voisinage de l'annulus $\{\rho' \le |z - \lambda| \le \rho\}$ (car elle sont des fonctions rationnelles). Il suit de la formule de Cauchy que
        \begin{equation*}
            \frac{1}{2\pi i} \int_{\Ccal_{\lambda,\rho}} (z - A)^{-1}\,dz = \frac{1}{2\pi i} \int_{\Ccal_{\lambda,\rho'}} (z - A)^{-1}\,dz.
        \end{equation*}
    
    \item \label{Partie3} Soient $\rho > \rho' > 0$ comme dans la partie \ref{Partie2}. On a
        \begin{align*}
            \Pi_\lambda^2 & = \tuple{\frac{1}{2\pi i} \int_{\Ccal_{\lambda,\rho}} (z - A)^{-1}\,dz}\tuple{\frac{1}{2\pi i} \int_{\Ccal_{\lambda,\rho'}} (w - A)^{-1}\,dw} \\
            & = \frac{1}{(2\pi i)^2} \iint_{\Ccal_{\lambda,\rho} \times \Ccal_{\lambda,\rho'}} (z - A)^{-1}(w - A)^{-1} \,dzdw \\
            & = \frac{1}{(2\pi i)^2}\iint_{\Ccal_{\lambda,\rho} \times \Ccal_{\lambda,\rho'}} \frac{(z - A)^{-1} - (w - A)^{-1}}{w - z}\,dzdw & \text{(partie {\bf\ref{Partie1}.})}\\
            & = \frac{1}{2\pi i}\int_{\Ccal_{\lambda,\rho}} \tuple{\frac{1}{2\pi i}\int_{\Ccal_{\lambda,\rho'}}\frac{dw}{w - z}} (z-A)^{-1}\,dz \\
            & + \frac{1}{2\pi i}\int_{\Ccal_{\lambda,\rho'}} \tuple{\frac{1}{2\pi i}\int_{\Ccal_{\lambda,\rho}}\frac{dz}{z - w}} (w-A)^{-1}\,dw.
        \end{align*}
    Or $\ds \frac{1}{2\pi i}\int_{\Ccal_{\lambda,\rho'}}\frac{dw}{w - z} = 0$ pour tout $z \in \Ccal_{\lambda,\rho}$ et $\ds \frac{1}{2\pi i}\int_{\Ccal_{\lambda,\rho'}}\frac{dz}{z - w} = 1$ pour tout $w \in \Ccal_{\lambda,\rho'}$ par la formule de Cauchy. Il suit que $\Pi_\lambda^2 = \Pi_\lambda$.
    
    \item \label{Partie4} On rappelle que la diff\'erentielle du d\'eterminant en $A \in \GL_n(\Cbb)$ est
        $$M_n(\Cbb) \owns H \mapsto \det(A)\tr(A^{-1}H).$$
    Ainsi, pour $d(z) = \det B(z)$, on a
        $$d'(z) = \det B(z) \tr(B(z)^{-1}B'(z)) = d(z)\tr(B(z)^{-1}B'(z)).$$
    
    \item \label{Partie5} Soit $B: \Cbb \to M_n(\Cbb)$ d\'efinie par $B(z):= z - A$. Elle est holomorphe de d\'eriv\'ee $B'(z) = \id_n$. Soit $N $ l'ordre d'annulation de $d(z) := \det B(z)$ en $\lambda$, qui est la m\^eme chose que $\dim_{\Cbb}C_{\lambda,\Cbb}$. 
    On a
        $$\tr \Pi_\lambda = \frac{1}{2\pi i}\int_{\Ccal_{\lambda,\rho}} \tr\tuple{(z-A)^{-1}}\,dz = \frac{1}{2\pi i}\int_{\Ccal_{\lambda,\rho}} \tr \tuple{B(z)^{-1}B'(z)} = \frac{1}{2\pi i} \int_{\Ccal_{\lambda,\rho}} \frac{d'(z)}{d(z)}\,dz$$
    (partie {\bf \ref{Partie4}.}). Par la formule de Cauchy cette valeur est $N$. Observons que les entr\'ees de $(z - \lambda)^N(z-A)^{-1}$   sont de la forme
        $$\dfrac{(z-\lambda)^N}{d(z)}P(z), \qquad P \in \Cbb[z].$$
    Il suit que ces entr\'ees sont holomorphes au voisinage de $\lambda$. Ainsi
        $$(\lambda - A)^N(z-A)^{-1} = (\lambda-z)^N(z-A)^{-1} + \sum_{k=1}^N \binom{N}{k}(\lambda - z)^{N-k}(z-A)^{k-1}$$
    a ses entr\'ees holomorphes au voisinage de $\lambda$. En prenant $\rho$ assez petit
        $$(\lambda - A)^N\Pi_{\lambda} = \frac{1}{2\pi i} \int_{\Ccal_{\lambda,\rho}} (\lambda - A)^N(z-A)^{-1}\,dz = 0.$$
    Il suit que l'image de $\Pi_\lambda$ est contenue dans $C_{\lambda,\Cbb}$. De {\bf \ref{Partie3}.}, on sait que $\Pi_\lambda$ est un projecteur. Comme $\rang \Pi_\lambda = \tr \Pi_\lambda = N$, on a $\Img \Pi_\lambda = C_{\lambda,\Cbb}$.
    
    \item \label{Partie6} De {\bf \ref{Partie5}.}, $\Pi_\lambda$ fixe les vecteurs dans $C_{\lambda,\Cbb}$. Soient $\lambda, \mu \in \spec(A)$, $\lambda \neq \mu$ et $\rho > 0$ tels que $\ol{D}(\lambda,\rho) \cap \spec(A) = \{\lambda\}$, que $\ol{D}(\mu,\rho) \cap \spec(A) = \{\mu\}$ et que $\ol{D}(\lambda,\rho) \cap \ol{D}(\mu,\rho) = \varnothing$. On a
        \begin{align*}
            \allowdisplaybreaks
            \Pi_\lambda \Pi_\mu & = \tuple{\frac{1}{2\pi i} \int_{\Ccal_{\lambda,\rho}} (z - A)^{-1}\,dz}\tuple{\frac{1}{2\pi i} \int_{\Ccal_{\mu,\rho}} (w - A)^{-1}\,dw} \\
            & = \frac{1}{(2\pi i)^2} \iint_{\Ccal_{\lambda,\rho} \times \Ccal_{\mu,\rho}} (z - A)^{-1}(w - A)^{-1} \,dzdw \\
            & = \frac{1}{(2\pi i)^2}\iint_{\Ccal_{\lambda,\rho} \times \Ccal_{\mu,\rho}} \frac{(z - A)^{-1} - (w - A)^{-1}}{w - z}\,dzdw & \text{(partie {\bf\ref{Partie1}.})}\\
            & = \frac{1}{2\pi i}\int_{\Ccal_{\lambda,\rho}} \tuple{\frac{1}{2\pi i}\int_{\Ccal_{\mu,\rho}}\frac{dw}{w - z}} (z-A)^{-1}\,dz \\
            & + \frac{1}{2\pi i}\int_{\Ccal_{\mu,\rho}} \tuple{\frac{1}{2\pi i}\int_{\Ccal_{\lambda,\rho}}\frac{dz}{z - w}} (w-A)^{-1}\,dw\\
            & = 0
        \end{align*}
    par la formule de Cauchy. Ainsi les $\Pi_\lambda$, $\lambda \in \spec(A)$ sont les matrices des projecteurs spectraux complexes associ\'es \`a $A$.
    
    \item \label{Partie7} Les coefficients de $(z-A)^{-1}$ sont holomorphes au voisinages de $U \setminus \bigcup\limits_{\lambda \in \spec(A) \cap U} \ol{D}(\lambda,\rho)$ pour $\rho$ assez petit. Par la formule de Cauchy
        $$\Pi_U := \sum_{\lambda \in \spec(A) \cap U} \Pi_\lambda = \sum_{\lambda \in \spec(A) \cap U} \frac{1}{2\pi i} \int_{\Ccal_{\lambda,\rho}} (z - A)^{-1}\,dz = \frac{1}{2\pi i} \int_{\partial U} (z - A)^{-1}\,dz.$$
    Comme $U$ est compact, il existe un voisinage $\Ucal$ de $A$ dans $M_n(\Cbb)$ tel que $\det(z-M)$ ne s'annule par sur $\partial U$ pour tout $M \in \Ucal$. Ainsi
        $$\Pi_U(M) = \sum_{\lambda \in \spec(M) \cap U} \Pi_\lambda(M) = \frac{1}{2\pi i} \int_{\partial U} (z - M)^{-1}\,dz$$
    d\'efinit une fonction \`a entr\'ees holomorphes.
    
    \item \label{Partie8} Pour $u \in C_{\lambda, \Rbb}$, on a $\Pi_\lambda u = u$ (on peut voir $u$ comme un vecteur dans $\Cbb^n$, alors $u \in C_{\lambda,\Cbb}$). De m\^eme, si $\mu \in 
    \spec(A) \cap \Rbb$ est diff\'erente de $\lambda$, on a $\Pi_\lambda u = 0$ pour tout $u \in C_{\mu,\Rbb}$. Finalement si $\mu \in \spec(A) \setminus \Rbb$ et $u \in C_{\mu,\ol{\mu}}$, il existera $v \in \Cbb^n$ tel que $u + iv \in C_{\mu,\Cbb}$ et $u - iv \in C_{\ol{\mu},\Cbb}$. Ainsi $\Pi_\lambda(u + iv) = \Pi_\lambda(u - iv) = 0$,  d'o\`u $\Pi_\lambda u = 0$. On a montr\'e que les vecteurs propres r\'eels g\'en\'eralis\'es de $A$ sont envoy\'es sur les vecteurs r\'eels. Ainsi l'image de $\Rbb^n$ par $\Pi_\lambda$ est contenu dans $\Rbb^n$, donc les entr\'ees de $\Pi_\lambda$ sont r\'eelles.
    
    \item \label{Partie9} Dans {\bf \ref{Partie6}.}, on a montr\'e que $\Pi_\lambda \Pi_\mu = 0$ si $\lambda \neq \mu$. Comme $\lambda \neq \ol{\lambda}$, on a
        $$\Pi_{\lambda,\ol{\lambda}}^2 = (\Pi_\lambda + \Pi_{\ol{\lambda}})^2 = \Pi_\lambda^2 + \Pi_{\ol{\lambda}}^2 =  \Pi_\lambda + \Pi_{\ol{\lambda}} = \Pi_{\lambda,\ol{\lambda}}.$$
        
    Soit $u \in C_{\lambda,\ol{\lambda}}$. Soit $v \in \Rbb^n$ tel que $u + iv \in C_{\lambda,\Cbb}$ et $u - iv \in C_{\ol{\lambda},\Cbb}$. On a
        \begin{align*}
         \Pi_{\lambda,\ol{\lambda}} u & = \Pi_\lambda u + \Pi_{\ol{\lambda}} u \\
          & = \frac{1}{2}\Pi_\lambda(u + iv) + \frac{1}{2}\Pi_\lambda(u - iv) + \frac{1}{2}\Pi_{\ol{\lambda}}(u + iv) + \frac{1}{2}\Pi_{\ol{\lambda}}(u - iv)\\
          & = \frac{1}{2}(u + iv) + \frac{1}{2} \cdot 0 + \frac{1}{2} \cdot 0 + \frac{1}{2}( u -iv) \\
          & = u.
        \end{align*}
    De m\^eme, si $\mu \in \spec(A) \setminus \Rbb$ diff\'erente de $\lambda$ et de $\ol{\lambda}$, on aura $\Pi_{\lambda,\ol{\lambda}} u = 0$ pour tout $u \in C_{\mu,\ol{\mu}}$. Si $\mu \in \spec(A) \cap \Rbb$, on aura $\Pi_{\lambda,\ol{\lambda}} u = 0$ pour tout $u \in C_{\mu,\Rbb}$. Avec le m\^eme argument comme celui dans {\bf \ref{Partie8}.}, on voit que les entr\'ees de $\Pi_{\lambda,\ol{\lambda}}$ sont r\'eelles.
    
    \item \label{Partie10} De {\bf \ref{Partie9}.}, pour $\lambda \in \spec(A) \setminus \Rbb$, $\Pi_{\lambda,\ol{\lambda}}$ est un projecteur d'image $C_{\lambda,\ol{\lambda}}$. Il suffit de montrer que le produit de deux matrices distinctes, choisies arbitrairement dans les matrices donn\'ees, est $0$.
    \begin{enumerate} 
        \item Si $\lambda,\mu \in \spec(A) \cap \Rbb$ et $\lambda \neq \mu$, on a bien $\Pi_\lambda \Pi_\mu = 0$ ({\bf \ref{Partie6}.}).
        
        \item  Si $\lambda \in \spec(A) \cap \Rbb$ et $\mu \in \spec(A)$ avec $\Im \mu > 0$, on a
            $$\Pi_\lambda \Pi_{\mu,\ol{\mu}} = \Pi_\lambda \Pi_\mu + \Pi_\lambda \Pi_{\ol{\mu}} = 0.$$
            
        \item Si $\lambda,\mu \in \in \spec(A) \cap \Rbb$ avec $\Im \lambda > 0$ et $\Im \mu > 0$, on a
            $$\Pi_{\lambda,\ol{\lambda}} \Pi_{\mu,\ol{\mu}} = \Pi_\lambda \Pi_\mu + \Pi_\lambda \Pi_{\ol{\mu}} + + \Pi_{\ol{\lambda}}\Pi_\mu +  \Pi_{\ol{\lambda}} \Pi_{\ol{\mu}} = 0.$$
    \end{enumerate}
    On en d\'eduit le r\'esultat.
\end{enumerate}



\section*{Classification topologique des flots contractants}


\begin{enumerate}
    \setcounter{enumi}{10}
    \item \label{Partie11} On fixe $A \in M_n^{-}(\Rbb)$. Soient $z_1,\ldots,z_k \in \Cbb$ les racines de 
        $$P(z) := \det(z - A) = z^n + a_1z^{n-1} + \cdots + a_n,$$
    de multiplicit\'es respectives $m_1,\ldots,m_k$. On fixe $\varepsilon > 0$ tels que les disques $\ol{D}(z_j,\varepsilon)$, $j = 1,\ldots,k$ soient deux \`a deux disjoints et n'intersectent pas l'axe imaginaire. Alors pour $j = 1,\ldots,k$, $P$ ne s'annule pas sur $\partial D(z_j,\varepsilon)$. On pose
        $$\delta:=\min_{1 \le j \le k}\min_{z \in \partial D(z_j,\varepsilon)}\frac{|P(z)|}{1 + \cdots + |z|^{n-1}} > 0.$$
    Soit $\Ucal$ un voisinage de $A$ dans $M_n(\Rbb)$ tel que pour tout $B \in \Ucal$, le polyn\^ome caract\'eristique de $B$ a pour la forme
        $$\det(z - B) = Q(z) = z^n + b_1z^{n-1} + \cdots + b_n, \qquad \forall 1 \le \ell \le n,\ |b_\ell - a_\ell| < \delta.$$
    Par cons\'equent, pour tous $1 \le j \le k$ et $z \in \partial D(z_j,\varepsilon)$
        \begin{align*}
            |Q(z) - P(z)| & \le \sum_{\ell = 1}^n |b_\ell - a_\ell||z|^{n - \ell}
            < \delta(|z|^{n-1} + \cdots + 1)
             \le |P(z)|.
        \end{align*}
    Il suit du th\'eor\`eme de Rouch\'e que $Q$ a $m_j$ racines dans $D(z_j,\varepsilon)$ (compt\'e avec multiplicit\'e) pour chaque $j = 1,\ldots,k$. Ce sont toutes les racines de $Q$ en raison de degr\'e. Ainsi, si $\lambda \in \spec(B)$, il existera $1 \le j \le k$ tels que $|\Re\lambda - \Re z_j| \le |\lambda - z_j| < \varepsilon$. Il suit que $\Re \lambda < \Re z_j + \varepsilon \le -\alpha(A) + \varepsilon$. Donc
        $$-\alpha(B) = \max_{\lambda \in \spec(B)} \Re \lambda < -\alpha(A) + \varepsilon < 0,$$
    i.e. $\alpha(B) > \alpha(A) - \varepsilon$. De plus, supposons sans perde de g\'en\'eralit\'e que que $\Re z_1 = \max\limits_{1 \le j \le k}\Re z_j = -\alpha(A)$. Soit $\lambda_1 \in D(z_1,\varepsilon) \cap \spec(B)$, alors $|\Re \lambda_1 - \Re z_1| \le |\lambda_1 - z_1| < \varepsilon$. Il suit que
        $$-\alpha(B) = \max_{\lambda \in \spec(B)}\Re \lambda \ge \Re \lambda_1 > \Re z_1 - \varepsilon = -\alpha(A) - \varepsilon,$$
    i.e. $\alpha(B) < \alpha(A) + \varepsilon$. On conclut que $\Ucal \subseteq M_n^{-}(\Rbb)$ et que pour tout $B \in \Ucal$, $|\alpha(B) - \alpha(A)| < \varepsilon$. Ainsi, $M_n^{-}(\Rbb)$ est une partie ouverte de $M_n(\Rbb)$ et l'application $\alpha: M_n^{-}(\Rbb) \to \Rbb_{> 0}$ est continue.
    
    \item \label{Partie12} Soit $\norm{\cdot}$ la norme euclidienne sur $\Rbb^n$. Pour $A \in M^-_n(\Rbb)$, on d\'efinit
        \begin{equation} \label{eq2}
            \forall x \in \Rbb^n, \qquad \norm{x}_A:= \sqrt{\int_{0}^\infty e^{2s\beta(A)} \norm{e^{sA}x}^2 \, ds},
        \end{equation}
    qui est bien d\'efini. En effet, soient $d \in ]\beta(A),\alpha(A)[$ et $C > 0$ tels que
        $$\forall x \in \Rbb^n,\forall s \ge 0, \qquad \norm{e^{sA}x} \le Ce^{-ds}\norm{x}.$$
    Alors 
        $$\int_{0}^\infty e^{2s\beta(A)} \norm{e^{sA}x}^2 \, ds \le C^2\norm{x}^2 \int_{0}^\infty e^{2s(\beta(A) - d)}\,ds = \frac{C^2\norm{x}^2}{2(d - \beta(A))} < +\infty.$$
    De plus, \eqref{eq2} d\'efinit une norme. En effet, cette norme est induite par un produit scalaire $\sprod{\cdot,\cdot}_A$ d\'efini par
        $$\forall x,y \in \Rbb^n, \qquad \sprod{x,y}_A = {\int_{0}^\infty e^{2s\beta(A)} \sprod{e^{sA}x,e^{sA}y} \, ds}.$$
    o\`u $\sprod{\cdot,\cdot}$ est le produit scalaire usuel. Soit $x \in \Rbb^n$ et $t \ge 0$, on a
        \begin{align*}
            \norm{e^{tA}x}_A & = \sqrt{\int_{0}^\infty e^{2s\beta(A)} \norm{e^{(t+s)A}x}^2 \, ds} \\
            & =  \sqrt{\int_{t}^\infty e^{2(u - t)\beta(A)} \norm{e^{uA}x}^2 \, du} \\
            & \le  \sqrt{e^{-2t\beta(A)}\int_{0}^\infty e^{2u\beta(A)} \norm{e^{uA}x}^2 \, du} \\
            & = e^{-t\beta(A)}\norm{x}_A.
        \end{align*}
    Montrons finalement que l'application 
        $$M_n^{-}(\Rbb) \times \Rbb^n \to \Rbb_{\ge 0},\qquad (A,x) \mapsto \norm{x}_A^2$$
    est continue. Fixons $A \in M_n^-(\Rbb)$, $x \in \Rbb^n$ et $\varepsilon > 0$. Contr\^olons tout d'abord le terme 
        $$\int_M^\infty e^{2s\beta(B)}\norm{e^{sB}y}^2\,ds$$
    pour $M$ assez grand et $(B,y)$ assez proche de $(A,x)$.
   
   La matrice $e^{A}$ est hyperbolique, \`a valeur propres ayant module au plus $e^{-\alpha(A)}$. D'apr\`es le cours, on sait (en utilisant la forme normale de Jordan) qu'il existe une norme adapt\'ee $\norm{\cdot }'$ sur $\Rbb^n$ tel que $\triplenorm{e^A}' < e^{-d}$ (rappelons que $\alpha(A) > d > \beta(A)$). Si $B$ est assez proche de $A$, on aura $\triplenorm{e^B}' < e^{-d}$. Posons
        $$C_1 := \max_{\substack{0 \le \gamma \le 1 \\ \norm{y}' = 1}}e^{\gamma d}\norm{e^\gamma y}' > 0.$$
    Pour $s \ge 0$, \'ecrivons $s = m + \gamma$ avec $m \in \Nbb$ et $\gamma \in [0,1[$. On a, pour tout $y$ tel que $\norm{y}' = 1$ et tout $B \in M_n(\Rbb)$ assez proche de $A$
        $$\norm{e^{sB}y}' \le \tuple{\triplenorm{e^B}'}^m \norm{e^\gamma y}' < e^{-md}C_1e^{-\gamma d} = C_1e^{-sd}.$$
    Il suit que, pour tout $B$ assez proche de $A$
        $$\forall y \in \Rbb^n, \qquad \norm{e^{sB}y}' \le C_1e^{-sd}\norm{y}'.$$
    Finalement, soit $C_2 > 0$ tel que 
        $$\forall y \in \Rbb^n, \qquad \norm{y} \le C_2\norm{y}'.$$
   Pour tout $B$ assez proche de $A$ est tout $y \in \Rbb^n$
        \begin{align*}
            \int_M^\infty e^{2s\beta(B)} \norm{e^{sB}y}^2 \,ds & \le C_2^2\int_M^\infty e^{2s\beta(B)} \tuple{\norm{e^{sB}y}'}^2 \,ds \\
            & \le C_1^2C_2^2 \int_M^\infty e^{2s(\beta(B) - d)}\norm{y}'\,ds\\
            & = \frac{C_1^2C_2^2 (\norm{y}')^2}{2(d - \beta(B))}e^{2M(\beta(B) - d)}.
        \end{align*}
    (car $\beta(B) < d$). Soit $\Ucal$ un voisinage de $A$ dans $M_n(\Rbb)$ tel que $d - \beta(B) \ge \rho$ pour tout $B \in \Ucal$ et un certain $\rho > 0$ ne d\'ependant pas de $B$. Soit $U$ un voisinage de $x$ dans $\Rbb^n$ tel que $\norm{y}' < 2\norm{x}'$ pour tout $y \in U$. On a
        $$\forall M > 0, \forall (B,y) \in \Ucal \times U, \quad \int_M^\infty e^{2s\beta(B)} \norm{e^{sB}y}^2 \,ds \le \frac{2C_1^2C_2^2 (\norm{x}')^2}{\rho}e^{-2M\rho}.$$
        
    On choisit $M_0 > 0$ tel que
        $$\frac{2C_1^2C_2^2 (\norm{x}')^2}{\rho}e^{-2M_0\rho} < \frac{\varepsilon}{3}.$$
        
    Pour tout $(B,y) \in \Ucal \times U$, on a
        $$\abs{\int_{M_0}^\infty e^{2s\beta(B)} \norm{e^{sB}y}^2 \,ds  - \int_{M_0}^\infty e^{2s\beta(A)} \norm{e^{sA}x}^2 \,ds } < \frac{2\varepsilon}{3}.$$
        
    Quitte \`a choisir $\Ucal$ et $U$ plus petit, on peut supposer que
        $$\forall (s,B,y) \in [0,M_0] \times \Ucal \times U, \quad \abs{e^{2s\beta(B)}\norm{e^{sB}y}^2 - e^{2s\beta(A)}\norm{e^{sA}x}^2} < \frac{\varepsilon}{3M_0}.$$
    Donc, pour $(B,y)$ assez proche de $(A,x)$
        $$\abs{\int_0^{M_0}e^{2s\beta(B)}\norm{e^{sB}y}^2\,ds - \int_0^{M_0}e^{2s\beta(A)}\norm{e^{sA}x}^2 \,ds} < \frac{\varepsilon}{3}.$$
    On conclut alors que $\abs{\norm{y}_B^2 - \norm{x}_A^2} < \varepsilon$ si $(B,y)$ est assez proche de $(A,x)$. Autrement dit, l'application $(A,x) \mapsto \norm{x}_A$ est continue.
    

    
    \item  \label{Partie13} Soit $A \in M_n^{-}(\Rbb)$, on a $\alpha(A) > 0$ et $\beta(A) > 0$. Il suit que pour tout $x \in \Rbb^{n} \setminus \{0\}$ et tous $t > s \in \Rbb$, on a
        $$\norm{e^{tA}x}_A = \norm{e^{(t-s)A}e^{sA}x}_A \le e^{(s-t)\beta(A)}\norm{e^{sA}x}_A < \norm{e^{sA}x}_A,$$
    i.e. la fonction $t \mapsto \norm{e^{tA}x}_A$ est d\'ecroissante et continue. De {\bf \ref{Partie12}.}, on a
        $$\forall t \ge  0, \qquad \norm{e^{tA}x}_A \le e^{-t\beta(A)}\norm{x}_A, \qquad \norm{e^{-tA}x}_A \ge e^{t\beta(A)}\norm{x}_A.$$
    Ainsi
        $$\lim_{t \to + \infty} \norm{e^{tA}x}_A = 0, \qquad \lim_{t \to -\infty} \norm{e^{tA}x}_A = +\infty.$$
    Il exite un unique $\tau_A(x) \in \Rbb$ tel que $\norm{e^{\tau_A(x)}x}_A = 1$, i.e. $e^{\tau_A(x)}x \in S_A$.
    
    \item \label{Partie14} Montrons que l'application $(A,x) \mapsto \tau_A(x)$ est continue de $M_n^-(\Rbb) \times (\Rbb^n \setminus 0)$ dans $\Rbb$. Soit $A_n \to A$ et $x_n \to x$. On pose $t_n:=\tau_{A_n}(x_n)$. Pour tout $n$, si $\norm{x_n}_{A_n} \ge 1$, alors $t_n \ge 0$, donc $1 = \norm{e^{t_nA_n}x_n}_{A_n} \le e^{-t_n\beta(A_n)}\norm{x_n}_{A_n}$. Il suit que $t_n \le \frac{\ln \norm{x_n}_{A_n}}{\beta(A_n)}$. Si $0 < \norm{x_n}_{A_n} < 1$, alors $t_n < 0$, donc $1 = \norm{e^{t_n A_n}x_n}_{A_n} \ge e^{-t_n\beta(A_n)}\norm{x_n}_{A_n}$. Il suit que $t_n \ge \frac{\ln\norm{x_n}_{A_n}}{\beta(A_n)}$. Dans tous cas, on a
        $$\forall n, \qquad |t_n| \le \frac{\abs{\ln\norm{x_n}_{A_n}}}{\beta(A_n)},$$
    (valable m\^eme si $x_n = 0$), qui converge vers $\frac{\abs{\ln\norm{x}_A}}{\beta(A)}$ (en particulier, il est born\'e). Donc la suite $(t_n)_n$ admet au moins une valeur d'adh\'erence. Soit $t$ une telle valeur. Alors $\norm{e^{tAx}x} = 1$. Par unicit\'e de $\tau_A(x)$, on a n\'ecessairement $t = \tau_A(x)$. Donc $\tau_A(x)$ est la seule valeur d'adh\'erence de la suite $(t_n)_n$. Ainsi $\tau_{A_n}(x_n) \to \tau_A(x)$ comme d\'esir\'e. \\
    La m\^eme preuve donne $\tau_{A_n}(x_n) \to -\infty$ si $x = 0$. Ainsi $(A,x) \mapsto \varphi(A)(x)$ est continue $M_n^{-}(\Rbb) \times \Rbb^n \to \Rbb^n$.
    
    
    \item \label{Partie15} {\it En effet, $ \varphi(A)(x) = e^{\tau_A(x)}e^{\tau_A(x)A}x$ pour tout $x \in \Rbb^{n} \setminus \{0\}$, car $e^{\tau_A(x)A}x$ est d\'ej\`a dans $S_A$}. 
    Soit $\psi(A): \Rbb^n \to \Rbb^n$ donn\'ee par $\psi(A)(0) = 0$ et $\psi(A)(y) = e^{-(\ln\norm{y}_A) A}h_A(y)$. Bien s\^ur, $\psi(A)$ est continue en tout point de $\Rbb^n \setminus \{0\}$. De plus, pour $y \in \Rbb^n \setminus \{0\}$ tel que $\norm{y}_A < 1$, on a $-\ln\norm{y}_A > 0$, donc
        $$\norm{\psi(A)(y)}_A \le  e^{\ln\norm{y}_A\beta(A)}\norm{h_A(y)}_A = \norm{y}_A^{\beta(A)} \to 0$$
    quand $y \to 0$. D'o\`u la continuit\'e de $\psi(A)$ en $0$. 
    
    Soit $x \in \Rbb^{n} \setminus \{0\}$. On a  $\norm{\varphi(A)(x)}_A = e^{\tau_A(x)}\norm{e^{\tau_A(x)A}}_A = e^{\tau_A(x)}$, donc
        \begin{align*}
            \psi(A)(\varphi(A)(x)) & = e^{-(\ln\norm{\varphi_A(x)}_A)A} h_A(\varphi_A(x)) \\
            & = e^{-\tau_A(x)A} \tuple{\frac{\varphi_A(x)}{e^{\tau_A(x)}}}\\
            & = e^{-\tau_A(x)A}\tuple{\frac{e^{\tau_A(x)}e^{\tau_A(x)A}x}{e^{\tau_A(x)}}}\\
            & = x.
        \end{align*}
    De m\^eme, pour tout $y \in \Rbb^n \setminus \{0\}$, on voit facilement que 
        $$\norm{e^{(\ln\norm{y}_A)A} \psi(A)(y)}_A = \norm{h_A(y)}_A = 1,$$
    donc $\tau_A\tuple{\psi(A)(y)} = \ln\norm{y}_A$. Il suit que
        \begin{align*}
            \allowdisplaybreaks
                \varphi(A)(\psi(A)(y)) & = e^{\ln\norm{y}_A} e^{(\ln\norm{y}_A)A} \psi(A)(y) \\
                & = \norm{y}_A h_A(y) \\
                & = y.
        \end{align*}
    Donc $\varphi(A)$ est un hom\'emorphisme. De plus, pour $x \in \Rbb^n \setminus \{0\}$ et $t \in \Rbb$
        $$\norm{e^{(\tau_A(x) - t)A}e^{tAx}}_A = 1,$$
    donc $\tau_A(e^{tA}x) = \tau_A(x) - t$. Il suit que
        $$\varphi(A)(e^{tA}x) = e^{\tau_A(x) - t}e^{(\tau_A(x) - t)A}e^{tA}x = e^{-t} e^{\tau_A(x)}e^{\tau_A(x)A}x = e^{-t}\varphi(A)(x).$$
    Ainsi, $e^{-t}\varphi(A) = \varphi(A) \circ e^{tA}$ pour tout $t \in \Rbb$.
    
    \item \label{Partie16} Il suit de {\bf \ref{Partie15}.} que $\varphi(A)$ est une conjugaison entre les flots $e^{tA}$ et $e^{-t\id_n}$.
\end{enumerate}






\section*{Stabilit\'e structurelle des flots lin\'eaires hyperboliques}
\begin{enumerate}
    \setcounter{enumi}{16}
    \item \label{Partie17} On pose
        $$\tilde{E}^s(A):=\tuple{\bigoplus_{\lambda \in \spec(A) \cap \Rbb_{> 0}} C_{\lambda,\Rbb}} \oplus \tuple{ \bigoplus_{\substack{\lambda \in \spec(A) \setminus \Rbb \\ \Re \lambda >0,\, \\ \Im \lambda > 0}} C_{\lambda,\ol{\lambda}}}$$
    et
        $$\tilde{E}^u(A):=\tuple{\bigoplus_{\lambda \in \spec(A) \cap \Rbb_{< 0}} C_{\lambda,\Rbb}} \oplus \tuple{ \bigoplus_{\substack{\lambda \in \spec(A) \setminus \Rbb \\ \Re \lambda < 0,\, \\ \Im \lambda > 0}} C_{\lambda,\ol{\lambda}}}.$$
    Alors $\Rbb^n = \tilde{E}^s(A) \oplus \tilde{E}^u(A)$. Posons $A_s:=A|_{\tilde{E}^s(A)}$ et $A_u:=A|_{\tilde{E}^u(A)}$. On obtient $A_s \in M^-_{m(A)}(\Rbb)$ et $-A_u \in M^-_{n - m(A)}(\Rbb)$. De {\bf \ref{Partie12}.}, on a $\tilde{E}^s(A) \subseteq E^s(A)$ et $\tilde{E}^u(A) \subseteq E^u(A)$, donc $\Rbb^n = E^s(A) + E^u(A)$. De plus, si $x \in E^s(A) \cap E^u(A)$, on peut \'ecrit $x = x_s + x_u$ avec $x_s \in \tilde{E}^s(A)$ et $x_u \in \tilde{E}^u(A)$. Il suit que
        $$e^{tA_u}x_u = e^{tA}x_u = e^{tA}x - e^{tA}x_s \xrightarrow[t \to + \infty]{} 0.$$
    Or $\norm{x_u}_{-A_u} \le e^{-t\beta(-A_u)}\norm{e^{tA_u}x_u}_{-A_u} \to 0$ quand $t \to +\infty$, donc $\norm{x_u}_{-A_u} = 0$, i.e. $x_u = 0$. De m\^eme, $x_s = 0$ et on conclut que $x = 0$. Il suit que $E^s(A) \cap E^u(A) = \{0\}$ et on a une somme directe
        $$\Rbb^n = E^s(A) \oplus E^u(A).$$
    En particulier, $E^s(A) = \tilde{E}^s(A)$ et $E^u(A) = \tilde{E}^u(A)$.
    
    \item \label{Partie18} Soit $U$ un ouvert born\'e de $\{z \in \Cbb,\ \Re z < 0\}$ qui contient $\spec(A) \cap \{z \in \Cbb,\ \Re z < 0\}$. En utilisant les notations comme celles dans {\bf \ref{Partie7}.}, on a $\pi_s(A) = \Pi_U(A)$. Le m\^eme argument avec le th\'eor\`eme de Rouch\'e comme celui dans {\bf \ref{Partie11}.} nous donne un voisinage $\Vcal$ de $A$ dans $M_n(\Cbb)$ tel que pour tout $B \in \Vcal$, on a $B \in \Hyp_n(\Rbb)$ et $\spec(B) \cap \{z \in \Cbb,\ \Re z < 0\} \subseteq U$. Soit $\Ucal$ un voisinage de $A$ dans $M_n(\Rbb)$ tel que toute matrice $B \in \Ucal$ appartienne aussi \`a $\Vcal$. Alors $\Pi_U(B) = \pi_s(B) \in M_n(\Rbb)$ pour $B \in \Ucal$, et $\Pi_U: \Ucal \to M_n(\Rbb)$ d\'efinit une fonction continue, car elle est la restriction d'une fonction holomorphe. On en d\'eduit que l'application $A \mapsto \pi_s(A)$ est continue de $\Hyp_n(\Rbb)$ dans $\Lcal(\Rbb^n)^2$. M\^eme argument pour $A \mapsto \pi_u(A)$.
    
    \item \label{Partie19} Soit $\Ucal$ comme dans {\bf \ref{Partie18}.}. Puisque $\dim E^s(M) = \tr \pi_s(M) \in \Nbb$ et que $M \mapsto \pi_s(M)$ est continue, on a que $M \mapsto \dim E^s(M)$ est localement constante. Soit $v_1, \dots, v_r$ une base de $E^s(A)$. Alors pour tout $M$ assez proche de $A$, on a que la famille $(\pi_s(M) v_i)_{i=1, \dots, r}$ est libre (cette famille d\'epend contin\^ument de $M$ et vaut $(v_i)_{i=1, \dots,r}$ pour $M = A$). Puisque $\dim E^s(M) = \dim E^s(A)$ pour $M$ assez proche de $A$, on a le r\'esultat.
    
    \item \label{Partie20} On fixe $(u_1, \dots, u_r)$ une base de $E^s(A)$, que l'on compl\`ete en une base $(u_1, \dots, u_n)$ de $\Rbb^n$. Alors pour tout $M$ proche de $A$, la famille 
    $$\beta(M) = \left(\pi_s(M)u_1, \dots, \pi_s(M)u_r, u_{r+1}, \dots, u_n\right)$$ reste libre, ainsi que la famille
    $$\tilde\beta(M) = \left(M\pi_s(M) u _1, \dots, M\pi_s(M) u_r, u_{r+1}, \dots, u_n\right)$$ puisque $M$ pr\'eserve $E^s(M)$ et est inversible. On note $P(M)$ la matrice de $\beta(M)$ dans la base $\tilde \beta(M)$. Notons
    $$
    P(M) = \begin{pmatrix} Q(M) & 0 \\ 0 & I_{n-r} \end{pmatrix}.
    $$
  Alors par d\'efinition, la matrice de $q_s(M) M \pi_s(M)|_{E^s(A)}$ dans la base $(u_1, \dots, u_r)$ est donn\'ee par $Q(M)$. Il suffit donc de montrer que $M \mapsto Q(M)$ est continue. Ceci sera vrai si $M \mapsto P(M)$ l'est. Or on a (en identifiant les bases $\beta(M), \tilde \beta(M)$ avec les matrices les repr\'esentant dans la base canonique de $\Rbb^n$)
  $$
  P(M) = \beta(M) \tilde \beta(M)^{-1},
  $$
  ce qui conclut puisque les applications $M \mapsto \beta(M), \tilde \beta(M)$ sont continues, et l'inversion est continue $\mathrm{GL}(n,\Rbb) \to \mathrm{GL}(n, \Rbb)$.
    \item \label{Partie21} Pour $M \in \Ucal$, les valeurs propres de $M|_{E^s(M)}$ ont parties r\'eelles n\'egatives, celles de $\wt{M}$ aussi. On d\'efinit $\wt{\Phi}_s: \Ucal \times E^{s}(A) \to E^s(A)$ par
        $$\forall (M, x_s) \in E^s(A), \qquad \wt{\Phi}_s(M,x_s) := \varphi(A)^{-1} \circ \varphi(\wt{M})(x_s)$$
    o\`u l'hom\'eomorphisme $\varphi(B): E^s(A) \to E^s(A)$ est d\'efini dans {\bf \ref{Partie15}.} pour chaque $B \in \Lcal(E^s(A))$ ayant seulement les valeurs propres de partie r\'eelle n\'egative. Alors $\wt{\Phi}_s(M,\cdot) = \varphi(A)^{-1} \circ \varphi(\wt{M})$ est un hom\'eomorphisme (en particulier, $\wt{\Phi}_s(A,\cdot) = \id_{E^s(A)}$). Soient $M \in \Ucal$, $t \in \Rbb$ et $x_s \in E^s(A)$. On a
        $$\varphi(A)(e^{tA} \wt{\Phi}_s(M,x_s)) = e^{-t}\varphi(A) (\wt{\Phi}_s(M,s)) = e^{-t}\varphi(\wt{M})(x_s) = \varphi(\wt{M})(e^{t\wt{M}}x_s).$$
    Il suit que
        $$e^{tA} \wt{\Phi}_s(M,x_s) = \varphi(A)^{-1} \circ \varphi(\wt{M})(e^{t\wt{M}}x_s) = \wt{\Phi}_s(M, e^{t\wt{M}}x_s).$$
    
    \item \label{Partie22} En rempla\c cant $A$ par $-A$ dans {\bf \ref{Partie21}.} (et choissisant $\Ucal$ plus petit si n\'ecessaire), on peut supposer que pour chaque $M \in \Ucal$, $\pi_u(M)|_{E^u(A)}:E^u(A) \to E^u(M)$ soit un isomorphisme d'inverse $q_u(M)$ et qu'il existe un application continue $\wt{\Phi}_u : \Ucal \times E^u(A) \to E^u(A)$ v\'erifiant
        $$\forall (M,t,x_u) \in \Ucal \times \Rbb \times E^u(A), \qquad e^{tA} \wt{\Phi}_u(M,x_u) = \wt{\Phi}_u(M, e^{t\wh{M}}x_u)$$
    o\`u $\wh{M} = q_u(M)M\pi_u(M)|_{E^u(A)}$, telle que $\wt{\Phi}_u(M,\cdot):E^u(A) \to E^u(A)$ soit un hom\'eomorphisme pour tout $M \in \Ucal$ et que $\wt{\Phi}_u(A,\cdot) = \id_{E^u(A)}$. On d\'efinit alors
        $$\Phi_s: \Ucal \times \Rbb^n \to E^s(A), \qquad (M,x) \mapsto \wt{\Phi}_s(M,q_s(M)\pi_s(M)x)$$
        $$\Phi_u: \Ucal \times \Rbb^n \to E^u(A), \qquad (M,x) \mapsto \wt{\Phi}_u(M,q_u(M)\pi_u(M)x)$$
    et
        $$\Phi: \Ucal \times \Rbb^n \to \Rbb^n, \qquad (M,x) \mapsto \Phi_s(M,x) + \Phi_u(M,x).$$
    Pour chaque $M \in \Ucal$, on note $\wt{\Psi}_s(M, \cdot): E^s(A) \to E^s(A)$ (resp. $\wt{\Psi}_u(M, \cdot): E^u(A) \to E^u(A)$) l'inverse de $\wt{\Phi}_s(M,\cdot)$ (resp. de $\wt{\Phi}_u(M, \cdot)$). D\'efinissons
        $$\Psi: \Ucal \times \Rbb^n \to \Rbb^n, \quad (M,x) \mapsto \pi_s(M) \wt{\Psi}_s(M, x_s) + \pi_u(M) \wt{\Psi}_u(M, x_u)$$
    o\`u $x_s = \pi_s(A)x$ et $x_u = \pi_u(A)x$, qui est continue. De plus
        \begin{align*}
            \Psi(M,\Phi(M,x)) & = \Psi(M,\Phi_s(M,x) +\Phi_u(M,x))\\
            & = \pi_s(M)\wt{\Psi}_s(M, \Phi_s(M,x)) + \pi_u(M)\wt{\Psi}_u(M, \Phi_u(M,x))\\
            & = \pi_s(M) q_s(M) \pi_s(M)x + \pi_u(M) q_u(M) \pi_u(M)x\\
            & = \pi_s(M)x + \pi_u(M)x \\
            & = x
        \end{align*}
    et
        \begin{align*}
             & \Phi(M,\Psi(M,x))\\
            =\ & \Phi_s(M,\Psi(M,x)) + \Phi_u(M,\Psi(M,x))\\
            =\ & \wt{\Phi}_s(M,q_s(M)\pi_s(M)\Psi(M,x)) + \wt{\Phi}_u(M,q_u(M)\pi_u(M)\Psi(M,x)) \\
            =\ & \wt{\Phi}_s(M,q_s(M)\pi_s(M)\wt{\Psi}_s(M,x_s)) + \wt{\Phi}_u(M,q_u(M)\pi_u(M)\wt{\Psi}(M,x_u))\\
            =\ & \wt{\Phi}_s(M,\wt{\Psi}_s(M,x_s)) + \wt{\Phi}_u(M,\wt{\Psi}(M,x_u)) \\
            =\ & x_s + x_u \\
            =\ & x.
        \end{align*}
    On conclut que $\Phi(M,\cdot): \Rbb^n \to \Rbb^n$ est un hom\'eomorphisme pour tout $M \in \Ucal$, et qu'on a une application continue $\Ucal \to \Homeo(\Rbb^n)$, $M \mapsto \Phi(M,\cdot)$ (la topologie sur $\Homeo(\Rbb^n)$ est la topologie compacte-ouverte), car $\Ucal$ et $\Rbb^n$ sont localement compacts. De plus, $\Phi(A,\cdot) = \id_{\Rbb^n}$. 
    
    Finalement, soit $t \in \Rbb$. On rappelle que
        $$\wt{\Psi}_s(M,e^{tA}x_s) = e^{t\wt{M}}\wt{\Psi}_s(M,x_s), \qquad \wt{\Psi}_u(M,e^{tA}x_u) = e^{t\wh{M}}\wt{\Psi}_u(M,x_u).$$
    Pour tout $M \in \Ucal$, les espaces $E^s(M)$ et $E^u(M)$ sont $M$-invariants, donc $M$ commute avec les projecteurs $\pi_s(M)$ et $\pi_u(M)$. Il suit que
        \begin{align*}
            & \Psi(M,e^{tA}x) \\
            = & \  \pi_s(M)\wt{\Psi}_s(M, \pi_s(A)e^{tA}x) + \pi_s(M)\wt{\Psi}_u(M, \pi_u(A)e^{tA}x) \\
            = & \  \pi_s(M)\wt{\Psi}_s(M, e^{tA}x_s) + \pi_u(M)\wt{\Psi}_u(M, e^{tA}x_u)\\
            = & \ \pi_s(M)e^{t\wt{M}}\wt{\Psi}_s(M,x_s) + \pi_u(M)e^{t\wh{M}}\wt{\Psi}_u(M,x_u)\\
            = & \ \pi_s(M)q_s(M)e^{tM}\pi_s(M)\wt{\Psi}_s(M,x_s) + \pi_u(M)q_u(M)e^{tM}\pi_u(M)\wt{\Psi}_u(M,x_u)\\
            = & \ e^{tM}\pi_s(M)\wt{\Psi}_s(M,x_s) + e^{tM}\pi_u(M)\wt{\Psi}_u(M,x_u)\\
            = & \ e^{tM}\Psi(M,x).
        \end{align*}
    Ainsi $e^{tA}\Phi(M,x) = \Phi(M,e^{tM}x)$, i.e. on a donc une conjugaison $\Phi(M,\cdot)$ entre les flots $e^{tA}$ et $e^{tM}$, qui varie continuement en $M \in \Ucal$ et $\Phi(A, \cdot) = \id_{\Rbb^n}$. En autres termes, le flot $e^{tA}$ est structurellement stable.
\end{enumerate}





\section*{Applications: conjugaisons en famille}
    \begin{enumerate}
        \setcounter{enumi}{22}
        \item \label{Partie23} On a discut\'e l'ouverture des $\Ucal_j$, $j = 0,\ldots,n$ dans {\bf \ref{Partie11}.}. Il reste \`a montrer leur connexit\'e. Soit $I_j \in \Ucal_j$ la matrice $\diag[1,\ldots,1,-1,\ldots,-1]$ ($j$ entr\'ees sont $1$). Toute matrice $ M \in \Ucal_j$ s'\'ecrit $M = PHP^{-1}$, o\`u $H = \diag[A_1,\ldots,A_r,B_1,\ldots,B_s,C_1\ldots,C_u,D_1,\ldots,D_v]$ telle que
        \begin{enumerate}
            \item Les $A_k$ $(1 \le k \le r)$ sont les blocs de Jordan r\'eels associ\'es aux valeurs propres positives de $M$.
            
            \item Les $B_k$ $(1 \le k \le s)$ sont les blocs de Jordan complexes associ\'es aux valeurs propres de partie r\'eelle positive de $M$.
            
            \item Les $C_k$ $(1 \le k \le u)$ sont les blocs de Jordan r\'eels associ\'es aux valeurs propres n\'egatives de $M$.
            
            \item Les $D_k$ $(1 \le k \le v)$ sont les blocs de Jordan complexes associ\'es aux valeurs propres de partie r\'eelle n\'egative de $M$.
            
            \item La somme des tailles des $A_k$ et $B_k$ est $m(A)$.
            
            \item $\det(P) > 0$ (on peut remplacer $P$ par $-P$ si n\'ecessaire).
        \end{enumerate}
        
        L'ensemble $\GL^+_n(\Rbb)$ des matrices de d\'eterminant positif est connexe par arcs, donc on peut trouver un chemin $p(t): [0,1] \to \GL^{+}_n(\Rbb)$ tel que $p(0) = P$ et $p(1) = \id_n$. Trouvons maintenant un chemin dans $\Ucal_j$ reliant $H$ \`a $I_j$.
        \begin{enumerate}
            \item Soit $1 \le k \le r$ et $A_k = \begin{bmatrix}
                \lambda &  1  \\
                &  \ddots \\
                & & \lambda
            \end{bmatrix}$, $\lambda > 0$. Le chemin
                $$\forall t \in [0,1] \qquad a_k(t):=  \begin{bmatrix}
                (1-t)\lambda + t & 1 - t  \\
                &  \ddots \\
                & & (1-t)\lambda + t
            \end{bmatrix}$$
        relie $a_k(0) = A_k$ \`a $a_k(1) = \diag[1,\ldots,1]$. De plus, les valeurs propres de $a_k(t)$ (qui sont $(1 - t)\lambda + t$) sont positives pour tout $t \in [0,1]$.
        
            \item Soit $1 \le k \le s$ et $B_k = \begin{bmatrix}
                Q &  \id_2  \\
                &  \ddots \\
                & & Q
            \end{bmatrix}$, $Q = \begin{bmatrix} a & -b \\
            b & a \end{bmatrix}$, $a > 0$, $b \neq 0$. Le chemin $q(t):=\begin{bmatrix}
                (1-t)a + t & -b(1-t)\\
                b(1-t) & (1-t)a + t
            \end{bmatrix}$ relie $q(0) = Q$ \`a $q(1) = \id_2$. Ainsi, le chemin
                $$\forall t \in [0,1] \qquad b_k(t):=              \begin{bmatrix}
                         q(t) & (1 - t)\id_2  \\
                         &  \ddots \\
                         & & q(t)
                    \end{bmatrix}$$
        relie $b_k(0) = B_k$ \`a $b_k(1) = \diag[1,\ldots,1]$. De plus, les valeurs propres de $b_k(t)$ (qui sont $(1 - t)a + t \pm ib(1-t)$) sont de partie r\'eelle positive pour tout $t \in [0,1]$.
        
        \item De m\^eme, on peut trouver les chemins $c_k$ reliant $C_k$ \`a $\diag[-1,\ldots,-1]$ ($1 \le k \le u$) tel que les valeurs propres de $c_k(t)$ sont n\'egatives pour tout $t \in [0,1]$; et les chemins $d_k$ reliant $D_k$ \`a $\diag[-1,\ldots,-1]$ ($1 \le k \le v$) tel que les valeurs propres de $d_k(t)$ sont de partie r\'eelle n\'egative pour tout $t \in [0,1]$.
        \end{enumerate}
        
        Ainsi, on a un chemin
            $$h:=\diag[a_1,\ldots,a_r,b_1,\ldots,b_s,c_1,\ldots,c_u,d_1,\ldots,d_v] : [0,1] \to \Ucal_j$$
        reliant $h(0) = H$ \`a $h(1) = I_j$. En cons\'equence, on a une chemin
            $$\forall t \in [0,1], \qquad m(t):=p(t)h(t)p(t)^{-1} \in \Ucal_j$$
        reliant $m(0) = PHP^{-1} = M$ \`a $m(1) = I_j$ dans $\Ucal_j$. Donc $\Ucal_j$ est connexe par arcs.
        
        \item \label{Partie24} De {\bf \ref{Partie22}.}, pour tout $s \in [0,1]$, il existe un voisinage $\Vcal_s \subseteq \Ucal_j$ de $M(s)$ et une application continue $\Phi_s: \Vcal_s \times \Rbb^n \to \Rbb^n$ telle que pour tout $s' \in [0,1]$ avec $M(s) \in \Vcal_s$, on ait une conjugaison $\Phi_s(s',\cdot)$ entre les flots $e^{tM(s)}$ et $e^{tM(s')}$; et que $\Phi_s(s,\cdot) = \id_{\Rbb^n}$.
        
        Soient $0 = s_0 < s_1 < \cdots < s_k = 1$ tels que pour chaque $1 \le k \le m$, $M([s_{k-1},s_k]) \subseteq \Vcal_k:=\Vcal_{s_{k-1}}$ (par compacit\'e et connexit\'e). On note $\Phi_k:=\Phi_{s_{k-1}}$ pour $1 \le k \le m$. On d\'efinit une application continue 
            $$F_k: [s_{k-1},s_k] \to \Homeo(\Rbb^n), \quad s \mapsto \Phi_1(s_1,\cdot) \circ \cdots \circ \Phi_{k-1}(s_{k-1},\cdot) \circ  \Phi_k(s,\cdot).$$
        pour chaque $1 \le k \le m$. Pour tout $(s,t) \in [s_{k-1}, s_k] \times \Rbb$, on a
            \begin{align*}
                F_k(s) \circ e^{tM(s)} & = \Phi_1(s_1,\cdot) \circ \cdots \circ \Phi_{k-1}(s_{k-1},\cdot) \circ  \Phi_k(s,\cdot) \circ e^{tM(s)} \\
                & = \Phi_1(s_1,\cdot) \circ \cdots \circ \Phi_{k-1}(s_{k-1},\cdot) \circ e^{tM(s_{k-1})} \circ  \Phi_k(s,\cdot)\\
                & \vdots \\
                & = \Phi_1(s_1,\cdot) \circ e^{tM(s_1)} \circ \cdots \circ  \Phi_k(s,\cdot)\\
                & = e^{tM(s_0)} \circ \Phi_1(s_1,\cdot) \circ \cdots \Phi_1(s_1,\cdot)\\
                & = e^{tA} \circ F_k(s).
            \end{align*}
        Or, pour $1 \le k \le m-1$, on a $\Phi_{k+1}(s_k) = \id_{\Rbb}^n$, donc $F_k(s_k) = F_{k+1}(s_k)$. On peut d\'efinir donc une application continue $\Psi:[0,1] \times \Rbb^n \to \Rbb^n$ en posant $\Psi(s,x):=F_k(s)(x)$ lorsque $s_{k-1} \le s \le s_k$. De plus,
            $$e^{tA}\Psi(s,x) = e^{tA}F_k(s)(x) = F_k(s)(e^{tM(s)}x) = \Psi(s,e^{tM(s)}x).$$
\end{enumerate}
\end{document}


