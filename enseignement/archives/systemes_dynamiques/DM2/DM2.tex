\documentclass[a4paper,12pt]{article}
\usepackage[T1]{fontenc}
\usepackage[margin=2.8cm]{geometry}
\usepackage[applemac]{inputenc}
\usepackage{lmodern}
\usepackage{enumitem}
\usepackage{microtype}
\usepackage{hyperref}
\usepackage{enumitem}
\usepackage{dsfont}
\usepackage{amsmath,amssymb,amsthm}
\usepackage{mathenv}
\usepackage{amsthm}
\usepackage{graphicx}
\usepackage[all]{xy}
\usepackage{mathrsfs}
\theoremstyle{plain}
\newtheorem{thm}{Theorem}[section]
\newtheorem*{thm*}{Th\'eor\`eme}
\newtheorem{prop}[thm]{Proposition}
\newtheorem{cor}[thm]{Corollary}
\newtheorem{lem}[thm]{Lemma}
\newtheorem{propr}[thm]{Propri\'et\'e}
\theoremstyle{definition}
\newtheorem{deff}[thm]{Definition}
\newtheorem{rqq}[thm]{Remark}
\newtheorem{ex}[thm]{Exercice}
\newcommand{\e}{\mathrm{e}}
\newcommand{\prodscal}[2]{\left\langle#1,#2\right\rangle}
\newcommand{\devp}[3]{\frac{\partial^{#1} #2}{\partial {#3}^{#1}}}
\newcommand{\w}{\omega}
\newcommand{\dd}{\mathrm{d}}
\newcommand{\x}{\times}
\newcommand{\ra}{\rightarrow}
\newcommand{\pa}{\partial}
\newcommand{\vol}{\operatorname{vol}}
\newcommand{\dive}{\operatorname{div}}
\newcommand{\tr}{\mathrm{tr}}
\newcommand{\T}{\mathbf{T}}
\newcommand{\R}{\mathbf{R}}
\newcommand{\Z}{\mathbf{Z}}
\newcommand{\C}{\mathbf{C}}
\newcommand{\N}{\mathbf{N}}
\renewcommand{\x}{x}
\newcommand{\Homeo}{\mathrm{Homeo}}
\renewcommand{\sp}{\mathrm{sp}}
\newcommand{\CC}{\mathscr{C}}

\title{\textsc{Syst\`emes dynamiques} \\ DM 2}
\date{{Pour le 18/11/2021}}
\author{}

\begin{document}
{\noindent \'Ecole Normale Sup\'erieure  \hfill Pour toute question :} \\
{2021/2022 \hfill  \texttt{chaubet@dma.ens.fr}}

{\let\newpage\relax\maketitle}
\maketitle

\noindent On note $M_n(k)$ l'espace vectoriel des matrices carr\'ees de taille $n$ \`a coefficients dans $k= \R$ ou $\C$. Pour tout $A \in M_n(\C)$, on note $\mathrm{sp}(M) \subset \C$ son spectre et
$$
C_{\lambda, k} = \Bigl\{u \in k^n ~:~\exists N \in \N,~(A-\lambda)^Nu = 0\Bigr\},
$$
le $k$-espace propre g\'en\'eralis\'e de $A$ associ\'e \`a $\lambda \in \C$. Si $\lambda \in \C \setminus \R$, on d\'efinit aussi
$$
C_{\lambda, \bar{\lambda}} = \Bigl\{u \in \R^n~:~ \exists N \in \N,~ (A-\lambda)^N(A-\bar{\lambda})^{N}u = 0 \Bigr\}
$$
l'espace propre g\'en\'eralis\'e r\'eel associ\'e \`a $\lambda$ et $\bar{\lambda}$. On note aussi $D(z, \rho) =  \{\zeta \in \C,~ |z-\zeta| < \rho\}$, sa fermeture $\bar{D}(z,\rho) = \overline{D(z,\rho)}$ et $\CC_{\lambda,\rho} = \partial D(z,\rho)$ pour tout $z \in \C$ et $\rho > 0$. \\

%
\noindent Le but du probl\`eme est de montrer le r\'esultat suivant. \\

\noindent \textbf{Proposition} (Stabilit\'e structurelle des flots lin\'eaires hyperboliques). \textit{Soit $A \in M_n(\R)$ une matrice dont toutes les valeurs propres ont une partie r\'eelle non nulle. Alors il existe un voisinage $\mathcal{U}$ de $A$ dans $M_n(\R)$ et une application continue $\Phi : \mathcal{U} \times \R^n \to \R^n$ telle que pour tout $M \in \mathcal{U}$, l'application $x \mapsto \Phi(M,x)$ est un hom\'eomorphisme de $\R^n$ et
$$
\Phi(M, \e^{tM}x) = \e^{tA}\Phi(M,x), \quad t \in \R, \quad x \in \R^n.
$$
}
\section*{Perturbation des valeurs propres}
Soit $A \in M_n(\C)$.

\begin{enumerate}[label=\textbf{\arabic*.}]
\item\label{q:resolventformula} Montrer que pour tous $\lambda, \mu \in \C\setminus \sp(A)$, on a 
$$
(A-\lambda)^{-1}(A-\mu)^{-1} = \frac{(A-\lambda)^{-1} - (A-\mu)^{-1}}{\lambda - \mu}.
$$
\end{enumerate}
Soit $\lambda \in \sp(A)$. On fixe $\rho > 0$ tel que
\begin{equation}\label{eq:petitrho}
\bar{D}(\lambda, \rho) \cap \sp(A) = \{\lambda\}.
\end{equation}
 On d\'efinit la matrice $\Pi_\lambda \in M_n(\C)$ par
$$
\Pi_\lambda = \frac{1}{2\pi i} \int_{\CC_{\lambda,\rho}} (z-A)^{-1} \dd z,
$$
o\`u l'on a int\'egr\'e sur le contour $\CC_{\lambda, \rho}$ dans le sens anti-horaire.
\begin{enumerate}[resume, label=\textbf{\arabic*.}]
\item Montrer que $\Pi_\lambda$ ne d\'epend pas du choix de $\rho$ v\'erifiant (\ref{eq:petitrho}).

\item En utilisant la question \ref{q:resolventformula}, montrer que $\Pi_\lambda^2 = \Pi_\lambda$.

\item Soit $z \mapsto B(z) \in M_n(\C)$ une application holomorphe sur un ouvert $U \subset \C$ et $d(z) = \det B(z)$.  Montrer que pour tout $z \in U$ tel que $d(z) \neq 0$,
$$
d'(z) = d(z) ~ \tr\bigl(B(z)^{-1} B'(z)\bigr).
$$
\item En d\'eduire que $\tr ~\Pi_\lambda = \dim_\C C_{\lambda,\C}$ et que l'application induite par $\Pi_\lambda$ est un projecteur d'image $C_{\lambda,\C}$.

\item Montrer que les matrices $\left\{\Pi_\mu, \mu \in \sp(M)\right\}$  sont les matrices des projecteurs spectraux complexes associ\'es \`a $A$, i.e. les projecteurs associ\'es \`a la d\'ecomposition
$$
\C^n = \bigoplus_{\mu \in \sp(M)} C_{\mu,\C}.
$$

\item Soit $U \subset \C$ un ouvert born\'e tel que $\partial U \cap \sp(A) = \emptyset$. Montrer qu'il existe un voisinage $\mathcal{U}$ de $A$ dans $M_n(\C)$ tel que l'application $\Pi_U : \mathcal{U} \to M_n(\C)$ d\'efinie par
$$
\Pi_U(M) = \sum_{\lambda \in \sp(M) \cap U} \Pi_{\lambda}(M), \quad M \in \mathcal{U},
$$
o\`u $\Pi_\lambda(M)$ est le projecteur spectral sur l'espace caract\'eristique de $M$ associ\'e \`a $\lambda$, est holomorphe en chaque coefficient de $M$.

\end{enumerate}
\noindent On suppose maintenant que $A \in M_n(\R)$.
\begin{enumerate}[resume, label=\textbf{\arabic*.}]
\item On suppose que $\lambda \in \R$. Montrer que $\Pi_\lambda$ est \`a coefficients r\'eels.
\item On suppose que $\lambda \in \C \setminus \R$. Montrer que
$$
\Pi_{\lambda, \bar{\lambda}} = \Pi_\lambda + \Pi_{\bar\lambda}.
$$
est \`a coefficients r\'eels et que $\Pi_{\lambda,\bar\lambda}^2 = \Pi_{\lambda, \bar \lambda}$.

\item Montrer que les matrices $\{\Pi_\mu\} \cup \left\{\Pi_{\mu, \bar\mu}\right\}$ sont les matrices des projecteurs spectraux r\'eels associ\'es \`a $A$, i.e. les projecteurs associ\'es \`a la d\'ecomposition
$$
\R^n = \left(\bigoplus_{\mu \in \R} C_\mu\right) \oplus \left(\bigoplus_{\Im \mu > 0} C_{\mu, \bar\mu}\right).
$$



\end{enumerate}

\section*{Classification topologique des flots contractants}

Soit $M_n^{-}(\R) = \{A \in M_n(\R),~\sp(A) \subset \R_{<0}\}$. Pour $A \in M_n^{-}(\R)$ on note
$$
\alpha(A)= - \sup_{\lambda \in \sp(A)} \Re{\lambda} > 0.
$$
Soit $f : \R_{>0} \to \R_{>0}$ une fonction continue telle que $f(t) < t$ pour tout $t > 0$. On note $\beta = f \circ \alpha : M_n^{-}(\R) \to \R_{>0}$.

\begin{enumerate}[resume, label=\textbf{\arabic*.}]
\item Montrer que $A \mapsto \alpha(A)$ est continue $M_n^{-}(\R) \to \R_{>0}$.
\item Montrer que l'on peut trouver une famille de normes $\{\|\cdot\|_A,~A \in M_n^{-}(\R)\}$ telle que l'application $(A,x) \mapsto \|x\|_A $ est continue $M_n^{-}(\R) \times \R^n \to \R_{\geqslant 0}$ et
$$
\left\|\e^{tA}x\right\|_A \leqslant \e^{-t\beta(A)}\|x\|_A, \quad t \geqslant 0.
$$
\end{enumerate}
Pour tout $A \in M_n^{-}(\R)$ on note $S_A = \{x \in \R^n,~\|x\|_A = 1\}$.
\begin{enumerate}[resume, label=\textbf{\arabic*.}]

\item Montrer que pour tout $A \in M_n^{-}(\R)$ et tout $x \in \R^n \setminus \{0\}$, il existe un unique $\tau_A(x) \in \R$ tel que $\e^{\tau_A(x)A}x \in S_A.$
\end{enumerate}
Pour $A \in M_n^{-}(\R)$ on d\'efinit $\varphi(A) : \R^n \to \R^n$ par $\varphi(A)(0) = 0$ et 
$$
\varphi(A)(x) = \e^{\tau_A(x)}\bigl(\e^{\tau_A(x)A}x\bigr), \quad x \in \R^n \setminus \{0\}.
$$ 
\begin{enumerate}[resume, label=\textbf{\arabic*.}]
\item Montrer que l'application $(A,x) \mapsto \varphi(A)(x)$ est continue $M_n^{-}(\R) \times \R^n \to \R^n$.

\item Montrer que $\varphi(A)$ est un hom\'eomorphisme de $\R^n$ dans lui-m\^eme et que
$$
\e^{-t} \varphi(A) = \varphi(A) \circ \e^{tA}, \quad t \in \R.
$$

\item En d\'eduire que pour toutes matrices $A,B \in M_n^{-}(\R)$, les flots $\e^{tA}$ et $\e^{tB}$ sont conjugu\'es.

\end{enumerate}

\section*{Stabilit\'e structurelle des flots lin\'eaires hyperboliques}

\noindent On note $\mathrm{Hyp}_n(\R) \subset \mathrm{GL}(n,\R)$ les matrices r\'eelles engendrant un flot hyperbolique, c'est-\`a-dire les matrices dont toutes les valeurs propres ont une partie r\'eelle non nulle. Pour $A \in \mathrm{Hyp}_n(\R)$ on note
$$
m(A) = \sum_{\Re(\lambda) > 0} \dim_\C C_{\lambda, \C},
$$
On note aussi
$$
E^s(A) = \left\{x \in \R^n, ~\e^{tA}x \underset{t \to +\infty}{\longrightarrow} 0\right\}, \quad E^u(A) = \left\{x \in \R^n, ~\e^{tA}x \underset{t \to -\infty}{\longrightarrow} 0\right\}.
$$


\begin{enumerate}[resume, label=\textbf{\arabic*.}]

\item Montrer que $\mathrm{Hyp}_n(\R)$ est ouvert dans $M_n(\R)$ et que 
$$\R^n = E^s(A) \oplus E^u(A)$$ pour tout $A \in \mathrm{Hyp}_n(\R)$. On notera $\pi_s(A)$ et $\pi_u(A)$ les projections associ\'ees \`a cette d\'ecomposition.


\item Montrer que $A \mapsto (\pi_{s}(A), \pi_u(A))$ est continue $\mathrm{Hyp}_n(\R) \to \mathcal{L}(\R^n)^2$.

\end{enumerate}
On fixe dans la suite $A \in \mathrm{Hyp}_n(\R)$.  

\begin{enumerate}[resume, label=\textbf{\arabic*.}]

\item Montrer qu'il existe un voisinage $\mathcal{U}$ de $A$ dans $\mathrm{Hyp}_n(\R)$ tel que pour tout $M \in \mathcal{U}$, l'application
$$
\pi_s(M)|_{E^s(A)} : E^s(A) \to E^s(M)
$$
est un isomorphisme. 

\end{enumerate}
Pour $M \in \mathcal{U}$ on note $q_s(M) : E^s(M) \to E^s(A)$ l'inverse de $\pi_s(M)|_{E^s(A)}$ et 
$$\widetilde{M} = q_s(M) M \pi_s(M)|_{E^s(A)} : E^s(A) \to E^s(A).$$

\begin{enumerate}[resume, label=\textbf{\arabic*.}]

\item Montrer que $M \mapsto \widetilde{M}$ est continue $\mathcal{U} \to \mathcal{L}(E^s(A)).$

\item Montrer qu'il existe une application continue $\widetilde{\Phi}_s : \mathcal{U} \times E^s(A) \to E^s(A)$ telle que $\widetilde{\Phi}_s(M, \cdot)$ est un hom\'eomorphisme de $E^s(A)$ et
$$
\e^{tA}\widetilde{\Phi}_s(M,x_s) = \widetilde{\Phi}_s(M, \e^{t\widetilde{M}}x_s), \quad t \in \R, \quad x_s \in E^s(A).
$$

\item D\'emontrer le r\'esultat voulu. \\
\textit{Indication : on pourra consid\'erer l'application $\Phi_s$ d\'efinie par
$$
\Phi_s(M,x) = \widetilde{\Phi}_s(M, q_s(M)\pi_s(M)x).
$$}
\end{enumerate}

\section*{Application : conjugaisons en famille}

\begin{enumerate}[resume, label=\textbf{\arabic*.}]
\item Montrer que les composantes connexes de $\mathrm{Hyp}_n(\R)$ sont exactement les ensembles $\mathcal{U}_j = \{A \in \mathrm{Hyp}_n(\R),~m(A) = j\}$ pour $j=0, \dots, n$.

\item Soient $j \in \{0, \dots, n\}$ et $A,B \in \mathcal{U}_j$ ; on se donne $M : [0,1] \to \mathcal{U}_j$ une application continue telle que $M(0) = A$ et $M(1) = B$. Montrer qu'il existe une application continue $\Psi : [0,1] \times \R^n \to \R^n$ telle que $\Psi(s, \cdot)$ est un hom\'eomorphisme de $\R^n$ pour tout $s \in [0,1]$ et 
$$
\e^{tA} \Psi(s, x) = \Psi(s, \e^{tM(s)}x), \quad t \in \R, \quad x \in \R^n.
$$

\end{enumerate}
\end{document} 
