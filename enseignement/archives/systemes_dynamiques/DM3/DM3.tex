\documentclass[a4paper,12pt]{article}
\usepackage[T1]{fontenc}
\usepackage[margin=2.8cm]{geometry}
\usepackage[applemac]{inputenc}
\usepackage{lmodern}
\usepackage{enumitem}
\usepackage{microtype}
\usepackage{hyperref}
\usepackage{enumitem}
\usepackage{dsfont}
\usepackage{amsmath,amssymb,amsthm}
\usepackage{mathenv}
\usepackage{amsthm}
\usepackage{graphicx}
\usepackage[all]{xy}
\usepackage{mathrsfs}
\theoremstyle{plain}
\newtheorem{thm}{Theorem}[section]
\newtheorem*{thm*}{Th\'eor\`eme}
\newtheorem{prop}[thm]{Proposition}
\newtheorem{cor}[thm]{Corollary}
\newtheorem{lem}[thm]{Lemma}
\newtheorem{propr}[thm]{Propri\'et\'e}
\theoremstyle{definition}
\newtheorem{deff}[thm]{Definition}
\newtheorem{rqq}[thm]{Remark}
\newtheorem{ex}[thm]{Exercice}
\newcommand{\e}{\mathrm{e}}
\newcommand{\prodscal}[2]{\left\langle#1,#2\right\rangle}
\newcommand{\devp}[3]{\frac{\partial^{#1} #2}{\partial {#3}^{#1}}}
\newcommand{\w}{\omega}
\newcommand{\dd}{\mathrm{d}}
\newcommand{\x}{\times}
\newcommand{\ra}{\rightarrow}
\newcommand{\pa}{\partial}
\newcommand{\vol}{\operatorname{vol}}
\newcommand{\dive}{\operatorname{div}}
\newcommand{\tr}{\mathrm{tr}}
\newcommand{\T}{\mathbf{T}}
\newcommand{\R}{\mathbf{R}}
\newcommand{\Z}{\mathbf{Z}}
\newcommand{\C}{\mathbf{C}}
\newcommand{\N}{\mathbf{N}}
\newcommand{\Q}{\mathbf{Q}}
\renewcommand{\x}{x}
\newcommand{\Homeo}{\mathrm{Homeo}}
\renewcommand{\sp}{\mathrm{sp}}
\newcommand{\CC}{\mathscr{C}}

\title{\textsc{Syst\`emes dynamiques} \\ DM 3}
\date{{Pour le 16/12/2021}}
\author{}

\begin{document}
{\noindent \'Ecole Normale Sup\'erieure  \hfill Pour toute question :} \\
{2021/2022 \hfill  \texttt{chaubet@dma.ens.fr}}

{\let\newpage\relax\maketitle}
\maketitle

\noindent On se propose ici de d\'emontrer le th\'eor\`eme suivant. \\

\noindent \textbf{Th\'eor\`eme.} \textit{Soit $f : \N \to \R_{\geq 0}$ une fonction. Alors
\begin{enumerate}[label=(\roman*)]
\item Si $\sum_n f(q)$ diverge et si $(qf(q))$ est d\'ecroissante, alors pour presque tout $x \in [0,1]$, il existe une infinit\'e de couples d'entiers $(p,q) \in \N_{\geq 1}^2$ tels que
\begin{equation}\label{eq:diop}
\left|x - \frac{p}{q}\right| < \frac{f(q)}{q}.
\end{equation}
\item Si $\sum_n f(q)$ converge, alors pour presque tout $x \in [0,1]$, le nombre de couples $(p,q)$ v\'erifiant (\ref{eq:diop}) est fini.
\end{enumerate}
}

\section*{\'Echauffement}
\begin{enumerate}[label=\textbf{\arabic*.}]

\item 

\begin{enumerate}

\item Pour tout $q \in \N_{\geq 1},$ on note $A_q \subset [0,1]$ l'ensemble des $x \in [0,1]$ tels que (\ref{eq:diop}) est vraie pour un $p \in \N$. Montrer que $\ell(A_q) \leq 2f(q),$ o\`u $\ell$ est la mesure de Lebesgue.

\item Montrer le second point du \textbf{Th\'eor\`eme}.
\end{enumerate}
\end{enumerate}

\section*{D\'eveloppement en fractions continues}

\noindent Dans tout ce qui suit, $I$ d\'esigne l'intervalle $[0,1]$. Pour tout $x \in \R$ on notera $[x]$ sa partie enti\`ere et $\{x\}$ sa partie fractionnaire, i.e.
$$
x = [x] + \{x\}, \quad [x] \in \N, \quad \{x\} \in [0,1).
$$
On d\'efinit deux applications $a : I \to \N_{\geq 1} \cup \{\infty\}$ et $T : I \to I$ par $a(0) = \infty$, $T(0) = 0$ et
$$
a(x) = \left[1/x\right], \quad T(x) = \left\{1/x\right\}, \quad x \neq 0.
$$
Pour tout $x \in I$ et $n \in \N_{\geq 1}$, on notera
$$
a_n(x) = a(T^{n-1}(x)).
$$
Enfin pour toute s\'equence $(a_1, \dots, a_m) \in (\N_{\geq 1} \cup \{\infty\})^m$ et $t \in [0,1)$ on notera
$$
[a_1, \dots, a_m; t] = \frac{1}{\displaystyle{a_1 + \frac{1}{\displaystyle{a_2 + \frac{1}{\displaystyle{\dots + {\displaystyle{\frac{1}{a_m + t}}}}}}}}},
$$
et $[a_1, \dots, a_m] = [a_1, \dots, a_m; 0].$


\begin{enumerate}[resume,label=\textbf{\arabic*.}]

\item Montrer que pour tout $m \geq 1$ et tout $x \in I$ on a 
$$
x = \left[a_1(x), \dots, a_m(x) ; T^m(x)\right].
$$

\item Montrer que $x \in I$ est rationnel si et seulement si il existe $n \geq 1$ tel que $T^n(x) = 0$.

\end{enumerate}

\noindent Soit $x \in I \setminus \Q$. On d\'efinit les suites d'entiers $(p_n(x))_{n \geq -1}, (q_n(x))_{n \geq -1}$, par $p_{-1}(x) = q_0(x) = 1$, $p_0(x) = q_{-1}(x) = 0$ et 
$$
p_n(x) = a_n(x) p_{n-1}(x) + p_{n-2}(x), \quad q_{n}(x) = a_n(x) q_{n-1}(x) + q_{n-2}(x), \quad n \geq 1.
$$
Si $x \in \Q$, on d\'efinit de m\^eme les nombres $p_n(x)$ et $q_n(x)$ pour tout $n$ tel que $n < n(x) = \min \{m \geq 1, ~T^m(x) = 0\}$.  \\

\noindent Dans la suite on fixe $x \in I$ et $1 \leq n < n(x)$.

\begin{enumerate}[label=\textbf{\arabic*.},resume]
\item 
\begin{enumerate}
\item Montrer que
$
p_{n-1}(x) q_n(x) - p_n(x) q_{n-1}(x) = (-1)^n.
$

\item Montrer que
$$
[a_1(x): \dots, a_n(x); t] = \frac{p_n(x) + tp_{n-1}(x)}{q_n(x) + tq_{n-1}(x)}, \quad t \in [0,1).
$$

\item En d\'eduire que
$$
\frac{1}{q_n(x)(q_n(x) + q_{n+1}(x))} \leq\left|x - \frac{p_n(x)}{q_n(x)}\right| \leq \frac{1}{\displaystyle{q_n(x)q_{n+1}(x)}}.
$$
\end{enumerate}

\item Montrer que 
$$
\left| \log \frac{x}{p_n(x)/q_n(x)}  \right| \leq \frac{1}{2^{n-2}}.
$$

\end{enumerate}

\section*{La mesure de Gauss}
On note $\mu$ la \textit{mesure de Gauss}, c'est-\`a-dire la mesure sur $I$ de densit\'e 
$$
\dd \mu(x) = \frac{1}{\log 2} \frac{\dd \ell (x)}{1+x},
$$
o\`u $\ell$ est la mesure de Lebesgue sur $I$.

\begin{enumerate}[label=\textbf{\arabic*.},resume]

\item Montrer que $T$ pr\'eserve la mesure de Gauss.

\end{enumerate}
\noindent Pour $a_1, \dots a_m \in \N_{\geq_1}$, on notera
$$
I_{a_1, \dots, a_m} = \left\{ x \in I, ~a_j(x) = a_j, ~ j = 1, \dots, m\right\}.
$$

\begin{enumerate}[label=\textbf{\arabic*.},resume]
\item
\begin{enumerate}
\item Montrer que $I_{a_1, \dots a_m}$ est l'image de $[0,1)$ par l'application $\psi_{a_1, \dots, a_m}$ d\'efinie par
$$
\psi_{a_1, \dots, a_m} (t) = [a_1, \dots, a_m ;t], \quad t \in [0,1).
$$
\item Montrer que $\displaystyle{\psi_{a_1, \dots, a_m}(t) = \frac{p_m + tp_{m-1}}{q_m + tq_{m-1}}}$, o\`u $(p_k)$ et $(q_k)$ sont d\'efinies par r\'ecurrence en terme des $a_k$ comme dans la partie pr\'ec\'edente.
\item Montrer que $\displaystyle{\ell(I_{a_1, \dots, a_m}) = \frac{1}{q_n(q_n +q_{n-1})}}$.
\item Montrer que la classe $\{I_{a_1, \dots, a_m}, ~m \in \N_{\geq 1},~a_1, \dots, a_m \in \N_{\geq 1}\} \cup \{I\}$ engendre la tribu des bor\'eliens sur $I$.
\end{enumerate}

\item Montrer que pour tout intervalle $J = [x,y) \subset I$ et tous $a_1, \dots, a_m \in \N_{\geq 1}$, on a 
$$
\frac{1}{2}\ell(J) \leq \frac{\ell(T^{-m}(J) \cap I_{a_1,\dots,a_m})}{\ell(I_{a_1,\dots,a_m})} \leq 2 \ell(J).
$$

\item Montrer que $\mu$ est ergodique pour $T$.
\end{enumerate}

\section*{Applications aux approximations diophantiennes}

\begin{enumerate}[label=\textbf{\arabic*.},resume]

\item Montrer que pour tout $x \in I \setminus \Q$ et tout $n \geq 1$
$$
\frac{1}{q_n(x)} = \prod_{k=1}^n [a_k(x), \dots, a_n(x)].
$$
\item En d\'eduire que pour tout $x \in I \setminus \Q$, on a
$$
\frac{1}{n} \log \frac{1}{q_n(x)} = \frac{1}{n} \sum_{k=1}^n \log T^{k-1}(x) + O\left(\frac{1}{n}\right), \quad n \to + \infty.
$$
\item En d\'eduire que, pour presque tout $x$ de $I$,
$$
\lim_{n \to \infty} \frac{1}{n} \log \left|x - \frac{p_n(x)}{q_n(x)}\right| = - \frac{\pi^2}{6 \log 2}.
$$

\end{enumerate}
Pour toute suite $\mathbf{a} = (a_n)_{n \geq 1}$ de r\'eels strictement positifs, on note 
$$
A(\mathbf{a}) = \Bigl\{ x \in I, ~\# \{n \in \N_{\geq 1},~ a_n(x) > a_n\} < +\infty \Bigr\}.
$$

\begin{enumerate}[label=\textbf{\arabic*.},resume]

\item 
\begin{enumerate}
\item Montrer que si $\sum 1 / a_n$ converge alors $\mu(A(\mathbf{a})) = 1$.
\item Montrer que si $\sum 1 / a_n$ diverge alors $\mu(A(\mathbf{a}))=0.$
\end{enumerate}
\end{enumerate}

Dans la suite, on se donne $f : \N \to \R_{\geq 0}$ une fonction.

\begin{enumerate}[label=\textbf{\arabic*.},resume]

\item On suppose dans cette question que $\sum f(q)$ diverge et que $(qf(q))$ est d\'ecroissante, et on note $\varphi(n) = 4^n f(4^n)$ pour tout $n \geq 1.$

\begin{enumerate}
\item Montrer que pour presque tout $x \in I$, on a 
$$
\varphi(n) \leq q_n(x) f(q_n(x)),
$$
sauf pour un nombre fini de valeurs de $n \in \N_{\geq 1}.$
\item Montrer le point \textit{(i)} du \textbf{Th\'eor\`eme.}
\end{enumerate}
\end{enumerate}

\end{document}
