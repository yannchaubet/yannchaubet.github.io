\documentclass[a4paper,10pt,openany]{article}
\usepackage{fancyhdr}
\usepackage[T1]{fontenc}
\usepackage[margin=1.8cm]{geometry}
\usepackage[applemac]{inputenc}
\usepackage{lmodern}
\usepackage{enumitem}
\usepackage{microtype}
\usepackage{hyperref}
\usepackage{enumitem}
\usepackage{dsfont}
\usepackage{amsmath,amssymb,amsthm}
\usepackage{mathenv}
\usepackage{amsthm}
\usepackage{graphicx}
\usepackage[all]{xy}
\usepackage{lipsum}       % for sample text
\usepackage{changepage}
\theoremstyle{plain}
\newtheorem{thm}{Theorem}[section]
\newtheorem*{thm*}{Th\'eor\`eme}
\newtheorem{prop}[thm]{Proposition}
\newtheorem{cor}[thm]{Corollary}
\newtheorem{lem}[thm]{Lemma}
\newtheorem{propr}[thm]{Propri\'et\'e}
\theoremstyle{definition}
\newtheorem{deff}[thm]{Definition}
\newtheorem{rqq}[thm]{Remark}
\newtheorem{ex}[thm]{Exercice}
\newcommand{\e}{\mathrm{e}}
\newcommand{\prodscal}[2]{\left\langle#1,#2\right\rangle}
\newcommand{\devp}[3]{\frac{\partial^{#1} #2}{\partial {#3}^{#1}}}
\newcommand{\w}{\omega}
\newcommand{\dd}{\mathrm{d}}
\newcommand{\x}{\times}
\newcommand{\ra}{\rightarrow}
\newcommand{\pa}{\partial}
\newcommand{\vol}{\operatorname{vol}}
\newcommand{\dive}{\operatorname{div}}
\newcommand{\T}{\mathbf{T}}
\newcommand{\R}{\mathbf{R}}
\newcommand{\Z}{\mathbf{Z}}
\newcommand{\N}{\mathbf{N}}
\newcommand{\F}{\mathcal{F}}
\newcommand{\Homeo}{\mathrm{Homeo}}
\newcommand{\Matn}{\mathrm{Mat}_{n \times n}}
\DeclareMathOperator{\tr}{tr}
\newcommand{\id}{\mathrm{id}}
\newcommand{\htop}{h_\mathrm{top}}


\title{\textsc{Syst\`emes dynamiques} \\ Feuille d'exercices 2}
\date{}
\author{}

\begin{document}

{\noindent \'Ecole Normale Sup\'erieure  \hfill Yann Chaubet } \\
{2021/2022 \hfill \texttt{chaubet@dma.ens.fr}}

{\let\newpage\relax\maketitle}
\maketitle
\noindent {\large \textbf{Exercice 1.} \textit{Familles d'applications transitives}} \vspace{0.15cm}

\noindent Soit $X$ un espace topologique s\'epar\'e, \`a base d\'enombrable, localement compact et sans points isol\'es.  Soit $(f_i)_{i \in \N}$ une famille d'applications continues et topologiquement transitives. Montrer qu'il existe $x \in X$ tel que $\omega_{f_i}(x) = X$ pour tout $i \in \N$.

\vspace{0.6cm}
%% \\

\noindent {\large \textbf{Exercice 2.} \textit{Transformations minimales}} \vspace{1.5mm}

\noindent Soit $X$ un espace topologique s\'epar\'e. On dira qu'une transformation continue $f : X \to X$ est \textit{minimale} si 
pour tout ferm\'e non vide $Y \subset X$ on a 
$$
f(Y) \subset Y \implies Y = X.
$$
On dira qu'une partie ferm\'ee invariante $Y \subset X$ est \textit{minimale} pour $f$ si $f|_Y : Y \to Y$ est minimale.
\begin{enumerate}
\item Montrer que si $X$ est compact, toute transformation minimale de $X$ est topologiquement transitive.
\item En utilisant l'axiome du choix, montrer que si $X$ est compact, alors toute transformation continue de $X$ admet une partie ferm\'ee minimale non vide.
\item En d\'eduire que si $X$ est compact, alors toute application continue de $X$ a un point positivement r\'ecurrent.
\end{enumerate}
\vspace{0.6cm}

\noindent {\large \textbf{Exercice 3.} \textit{Ensemble non-errant}} \vspace{1.5mm} 

\noindent Soit $X$ un espace topologique s\'epar\'e et $f : X \to X$ une transformation continue. On dira que $x \in X$ est \textit{non errant} si pour tout voisinage $U$ de $x$, il existe $n \in \N^*$ tel que $f^n(U) \cap U \neq \emptyset$. On note $\Omega(f)$ l'ensemble des points non errants.
\begin{enumerate}
\item Montrer que si $x \in X$ est non errant et $U$ un voisinage de $x$, alors pour tout $m \in \N$ il existe $n > m$ tel que $f^{n}(U)\cap U \neq \emptyset$.
\item Montrer que $\Omega(f)$ est un ferm\'e invariant et qu'il contient tous les ensembles $\omega$-limites (et $\alpha$-limites si $f$ est inversible) de tous les points.
\item Montrer que l'on a 
$$
\mathrm{Per}(f) \subset M(f) \subset R(f) \subset \Omega(f),
$$
o\`u $\mathrm{Per}(f)$ est l'ensemble des points p\'eriodiques de $f$, $M(f)$ est la fermeture de l'union de toutes les parties minimales pour $f$ et $R(f)$ est la fermeture de l'ensemble des points r\'ecurrents pour $f$.
\end{enumerate}

\vspace{0.6cm}

\noindent {\large \textbf{Exercice 4.} \textit{Entropie d'un flot}} \vspace{1.5mm} \\
\noindent Soit $(X, \dd)$ un espace m\'etrique compact et $\Phi = \{\varphi^t\}_{t \in \R}$ un flot continu sur $X$. On d\'efinit l'entropie $\htop(\Phi)$ du flot $\Phi$ de la m\^eme mani\`ere que dans le cas discret, en consid\'erant les distances
$$
\dd_T(x,y) = \max_{0 \leq t \leq T-1} \dd\left(\varphi^t(x), \varphi^t(y)\right).
$$
Montrer que 
$$
\htop(\Phi) = \htop(\varphi^1).
$$
\vspace{0.6cm}

\noindent {\large \textbf{Exercice 5.} \textit{Propri\'et\'es de l'entropie topologique}} \vspace{1.5mm} 

\noindent Soient $(X, \dd_X), (Y, \dd_Y)$ des espaces m\'etriques compacts et des transformations continues $ f : X \to X$ et $g : Y \to Y$. 
\begin{enumerate}
\item Soit $\Lambda \subset X$ un ferm\'e $f$-invariant. Montrer que $\htop(f|_\Lambda) \leq \htop(f)$.
\item Soient $\Lambda_1, \dots, \Lambda_m$ des ferm\'es $f$-invariants de $X$ tels que $X = \bigcup_{j=1}^m \Lambda_j$. Montrer que 
$
\htop(f) = \max_{1\leq j \leq m} \htop(f|_{\Lambda_j}).
$
\item Montrer que $\htop(f^k) = |k| \htop(f)$ pour tout $k \in \Z$ ($k \in \N$ si $f$ n'est pas inversible).
\item Montrer que si $\dd_X'$ est une autre m\'etrique sur $X$ engendrant la m\^eme topologie que $\dd_X$, alors $\htop^{\dd_X}(f) = \htop^{\dd_X'}(f).$
\item Montrer que $\htop(f \times g) = \htop(f) + \htop(g)$ o\`u $f \times g : X \times Y \to X \times Y$ est donn\'ee par $(f \times g) (x,y) = (f(x),f(y))$ et o\`u $X \times Y$ est muni de la distance $\dd_{X \times Y}\left((x,y),(x',y')\right) = \max \left(\dd_X(x,x'), \dd_Y(y,y')\right).$
\end{enumerate}
\vspace{0.6cm}


\noindent {\large \textbf{Exercice 6.} \textit{Entropie des transformations Lipschitziennes}} \vspace{1.5mm} 

\noindent Soit $(X,\dd)$ un espace m\'etrique compact. On d\'efinit
$$
\mathrm{bdim}(X) = \limsup_{\varepsilon \to 0}\frac{\log M(X,\varepsilon)}{\log1/\varepsilon}
$$
o\`u $M(X,\varepsilon)$ est le nombre minimal de $\varepsilon$-boules (pour la distance $\dd$) qu'il faut pour recouvrir $X$. 
\begin{enumerate}
\item Montrer que $\mathrm{bdim} \left([0,1]^n\right) = n$. 
\end{enumerate}
Soit $f : X \to X$ une application Lipschitzienne et
$$
L(f) = \sup_{x \neq y} \frac{\dd(f(x), f(y))}{\dd(x,y)}
$$
sa constante de Lipschitz.

\begin{enumerate}[resume]
\item Montrer que 
\begin{equation}\label{eq:upperboundentropy}
h_{\mathrm{top}}(f) \leq \mathrm{bdim}(X) \max(0, \log L(f)).
\end{equation}
\item Donner un exemple d'application $f$ telle que (\ref{eq:upperboundentropy}) soit une \'egalit\'e.
\end{enumerate}
\vspace{0.6cm}

\noindent {\large \textbf{Exercice 7.} \textit{Entropie alg\'ebrique}} \vspace{1.5mm} 

\noindent Soit $G$ un groupe finiment engendr\'e et $\Gamma = \{\gamma_1, \dots, \gamma_s\}$ un syst\`eme de g\'en\'erateur. Pour $\gamma \in G$ on d\'efinit
$$
L(\gamma, \Gamma) = \min \left\{\left. \sum_{j=1}^{ks}|i_j|~\right| ~\gamma = \gamma_1^{i_1} \cdots \gamma_s^{i_s} \gamma_1^{i_{s+1}} \cdots \gamma_s^{i_{2s}} \cdots \gamma_s^{i_{ks}}, ~i_j \in \Z,~ k \in \N\right\}.
$$
Si $F \in \mathrm{Hom}(G,G)$ est un morphisme de groupe on note
$$
L_n(F,\Gamma) = \max_{1 \leq i \leq s} L(F^n\gamma_i, \Gamma), \quad n \in \N.
$$
\begin{enumerate}
\item Montrer que la limite
$$h(F, \Gamma) = \lim_{n \to \infty} \frac{1}{n} \log L_n(F, \Gamma)$$
existe. 
\item Montrer que si $\Gamma'$ est un autre syst\`eme de g\'en\'erateurs, alors $h(F, \Gamma) = h(F, \Gamma')$. 
\end{enumerate}
\noindent On d\'efinit l'\textit{entropie alg\'ebrique} $h_\mathrm{alg}(f)$ de $f$ par $h_\mathrm{alg}(F) = h(F,\Gamma)$ pour n'importe quel syst\`eme de g\'en\'erateur $\Gamma$.
\begin{enumerate}[resume]
\item Montrer que $h_\mathrm{alg}(I_{\gamma_0} F) = h_\mathrm{alg}(F)$ pour tout $\gamma_0 \in G$ o\`u $I_{\gamma_0} \in \mathrm{Hom}(G,G)$ est d\'efini par $I_{\gamma_0}(\gamma) = \gamma_0^{-1} \gamma \gamma_0.$
\end{enumerate}
\noindent Soit $M$ une vari\'et\'e connexe compacte, $x_\star \in M$ et $G = \pi_1(M, x_\star)$. Soit $\alpha$ un chemin dans $M$ joignant $x_\star$ \`a $f(x_\star).$ Soit $f$ une transformation continue de $M$ ; on d\'efinit $F_{x_\star, \alpha} \in \mathrm{Hom}(G,G)$ par
$$
F_{x_\star, \alpha} \gamma = \alpha^{-1} (f \circ\gamma) \alpha.
$$
\begin{enumerate}[resume]
\item On admet que $G$ est finiment engendr\'e. Montrer que $h_\mathrm{alg}(F_{x_\star, \alpha})$ ne d\'epend pas des choix de $x_\star$ et de $\alpha$. 
\end{enumerate}
\noindent Le nombre $h_\mathrm{alg}(f)$ d\'efini par $h_\mathrm{alg}(f) = h_\mathrm{alg}(F_{x_\star, \alpha})$ pour n'importe quel choix de $x_\star, \alpha$ est appel\'e \textit{entropie alg\'ebrique} de $f$. On peut montrer que 
$$
h_\mathrm{alg}(f) \leq h_\mathrm{top}(f).
$$

\iffalse
\noindent {\large \textbf{Exercice 5.} \textit{Croissance des orbites p\'eriodiques et entropie des applications expansives}} \vspace{1.5mm} 

\noindent Soit $(X, \dd)$ un espace m\'etrique compact et $f : X \to X$ une application expansive, c'est-\`a-dire qu'il existe $\delta > 0$ tel que pour tous $x, y \in X$,
$$
\quad \sup_{n\in \Z} \dd(f^n(x),f^n(y)) \leq \delta \implies x = y.
$$
Pour tout $n \in \N$, on note
$$
p_n(f) = \#\{x \in X, f^n(x) = x\}.
$$
On d\'efinit aussi le taux de croissance exponentielle de la s\'equence $p_n(f)$,
$$
p(f) = \limsup_{n \to \infty} \frac{\log(1 + p_n(f))}{n}.
$$
\begin{enumerate}
\item Montrer que $p_n(f)$ est fini pour tout $n \in \N.$
\item Montrer que 
\begin{equation}\label{eq:lowerboundentropy}
p(f) \leq h_{\mathrm{top}}(f),
\end{equation}
\item Donner un exemple d'application $f$ telle que (\ref{eq:lowerboundentropy}) soit une \'egalit\'e.
\item Montrer que pour toute matrice $A \in \mathrm{GL}(m,\mathbb{Z})$ hyperbolique (i.e. dont les valeurs propres sont toutes de module diff\'erent de $1$), on a
$$
\sum_{\substack{\lambda \in \mathrm{sp}(A) \\ |\lambda| > 1 }} \log |\lambda| \leq h_\mathrm{top}(f_A),
$$
o\`u $f_A : \T^m \to \T^m$ est l'automorphisme toral associ\'e \`a $A$.
\end{enumerate}


\vspace{0.6cm}
\fi



%%%


\end{document} 
