\documentclass[a4paper,12pt,openany]{article}
\usepackage{fancyhdr}
\usepackage[T1]{fontenc}
\usepackage[margin=1.8cm]{geometry}
\usepackage[applemac]{inputenc}
\usepackage{lmodern}
\usepackage{enumitem}
\usepackage{microtype}
\usepackage{hyperref}
\usepackage{enumitem}
\usepackage{dsfont}
\usepackage{amsmath,amssymb,amsthm}
\usepackage{mathenv}
\usepackage{amsthm}
\usepackage{graphicx}
\usepackage{mathrsfs}
\usepackage[all]{xy}
\usepackage{lipsum}       % for sample text
\usepackage{changepage}
\theoremstyle{plain}
\newtheorem{thm}{Theorem}[section]
\newtheorem*{thm*}{Th\'eor\`eme}
\newtheorem{prop}[thm]{Proposition}
\newtheorem{cor}[thm]{Corollary}
\newtheorem{lem}[thm]{Lemma}
\newtheorem{propr}[thm]{Propri\'et\'e}
\theoremstyle{definition}
\newtheorem{deff}[thm]{Definition}
\newtheorem{rqq}[thm]{Remark}
\newtheorem{ex}[thm]{Exercice}
\newcommand{\e}{\mathrm{e}}
\newcommand{\prodscal}[2]{\left\langle#1,#2\right\rangle}
\newcommand{\devp}[3]{\frac{\partial^{#1} #2}{\partial {#3}^{#1}}}
\newcommand{\w}{\omega}
\newcommand{\dd}{\mathrm{d}}
\newcommand{\x}{\times}
\newcommand{\ra}{\rightarrow}
\newcommand{\pa}{\partial}
\newcommand{\vol}{\operatorname{vol}}
\newcommand{\dive}{\operatorname{div}}
\newcommand{\T}{\mathbf{T}}
\newcommand{\R}{\mathbf{R}}
\newcommand{\Z}{\mathbf{Z}}
\newcommand{\N}{\mathbf{N}}
\newcommand{\C}{\mathbf{C}}
\newcommand{\F}{\mathcal{F}}
\newcommand{\Homeo}{\mathrm{Homeo}}
\newcommand{\Matn}{\mathrm{Mat}_{n \times n}}
\DeclareMathOperator{\tr}{tr}
\newcommand{\id}{\mathrm{id}}
\newcommand{\Id}{\mathrm{Id}}
\newcommand{\htop}{h_\mathrm{top}}


\title{\textsc{Syst\`emes dynamiques} \\ Corrig\'e 4}
\date{}
\author{}

\begin{document}

{\noindent \'Ecole Normale Sup\'erieure  \hfill \texttt{chaubet@dma.ens.fr}} \\
{2021/2022 \hfill}

{\let\newpage\relax\maketitle}
\maketitle

\noindent {\large \textbf{Exercice 1.} \textit{Exposants de Lyapunov pour les syst\`emes lin\'eaires}} \vspace{1.5mm} 


\begin{enumerate}
\item On a 
$$
\exp(-|t| \|A\|) \leq \|\exp(tA)\| \leq \exp(|t|\|A\|), \quad t \in \R.
$$
En particulier si $x_0 \neq 0$ on a 
$$
-\|A\| + \frac{1}{|t|} \log \|x_0\| \leq \frac{1}{t} \log \|\e^{tA}x_0\| \leq \|A\| + \frac{1}{|t|} \log \|x_0\|, \quad t \in \R.
$$
Ceci montre que $\lambda(x_0, A)$ est fini. Si $\|\cdot\|'$ est une autre norme sur $\R^n$, alors il existe des constantes $c,C > 0$ telles que
$$
\log c + \log \|\e^{tA}x_0\| \leq \log \|\e^{tA}x_0\|' \leq \log C + \log \|\e^{tA}x_0\|, \quad t \in \R,
$$
ce qui conclut.
\item On a 
$$
\begin{aligned}
\lambda(y_0, B) &= \limsup \frac{1}{t} \log \|\e^{tB}y_0\| \\
&= \limsup \frac{1}{t} \log \|\e^{tP^{-1}AP}y_0 \| \\
&= \limsup \frac{1}{t} \log \|P^{-1}\e^{tA}Py_0\| \\
&= \lambda(Py_0, A),
\end{aligned}
$$
par la question 1., puisque $\|P^{-1} \cdot\|$ est une norme sur $\R^n$.
\end{enumerate}

\begin{enumerate}[resume]
\item Soit $\lambda \in \R$ ; on suppose que $A$ est un bloc de Jordan pour $\lambda$, de sorte que
\begin{equation}\label{eq:jordan1}
A = \begin{pmatrix} \lambda & 1 & & (0) \\  & \ddots & \ddots&  \\ & & \ddots & 1 \\ (0) & & & \lambda \end{pmatrix},
 \quad  \quad
\e^{tA} = \exp(\lambda t) 
\begin{pmatrix}
1 & t & \cdots &\displaystyle{ \frac{t^{m-1}}{(m-1)!}} \\
& \ddots & & \vdots \\
& & \ddots & t \\
(0) & & & 1
\end{pmatrix}.
\end{equation}
Soit $x_0 = \begin{pmatrix} \alpha_1 \\ \vdots \\ \alpha_n \end{pmatrix} \in \R^n \setminus 0$. On note $\e^{tA}x_0 = \begin{pmatrix} \alpha_1(t) \\ \vdots \\ \alpha_n(t) \end{pmatrix}$ ; on a 
$$
\alpha_j(t) = \e^{\lambda t}\sum_{k=j}^n \frac{t^{k-j}}{(k-j)!}\alpha_k, \quad j = 1, \dots, n, \quad t \in \R.
$$
Soit $i = \max\{j \in \{1, \dots, n\}, ~\alpha_j \neq 0\}.$ Alors $\alpha_i(t) = \e^{\lambda t} \alpha_i$ pour tout $t$, et donc
$$
\log  \|\e^{tA}x_0\|_\infty \geq \lambda t + \log |\alpha_i|.
$$
On a aussi, pour un certain polyn\^ome $P$,
$$
\log \|\e^{tA}x_0\|_\infty \leq \lambda t + \log |P(t)| + \log \|x_0\|_\infty, \quad t \in \R.
$$
On obtient bien
$$
\lim_{|t|\to \infty} \frac{1}{t} \log \|\e^{tA}x_0\| = \lambda.
$$
On suppose maintenant que $\lambda = r + i \nu$ avec $r, \nu \in \R$, et que $A$ est un bloc de Jordan pour $\lambda$, i.e. $n=2m$ est pair et
\begin{equation}\label{eq:jordan2}
A = \begin{pmatrix} D & I_2 & & (0) \\  & \ddots & \ddots&  \\ & & \ddots & I_2 \\ (0) & & & D \end{pmatrix}, \quad D = \begin{pmatrix} r & -\nu \\ \nu & r \end{pmatrix}.
\end{equation}
Soit $x_0 = \begin{pmatrix} \alpha_1 \\ \beta_1 \\ \vdots \\ \alpha_m \\ \beta_m \end{pmatrix}  \in \R^n \setminus 0$. Alors en notant $\alpha_j(t)$ et $\beta_j(t)$ les coordonn\'ees $2j-1$ et $2j$ de $\e^{tA}x_0$ ($j = 1, \dots, m$), on a, pour tout $t \in \R$,
$$
\begin{aligned}
\alpha_j(t) = \e^{rt}\sum_{k=j}^m (\alpha_k \cos(\nu t) - \beta_k \sin(\nu t)) \frac{t^{k-j}}{(k-j)!}, \\
\beta_j(t) = \e^{rt}\sum_{k=j}^m (\alpha_k \sin(\nu t) + \beta_k \cos(\nu t)) \frac{t^{k-j}}{(k-j)!}.
\end{aligned}
$$
Soit $i = \max\{j \in \{1, \dots, m\},~(\alpha_j, \beta_j) \neq (0, 0)\}.$ Alors
$$
\alpha_i(t) = \e^{rt}(\alpha_i \cos(\nu t) - \beta_i \sin(\nu t)), \quad \beta_i(t) = \e^{rt}(\alpha_i \sin(\nu t) + \beta_i \cos(\nu t)), \quad t \in \R.
$$
En particulier $\|(\alpha_i(t), \beta_i(t))\|_2= \e^{rt} \|(\alpha_i, \beta_i)\|_2$, ce qui donne 
$$
\log\|e^{tA}x_0\|_2 \geq rt + \log \|(\alpha_i, \beta_i)\|_2, \quad t \in \R.
$$
On a par ailleurs, pour un certain polyn\^ome $P$,
$$
\log\|\e^{tA}x_0\|_2 \leq rt + \log |P(t)|, \quad t \in \R.
$$
Cela donne encore une fois 
$$
\lim_{|t|\to \infty} \frac{1}{t} \log \|\e^{tA}x_0\| = r.
$$
Dans le cas o\`u $A$ n'est pas un bloc de Jordan, le th\'eor\`eme de d\'ecomposition de Jordan et la question 2. permettent de se ramener aux cas pr\'ec\'edents.

\item Soit $v \in V_j \setminus V_{j+1}$. Alors $v$ est de la forme $v = v' + w$ avec $v' \in L_j \setminus 0$ et $w \in V_{j+1}.$ On sait par la question pr\'ec\'edente que
$$
\lim_{t\to+\infty} \frac{1}{t} \log \|\e^{tA}v'\| = r_j.
$$
En particulier, pour tout $r < r_j$, il existe $C>0$ telle que
$$
\|\e^{tA}v'\|\geq C\e^{rt}, \quad t \in \R.
$$
On fixe deux r\'eels $r,r'$ tels que $r_{j+1} < r' < r < r_j$. Alors il existe $C'>0$ telle que $\|\e^{tA}w\|\leq C' \e^{r't}$ pour tout $t\geq 0$. En particulier on a $\|\e^{tA}w\|/\|\e^{tA}v'\|\to 0$ quand $t \to + \infty$. Ceci donne alors
$$
\frac{1}{t}\log\|\e^{tA}v\| = \frac{1}{t} \log \|\e^{tA}v' + \e^{tA}w\|\to r_j, \quad t \to +\infty.
$$
La r\'eciproque est alors imm\'ediate puisque $\R^n = \bigoplus_j L_j.$ 

On proc\`ede identiquement pour la seconde \'equivalence.

\item Soit $M \in U_{a,b}$ et
$$
d_M(z) = \frac{\chi_M'(z)}{\chi_M(z)}, \quad z \in \C \setminus \mathrm{sp}(M).
$$
o\`u $\chi_M$ est le polyn\^ome caract\'eristique de $M$. Alors $d$ est m\'eromorphe sur $\C$ et a un p\^ole simple en $\lambda$ pour tout $\lambda \in \mathrm{sp}(M)$, avec r\'esidu \'egal \`a $\dim_\C C_{\lambda,\C}$, o\`u 
$$
C_{\lambda, \C} = \{u \in \C^n,~\exists N \in \N,~(A-\lambda)^Nu = 0\}.
$$
En particulier, puisque pour tout $\lambda \notin \R$ on a $\dim_\C C_{\lambda, \C} = \frac{1}{2} \dim_\R C_{\lambda, \bar \lambda},$ on obtient
$$
\dim L(a, b, M) = \int_{\mathscr{C}_{a,b}} d_M(z) \dd z,
$$
o\`u $\mathscr{C}_{a,b}$ est un chemin lisse ferm\'e entourant dans le sens direct les valeurs propres de $M$ qui ont une partie r\'eelle dans $]a,b[$, et qui ne rencontre pas le spectre de $M$. On choisit aussi deux autres lacets $\mathscr{C}_{<a}$ et $\mathscr{C}_{>b}$ qui entourent dans le sens direct les valeurs propres de $M$ ayant une partie r\'eelle respectivement dans $]-\infty, a[$ et dans $]b, +\infty[$, et n'intersectant pas le spectre de $M$. 

En particulier on a $|\chi_M|>\varepsilon$ sur $\mathscr{C}_{<a} \cup \mathscr{C}_{a,b} \cup \mathscr{C}_{>b} $ pour un $\varepsilon > 0$. Par continuit\'e de $M' \mapsto \chi_M'$, on sait qu'il existe un voisinage connexe $U$ de $M$ tel que pour toute matrice $M' \in U$, on a $|\chi_{M'}|>\varepsilon/2$ sur $\mathscr{C}_{a,b}$. 

En particulier, les applications
$$
M' \mapsto \int_{\mathscr{C}} d_{M'}(z) \dd z, \quad \mathscr{C} = \mathscr{C}_{<a}, ~\mathscr{C}_{a,b},~\mathscr{C}_{>b},
$$
sont bien d\'efinies et continues $U \to \C$. Comme elles sont \`a valeurs dans $\Z$, elles sont constantes. En particulier on a
$$
n = \int_{\mathscr{C}_{<a}} d_{M'}(z) \dd z + \int_{\mathscr{C}_{a,b}} d_{M'}(z) \dd z + \int_{\mathscr{C}_{>b}} d_{M'}(z) \dd z,
$$
puisque cette \'egalit\'e est vraie pour $M' = M$ (par le lemme des noyaux). Ceci implique que toutes les valeurs propres de $M'$ sont contenues dans les zones d\'elimit\'ees par $\mathscr{C}_{<a}$, $\mathscr{C}_{a,b}$ et $\mathscr{C}_{>b}$, ce qui conclut.

\end{enumerate}
\vspace{0.6cm}

\noindent {\large \textbf{Exercice 2.} \textit{Stabilit\'e de $0$ pour les syst\`emes lin\'eaires}} \vspace{1.5mm} 


\begin{enumerate}
\item \underline{(i) $\implies$ (iii)}. Supposons que $0$ est un point fixe asymptotiquement stable. Soit $\lambda$ une valeur propre de $A$. Alors tout $u \in C_{\lambda, \bar \lambda}$ (avec les notations de l'exercice pr\'ec\'edent) v\'erifie $\frac{1}{t} \log \|\e^{tA}u\| \to \Re(\lambda)$ quand $t \to +\infty$. Puisque $\e^{tA}u \to 0$ on a n\'ecessairement $\Re(\lambda) \leq 0$. Supposons $\Re(\lambda) = 0$. Alors la correction de la question 3. de l'\textbf{Exercice 1.} montre qu'on peut trouver $u \in \R^n$ tel que $\|\e^{tA} u\| \geq \delta$ pour tout $t$, pour un $\delta > 0$. C'est absurde, donc $\Re(\lambda) < 0$. \newline

\underline{(iii) $\implies$ (ii)}. Pour tout $\lambda \in \mathrm{sp}(A)$ et $u \in C_{\lambda, \bar \lambda}$ on a pour un certain polyn\^ome $P$
$$
\|\e^{tA}u\| \leq C \e^{\Re(\lambda) t} |P(t)|, \quad t \geq 0.
$$
Puisque $\R^n = \bigoplus_\lambda C_{\lambda, \bar \lambda}$, on que pour tout $0 > a > \sup_{\lambda \in \mathrm{sp}(A)} \Re \lambda$, il existe $c > 0$ tel que 
\begin{equation}\label{eq:exp}
\|\e^{tA}x\| \leq c \e^{-at} \|x\|, \quad x \in \R^n, \quad t \geq 0.
\end{equation}

\underline{(ii) $\implies$ (iv)}. Par lin\'earit\'e de $\e^{tA}$, il existe $c, a >0$ tels que (\ref{eq:exp}) est v\'erifi\'ee. On pose alors
$$
\|x\|_A = \int_0^\tau \e^{bt} \|\e^{tA}x\| \dd t, \quad x \in \R^n,
$$
o\`u $b,\tau > 0$ satisfont 
$$
c\e^{-(a-b)\tau} < 1.
$$
On peut supposer $c \geq 1$, sinon la norme $\|\cdot\|$ convient. Ceci implique $a \geq b$. Alors
$$
\begin{aligned}
\|\e^{TA}x\|_A &= \int_0^\tau \e^{bt}\|\e^{(t+T)A}x\| \dd t \\
&= \int_T^{T+ \tau} \e^{b(t-T)}\|\e^{tA}x\| \dd t \\
&= \e^{-bT}\int_T^{T+\tau} \e^{bt}\|\e^{tA}x\| \dd t \\
&= \e^{-bT} \left( \int_T^0 \e^{bt} \|\e^{tA}x\| \dd t + \int_0^\tau \e^{bt} \|\e^{tA}x\| \dd t+ \int_\tau^{\tau + T} \e^{bt} \|\e^{tA}x\| \dd t\right) \\
&= \e^{-bT}\|x\|_A + \e^{-bT} \left(\int_\tau^{\tau+T} \e^{bt} \|\e^{tA}x\| \dd t - \int_0^T \e^{bt} \|\e^{tA}x\| \dd t\right).
\end{aligned}
$$
On a 
\iffalse
$$
\int_\tau^{\tau+T} \e^{bt} \|\e^{tA}x\| \dd t \leq T\e^{b(\tau+T)} c \e^{-a\tau} \|x\| \leq Tc \e^{-(a-b)\tau} \e^{bT} \|x\| \leq T\e^{bT} \|x\|.
$$
\fi
$$
\begin{aligned}
\int_\tau^{\tau+T} \e^{bt} \|\e^{tA}x\| \dd t &\leq c\int_{\tau}^{\tau + T} \e^{bt} \e^{-at} \|x\| \dd t \\
&\leq c\e^{-(a-b)\tau}\frac{\left(1- \e^{-(a-b)T}\right)}{a-b}\|x\|
\end{aligned}
$$
D'autre part on a
$$
\int_0^T \e^{bt} \|\e^{tA}x\| \dd t \geq \int_0^T \e^{bt} \e^{-t\|A\|} \|x\| \dd t \geq \frac{1 - \e^{-T(\|A\|-b)}}{\|A\|-b}\|x\|.
$$
On a
$$
c\e^{-(a-b)\tau}\frac{1 - \e^{-(a-b)T}}{a-b} - \frac{1 - \e^{-T(\|A\|-b)}}{\|A\|-b} =\left(c\e^{-(a-b)\tau} - 1\right) T + \mathcal{O}(T^2), \quad T \to 0.
$$
En particulier cette quantit\'e est n\'egative pour $T > 0$ assez petit. On a donc obtenu $\delta > 0$ tel que pour $T \leq \delta$ on a 
$$
\|\e^{TA}\|_{A} \leq \e^{-bT} \|x\|_A, \quad x \in \R^n.
$$
Soit $T' > 0$. On \'ecrit $T' = n\delta + T$ avec $0 \leq T < \delta$ et $n \in \N$. Alors pour tout $x \in \R^n$,
$$
\|\e^{T'A}x\|_A \leq \|\e^{\delta A} \cdots \e^{\delta A} \e^{TA} x\|_A \leq \e^{- bn\delta} \e^{-bT} \|x\|_A \leq \e^{-bT'} \|x\|_A,
$$
ce qui conclut.
\newline

\underline{(iv) $\implies$ (i)}. \'evident.

\item Supposons que $\lambda = r+ i \nu \in \mathrm{sp}(A)$ avec $\nu \in \R$ et $r\geq 0$. On suppose que $A$ est un bloc de Jordan associ\'e \`a $\lambda$, qui est non trivial (i.e. de la forme (\ref{eq:jordan1}) et de taille $>1$ ou de la forme (\ref{eq:jordan2}) et de taille $>2$ selon que $\nu$ soit nul ou non). Alors si $x = \begin{pmatrix} 0, \dots, 0, 1 \end{pmatrix}^{\perp}$, on v\'erifie ais\'ement que $ \limsup_{t\to +\infty}\|\e^{tA}x\| = +\infty$.

R\'eciproquement, supposons que les valeurs propres de $A$ ont toutes des parties r\'eelles n\'egatives ou nulles, et que les valeurs propres ayant une partie r\'eelle nulle v\'erifient la condition de semi-simplicit\'e. Alors on v\'erifie ais\'ement que $\sup_{t\in \R} \|\e^{tA}\| < + \infty$ (en regardant la d\'ecomposition de Jordan), ce qui conclut.
\end{enumerate}
\vspace{0.6cm}

\noindent {\large \textbf{Exercice 3.} \textit{Syst\`emes linaires topologiquement conjugu\'es}} \vspace{1.5mm} 

\begin{enumerate}
\item Soit $x \in \R^n$ non nul. Alors $\lim_{t \to \pm \infty} \log \|\e^{tA}x\|_A = \mp \infty$ ; de plus $t \mapsto \log \|\e^{tA}x\|_A$ est strictement d\'ecroissante sur $\R$ puisque
$$
\|\e^{tA}y\|_A \leq \e^{-at} \|y\|_A, \quad t \geq 0, \quad y \in \R^n,
$$
pour un certain $a>0$, ce qui implique aussi
$$
\|\e^{-tA}y\|_A \geq \e^{at} \|y\|_A, \quad t \geq 0, \quad y \in \R^n.
$$
Par le th\'eor\`eme des valeurs interm\'ediaires il existe un unique $\tau(x) \in \R$ tel que $\e^{\tau(x)A}x \in S_A.$

Montrons que $\tau$ est continue. Soit $x_0 \in \R^n \setminus 0$. Il existe $C>0$ tel que pour tout $x \in \R^n$,
$$
\|\e^{\tau(x_0)A}x_0 - \e^{\tau(x_0)}x\|_A \leq C \|x-x_0\|_A.
$$
Soit $\varepsilon >0$ ; si $\|x-x_0\|_A \leq \varepsilon$ on a
$$
1- C\varepsilon \leq \|\e^{\tau(x_0)A}x\| \leq 1 + C\varepsilon.
$$
Posons 
$$
t_\pm = \pm \frac{1}{a} \log(1 \pm C \varepsilon).
$$
Alors
$$
\|\e^{t_+A} \e^{\tau(x_0)A}x \|_A \leq \e^{-at_+}(1+C\varepsilon) = 1
$$
et de m\^eme
$$
\|\e^{-t_-A} \e^{\tau(x_0)A} \|\geq 1
$$
En particulier, on a que $|\tau(x)-\tau(x_0)| \leq \max(t_+, t_-)$, avec $t_\pm \to 0$ quand $\varepsilon \to 0$, et donc $\tau$ est continue.


\item On a que $\tau(\e^{tA}x) = \tau(x)-t$ pour tout $x \in \R^n$ non nul et tout $t \in \R$. Ceci implique que
$$
\begin{aligned}
\e^{-\tau(\e^{tA}x)B}\varphi(\e^{\tau(\e^{tA}x)A}\e^{tA}x) &= \e^{-(\tau(x)-t)B}\varphi(\e^{(\tau(x)-t)A}\e^{tA}x) \\
&= \e^{tB} \e^{-\tau(x)B}\varphi(\e^{\tau(x)A}x).
\end{aligned}
$$
On a bien $\Phi \circ \e^{tA} = \e^{tB} \circ \Phi$. De plus $\Phi$ est continue sur $\R^n\setminus 0$ par continuit\'e de $\tau$ et de $\varphi$. Elle est continue en $0$ car $\tau(x) \to -\infty$ quand $x \to 0$, et donc pour un $b>0$ on a
$$
\|\Phi(x)\|_B \leq \e^{-\tau(x)b} \|\varphi(\e^{\tau(x)A}x)\|_B \leq C \e^{-\tau(x)b} \to 0, \quad x \to 0.
$$
Enfin, si $\Psi : \R^n \to \R^n$ est d\'efinie comme $\Phi$ en interchangeant les r\^oles de $A$ et de $B$, on obtient $\Phi \circ \Psi = \Psi \circ \Phi = \Id_{\R^n}$, ce qui conclut.

\item On pose pour $M = A,B$, avec les notations de l'\textbf{Exercice 1.},
$$
E^s(M) = \bigoplus_{\Re(\lambda) < 0} C_{\lambda, \bar \lambda}(M), \quad E^u(M) = \bigoplus_{\Re(\lambda) > 0} C_{\lambda, \bar \lambda}(M).
$$

Par la question \textbf{1.}5., on a que $\dim E^s(A_t) = \dim E^s(A)$ pour tout $t$ et donc $\dim E^s(A) = \dim E^s(B)$. En particulier, par la question pr\'ec\'edente, il existe des isomorphismes $\Phi^\bullet : E^\bullet(A) \to E^\bullet(B),~\bullet = s,u$ qui conjuguent $\exp(tA|_{E^\bullet(A)})$ \`a $\exp(tB|_{E^\bullet(B)})$, pour $\bullet = s,u.$ On note
$$
\pi_\bullet : \R^n \to E^\bullet(A), \quad \bullet = s,u,
$$
les projecteurs spectraux associ\'es \`a la d\'ecomposition $\R^n = E^s(A) \oplus E^u(A).$ On pose
$$
\Phi = \Phi^u \circ \pi_u + \Phi^s \circ \pi_s : \R^n \to \R^n.
$$
Alors on v\'erifie ais\'ement que $\Phi$ conjugue $\e^{tA}$ \`a $\e^{tB}$.

\end{enumerate}

\vspace{0.6cm}

\noindent {\large \textbf{Exercice 4.} \textit{Syst\`emes lin\'eaires avec second membre}} \vspace{1.5mm} 

\begin{enumerate}
\item En cherchant une solution particuli\`ere sous la forme $t \mapsto \e^{tA}c(t)$, on trouve que les solutions sont de la forme
$$
x(t) = \e^{tA} \left(x_0 + \int_0^t \e^{-sA}z(s) \dd s\right), \quad t \in \R^n,
$$
o\`u $x_0 \in \R^n$.
\item Soit $\varepsilon > 0$ et $x_0 \in \R^n$. Soit $T > 0$ tel que pour tout $t \geq T$ on a $\|z(t)-z_\infty\|_A \leq \varepsilon$, o\`u $\|\cdot\|_A$ est une norme adapt\'ee \`a $A$. Alors 
$$
\int_0^t\e^{(t-s)A}z(s)\dd s = \int_0^T \e^{(t-s)A}z(s)\dd s + \int_T^t \e^{(t-s)A}z(s)\dd s.
$$
On a 
$$
\int_T^t \e^{(t-s)A}z(s)\dd s = \int_T^t\e^{(t-s)A}(z(s) - z_\infty) \dd s + \left(\int_T^t \e^{(t-s)A}\dd s \right)z_\infty.
$$
Or pour tout $t\geq T$ on a 
$$
\left\|\int_T^t\e^{(t-s)A}(z(s) - z_\infty) \dd s\right\|_A \leq \varepsilon \int_T^t \e^{-a(t-s)} \dd s \leq \frac{\varepsilon}{a}.
$$
D'autre part, 
$$
\int_{T}^t\e^{(t-s)A} \dd s = \e^{tA}\left[-A^{-1}\e^{-sA}\right]_{s=T}^{s=t} = -A^{-1} + A^{-1}\e^{(t-T)A}.
$$
En particulier puisque $A$ est une contraction on a
$$
\left(\int_T^t \e^{(t-s)A}\dd s \right)z_\infty \to -A^{-1}z_\infty, \quad t \to +\infty.
$$
On a aussi que $\displaystyle{\e^{tA} \int_0^T \e^{-s}z(s) \dd s + \e^{tA}x_0 \to 0}$ quand $t \to +\infty$. Tout ce qui pr\'ec\`ede montre que pour $t$ assez grand on a (pour une constante $C$ d\'ependant seulement de $a$)
$$
\left\|x(t) +A^{-1}z_\infty\right\| \leq C \varepsilon.
$$
On a obtenu que 
$$
\lim_{t \to +\infty} x(t) = -A^{-1} z_\infty.
$$
\end{enumerate}

\vspace{0.6cm}


\iffalse
\noindent Soit $A$ une matrice carr\'ee d'ordre $n$. 
\begin{enumerate}
\item Montrer que les conditions suivantes sont \'equivalentes :
\begin{enumerate}[label=(\roman*)]
\item Toutes les valeurs propres de $A$ ont un module strictement inf\'erieur \`a $1$ ;
\item Il existe une norme adapt\'ee pour $A$, c'est \`a dire une norme $\|\cdot\|_A$ sur $\R^n$ telle que pour un certain $a < 1$,
$$
\left\|A^Nx\right\|_A \leq a^N \|x\|_A, \quad x \in \R^n, \quad N \in \mathbb{N}.
$$
\end{enumerate}
\end{enumerate}
On se donne dans la suite deux matrices $A$ et $B$ v\'erifiant ces conditions. On suppose qu'il existe une famille continue de matrices inversibles $A_t,  ~t \in [0,1]$, telle que $A_0 = B$, $A_1 = A$ et $\mathrm{sp}(A_t) \subset \{z\in \C,~ |z| < 1\}$ pour tout $t \in [0,1]$. On se donne deux normes $\|\cdot\|_A$ et $\|\cdot\|_B$, adapt\'ees \`a $A$ et $B$. On note
$$
D_A = \{x \in \R^n,~\|x\|_A < 1\}, \quad S_A = \{x \in \R^n, ~ \|x\|_A = 1\}
$$
et on d\'efinit $D_B$ et $S_B$ identiquement. Finalement on note
$$
C_A = \overline{D_A \setminus AD_A}, \quad C_B = \overline{D_B \setminus BD_B}.
$$
\begin{enumerate}[resume]
\item Soit $\tau_A : [0,1] \times \mathbf{S}^{n-1} \to \R_+$ et $h_A : [0,1] \times \mathbf{S}^{n-1} \to F_A$ les applications d\'efinise par
$$
h_A(t,x) = \tau_A(t,x)x, \quad \tau_A(t,x) = \frac{t}{\|x\|_A} + \frac{1-t}{\left\|A^{-1}x\right\|_A}, \quad t \in [0,1], \quad x \in \mathbf{S}^{n-1},
$$
o\`u $\mathbf{S}^{n-1}$ est la sph\`ere de rayon $1$ dans $\R^n$. On d\'efinit identiquement $\tau_B, h_B$. En utilisant les applications $h_A$ et $h_B$, montrer qu'il existe un hom\'eomorphisme $h : C_A \to C_B$ tel que $h \circ A = B \circ h$
\item Montrer que les syst\`emes dynamiques discrets associ\'es \`a $A$ et $B$ sont topologiquement conjugu\'es.
\end{enumerate}
\fi

\vspace{0.6cm}

\end{document} 
