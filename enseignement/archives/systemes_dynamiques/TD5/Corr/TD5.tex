\documentclass[a4paper,12pt,openany]{article}
\usepackage{fancyhdr}
\usepackage[T1]{fontenc}
\usepackage[margin=1.8cm]{geometry}
\usepackage[applemac]{inputenc}
\usepackage{lmodern}
\usepackage{enumitem}
\usepackage{microtype}
\usepackage{hyperref}
\usepackage{enumitem}
\usepackage{dsfont}
\usepackage{amsmath,amssymb,amsthm}
\usepackage{mathenv}
\usepackage{amsthm}
\usepackage{graphicx}
\usepackage[all]{xy}
\usepackage{lipsum}       % for sample text
\usepackage{changepage}
\theoremstyle{plain}
\newtheorem{thm}{Theorem}[section]
\newtheorem*{thm*}{Th\'eor\`eme}
\newtheorem{prop}[thm]{Proposition}
\newtheorem{cor}[thm]{Corollary}
\newtheorem{lem}[thm]{Lemma}
\newtheorem{propr}[thm]{Propri\'et\'e}
\theoremstyle{definition}
\newtheorem{deff}[thm]{Definition}
\newtheorem{rqq}[thm]{Remark}
\newtheorem{ex}[thm]{Exercice}
\newcommand{\e}{\mathrm{e}}
\newcommand{\prodscal}[2]{\left\langle#1,#2\right\rangle}
\newcommand{\devp}[3]{\frac{\partial^{#1} #2}{\partial {#3}^{#1}}}
\newcommand{\w}{\omega}
\newcommand{\dd}{\mathrm{d}}
\newcommand{\x}{\times}
\newcommand{\ra}{\rightarrow}
\newcommand{\pa}{\partial}
\newcommand{\vol}{\operatorname{vol}}
\newcommand{\dive}{\operatorname{div}}
\newcommand{\T}{\mathbf{T}}
\newcommand{\R}{\mathbf{R}}
\newcommand{\Z}{\mathbf{Z}}
\newcommand{\N}{\mathbf{N}}
\newcommand{\C}{\mathbf{C}}
\newcommand{\F}{\mathcal{F}}
\newcommand{\Homeo}{\mathrm{Homeo}}
\newcommand{\Matn}{\mathrm{Mat}_{n \times n}}
\DeclareMathOperator{\tr}{tr}
\newcommand{\id}{\mathrm{id}}
\newcommand{\Id}{\mathrm{Id}}
\newcommand{\htop}{h_\mathrm{top}}


\title{\textsc{Syst\`emes dynamiques} \\ Corrig\'e 5}
\date{}
\author{}

\begin{document}

{\noindent \'Ecole Normale Sup\'erieure  \hfill \texttt{chaubet@dma.ens.fr} } \\
{2021/2022 \hfill}

{\let\newpage\relax\maketitle}
\maketitle

\noindent {\large \textbf{Exercice 1.}  \vspace{1.5mm} }


\begin{enumerate}
\item 
\begin{enumerate}
\item Si $x$ est un point p\'eriodique pour $\varphi_t$, i.e. $\varphi_{t_0}(x) = x$ pour un $t_0 \in \R$, alors 
$$
\phi_{t_0}(h(x)) = h (\varphi_{t_0}(x)) = h(x),
$$
i.e. $h(x)$ est un point p\'eriodique de p\'eriode $t_0$ pour $(\phi_t)$.
\item Supposons que $\mathcal{O}_\varphi(x)$ soit ferm\'ee. Soit $(y_n)$ une suite \`a valeurs dans $\mathcal{O}_\phi(h(x))$ qui converge vers $y \in \R^n$. Alors par la question pr\'ec\'edente, $h^{-1}(y_n) \in \mathcal{O}_\varphi(x)$ pour tout $n$. Puisque la suite $(h^{-1}(y_n))$ converge vers $h^{-1}(y)$ (par continuit\'e de $h^{-1}$) on a $h^{-1}(y) \in \mathcal{O}_\varphi(x)$, car $\mathcal{O}_\varphi(x)$ est ferm\'ee. Ainsi $y = h(h^{-1}(y)) \in \mathcal{O}_\phi(h(x))$ par la question pr\'ec\'edente, ce qui conclut (on peut renverser les r\^oles de $\varphi$ et $\phi$ pour avoir la r\'eciproque).

\item Soit $x \in \R^n$ et $y \in \omega(x)$. Alors il existe une suite $(t_k)_{k \in \N}$ \`a valeurs dans $\R_+$ telle que $t_k \to +\infty$ et $\varphi_{t_k}(x) \to y$ quand $k \to +\infty$. Par continuit\'e de $h$, on obtient que $h\circ \varphi_{t_k}(y) \to h(y)$ quand $k \to +\infty$, c'est-\`a-dire que $\phi_{t_k}(h(x)) \to h(y).$ En particulier $h(y) \in \omega(h(x))$, ce qui conclut.

\end{enumerate}
\item 
\begin{enumerate}
\item On a $\exp(tA) = \mathrm{diag}(\e^{t}, \e^{t})$ pour tout $t$. On a aussi
$$\exp(tB) = \e^{t} \begin{pmatrix} \cos(t) & \sin(t) \\ -\sin(t) & \cos(t) \end{pmatrix}, \quad t \in \R.$$
\item Pour tout $x \in \R^n$ on a $\|\e^{tA}x\| = \e^{t} \|x\|$. La conclusion est imm\'ediate.
\item Pour tout $x \neq 0$ on note $\tau(x)$ le temps obtenu \`a la question pr\'ec\'edente. On pose $\Phi(0)=0$
$$
\Phi(x) = \e^{-\tau(x)B} \e^{\tau(x)A} x, \quad x \in \R^n.
$$
Alors on v\'erifie que $\Phi$ conjugue $\e^{tA}$ \`a $\e^{tB}$ (cf. le corrig\'e du TD 4).
\end{enumerate}

\item Il est clair que pour tout $x \in \R^2 \setminus 0$ on a $\|\e^{tB}x\| \to +\infty$ quand $t \to +\infty$. Soit $\Phi : \R^2 \to \R^2$ un hom\'eomorphisme tel que $\e^{tB} \circ \Phi = \Phi \circ \e^{tC}$ pour tout $t \in \R$. Puisque $\exp(tC) = \begin{pmatrix} \cos(t) & -\sin(t) \\ \sin(t) & \cos(t) \end{pmatrix}$, on a $\|\e^{tC}x\| = \|x\|$ pour tout $x$ et tout $t \in \R$, et en particulier pour tout $r>0$ il existe $C>0$ tel que
$$
\|\Phi(\e^{tC}x)\|\leq C, \quad x \in B(0,r).
$$
Soit $x \in B(0,r)$ tel que $\Phi(x) \neq 0$. On a $\|\Phi(\e^{tC}x)\| = \|\e^{tB} \Phi(x)\| \to + \infty$ quand $t \to +\infty$. C'est absurde. 

La m\^eme d\'emonstration montre que $\e^{B}$ et $\e^C$ ne sont pas conjugu\'ees.

\end{enumerate}
\vspace{0.6cm}
\newpage

\noindent {\large \textbf{Exercice 2.} \vspace{1.5mm} }

\begin{enumerate}

\item On a que $\mathcal{H}(\R^n)$ est ouvert : cf. question \textbf{1}.5 du TD 4. Soit $A \in \mathrm{GL}(\R^n)$, et 
$$
\delta = \inf \{|\Re(\lambda)|,~\lambda \in \mathrm{sp}(A) \setminus 0\} > 0.
$$
Alors pour tout $|t|< \delta$, les valeurs propres de $A + t \Id$ ont toutes une partie r\'eelle non nulle. Ceci montre que $\mathcal{H}(\R^n)$ est dense dans $\mathrm{GL}(\R^n).$

 On a que $\mathrm{GL}(\R^n) = \{M \in \mathcal{L}(\R^n),~\det(M) \neq 0\}$, qui est donc ouvert. La m\^eme d\'emonstration que pr\'ec\'edemment montre que $\mathrm{GL}(\R^n)$ est dense dans $\mathcal{L}(\R^n)$.


\item Cela est imm\'ediat par un lemme du cours qui dit qu'il existe $\delta > 0$ tel que pour toute fonction $\varphi : \R^n \to \R^n$ continue, born\'ee, et $\delta$-lipschitzienne, alors les syt\`emes dynamiques $A$ et $A + \varphi$ sont topologiquement conjugu\'es.


\item Soit $\varepsilon > 0$. En regardant la d\'ecomposition de Jordan de $A$, on obtient que 
$$
\left\|A^n\right\|\leq \rho(A)^n |P(n)|, \quad n \in \N,
$$
pour un certain polyn\^ome $P \in \R[X].$ Cette estim\'ee implique que l'expression
$$
\|x\|' = \sum_{n=0}^{+\infty} b^{-n} \|A^nx\|, \quad x \in \R^n,
$$
o\`u $b>0$ v\'erifie $\rho(A) < b < \rho(A) + \varepsilon$, d\'efinit bien une norme. On a
$$
\|Ax\|' = \sum_{n=0}^{+\infty} b^{-n}\|A^{n+1}x\| = b \sum_{n=1}^{+\infty}b^{-n} \|A^{n}x\| \leq b\|x\|',
$$
ce qui donne (au sens de la norme d'op\'erateur) $\|A\|'\leq b<\rho(A) + \varepsilon$.

\end{enumerate}

\vspace{0.6cm}

\noindent {\large \textbf{Exercice 3.} \vspace{1.5mm} }

\noindent
Puisque $x$ est de p\'eriode $n$, on a que tout $y$ assez proche de $x$ ne peut pas \^etre $k$ p\'eriodique avec $k<n$ (puisque $f^k(x) \neq x$ pour tout $k \in \{1, \dots, n\}$). \vspace{0.2cm}

\noindent Soit $A = \dd f^n(x)$. Par le th\'eor\`eme de Grobman-Hartman, il existe un voisinage $V$ de $x$, un voisinage $U$ de $0$ et un hom\'eomorphisme $h : U \to V$ tel que $h \circ f^n = A \circ h$ pour tout $x \in f^{-n}(U)$. Si $y \in f^{-n}(U)$ v\'erifie $f^n(y) = y$, alors $h(y)$ v\'erifie $Ah(y) = h(y).$ Puisque $x$ est hyperbolique on a $1 \notin \mathrm{sp}(A)$ et donc $h(y) = 0$ ce qui implique que $y = x$. Ceci conclut.

\vspace{0.6cm}

\noindent {\large \textbf{Exercice 4.} \vspace{1.5mm} }

\begin{enumerate}
\item Pour tout $x \in E$, on a 
$$
A^{n} \pi_s(x) \to 0, \quad A^{-n} \pi_u(x) \to 0, \quad n \to +\infty.
$$
Soit $x \in E^s$. Alors $\pi_u(A^n x) = 0$ pour tout $n \geq 0$, et en particulier pour tout $\gamma > 0$ on a $x \in A^{-n}(C^s_\gamma).$ \vspace{0.2cm}

\noindent R\'eciproquement, supposons que $x \in A^{-n}(C^s_\gamma)$ pour tout $n \geq 0$ pour un certain $\gamma > 0$. En particulier
$$
\|A^n\pi_u(x)\|\leq \gamma \|A^n \pi_s(x)\| \to 0, \quad n \to +\infty.
$$
Par l'exercice \textbf{4.}, il existe une norme $\|\cdot\|_u$ sur $E_u$ et $a > 1$ tels que $\|(A|_{E_u})^{-1}\|_u \leq a^{-1} < 1$, puisque $\rho((A|_{E_u})^{-1}) < 1.$ En particulier $\|\pi_u(x)\|_u \leq a^{-n} \|A^n \pi_u(x)\|_u \to 0$ quand $n \to +\infty$. Il suit que $\pi_u(x) = 0.$

\vspace{0.2cm}
\noindent On montre de m\^eme que $E^u = \bigcup_{\gamma > 0} \bigcap A^n(C^u_\gamma).$

\item Soit $x$ tel que  $\|A^nx\| \leq C$ pour tout $n \geq 0$. Puisque $A^nx = A^n\pi_s(x) + A^n\pi_u(x)$ et que $A^n\pi_s(x) \to 0$ quand $n \to +\infty$, on a $A^n \pi_u(x) \leq C$ pour tout $n \geq 0$. De m\^eme qu'\`a la question pr\'ec\'edente, on obtient
$$
 C \geq \| A^n \pi_u(x)\|_u \geq a^n \|\pi_u(x)\|_u, \quad n \geq 0,
$$
ce qui implique que $\pi_u(x) = 0$. L'autre inclusion et claire et on proc\`ede identiquement pour l'autre \'egalit\'e.
\end{enumerate}
\vspace{0.6cm}

\noindent {\large \textbf{Exercice 5.} \vspace{1.5mm} }

\begin{enumerate}
\item La fonction $f$ est lisse sur $\R_{>0}$ avec
$$
f^{(k)}(x) = Q_k(x) \exp\left(-\frac{1}{x^2}\right), \quad x > 0, \quad k \in \N,
$$
o\`u les $Q_k$ sont des fractions rationnelles n'ayant des p\^oles qu'en $x = 0$. En particulier on $f^{(k)}(x) \to 0$ quand $x \to 0^+$ pour tout $k \in \N$. Ceci implique que $f$ est lisse par le th\'eor\`eme de la limite de la d\'eriv\'ee.
\item Le champ $X$ est lisse puisque $\rho$ et $r$ sont lisses. On a 
$$
\dd X(x,y) = \begin{pmatrix} \rho(r^2) + 2x^2 \rho'(r^2) & 1 \\ -1 & \rho(r^2) + 2y^2\rho'(r^2)\end{pmatrix},
$$
et en particulier
$$
\dd X(0) = \begin{pmatrix} 0 & 1 \\ -1 & 0 \end{pmatrix}.
$$
\item On choisit $\tilde f : \R \to [0,1]$ lisse telle que $\tilde f(x) = 0$ si $x \notin ]0,1[$, et $\tilde f(x) > 0$ si $x \in ]0,1[.$ On peut \'ecrire 
$$\R \setminus K = \left(\bigcup_{j \in J} ]a_j, b_j[ \right) \cup ]b, +\infty[$$
o\`u l'union est d\'enombrable, $a_j < b_j$ pour tout $j$ et o\`u $]a_j, b_j[ \cap ]a_{j'}, b_{j'}[ = \emptyset$ si $j \neq j'$. On d\'efinit $\rho_K : \R \to [0,1]$ par $\rho_K(t) = 0$ si $t \in K$,
$$
\rho_K(t) = \tilde f\left( \frac{t-a_j}{b_j - a_j} \right) \exp \left(-\frac{1}{(b_j - a_j)^2}\right), \quad t \in [a_j, b_j], \quad j \in J,
$$
et $\rho_K(t) = f(\mathrm{dist}(t, K))$ si $t > b$.
Alors $\rho_K$ v\'erifie les conditions demand\'ees.
\item En rempla\c cant $\rho_K$ par $\varepsilon \rho_K /2r$, on a les conditions demand\'ees, puisque $\rho_K \leq 1.$
\item Si $r^2 = x^2 + y^2 \in K$ alors $\rho_K(x,y) = (-y,x)$ Par suite l'orbite de $(x,y)$ est le cercle de rayon $r$.
\item Les trajectoires des points $(x,y)$ dans la couronne $C_{a,b}\{a^2< x^2 + y^2 < b^2\}$ restent \`a l'int\'erieur de la couronne ; en effet les cercles de rayon $a$ et $b$ sont des trajectoires p\'eriodiques de $X_K$, et les trajectoires ne peuvent pas s'intersecter. De plus, on calcule
$$
\begin{aligned}
X_K r^2(x,y) &= (y + \rho_K(r^2)x)\partial_x r^2(x,y) + (-x + \rho_K(r^2)y) \partial_y r^2(x,y) \\
&= 2yx + 2x^2\rho_K(r^2) -2xy + 2y^2 \rho_K(r^2) = 2r^2 \rho_K(r^2).
\end{aligned}
$$
Cette \'equation montre que $t \mapsto r^2(\varphi_K^t(x,y))$ est strictement croissante pour tous $(x,y) \in C_{a,b}$, o\`u $(\varphi_K^t)$ est le flot associ\'e \`a $X_K$, et que 
$$
\partial_t r^2(\varphi_K^t(x,y)) \geq c \tilde f\left(\frac{r^2(\varphi_K^t(x,y)) - a^2}{b^2-a^2}\right), \quad t \in \R
$$
par construction de $\rho_K$. En particulier on a $r^2(\varphi_K^t(x,y)) \to b^2$ quand $t \to +\infty$.

\item Supposons que $X_K$ et $X_{K'}$ soient conjugu\'es : il existe un hom\'eomorphisme $h : \R^2 \to \R^2$ tel que $h \circ \varphi_K^t = \varphi_{K'}^{t} \circ h$ pour tout $t \in \R.$ Soit $(x,y) \in \R^2$ tel que $x^2 + y^2 \in K$. Alors la trajectoire de $t \mapsto \varphi_K^t(x,y)$ est p\'eriodique, et par conjugaison la trajectoire $t \mapsto \varphi_{K'}^t(h(x,y))$ aussi. Par la question pr\'ec\'edente, cette trajectoire est un cercle, mettons de rayon $r'$, et on a $r'^2 \in K'$. On pose $\psi(r^2) = r'^2.$ Alors $\psi : K \to K'$ est continue, puisqu'elle co\"incide avec l'application
$$
K \ni \alpha \mapsto \|h(0, \sqrt{\alpha})\|^2 \in K'
$$
En renversant les r\^oles de $K$ et de $K'$, on obtient $\phi : K' \to K$ continue telle que $\psi \circ \phi =  \Id_{K'}$ et $\phi \circ \psi = \Id_K$. Cela conclut.

\end{enumerate}


\end{document} 
