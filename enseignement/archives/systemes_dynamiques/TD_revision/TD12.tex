\documentclass[a4paper,10pt,openany]{article}
\usepackage{fancyhdr}
\usepackage[T1]{fontenc}
\usepackage[margin=1.8cm]{geometry}
\usepackage[applemac]{inputenc}
\usepackage{lmodern}
\usepackage{enumitem}
\usepackage{microtype}
\usepackage{hyperref}
\usepackage{enumitem}
\usepackage{dsfont}
\usepackage{amsmath,amssymb,amsthm}
\usepackage{mathenv}
\usepackage{mathrsfs}
\usepackage{amsthm}
\usepackage{graphicx}
\usepackage[all]{xy}
\usepackage{lipsum}       % for sample text
\usepackage{changepage}
\theoremstyle{plain}
\newtheorem{thm}{Theorem}[section]
\newtheorem*{thm*}{Th\'eor\`eme}
\newtheorem{prop}[thm]{Proposition}
\newtheorem{cor}[thm]{Corollary}
\newtheorem{lem}[thm]{Lemma}
\newtheorem{propr}[thm]{Propri\'et\'e}
\theoremstyle{definition}
\newtheorem{deff}[thm]{Definition}
\newtheorem{rqq}[thm]{Remark}
\newtheorem{ex}[thm]{Exercice}
\newcommand{\e}{\mathrm{e}}
\newcommand{\prodscal}[2]{\left\langle#1,#2\right\rangle}
\newcommand{\devp}[3]{\frac{\partial^{#1} #2}{\partial {#3}^{#1}}}
\newcommand{\w}{\omega}
\newcommand{\dd}{\mathrm{d}}
\newcommand{\x}{\times}
\newcommand{\ra}{\rightarrow}
\newcommand{\pa}{\partial}
\newcommand{\vol}{\operatorname{vol}}
\newcommand{\dive}{\operatorname{div}}
\newcommand{\T}{\mathbf{T}}
\newcommand{\R}{\mathbf{R}}
\newcommand{\Q}{\mathbf{Q}}
\newcommand{\Z}{\mathbf{Z}}
\newcommand{\N}{\mathbf{N}}
\newcommand{\C}{\mathbf{C}}
\newcommand{\Pcal}{\mathcal{P}}
\newcommand{\F}{\mathcal{F}}
\newcommand{\Homeo}{\mathrm{Homeo}}
\renewcommand{\x}{\mathbf{x}}
\newcommand{\Matn}{\mathrm{Mat}_{n \times n}}
\DeclareMathOperator{\tr}{tr}
\newcommand{\id}{\mathrm{id}}
\newcommand{\htop}{h_\mathrm{top}}


\title{\textsc{Syst\`emes dynamiques} \\ Feuille de r\'evision}
\date{}
\author{}

\begin{document}

{\noindent \'Ecole Normale Sup\'erieure  \hfill Yann Chaubet } \\
{2021/2022 \hfill \texttt{chaubet@dma.ens.fr}}

{\let\newpage\relax\maketitle}
\maketitle

\noindent {\large \textbf{Exercice 1.} \textit{Entropie topologique des applications non dilatantes}} \vspace{1.5mm} 

\noindent
Soit $(X, \dd)$ un espace m\'etrique compact et $f : X \to X$ une transformation continue telle que 
$$
\dd(f(x), f(y)) \leq \dd(x,y), \quad x, y \in X.
$$
Montrer que $h_\mathrm{top}(f) = 0.$
\vspace{0.6cm}

\noindent {\large \textbf{Exercice 2.} \textit{Ergodiciti\'e et m\'elange au sens de C\'esaro}} \vspace{1.5mm} 

\noindent Soit $(X, \mathscr{A}, \mu)$ un espace probabilis\'e, et $f : X \to X$ une application pr\'eservant $\mu$. On suppose que $\mu$ est ergodique pour $f$. Montrer que pour tous $A, B \in \mathscr{A}$ on a 
$$
\lim_{n \to \infty} \frac{1}{n} \sum_{k=0}^{n-1} \mu\left(f^{-k}(A) \cap B\right) = \mu(A) \mu(B).
$$

\vspace{0.6cm}

\noindent {\large \textbf{Exercice 3.} \textit{Mesures ergodiques et points extr\'emaux}} \vspace{1.5mm} 

\noindent Soit $(X, \dd)$ un espace m\'etrique compact et $f : X \to X$ une application continue. On note $\mathcal M(X,f)$  les mesures de probabilit\'es bor\'eliennes $f$-invariantes. Les \textit{points extr\'emaux} de $\mathcal{M}(X,f)$ sont les mesures $\mu$ v\'erifiant la propri\'et\'e suivante. Pour toutes mesures $\mu_1, \mu_2 \in \mathcal{M}(X,f)$ et tout $t \in \left]0, 1\right[$, on a
$$
 \mu = t\mu_1 + (1-t) \mu_2 \quad \implies \quad \mu = \mu_1 = \mu_2.
$$
\begin{enumerate}
\item
\begin{enumerate}
\item Montrer que si $\mu$ est un point extr\'emal de $\mathcal M(X,f)$, alors $\mu$ est ergodique.
\item Soient $\mu \in \mathcal{M}(X,f)$ et $\varphi \in L^1(\mu)$ positive telle que $\int \varphi \dd \mu = 1$. Montrer que la mesure $\nu = \varphi \mu$ est invariante par $f$ si, et seulement si, $\varphi$ est invariante par $f$. \\
\textit{Indication. Pour le sens "si", on pourra consid\'erer les ensembles $\{\varphi > t\}$ et montrer qu'ils sont invariants.}
\item En d\'eduire que si $\mu$ est ergodique, alors c'est un point extr\'emal de $\mathcal{M}(X,f)$.
\end{enumerate}
\item En d\'eduire le fait suivant. Si $\mu, \nu$ sont deux mesures ergodiques, alors on a deux possibilit\'es: ou $\mu = \nu$, ou $\mu$ et $\nu$ sont \'etrang\`eres (ce qui signifie qu'il existe un bor\'elien $A$ tel que $\mu(A) = 1$ et $\nu(A) = 0$).
\end{enumerate}

\vspace{0.6cm}


\noindent {\large \textbf{Exercice 4.} \textit{Syst\`emes lin\'eaires avec second membre}} \vspace{1.5mm} 

\noindent Soit $A$ une matrice carr\'ee d'ordre $n$, et $z : \R \to \R^n$ une application continue. 

\begin{enumerate}
\item R\'esoudre l'\'equation diff\'erentielle 
\begin{equation}\label{eqdiff}
\dot{x}(t) = A x(t) + z(t).
\end{equation}
\item On suppose que $A$ est une contraction lin\'eaire et que $z(t) \to z_\infty \in \R^n$ quand $t \to +\infty$. Montrer que toute solution de (\ref{eqdiff}) converge en grand temps vers une limite \`a d\'eterminer.\\
\textit{Indication. On commencer par montrer l'existence d'une norme $\| \cdot\|_A$ sur $\R^n$ telle que $$ \|\e^{tA} x\|_A \leqslant \e^{-at} \|x\|_A, \quad x \in \R^n, \quad t \geqslant 0,$$ o\`u $a \in \left]0, 1\right[$.}
\end{enumerate}

\vspace{0.6cm}



\noindent {\large \textbf{Exercice 5.} \textit{Entropie des transformations Lipschitziennes}} \vspace{1.5mm} 

\noindent Soit $(X,\dd)$ un espace m\'etrique compact. On d\'efinit
$$
\mathrm{bdim}(X) = \limsup_{\varepsilon \to 0}\frac{\log M(X,\varepsilon)}{\log1/\varepsilon}
$$
o\`u $M(X,\varepsilon)$ est le nombre minimal de $\varepsilon$-boules (pour la distance $\dd$) qu'il faut pour recouvrir $X$. 
\begin{enumerate}
\item Montrer que $\mathrm{bdim} \left([0,1]^n\right) = n$. 
\end{enumerate}
Soit $f : X \to X$ une application Lipschitzienne et
$$
L(f) = \sup_{x \neq y} \frac{\dd(f(x), f(y))}{\dd(x,y)}
$$
sa constante de Lipschitz.

\begin{enumerate}[resume]
\item Montrer que 
\begin{equation}\label{eq:upperboundentropy}
h_{\mathrm{top}}(f) \leq \mathrm{bdim}(X) \max(0, \log L(f)).
\end{equation}
\item Donner un exemple d'application $f$ telle que (\ref{eq:upperboundentropy}) soit une \'egalit\'e.
\end{enumerate}
\vspace{0.6cm}



\noindent {\large \textbf{Exercice 6.} \textit{Le th\'eor\`eme de Von Neumann via le th\'eor\`eme de Birkhoff}} \vspace{1.5mm} 

\noindent Soit $(X, \mathscr{A}, \mu)$ un espace probabilis\'e, $f : X \to X$ une application pr\'eservant $\mu$, $\varphi \in L^2(\mu)$ et $\bar \varphi \in L^1(\mu)$ sa fonction associ\'ee dans le th\'eor\`eme de Birkhoff. On note aussi $\displaystyle{S_n\varphi = \frac{1}{n} \sum_{k=0}^{n-1} \varphi \circ f^k}$. On cherche \`a retrouver le th\'eor\`eme de Von Neumann.

\begin{enumerate}
\iffalse
\item Montrer que 
$\displaystyle{
\left(\int_X |\bar \varphi|^2 \dd \mu \right)^{1/2} \leq \liminf_n \left(\int_X \left|S_n\varphi\right|^2\right)^{1/2}.}
$
\fi
\item Montrer que $\bar \varphi \in L^2(\mu)$ et que $\|\bar \varphi\|_{L^2(\mu)} \leq \|\varphi\|_{L^2(\mu)}$.
\item Montrer que $S_n \varphi \to \bar \varphi$ dans $L^2(\mu).$ \\
\textit{Indication. On pourra consid\'erer le cas o\`u $\varphi \in L^\infty(\mu)$ puis conclure par un argument d'approximation, en consid\'erant les fonctions
$$
\varphi_k = \varphi 1_{|\varphi| \leqslant k}.
$$}
\end{enumerate}

\vspace{0.6cm}

\noindent {\large \textbf{Exercice 7.} \textit{Moyenne temporelle des temps de retour}} \vspace{1.5mm} 

\noindent Soit $(X, \mu)$ un espace de probabilit\'es et $f : X \to X$ une transformation ergodique pour $\mu$. Soit $A \subset X$ un ensemble mesurable de mesure non nulle, $\tau : A \to \N_{\geqslant 1} \cup \{+\infty\}$ le temps de premier retour dans $A$, et 
$$g : x \mapsto f^{\tau(x)}(x)$$l'application de premier retour associ\'ee (qui est d\'efinie presque partout sur $A$ par le th\'eor\`eme de Poincar\'e). 
\begin{enumerate}
\item Montrer le th\'eor\`eme de Kac : 
$$\displaystyle{\int_A \tau \dd \mu = 1}.$$
\item En d\'eduire que pour $\mu$ presque tout $x$ de $A$,
$$
\lim_{n \to + \infty}\frac{1}{n} \sum_{k=0}^{n-1} \tau\left(g^k(x)\right) = \frac{1}{\mu(A)}.
$$
\end{enumerate}




\end{document}
 
 
 

