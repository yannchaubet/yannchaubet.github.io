\documentclass[a4paper,12pt]{article}
\usepackage[T1]{fontenc}
\usepackage[margin=2.8cm]{geometry}
\usepackage[applemac]{inputenc}
\usepackage{lmodern}
\usepackage{enumitem}
\usepackage{microtype}
\usepackage{hyperref}
\usepackage{enumitem}
\usepackage{dsfont}
\usepackage{amsmath,amssymb,amsthm}
\usepackage{mathenv}
\usepackage{amsthm}
\usepackage{tikz,tkz-tab}
\usepackage{graphicx}
\usepackage[all]{xy}
\theoremstyle{plain}
\newtheorem{thm}{Theorem}[section]
\newtheorem*{thm*}{Th�or�me}
\newtheorem{prop}[thm]{Proposition}
\newtheorem{cor}[thm]{Corollary}
\newtheorem{lem}[thm]{Lemma}
\newtheorem{propr}[thm]{Propri�t�}
\theoremstyle{definition}
\newtheorem{deff}[thm]{Definition}
\newtheorem{rqq}[thm]{Remark}
\newtheorem{ex}[thm]{Exercice}
\newcommand{\e}{\mathrm{e}}
\newcommand{\prodscal}[2]{\left\langle#1,#2\right\rangle}
\newcommand{\devp}[3]{\frac{\partial^{#1} #2}{\partial {#3}^{#1}}}
\newcommand{\w}{\omega}
\newcommand{\dd}{\mathrm{d}}
\newcommand{\x}{\times}
\newcommand{\ra}{\rightarrow}
\newcommand{\pa}{\partial}
\newcommand{\vol}{\operatorname{vol}}
\newcommand{\dive}{\operatorname{div}}
\newcommand{\T}{\mathbf{T}}
\newcommand{\N}{\mathbf{N}}
\newcommand{\Id}{\mathrm{Id}}
\newcommand{\R}{\mathbf{R}}
\newcommand{\wto}{\rightharpoonup}
\newcommand{\Z}{\mathbf{Z}}
\renewcommand{\H}{\mathcal{H}}
\newcommand{\SL}{\mathrm{SL}}
\newcommand{\Homeo}{\mathrm{Homeo}}
\newcommand{\Isom}{\mathrm{Isom}}
\newcommand{\Leb}{\mathrm{Leb}}

\title{\textsc{Syst�mes dynamiques} \\�DM n�1}
\date{{Pour le 06/10/20}}
\author{}

\begin{document}
{\noindent �cole Normale Sup�rieure  \hfill� \texttt{chaubet@dma.ens.fr} \\
{2020/2021}}

{\let\newpage\relax\maketitle}
\maketitle

Soit $G_n = \mathrm{SL}(n, \R)$ que l'on munit de la topologie naturelle, et $\H$ un espace de Hilbert s�parable. Une action unitaire $\rho$ de $G_n$ sur $\H$ est un morphisme
$
\rho : G_n \to \Isom(\H),
$
o� $\Isom(\H)$ d�signe l'espace des isom�tries de $\H$. On dira qu'une telle action est fortement continue si l'application
$$
\begin{aligned}
G_n \times \H &\longrightarrow \H \\
(g, v) &\longmapsto \rho(g) \cdot v
\end{aligned}
$$
est continue.

\section*{Pr�liminaires}
On note $K = \mathrm{SO}(n, \R)$ et 
$$
A_{+}=\left\{\begin{pmatrix} t_1 & & 0 \\ & \ddots & \\ 0 & & t_{N}\end{pmatrix}, ~t_{1} \geq \ldots \geq t_{n}>0,~\prod_{i=1}^{n} t_{i}=1\right\}.
$$
\begin{enumerate}[label=\textbf{\arabic*.}]
\item Montrer que pour tout $g \in G_n$, il existe $k_1, k_2 \in K$ et $a \in A_+$ tels que $g = k_1 a k_2$.
\end{enumerate}
Pour tous $1 \leq i,j \leq n$ on note $E_{ij} \in \mathrm{Mat}_{n \times n}(\R)$ la matrice dont les coefficients sont nuls sauf le coefficient en place $(i,j)$ qui vaut $1$, et
$$
\begin{aligned}
U_{ij} &= \{\Id + \tau E_{ij},~\tau \in \R\}, \\
A_{ij} &= \{\Id + (t-1)E_{ii} + (t^{-1} - 1) E_{jj},~t > 0 \}, \\
S_{ij} &= \{\Id + (a-1)E_{ii} + b E_{ij} + c E_{ji} + (d-1)E_{jj},~ad-bc = 1\}.
\end{aligned}
$$
\begin{enumerate}[label=\textbf{\arabic*.}, resume]
\item Montrer que le sous-groupe $S_{ij}$ de $G_n$ est isomorphe � $G_2$.
\item Montrer que $G_n$ est engendr� par les sous-groupes $A_{ij}$ et $U_{ij}$ avec $1 \leq i,j \leq n$ et $i \neq j$.
\end{enumerate}

\section*{Vecteurs $G_2$-invariants}
Pour tout $t > 0$ et $\tau \in \R$ on note
$$
a_t = \begin{pmatrix} t & 0 \\ 0 & t^{-1} \end{pmatrix}, \quad u_\tau = \begin{pmatrix} 1 & \tau \\�0 & 1 \end{pmatrix}, \quad s_\tau = \begin{pmatrix} 1 & 0 \\ \tau & 1 \end{pmatrix}.
$$
Soit $(\H, \rho)$ une repr�sentation unitaire et fortement continue de $G_2$. Pour tout $v \in \H$, on note $G_v = \{g \in G_2,~\rho(g)v = v\}$.
\begin{enumerate}[label=\textbf{\arabic*.}, resume]
\item Montrer que $G_v$ est un sous-groupe ferm� de $G$ et que pour tout $v \in \H$,
$$
g \in G_v \iff \langle \rho(g)v, v \rangle = \|v\|^2.
$$
\item Soit $g \in G_2$ tel qu'il existe des suites $(g_m)$ dans $G$ et $(s_m), (s_m')$ dans $G_v$ avec 
$$
\lim_m g_m = g, \quad \lim_m s_m g_m s_m' = 1.
$$
Montrer que $g \in G_v$.
\end{enumerate}
Pour $g \in G_2$, on dira que $v \in \H$ est $g$-invariant si $\rho(g)v = v$.
\begin{enumerate}[label=\textbf{\arabic*.}, resume]
\item Soit $v \in \H$ et $t \neq 1$. Montrer que si $v$ est $a_t$-invariant, alors pour tout $\tau \in \R$, $v$ est $u_\tau$-invariant et $s_\tau$-invariant.
\item Soit $v \in \H$ et $\tau \neq 0$. Montrer que si $v$ est $u_\tau$-invariant ou $s_\tau$-invariant, alors pour tout $t > 0$, $v$ est $a_t$-invariant.
\item En d�duire que si $v \in \H$ est invariant par $a_t$ pour un $t \neq 1$ (ou par $u_\tau$, ou par $s_\tau$, pour $\tau \neq 0$), alors $G_v = G$.
\end{enumerate}





\section*{Th�or�me de Howe--Moore}
Soit $\rho : G_n \to \H$ une repr�sentation unitaire fortement continue telle que
$$
\{v \in \H,~�\forall g \in G_n,~\rho(g)v = v \} = \{0\}.
$$
Le but est de montrer que pour tous $u,v \in \H$,
$$
\lim_{\|g\|\to \infty} \langle \rho(g) u, v \rangle = 0.
$$
Pour cela on raisonne par l'absurde et on suppose qu'il existe $\varepsilon > 0$, $v,w \in \H$ et une suite $(g_m)$ de $G_n$ tels que $\|g_m\| \to \infty$ et
$$
|\langle \rho(g_m)v, w \rangle|>\varepsilon, \quad k \in \N.
$$
Pour tout $m \in \N$, on se donne $k_m, k'_m \in K$ et $a_m \in A_+$ tels que $g_m = k_m a_m k'_m.$ \\


On dit qu'une suite $(v_m)$ de $\H$ converge faiblement vers $v \in \H$, ce qu'on notera $v_m \wto v$, si pour tout $w \in \H$ on a quand $m \to \infty$,
$$
\langle v_m, w \rangle \to \langle v, w \rangle.
$$
\begin{enumerate}[label=\textbf{\arabic*.}, resume]
\item Montrer que toute suite born�e de $\H$ admet une sous-suite qui converge faiblement.
\item Montrer que quitte � extraire, on peut supposer qu'il existe $v_0 \in \H$ et $k, k' \in K$ tels que, quand $m \to \infty,$
\begin{enumerate}[label=(\roman*)]
\item $k_m \to k$ et $k'_m \to k'$,
\item $\rho(a_mk')v \wto v_0$,
\item $\rho(g_m)v \wto \rho(k) v_0.$
\end{enumerate}
\item Montrer qu'il existe $k \in \{1, \dots, n-1\}$ tel que pour tous $1 \leq i \leq k < j \leq n$, on a 
$$
\rho(g)v_0 = v_0, \quad g \in U_{ij}.
$$
\textit{Indication : on pourra montrer l'existence d'un $k \in \{1, \dots, n-1\}$ tel que $\lim_m t_{m}^{(k)} / t_m^{(k+1)} = \infty$ o� $t_m^{(j)}$ d�signe le coefficient en place $(j,j)$ de la matrice $a_m$.}
\item Conclure.
\end{enumerate}


\section*{Application au m�lange}
Pour tout bor�lien $A \subset G_n$, on note
$$
\nu(A)=\mathrm{Leb}\left(\left\{x \in \mathrm{Mat}_{n \times n}(\R),~ 1 \leq \operatorname{det} (x) \leq 2,~\det(x)^{-1/n} x \in A\right\}\right),
$$
o� $\Leb$ d�signe la mesure de Lebesgue sur l'espace $\mathrm{Mat}_{n \times n}(\R) \simeq \R^{n^2}$ des matrices $n \times n$. On note
$$
R_g(x) = xg^{-1}, \quad L_g(x) = gx, \quad g,x \in G_n.
$$
\begin{enumerate}[label=\textbf{\arabic*.}, resume]
\item Montrer que $\nu$ d�finit une mesure bor�lienne sur $G_n$ telle que $(R_g)_* \nu = (L_g)_* \nu = \nu.$
\end{enumerate}
Soit $\Gamma \subset G_n$ un sous-groupe discret. La mesure $\nu$ �tant $G_n$-invariante, elle induit une mesure $\mu$ sur $X = \Gamma \backslash G_n.$ On suppose que $\Gamma$ est un r�seau, c'est-�-dire que $\mu(\Gamma \backslash G_n) < +\infty$. 
\begin{enumerate}[label=\textbf{\arabic*.}, resume]
\item Montrer que pour tout $g \in G_n$ tel que $\displaystyle{\lim_{k \to +\infty} \|g^k\| = + \infty}$, on a 
$$
\lim_{k \to \infty} \int_X \varphi(x) \psi(xg^k) \dd \mu(x) = \left(\int_X \varphi ~\dd \mu \right)\left(\int_X \psi~ \dd \mu \right), \quad \varphi, \psi \in L^2(X, \mu).
$$
\end{enumerate}


\end{document} 
\textit{Indication : on pourra consid�rer $\H = \{\varphi \in L^2(X, \mu),~\int_X \varphi~ \dd \mu = 0\}$ et la repr�sentation $\rho$ de $G_n$ sur $\H$ d�finie par 
$$(\rho(g) \varphi)(x) = \varphi(xg), \quad x \in X, \quad g \in G_n, \quad \varphi \in \H.$$
}
