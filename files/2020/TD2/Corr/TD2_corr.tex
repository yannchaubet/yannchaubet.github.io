\documentclass[a4paper,12pt,openany]{article}
\usepackage{fancyhdr}
\usepackage[T1]{fontenc}
\usepackage[margin=1.8cm]{geometry}
\usepackage[applemac]{inputenc}
\usepackage{lmodern}
\usepackage{enumitem}
\usepackage{microtype}
\usepackage{hyperref}
\usepackage{enumitem}
\usepackage{dsfont}
\usepackage{amsmath,amssymb,amsthm}
\usepackage{mathenv}
\usepackage{amsthm}
\usepackage{graphicx}
\usepackage[all]{xy}
\usepackage{lipsum}       % for sample text
\usepackage{changepage}
\theoremstyle{plain}
\newtheorem{thm}{Theorem}[section]
\newtheorem*{thm*}{Th\'eor\`eme}
\newtheorem{prop}[thm]{Proposition}
\newtheorem{cor}[thm]{Corollary}
\newtheorem{lem}[thm]{Lemma}
\newtheorem{propr}[thm]{Propri\'et\'e}
\theoremstyle{definition}
\newtheorem{deff}[thm]{Definition}
\newtheorem{rqq}[thm]{Remark}
\newtheorem{ex}[thm]{Exercice}
\newcommand{\e}{\mathrm{e}}
\newcommand{\prodscal}[2]{\left\langle#1,#2\right\rangle}
\newcommand{\devp}[3]{\frac{\partial^{#1} #2}{\partial {#3}^{#1}}}
\newcommand{\w}{\omega}
\newcommand{\dd}{\mathrm{d}}
\newcommand{\x}{\times}
\newcommand{\ra}{\rightarrow}
\newcommand{\pa}{\partial}
\newcommand{\vol}{\operatorname{vol}}
\newcommand{\dive}{\operatorname{div}}
\newcommand{\T}{\mathbf{T}}
\newcommand{\R}{\mathbf{R}}
\newcommand{\Z}{\mathbf{Z}}
\newcommand{\N}{\mathbf{N}}
\newcommand{\F}{\mathcal{F}}
\newcommand{\Homeo}{\mathrm{Homeo}}
\newcommand{\Matn}{\mathrm{Mat}_{n \times n}}
\DeclareMathOperator{\tr}{tr}
\newcommand{\id}{\mathrm{id}}
\newcommand{\htop}{h_\mathrm{top}}


\title{\textsc{Syst\`emes dynamiques} \\�Corrig\'e 2}
\date{}
\author{}

\begin{document}

{\noindent \'Ecole Normale Sup\'erieure  \hfill Yann Chaubet } \\
{2020/2021 \hfill�\texttt{chaubet@dma.ens.fr}}

{\let\newpage\relax\maketitle}
\maketitle
\noindent {\large \textbf{Exercice 1.} \textit{Familles d'applications transitives}} \vspace{0.15cm}

\noindent Soit $(U_k)_{k \in \N}$ une base d'ouverts de $X$. Pour tous $k,i \in \N$, l'ensemble
$$
A_{i,k}= \bigcup_{n \in \N} f_i^{-n}(U_k)
$$
est un ouvert dense, par transitivit\'e des $f_i$. D\`es lors l'ensemble 
$$
Y = \bigcap_{i \in \N} \bigcap_{k \in \N} A_{i,k}
$$
est dense, par le th\'eor\`eme de Baire, puisque $X$ est localement compact. Tout \'el\'ement $y \in Y$ v\'erifie $\omega_{f_i}(y) = X.$

\vspace{0.6cm}
%% \\

\noindent {\large \textbf{Exercice 2.} \textit{Transformations minimales}} \vspace{1.5mm}

\begin{enumerate}
\item Soient $U,V$ deux ouverts non vides, et $Y = \overline{\bigcup_{n \geq 0} f^{-n}(V)}.$ Alors $Y$ est ferm\'e et $f(Y) \subset Y$ ; ainsi $Y=X$ et il existe $n \in \N$ tel que $f^{-n}(U) \cap V = \emptyset$.
\item Soit 
$
\mathcal{F} = \{F \subset X,~�F \text{ est un ferm\'e non vide tel que } f(F) \subset F\}.
$
Alors $\mathcal{F}$ est partiellement ordonn\'e pour l'inclusion. Soit $\mathcal{C} \subset \mathcal{F}$ une famille totalement ordonn\'ee. Alors
$$
G = \bigcap_{F \in \mathcal{C}} F
$$
est non vide. En effet, supposons que ce ne soit pas le cas. Alors $X = \bigcup_F \complement F$ et par compacit\'e, il existe $F_1, \dots, F_N \in \mathcal{C}$ tels que
$$
X = \bigcup_{i = 1}^N \complement F_i.
$$
Par suite $\bigcap_{j=1}^N F_j = \emptyset$, ce qui est absurde puisque la famille $\{F_1, \dots, F_N\}$ est totalement ordonn\'ee. 

Ainsi $G$ est non vide et c'est un minorant pour $\mathcal{C}$. Par le lemme de Zorn, il existe un \'el\'ement minimal de $\mathcal{F}$, not\'e $Y$. Soit $F \subset Y$ un ferm\'e non vide tel que $f(F) \subset F$. Alors $F = Y$ par minimalit\'e de $Y$ et donc $f|_Y$ est minimale.
\item Soit $f : X \to X$ continue. Alors par la questions pr\'ec\'edente, $f$ admet une partie ferm\'ee minimale non vide $Y$. Soit $y \in Y$ ; on consid\`ere l'orbite positive
$
\mathcal{O}_+(y) = \{f^n(y),~n \in \N\}.
$
Alors $\overline{\mathcal{O}_+(y)}$ est une partie ferm\'ee non vide de $Y$, invariante par $f$. Par minimalit\'e, $\overline{\mathcal{O}_+(y)} = Y$ et $y$ est positivement r\'ecurrent. 
\end{enumerate}
\vspace{0.6cm}

\noindent {\large \textbf{Exercice 3.} \textit{Ensemble non-errant}} \vspace{1.5mm} 

\begin{enumerate}
\item  Supposons que pour tout $n > m$, $f^n(U) \cap U = \emptyset$. Alors $x$ n'est pas p\'eriodique et il existe un voisinage $V$ de $x$ tel que $f^j(V) \cap V = \emptyset$ pour tout $j = 1, \dots, m.$ Quitte \`a r\'eduire $V$, on peut supposer $V \subset U$. On a donc $f^k(V) \cap V = \emptyset$ pour tout $k \in \N_{\geq 1}$ et $x \notin \Omega(f)$.


\item Soit $x \notin \Omega(f)$ et $U$ un voisinage ouvert de $x$ tel que $f^n(U) \cap U = \emptyset$ pour tout $n \geq 1$. Alors tout $y \in U$ est errant, et donc $\Omega(f)$ est ferm\'e. Il est invariant : si $x \in \Omega(f)$ et $y = f(x)$, on a un voisinage $U$ de $x$ et $n > 0$ tel que $f^n(U) \cap U \neq \emptyset$. Alors $V = f^{-1}(U)$ est un voisinage de $y$ et v\'erifie $f^n(V) \cap V \neq \emptyset$.

Enfin, soit $x \in X$ et $y \in \omega(x)$. Soit $U$ un voisinage ouvert de $y$. Alors il existe $m > n> 0$ tels que $f^m(x), f^n(x) \in U$. Il suit que $f^{m-n}(U) \cap U \neq \emptyset$, donc $y \in \Omega(f).$

\item Pour $x \in \mathrm{Per}(f)$, l'orbite $\mathcal{O}_+(x)$ est un ensemble minimal et donc $\mathrm{Per}(f) \subset M(f)$.

Soit $F\subset X$ un sous-ensemble minimal pour $f$. Alors tout point de $F$ est r\'ecurrent (cf. \textbf{Exercice 2.}) et donc $M(f) \subset R(f)$.

Si $x$ est r\'ecurrent, on a $x \in \omega(x) \subset \Omega(f)$. Comme $\Omega(f)$ est ferm\'e, on a $R(f) \subset \Omega(f)$.

\end{enumerate}

\vspace{0.6cm}

\noindent {\large \textbf{Exercice 4.} \textit{Entropie d'un flot}} \vspace{1.5mm} \\
\noindent La compacit\'e de $X$ et la continuit\'e de $\Phi$ donnent
$$
\forall \varepsilon > 0, \quad \exists \delta(\varepsilon) > 0, \quad \forall x,y \in X, \quad \mathrm{dist}(x,y) \leq \delta(\varepsilon) \implies \mathrm{d}^\Phi_1(x,y) \leq \varepsilon.
$$
Pour $f : X\to X$, on rappelle que
$$
\dd^{f}_n(x,y) = \max \{ \dd(f^k(x), f^k(y)),~k = 0, \dots, n-1\}.
$$
De plus, on peut supposer que $\delta(\varepsilon) \to 0$ quand $\varepsilon \to 0$. Par cons\'equent, puisque $\dd_T^\Phi \geq \dd_{\lfloor T \rfloor}^{\varphi^1}$, on a pour tout $\varepsilon > 0$ et tout $T > 1$
$$
B_{\dd^{\varphi^1}_{\lfloor T \rfloor}}(x, \varepsilon) \supset B_{\dd^\Phi_T}(x, \varepsilon) \supset B_{\dd^{\varphi^1}_{\lfloor T \rfloor}}(x, \delta(\varepsilon)).
$$
Ainsi, en notant pour toute application $f : X \to X$
$$
M^{f}(n, \varepsilon) = \min \left\{m \geq 1,~ \exists x_1, \dots, x_m \in X,~\bigcup_{i=1}^m B_{d^{f}_n}(x_i, \varepsilon) = X\right\},
$$
on a
$$
M^{\varphi^1}(\lfloor T \rfloor, \varepsilon) \leq M^{\Phi}(T, \varepsilon) \leq M^{\varphi^1}(\lfloor T \rfloor, \delta(\varepsilon)),
$$
o\`u $M^{\Phi}(T, \varepsilon)$ est d\'efini comme $M^f(n, \varepsilon)$ en rempla\c ant $\dd^f_n$ par $\dd^\Phi_T$. Il suit que 
$$
\limsup_n \frac{1}{n} \log M^{\varphi^1}(n, \varepsilon) \leq \limsup_T \frac{1}{T}�\log  M^{\Phi}(T, \varepsilon) \leq \limsup_n \frac{1}{n} \log M^{\varphi^1}(n, \delta(\varepsilon)),
$$
ce qui conclut.

\vspace{0.6cm}

\noindent {\large \textbf{Exercice 5.} \textit{Propri\'et\'es de l'entropie topologique}} \vspace{1.5mm} 

On d\'efinit comme dans le cours, pour tout $f : X \to X$, tout $n \in \N_{\geq 1}$ et tout $\varepsilon > 0$,
$$
C^f(n, \varepsilon) = \min \left\{ m \geq 1,~\exists U_1, \dots, U_m \subset X,~\forall j,~\mathrm{diam}_{\dd^f_n}(U_i) \leq \varepsilon, ~X \subset \bigcup_{i=1}^m U_i \right\},$$
et 
$$
N^f(n, \varepsilon) = \max \left\{ m \in \N,~ \exists x_1, \dots, x_m \in X,~\forall i \neq j,~\dd^f_n(x_i, x_j) \leq \varepsilon \right\}.
$$

\begin{enumerate}
\item On a $C^f(n, \varepsilon) \geq C^{f|_\Lambda}(n, \varepsilon)$ pour tous $n, \varepsilon$, ce qui conclut.
\item Par la question pr\'ec\'edente on a $\htop(f_j) \leq \htop(f)$ pour tout $j$. De plus, on a que
$$
C^f(n, \varepsilon) \leq \sum_{i=1}^m C^{f|_{\Lambda_i}}(n, \varepsilon).
$$
Ceci implique qu'il existe $i \in \{1, \dots, m\}$ tel que 
$$C^{f|_{\Lambda_i}}(n, \varepsilon) \geq \frac{1}{m}C^f(n, \varepsilon),$$
qui v\'erifie donc $\htop(f|_{\Lambda_i}) \geq \htop(f)$.
\item On a que $\dd^{f^m}_n \leq \dd^{f}_{mn-m+1}$ pour tous $m,n \geq 1$. Par suite, $M^{f^m}(n, \varepsilon) \leq M^{f}(mn -m + 1, \varepsilon) \leq M^f(mn, \varepsilon).$ 

Par continuit\'e de $f$, pour tout $\varepsilon > 0$, il existe $\delta(\varepsilon) > 0$ tel que $B(x, \delta(\varepsilon)) \subset B_{\dd^f_m}(x, \varepsilon).$ Alors 
$$
\begin{aligned}
B_{\dd^{f^m}_n}(x, \delta(\varepsilon)) &= \bigcap_{i=0}^{n-1} f^{-im} B(f^{im}(x), \delta(\varepsilon)) \\
&\subset \bigcap_{i=0}^{n-1} f^{-im} B_{\dd^f_m}(f^{im}(x), \varepsilon) \\
&= B_{\dd^{f}_{mn}}(x, \varepsilon).
\end{aligned}
$$
Ici, on a utilis\'e que
$$
B_{\dd^f_n}(x) = \bigcap_{k=0}^{n-1} f^{-k}B(f^k(x), \varepsilon).
$$
Il suit que $M^{f^m}(n, \delta(\varepsilon)) \geq M^{f}(mn, \varepsilon)$, et donc 
$$
M^{f}(mn, \varepsilon) \leq M^{f^m}(n, \delta(\varepsilon)) \leq M^f(mn, \delta(\varepsilon)),
$$
ce qui donne $\htop(f^m) = m\htop(f)$. 

Si $f$ est inversible on a $B_{\dd^f_n}(x, \varepsilon) = B_{\dd^{f^{-1}}_n}(f^{n-1}(x), \varepsilon)$ pour tous $n, x, \varepsilon$, ce qui conclut.

\item On a que $\id : (X, \dd) \to (X, \dd')$ est un hom\'eomorphisme puisque $\dd$ et $\dd'$ engendrent la m\^eme topologie. De plus $f \circ \id = \id \circ f$ donc les syst\`emes dynamiques topologiques $(X, \dd, f)$ et $(X, \dd', f)$ sont conjugu\'es. Cela conclut par un th\'eor\`eme du cours.


\item On a que
$
B_{\dd^{f\times g}_n}((x,y), \varepsilon) = B_{\dd^f_n}(x, \varepsilon) \times B_{\dd^g_n}(y, \varepsilon).
$
Ceci implique que $M^{f \times g}(n, \varepsilon) \leq M^f(n, \varepsilon) M^g(n, \varepsilon)$, et donc $\htop(f\times g) \leq \htop(f) + \htop(g)$.

Soient $x_1, \dots, x_m \in X$ (resp. $y_1, \dots y_p \in Y$) tels que pour tous $1 \leq i \neq i' \leq m$ (resp. $1 \leq j \neq j' \leq p$) on ait
$
\dd_{n}^f(x_i, x_j) \geq \varepsilon$ (resp. $\dd^g_n(y_i, y_j) \geq \varepsilon$). Alors pour tous $(i,j) \neq (i',j')$ on a 
$$
\dd^{f\times g}_{n}((x_i, y_j), (x_{i'}, y_{j'})) \geq \varepsilon.
$$
Par cons\'equent $N^{f\times g}(n, \varepsilon) \geq N^f(n, \varepsilon) N^g(n, \varepsilon)$, et donc $\htop(f \times g) \geq \htop(f) + \htop(g)$.

\end{enumerate}
\vspace{0.6cm}


\noindent {\large \textbf{Exercice 6.} \textit{Entropie des transformations Lipschitziennes}} \vspace{1.5mm} 


\begin{enumerate}
\item Soit $n \geq 1$. Il existe $c > 0$ telle que pour tout $\varepsilon > 0$ on a
$$
c^{-1} \varepsilon^{-n} \leq M([0,1]^n, \varepsilon) \leq c  \varepsilon^{-n}.
$$
Par suite
$$
\frac{-c + n \log 1/ \varepsilon}{\log 1 / \varepsilon} \leq \frac{\log M([0,1]^n, \varepsilon)}{\log 1/\varepsilon} \leq \frac{n \log 1/\varepsilon}{\log 1 / \varepsilon},
$$
ce qui conclut.

\item Soit $L > \max(1, L(f)).$ Alors $\dd(f(x), f(y)) \leq L \dd(x,y)$ pour tous $x,y \in X$. Cela implique que
$$
f^m\left(B(x, \varepsilon / L^n)\right) \subset B(f^m(x), \varepsilon), \quad 0 \leq m \leq n,
$$
et donc 
$$
B(x, \varepsilon / L^n) \subset \bigcap_{m=0}^{n-1} f^{-m} B(f^m(x), \varepsilon) = B_{\dd^f_n}(x, \varepsilon), \quad \forall x, \varepsilon.
$$
Ainsi on obtient
$$
\begin{aligned}
\frac{1}{n} \log M^f(n, \varepsilon) &\leq \frac{1}{n} \log M(X, \varepsilon / L^n) \\
&= \frac{\log(L^n/ \varepsilon)}{n} \frac{\log M(X, \varepsilon / L^n)}{\log(L^n/ \varepsilon)} \\
&= \left(\log L - \frac{\log \varepsilon}{n}\right) \frac{\log M(X, \varepsilon / L^n)}{\log(L^n/ \varepsilon)}.
\end{aligned}
$$
Puisque $\log L > 0$ on obtient 
$$
\limsup_n \frac{1}{n} M^f(n, \varepsilon) \leq \log(L) \mathrm{bdim}(X),
$$
et donc $\htop(f) \leq \log(L) \mathrm{bdim}(X)$.

\item Par le cours, l'application doublante $E_2 : [x] \mapsto [2x]$ sur $X = S^1$ satisfait cette \'egalit\'e, puisque $\mathrm{bdim}(S^1) = 1$, et $\htop(E_2) = \log 2$. 
\end{enumerate}
\vspace{0.6cm}

\noindent {\large \textbf{Exercice 7.} \textit{Entropie alg\'ebrique}} \vspace{1.5mm} 

\begin{enumerate}
\item Soit $i \in \{1, \dots, s\}$. et $m,n \geq 0$. Alors $F^n(\gamma_i)$ peut s'\'ecrire 
$$
F^n(\gamma_i) = \lambda_1 \cdots \lambda_{L(n, \Gamma)}, \quad \lambda_j \in \Gamma.
$$
On a donc
$$
F^{m+n}(\gamma_i) = F^m(\lambda_1) \cdots F^m(\lambda_{L(n,\Gamma)}).
$$
Chaque $F^m(\lambda_j)$ peut s'\'ecrire comme un produit d'\'el\'ements $\Gamma$ de $L(m, \Gamma)$ termes. Ceci montre que
$$
L_{n+m}(F, \Gamma) \leq L_n(F, \Gamma)L_m(F, \Gamma).
$$
Ainsi la suite $(\log L_n(F,\Gamma))_n$ est sous-additive, ce qui conclut.


\item Soit $\Gamma' = \{\gamma_1', \dots, \gamma_r'\}$ est un autre syst\`eme de g\'en\'erateurs. Soient 
$$
k = \max_{1 \leq j \leq r} L(\gamma_j', \Gamma), \quad k' = \max_{1 \leq i \leq s} L(\gamma_i, \Gamma').
$$
Alors pour tout $g \in G$ on a 
$$
L(g, \Gamma) \leq k' L(g, \Gamma') \leq kk' L(g, \Gamma).
$$
En particulier $L_n(F, \Gamma) \leq k' L_n(F, \Gamma') \leq kk' L_n(F, \Gamma)$, ce qui conclut.

\item Soit $\Gamma$ un syst\`eme de g\'en\'erateurs de $G$. Alors on a
$$
L_n(I_{\gamma_0} F, \Gamma) -2c \leq  L_n(F, \Gamma) \leq L_n(I_{\gamma_0} F, \Gamma) + 2c
$$
o\`u $c = \max(L(\gamma_0, \Gamma), L(\gamma_0^{-1}, \Gamma)).$ Cela conclut.


\item Soit $x_\star' \in X$ un autre point base et $\alpha'$ un chemin joignant $x_\star'$ \`a $f(x_\star')$. Soit $G' = \pi_1(M, x_\star')$. Soit $\beta$ un chemin joignant $x_\star$ \`a $x_\star'$ Alors l'application $\psi : G \to G'$ d\'efinie par $\psi(\gamma) = \beta^{-1} \gamma \beta$ est un isomorphisme de groupes. On a
$$
\begin{aligned}
F_{x_\star, \alpha}(\gamma) &=  \alpha^{-1}(f \circ \gamma) \alpha \\
&= \alpha^{-1}(f \circ \beta)^{-1}(f \circ \beta) (f \circ \gamma) (f \circ \beta)^{-1}(f \circ \beta) \alpha \\
&= \alpha^{-1} (f \circ \beta)^{-1} \alpha' \alpha'^{-1}(f \circ (\beta \gamma \beta^{-1})) \alpha' \alpha'^{-1}(f \circ \beta) \alpha,
\end{aligned}
$$
ce qui montre que $F_{x_\star, \alpha} = \phi^{-1} F_{x_\star', \alpha'} \psi$ o\`u $\phi : G \to G'$ est un isomorphisme de la forme $\gamma \mapsto \beta'^{-1} \gamma \beta'$ o\`u $\beta'$ est un chemin joignant $x_\star$ \`a $x_\star'$. En proc\'edant comme \`a la question pr\'ec\'edente, on conclut.


\end{enumerate}


\iffalse
\noindent {\large \textbf{Exercice 5.} \textit{Croissance des orbites p\'eriodiques et entropie des applications expansives}} \vspace{1.5mm} 

\noindent Soit $(X, \dd)$ un espace m\'etrique compact et $f : X \to X$ une application expansive, c'est-\`a-dire qu'il existe $\delta > 0$ tel que pour tous $x, y \in X$,
$$
\quad \sup_{n\in \Z} \dd(f^n(x),f^n(y)) \leq \delta \implies x = y.
$$
Pour tout $n \in \N$, on note
$$
p_n(f) = \#\{x \in X, f^n(x) = x\}.
$$
On d\'efinit aussi le taux de croissance exponentielle de la s\'equence $p_n(f)$,
$$
p(f) = \limsup_{n \to \infty} \frac{\log(1 + p_n(f))}{n}.
$$
\begin{enumerate}
\item Montrer que $p_n(f)$ est fini pour tout $n \in \N.$
\item Montrer que 
\begin{equation}\label{eq:lowerboundentropy}
p(f) \leq h_{\mathrm{top}}(f),
\end{equation}
\item Donner un exemple d'application $f$ telle que (\ref{eq:lowerboundentropy}) soit une \'egalit\'e.
\item Montrer que pour toute matrice $A \in \mathrm{GL}(m,\mathbb{Z})$ hyperbolique (i.e. dont les valeurs propres sont toutes de module diff\'erent de $1$), on a
$$
\sum_{\substack{\lambda \in \mathrm{sp}(A) \\ |\lambda| > 1 }} \log |\lambda| \leq h_\mathrm{top}(f_A),
$$
o\`u $f_A : \T^m \to \T^m$ est l'automorphisme toral associ\'e \`a $A$.
\end{enumerate}


\vspace{0.6cm}
\fi



%%%


\end{document} 
