\documentclass[a4paper,12pt,openany]{article}
\usepackage{fancyhdr}
\usepackage[T1]{fontenc}
\usepackage[margin=1.8cm]{geometry}
\usepackage[applemac]{inputenc}
\usepackage{lmodern}
\usepackage{enumitem}
\usepackage{microtype}
\usepackage{hyperref}
\usepackage{enumitem}
\usepackage{dsfont}
\usepackage{amsmath,amssymb,amsthm}
\usepackage{mathenv}
\usepackage{amsthm}
\usepackage{graphicx}
\usepackage[all]{xy}
\usepackage{lipsum}       % for sample text
\usepackage{changepage}
\theoremstyle{plain}
\newtheorem{thm}{Theorem}[section]
\newtheorem*{thm*}{Th\'eor\`eme}
\newtheorem{prop}[thm]{Proposition}
\newtheorem{cor}[thm]{Corollary}
\newtheorem{lem}{Lemme}
\newtheorem{propr}[thm]{Propri\'et\'e}
\theoremstyle{definition}
\newtheorem{deff}[thm]{Definition}
\newtheorem{rqq}[thm]{Remark}
\newtheorem{ex}[thm]{Exercice}
\newcommand{\e}{\mathrm{e}}
\newcommand{\prodscal}[2]{\left\langle#1,#2\right\rangle}
\newcommand{\devp}[3]{\frac{\partial^{#1} #2}{\partial {#3}^{#1}}}
\newcommand{\w}{\omega}
\newcommand{\dd}{\mathrm{d}}
\newcommand{\x}{\times}
\newcommand{\ra}{\rightarrow}
\newcommand{\pa}{\partial}
\newcommand{\vol}{\operatorname{vol}}
\newcommand{\dive}{\operatorname{div}}
\newcommand{\T}{\mathbf{T}}
\newcommand{\R}{\mathbf{R}}
\newcommand{\Z}{\mathbf{Z}}
\newcommand{\N}{\mathbf{N}}
\newcommand{\C}{\mathbf{C}}
\newcommand{\F}{\mathcal{F}}
\newcommand{\Homeo}{\mathrm{Homeo}}
\newcommand{\Matn}{\mathrm{Mat}_{n \times n}}
\DeclareMathOperator{\tr}{tr}
\newcommand{\id}{\mathrm{id}}
\newcommand{\htop}{h_\mathrm{top}}


\title{\textsc{Syst\`emes dynamiques} \\ÊCorrig\'e 3}
\date{}
\author{}

\begin{document}

{\noindent \'Ecole Normale Sup\'erieure  \hfill \texttt{chaubet@dma.ens.fr}} \\
{2021/2022 \hfill}

{\let\newpage\relax\maketitle}
\maketitle

\noindent {\large \textbf{Exercice 1.} \textit{Ensemble $\omega$-limite non minimal}} \vspace{1.5mm} 

\noindent Soit $X = \{0, 1\}^\N$ et $\sigma : X \to X$ le d\'ecalage. Soit
$$
x = 010011000111...
$$
Alors les singletons $\{000...\}$ et $\{111...\}$ sont deux parties ferm\'ees invariantes distinctes qui sont contenues dans $\omega(x)$.
\vspace{0.6cm}

\noindent {\large \textbf{Exercice 2.} \textit{Croissance des orbites p\'eriodiques et entropie des applications expansives}} \vspace{1.5mm} 

\begin{enumerate}
\item Soient $n \in \N^*$ et $x \in X$ tels que $f^n(x) = x$. Soit $\varepsilon > 0$ tel que pour tous $y \in B(x, \varepsilon)$ on a
$$
\dd(f^k(x), f^k(y)) \leq \delta, \quad 0 \leq k \leq n.
$$
Alors si $f^n(y) = y$ et $\dd(x,y) \leq \varepsilon$, on a $\dd(f^k(x), f^k(y)) \leq \delta$ pour tout $k \in \N$, et donc $x = y$. En particulier l'ensemble $\mathcal{P}_n(f) = \{x  \in X,~Êf^n(x) = x\}$ ne contient que des points isol\'es et donc $p_n(f)$ est fini par compacit\'e. 

De plus, pour tous $x\neq y \in \mathcal{P}_n(f)$, on a $\dd^f_n(x,y) > \delta$ (sinon on aurait $x=y$ par expansivit\'e). Ainsi $\mathcal{P}_n(f)$ est une famille de points qui est $\delta$-s\'epar\'ee pour la distance $\dd^f_n$. Ceci donne que pour tout $\alpha < \delta$ on a (avec les notations du cours)
$$
N(\alpha, n) \geq N(\delta, n) \geq p_n(f).
$$
Cela conclut.
\item On consid\`ere $E_m : \R/\Z \to \R/\Z$ d\'efinie par $x \mapsto mx$, pour $m \geq 2$. Alors
$$
p_n(E_m) = m^n-1
$$
ce qui implique que $p(E_m) = \log(m) = \htop(E_m)$ (cf. le cours).

\item On a
$$
\#\{ x \in \T^m,~A^n(x) = x\} = \#\{x \in \T^m,~(A^n-\mathrm{Id})x = 0\} = |\mathrm{det}(A^n-1)|
$$
par le TD n¡1. On a 
$$
\mathrm{sp}(A^n-1) = \{ \lambda^n - 1,~\lambda \in \mathrm{sp}(A)\}.
$$
 Il suit que (la somme porte sur les valeurs propres compt\'ees avec multiplicit\'e alg\'ebrique)
 $$
\log|\mathrm{det} (A^n-1)| = \sum_{\substack{\lambda \in \mathrm{sp}(A)}} \log |\lambda^n-1|.
$$
On a
$$
\lim_n \frac{1}{n} \log |\lambda^n-1| = \left\{\begin{matrix} \log |\lambda| \text{ si }|\lambda| > 1, \\ 0 \text{ sinon.}
\end{matrix} \right.
$$
Le lemme suivant permet de conclure par la question 1.
\begin{lem}
$f_A$ est expansive dans le sens o\`u il existe $\delta > 0$ tel que pour tous $x,y \in X$,
$$
\sup_{n \in \Z} \dd(f^n_A(x), f^n_A(y)) \leq \delta \quad \implies \quad x = y.
$$
\end{lem}
\begin{proof}
Supposons d'abord que $A \in \mathrm{Mat}_{n \times n}(\Z)$ est une matrice telle que $\mathrm{sp}(A) \subset \{z \in \C,~|z|> 1\}$. On pose $\mu = \min_{\lambda \in \mathrm{sp}(A)} |\lambda|$. Alors pour tout $\nu \in ]1, \mu[$, il existe une constante $C > 0$ telle que pour tout $N \geq 0$ on a 
$$
\|A^NX \|Ê\geq C \nu^N \|X\|, \quad X \in \R^n.
$$
Cela se v\'erifie ais\'ement en utilisant la d\'ecomposition de Jordan. Soit $0 < \varepsilon < \displaystyle{ \frac{1}{4 \|A\|}}$ (ici $\|A\|$ d\'esigne la norme d'op\'erateur de $A$), et $y \in \R^n \setminus 0$ tel que $\|y \|\leq \varepsilon.$ Soit 
$$N_0 = \max\{N \geq 0,~\|A^nY\|\leq 2 \varepsilon\} + 1.$$
Alors $\|A^{N_0}Y\| \geq 2 \varepsilon$ et 
$$
\|A^{N_0}Y\|Ê\leq \|A\|\|A^{N_0-1}Y\|\leq \|A\| (2\varepsilon) \leq \frac{1}{2}.
$$
On a montr\'e que pour tout $X \in \R^{n}\setminus 0$
$$
 \|X\|\leq \varepsilon \quad \implies \quad \exists n \in \N, \quad 2\varepsilon \leq \|A^nX\| \leq 1/2.
$$
Ceci implique que pour tous $x\neq y \in \T^n$,
\begin{equation}\label{eq:exp}
\dd(x, y) \leq \varepsilon \quad \implies \quad \exists n \in \N, \quad \dd(f^n_A(x), f^n_A(y)) \geq 2\varepsilon,
\end{equation}
et donc $f_A$ est expansive.

Si on suppose juste que $\mathrm{sp}(A) \cap \{zÊ\in \C, |z|Ê= 1\} = \emptyset$ alors on peut appliquer le raisonnement pr\'ec\'edent \`a $(A^{\pm 1})|_{E_\pm}$ o\`u
$$
E_\pm = \lim_{N \to \infty} \bigoplus_{|\lambda|^{\pm 1} > 1} \mathrm{ker}(A - \lambda\mathrm{Id})^N ;
$$
on obtient alors (\ref{eq:exp}) en rempla\c ant $\N$ par $\Z$, ce qui conclut.
\end{proof}
\end{enumerate}


\vspace{0.6cm}


\noindent {\large \textbf{Exercice 3.} \textit{Codage symbolique de l'application du Chat d'Arnold}} \vspace{1.5mm} 

\begin{enumerate}
\item On consid\`ere $B = (b_{ij}) \in \mathrm{Mat}_{5\times 5}(\Z)$ d\'efinie par
$$
B = \begin{pmatrix}
1 & 1 & 0 & 1 & 0 \\
1 & 1 & 0 & 1 & 0 \\
1 & 1 & 0 & 1 & 0 \\
0 & 0 & 1 & 0 & 1 \\
0 & 0 & 1 & 0 & 1
\end{pmatrix}.
$$
On note 
$\Sigma_B = \{\omega = (\omega_n)_{n \in \Z},~ b_{\omega_n \omega_{n+1}} = 1\}$
l'alphabet associ\'e \`a $B$. Alors pour tout $\omega \in \Sigma_B$, on a 
$$
\Delta(\omega) = \bigcap_{n \in \Z} f^{-n}( \Delta_{\omega_n}) \neq \emptyset,
$$
puisque pour tout $N \geq 1,$ l'intersection $ \Delta(\omega, N) = \bigcap_{|n|\leq N} f^{-n}(\Delta_{\omega_n})$ est non vide. De plus, il existe une constante $C>0$ telle que pour tout $N$, on a que $\Delta(\omega, N)$ est un rectangle qui v\'erifie que
\begin{equation}\label{eq:deltaw}
\mathrm{diam}(\Delta(\omega, N)) \leq C \lambda^{-N}.
\end{equation}
Il s'en suit que $\Delta(\omega)$ est r\'eduit \`a un point, not\'e $x(\omega)$. Soit $h : \Sigma_B \to \T^2$ d\'efinie par $h(\omega) = x(\omega).$ 

Alors $h$ est surjective. En effet, si $x \in \T^2$, on choisit pour tout $n \in \Z$ un $\omega_n \in \{0,1\}$ tel que $f^n(x) \in \Delta_{\omega_n}$ ; ceci implique que $h((\omega_n)_n) = x$.

L'application $h$ est aussi continue. En effet, soit $\varepsilon > 0$ et $N \geq 1$ tels que $C\lambda^{-N} \leq \varepsilon.$ Soient $\omega, \omega' \in \Sigma_B$ tels que $\omega_j = \omega_j'$ pour tout $|j|Ê\leq N$. Alors $h(\omega), h(\omega') \in \Delta(\omega, N)$, et (\ref{eq:deltaw}) implique que $\dd(h(\omega), h(\omega')) \leq \varepsilon.$

On a bien sžr $f_L\circ h = h \circ \sigma_B$, o\`u $\sigma_B$ est le d\'ecalage sur $\Sigma_B$, ce qui montre que $(\T^2, f_L)$ est un facteur topologique de $(\Sigma_B, \sigma_B)$.

\item Par la question pr\'ec\'edente et le cours on obtient pour $\lambda = (3 + \sqrt{5})/2$
$$
\htop(f_L) \leq \htop(\sigma_B) = \log \rho(B) = \log \lambda.
$$
On a aussi par l'exercice pr\'ec\'edent 
$$
\htop(f_L) \geq p(f_L) = \log \lambda.
$$
Ainsi 
$$\htop(f_L) = \log \frac{3 + \sqrt{5}}{2}.$$
\end{enumerate}


\vspace{0.6cm}
%%%

\noindent {\large \textbf{Exercice 4.} \textit{Fonctions z\^eta dynamiques}} \vspace{1.5mm} 

\noindent

\begin{enumerate}
\item Soit $|z|<\exp(-p(f))$ et $\varepsilon > 0$ tel que $|z|<\exp(-p(f)-\varepsilon)$. Il existe $C$ tel que tout $n$ assez grand $p_n(f) \leq C \exp((p(f) + \varepsilon/2)n)$ pour tout $n$ assez grand, par d\'efinition de $p(f)$. Alors pour tout $n$ assez grand on a
$$
\left|\frac{p_n(f)}{n}z^n\right| \leq \frac{C}{n} \e^{-n\varepsilon/2},
$$
et donc $\zeta_f(z)$ est bien d\'efinie.
\item Montrer, dans les cas suivants, que $\zeta_f$ est une fonction rationnelle admettant un p™le simple au point $z = \exp \left(-h_\mathrm{top}(f)\right)$, et que
$$
p_n(f) \sim \exp\left(n\htop(f)\right) \quad (n \to \infty).
$$
\begin{enumerate}
\item On a $p_n(E_m) = m^n-1$ pour tout $n \geq 1$. On calcule
$$
\begin{aligned}
\zeta_{E_m}(z) = \exp \sum_{n=1}^\infty \frac{m^n-1}{n}z^n 
= \frac{1-z}{1-mz}.
\end{aligned}
$$
Ainsi $\zeta_{E_m}$ a un p™le simple en $z = 1/m = \exp(-\log m) = \exp(-\htop(E_m))$.

\item On a $p_n(f_L) = |\mathrm{det}(L^n-1)| = -\mathrm{det}(L^n-1)$ et donc, si $\lambda = (3+\sqrt{5})/2$,
$$
\begin{aligned}
\zeta_{f_L}(z) &= \exp -\sum_{n=1}^\infty \frac{\mathrm{det}(L^n-1)}{n} z^n \\
&= \exp -\sum_{n=1}^\infty \frac{(\lambda^n-1)(\lambda^{-n}-1)}{n} z^n \\
&= \exp-\sum_{n=1}^\infty \frac{2 - \lambda^n - \lambda^{-n}}{n}z^n,
\end{aligned}
$$
ce qui donne comme \`a la question pr\'ec\'edente
$$
\zeta_{f_L}(z) = \frac{(1-z)^2}{(1-z\lambda)(1-z/\lambda)}.
$$
Encore une fois, $\zeta_{f_L}$ a un p™le simple en $\lambda^{-1} = \exp(-\log \lambda) = \exp(-\htop(f_L))$.

\item Par le cours on a $p_n(\sigma_A) = \tr A^n.$ Ainsi
$$
\zeta_{\sigma_A}(z) = \exp \sum_{n=1}^\infty \frac{\tr A^n}{n}z^n = \frac{1}{\det(1-zA)}.
$$
Par le th\'eor\`eme de Perron-Frobenius, on a $\det(1-zA) = (1-z\rho(A))P(z)$ o\`u $P$ est un polyn™me qui n'admet que des racines de modules strictement inf\'erieurs \`a $\rho(A)$. Cela conclut puisque $\htop(\sigma_A) = \log \rho(A)$. 

\end{enumerate}

\item On note $\mathcal{Q}_n(f)$ l'ensemble des points p\'eriodique de p\'eriode exactement $n$, et $\mathcal{O}(n)$ l'ensemble des orbites de p\'eriode $n$. Alors $\# \mathcal{Q}_n(f) = n \# \mathcal{O}(n)$ et (toutes les op\'erations sont licites pour $|z| < \e^{-p(f)}$)
$$
\begin{aligned}
\zeta_f(z) &= \exp \sum_{n=1}^\infty \frac{\#\mathcal{P}_n(f)}{n} z^n \\
&= \exp \sum_{n=1}^\infty \frac{\sum_{k|n} \#\mathcal{Q}_k(f)}{n}z^n \\
&= \exp \sum_{k=1}^\infty \sum_{j=1}^\infty \#\mathcal{Q}_k(f)\frac{z^{kj}}{kj} \\
& = -\exp \sum_{k=1} ^\infty  \#\mathcal{O}(k) \log(1-z^k) \\
& = \prod_{p \in \mathcal{P}} \frac{1}{1 - z^{|p|}}.
\end{aligned}
$$

\end{enumerate}

\vspace{0.6cm}

\noindent {\large \textbf{Exercice 5.} \textit{Toute transformation continue surjective est facteur d'un hom\'eomorphisme}} \vspace{1.5mm} 


Soit $\hat X = X^{-\N}$ et $\hat f : \hat X \to \hat X$ d\'efinie par 
$$
\hat f : (x_n)_{n \leq 0} \mapsto (f(x_n))_{n \leq 0}.
$$
On note 
$$
\tilde X = \{\tilde x = (x_n)_{n\leq 0} \in X^{-\N},~k<0 \implies x_{k+1} = f(x_k) \}.
$$
Alors $\tilde X$ est une partie ferm\'ee et positivement invariante par $\hat f$. On note $\widetilde f$ la restriction de $\hat f$ \`a $\tilde X$. Soit $h : \tilde X \to X$ d\'efinie par 
$$h : (x_n)_{n\leq0} \mapsto x_0.$$
 Alors $h$ est continue et v\'erifie $h \circ \tilde f = f \circ h$. 

Montrons que $h$ est surjective. Soit $x \in X$. Par surjectivit\'e de $f$, il existe $x_{-1}$ tel que $f(x_{-1}) = x$. En it\'erant ce processus, on obtient $\tilde x = (x_{-n})_{n \geq 0} \in \tilde X$ tel que $\tilde f( \tilde x) = x$. 

Il reste \`a montrer que $\tilde f$ est un hom\'eomorphisme. On d\'efinit $ \tilde g : \tilde X \to \tilde X$ par 
$$
\tilde g : (x_k)_{k \leq 0} \mapsto (x_{k-1})_{k \leq 0}.
$$
Alors $g$ est continue et v\'erifie $\tilde g \circ \tilde f = \tilde f \circ \tilde g = \mathrm{id}_{\tilde X}$, ce qui conclut.
\end{document} 
