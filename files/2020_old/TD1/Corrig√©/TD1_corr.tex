\documentclass[a4paper,12pt,openany]{article}
\usepackage{fancyhdr}
\usepackage[francais]{babel}
\usepackage[T1]{fontenc}
\usepackage[margin=1.8cm]{geometry}
\usepackage[applemac]{inputenc}
\usepackage{lmodern}
\usepackage{enumitem}
\usepackage{microtype}
\usepackage{hyperref}
\usepackage{enumitem}
\usepackage{dsfont}
\usepackage{amsmath,amssymb,amsthm}
\usepackage{mathenv}
\usepackage{amsthm}
\usepackage{graphicx}
\usepackage[all]{xy}
\usepackage{lipsum}       % for sample text
\usepackage{changepage}
\theoremstyle{plain}
\newtheorem{thm}{Theorem}[section]
\newtheorem*{thm*}{Th\'eor\`eme}
\newtheorem{prop}[thm]{Proposition}
\newtheorem{cor}[thm]{Corollary}
\newtheorem{lem}[thm]{Lemma}
\newtheorem{propr}[thm]{Propri\'et\'e}
\theoremstyle{definition}
\newtheorem{deff}[thm]{Definition}
\newtheorem{rqq}[thm]{Remark}
\newtheorem{ex}[thm]{Exercice}
\newcommand{\e}{\mathrm{e}}
\newcommand{\prodscal}[2]{\left\langle#1,#2\right\rangle}
\newcommand{\devp}[3]{\frac{\partial^{#1} #2}{\partial {#3}^{#1}}}
\newcommand{\w}{\omega}
\newcommand{\dd}{\mathrm{d}}
\newcommand{\x}{\times}
\newcommand{\ra}{\rightarrow}
\newcommand{\pa}{\partial}
\newcommand{\vol}{\operatorname{vol}}
\newcommand{\dive}{\operatorname{div}}
\newcommand{\T}{\mathbf{T}}
\newcommand{\R}{\mathbf{R}}
\newcommand{\Z}{\mathbf{Z}}
\newcommand{\N}{\mathbf{N}}
\newcommand{\F}{\mathcal{F}}
\newcommand{\Homeo}{\mathrm{Homeo}}
\newcommand{\Matn}{\mathrm{Mat}_{n \times n}}
\DeclareMathOperator{\tr}{tr}
\newcommand{\id}{\mathrm{id}}


\title{\textsc{Syst\`emes dynamiques} \\�Corrig\'e 1}
\date{}
\author{}

\begin{document}

{\noindent \'Ecole Normale Sup\'erieure  \hfill Yann Chaubet } \\
{2020/2021 \hfill�\texttt{chaubet@dma.ens.fr}}

{\let\newpage\relax\maketitle}
\maketitle
\noindent {\large \textbf{Exercice 1.} \textit{Points pr\'ep\'eriodiques}} 

{\begin{enumerate}
\item Soit $X = \{0, 1\}$ et $f : X \to X$ d\'efinie par $f(0) = f(1) = 1$. Alors $0$ est pr\'ep\'eriodique pour $f$. Si $g : Y \to Y$ est bijective, et $g^m(y) = g^n(y)$ o\`u $m \geq n \geq 0$, alors $g^{m-n}(y) = y$ donc $y$ est p\'eriodque.
\item Soit $x \in X$. Alors le cardinal de l'orbite de $x$ est fini puisque $X$ est fini. Donc il existe $m, n \geq 0$ tels que $f^m(x) = f^n(x)$. Donc $x$ est pr\'ep\'eriodique ou p\'eriodique. 
\end{enumerate}
}
\hfill \break 
%% \\

\noindent {\large \textbf{Exercice 2.} \textit{Lemme de prolongation}} \\

\noindent On fixe $c \in (a,b)$ et on \'ecrit 
$$
x(t) = x(c) + \int_c^t \dot x(s) \dd s = x(c) + \int_c^t V(x(s)) \dd s.
$$
On a $x(s) \in K$ pour tout $s \in (a,b)$. En particulier $s \mapsto V(x(s))$ est born\'ee sur $[c,b)$ et donc int\'egrable. Ainsi
$
\lim_b x = x(c) + \int_c^b V(x(s)) \dd s.
$
\hfill \break \\

\noindent {\large \textbf{Exercice 3.} \textit{Automorphismes lin\'eaires du tore de dimension 2}} \\ 

\begin{enumerate}
\item Une condition n\'ecessaire et suffisante est $|\mathrm{det}(A)| = 1$. En effet si $\det(A) = \pm 1$ on a que $A^{-1}$ est \`a coefficients entiers par la formule de la comatrice. Donc $f_A \circ f_{A^{-1}} = \id_{\T^2}$. La r\'eciproque d\'ecoule directement du fait suivant. \\ \\
\textbf{Fait.} Pour tout $A \in \mathrm{Mat}_{n \times n}(\mathbb{Z})$ et tout $p \in \T^n$ on a $ \# f_A^{-1}(\{p\}) = |{\det}(A)|.$
\begin{proof}
On note $C = [0, 1[^n$. Alors le nombre de pr\'eimages de tout point de $\T^n$ par $f_A$ est le nombre de points \`a coordonn\'ees enti\`eres de l'image de $C$ par $A$.  Puisque $A$ est \`a coefficients entiers, on peut d\'ecouper $A(C)$ en un nombre fini de polytopes, puis appliquer des translations enti\`eres \`a ces morceaux pour obtenir $[0, \det(A)[ \times [0, 1[^{n-1}$. Le nombre de points entiers est pr\'eserv\'e au cours des transformations et vaut donc $|{\det(A)}|$.
\end{proof}
\item \textbf{Cas $\tr(A) = 0$.} Alors les racines du polyn�me caract\'eristique de $A$ satisfont $\lambda \mu = 1$ et $\lambda = - \mu$, soit $\lambda = \pm i$. En particulier les valeurs propres de $A$ sont $\pm i$ est $A^4 = I_2$. En particulier $(f_A)^4 = \id_{\T^2}$.\\
\textbf{Cas $\tr(A) = 1$.} Alors les racines du polyn�me caract\'eristique de $A$ satisfont $\lambda \mu = 1$ et $\lambda = - \mu + 1$, soit $\mu^2 - \mu + 1 = 0$. On obtient $\mu = \pm (1/2 + i \sqrt{3}/2)$ et donc $A^6 = I_2$, soit $(f_A)^6 = \id_{\T^2}$.
\item On suppose $\tr(A) = 2$. Le polyn�me caract\'eristique de $A$ est alors donn\'e par $X^2 - 2X +1$, donc $A$ admet un vecteur propre $u$ tel que $Au = u$, qu'on peut supposer (quitte \`a appliquer la transformation $(x, y) \mapsto (y,x)$) de la forme $\begin{pmatrix} 1 \\\alpha \end{pmatrix}$ avec $\alpha \in \mathbb{Q}$ (car $A$ est \`a coefficients entiers). Soit $y \in [0,1[$ et $t \in \R$. Notons $A = \begin{pmatrix} a & b \\ c & d \end{pmatrix}$ et calculons
$$
A \left[ \begin{pmatrix} 0 \\ x  \end{pmatrix} + t u \right] = y \begin{pmatrix} b \\ d  \end{pmatrix} + t \begin{pmatrix} 1 \\ \alpha  \end{pmatrix}.
$$
On a $A u = u$ donc $a + b \alpha = 1$ et $\tr(A) = 2$ donc $a = 2 - d$. Par suite $d = 1 + b \alpha$ et donc si $v = \begin{pmatrix} 0 \\ x \end{pmatrix}$,
\begin{equation}\label{eq:rota}
A(v + tu) = x \begin{pmatrix} b \\ 1 + b \alpha  \end{pmatrix} + t u = v + (t + by) u.
\end{equation}
Pour $x \in [0,1[$ on note $C_y \subset \T^2$ l'image l'image de
$$
\left \{\begin{pmatrix} t \\ y + t \alpha \end{pmatrix},~t \in \R \right\}
$$
par la projection $\R^2 \to \T^2.$ Si $\alpha = p/q$ avec $p \in \Z$ et $q > 0$ premiers entre eux, on a $C_y \simeq \R/q\Z$ et sous cette identification, (\ref{eq:rota}) montre que $f_A|_{C_x}$ est la rotation $[\theta]�\mapsto [\theta + by].$ De plus on a la partition
$$
\T^2 = \bigsqcup_{[y]} C_y
$$
o\`u l'union porte sur les classes d'\'equivalences $[y] \in [0,1] / \sim$ o\`u $y \sim y'$ si $C_y = C_{y'}$
\\
Si $\tr(A) = -2$, on applique le raisonnement pr\'ec\'edent \`a $A^2$.
\item On suppose que $|{\tr A}| > 2$.
\begin{enumerate}
\item Le polyn�me caract\'eristique $X^2 - \tr(A) X +1$ a deux racines r\'eelles puisque $\tr(A)^2 - 4 > 0$. De plus $\tr(A)^2 - 4$ n'est pas un carr\'e parfait car $\tr(A) \in \Z$ avec $|\tr(A)| > 2$ (en effet l'\'equation $p^2 - 4 = q^2$ avec $p,q \in \Z$ n'admet que les solutions $p = \pm 2$ et $q=0$). Les valeurs propres de $A$ sont donc irrationnelles, et les vecteurs propres associ\'es ont une pente irrationnelle.
\item On note $u$ et $v$ les vecteurs propres associ\'es \`a $\lambda$ et $\lambda^{-1}$ avec $|\lambda| > 1$. Pour tout $y \in [0,1[$ on note $z_y = \begin{pmatrix} 0 \\ y  \end{pmatrix}$. Alors les images $F^u_y$ et $F^s_y$ des droites affines
$$
\left\{ z_y + t u, ~t \in \R\right\} \quad \text{et} \quad \left\{ z_y + t v, ~t \in \R\right\}
$$
par la projection naturelle $\R^2 \to \T^2$ forment des partitions
$$
\T^2 = \bigsqcup_{[y]} F^u_y = \bigsqcup_{[y]} F^s_y,
$$
o\`u l'union porte sur les classe d'\'equivalences $[y] \in [0,1] / \sim$, o\`u $y \sim y'$ si $F^u_y = F^u_{y'}$ (pour la premi\`ere) ou $F^s_y = F^s_{y'}$ (pour la deuxi\`eme). Les propri\'et\'es de contraction et de dilatation sont claires.

\end{enumerate}
\end{enumerate}
\hfill \break 

%%%
\noindent {\large \textbf{Exercice 4.} \textit{Persistance des orbites p\'eriodiques non d\'eg\'en\'er\'ees pour les flots}} \\�

\begin{enumerate}
\item Soit $\Sigma$ un hyperplan affine avec $x_0 \in \Sigma$ et tel que $x_0 + V(x_0) \not\in\Sigma$. Soit $\sigma \in \R^n$ un vecteur normal \`a $\Sigma$. Soit $\psi : \R \times \R^n \to \R$ d\'efinie par
$$
\psi(t, x) = \langle \varphi(t,x) - x_0, ~\sigma \rangle.
$$
Alors $\psi(\tau_0, x_0) = 0$. De plus, puisque $V(x_0)$ est transverse \`a $\Sigma$,
$$
\partial_t \psi(t,x) = \langle V(\varphi(t,x)), ~\sigma \rangle \neq 0;
$$
le th\'eor\`eme des fonctions implicites permet alors de conclure, puisque pour tout $z \in \R^n$ on a 
$$
z \in \Sigma \iff \langle z - x_0, ~\sigma \rangle = 0.
$$

\end{enumerate}

\begin{enumerate}[resume]
\item En appliquant la m\^eme construction qu'\`a la question pr\'ec\'edente en rempla�ant $\tau_0$ par $0$, on obtient (quitte \`a r\'eduire $U$) une application lisse $\tilde \tau : U \to \R$, avec $\tilde \tau(x_0) = 0$ et telle que 
$$
\varphi(\tilde \tau(x), x) \in \Sigma, \quad x  \in U.
$$
Soit $\Sigma'$ une autre hypersurface affine passant par $x_0$ et transverse \`a $V(x_0)$. On obtient de m\^eme que pr\'ec\'edemment deux applications $\tau', \tilde \tau' : U \to \R$ telles que $\tau'(x_0) = \tau_0$, $\tilde \tau'(x_0) = 0$ et
$$
\varphi(\tau'(x), x) \in \Sigma', \quad \varphi(\tilde \tau'(x), x) \in \Sigma',  \quad x  \in U.
$$
Soit $P_{\Sigma, \Sigma'} : U \cap \Sigma \to U \cap \Sigma'$ donn\'ee par
$
P_{\Sigma, \Sigma'}(x) = \varphi(\tilde \tau'(x), x).
$
Alors $P_{\Sigma, \Sigma'}$ est inversible d'inverse $x \mapsto P_{\Sigma', \Sigma}(x) = \varphi(\tilde \tau(x), x)$. On v\'erifie alors que pour tout $x$ assez proche de $x_0$,
$$
P_\Sigma(x) = \left(P_{\Sigma, \Sigma'} \circ P_{\Sigma'} \circ (P_{\Sigma, \Sigma'})^{-1}\right) (x),
$$
et donc 
$
(\dd P_\Sigma)_{x_0}
$
est conjugu\'ee \`a $(\dd P_\Sigma')_{x_0}$, ce qui conclut.
\item On consid\`ere $\Psi : \R \times \R \times \R^n \to \R^n$ d\'efinie par
$$
\Psi(s, t, x) = \langle \varphi_s(t, x) - x_0, \sigma \rangle.
$$
Alors 
$$
\partial_t \Psi(s, t, x) =  \langle V(\varphi_s(t,x)), \sigma \rangle,
$$
et donc $\partial_t \Psi(0, \tau_0, x_0) \neq 0$. Par le th\'eor\`eme des fonctions implicites, il existe une application lisse $T : (-\varepsilon, \varepsilon) \times U \to \R$ avec $T(0, x_0) = \tau_0$ et
$$
\varphi_s(T(s,x), x) \in \Sigma, \quad s \in (-\varepsilon, \varepsilon), \quad x \in U.
$$
On d\'efinit alors $\Phi : (-\varepsilon, \varepsilon) \times (U \cap \Sigma) \to \Sigma$ par
$$
\Phi(s, x) = \varphi_s(T(s,x), x) - x.
$$
On a $\Phi(0, x_0) = 0$ et
$$
\partial_x \Phi(0, x_0) = \dd (P_{\Sigma})_{x_0} - \mathrm{Id}_{T_{x_0}\Sigma}
$$
puisque $T(0, x) = \tau(x)$ pour tout $x \in U\cap \Sigma$. Ainsi $\partial_x \Phi(0, x_0) : T_{x_0}\Sigma \to T_{x_0} \Sigma$ est inversible et le th\'eor\`eme des fonctions implicites donne une application lisse $s \mapsto x_s$ d\'efinie pr\`es de $s=0$ telle que $\Phi(s, x_s)= 0$, ce qui \'equivaut \`a 
$$
\varphi_s(T(x_s,s), x_s) = x_s.
$$
On pose alors $\tau_s = T(s, x_s)$. Les applications $s \mapsto x_s$ et $s \mapsto \tau_s$ v\'erifient les conditions demand\'ees.
\end{enumerate}

\hfill \break 


\noindent {\large \textbf{Exercice 5.} \textit{Classes de conjugaison des applications expansives du cercle}} \\ 

\begin{enumerate}
\item On choisit $y_0 \in \R$ tel que $[y] = f([0])$. On pose alors $F(0) = y_0$. La fonction $f$ \'etant continue $\T \to \T$, elle l'est uniform\'ement, et il existe $1/2 > \varepsilon > 0$ tel que 
\begin{equation}\label{eq:unif}
\mathrm{dist}([y'], [y]) < \varepsilon \implies \mathrm{dist}(f([y']), f([y])) < 1/4.
\end{equation}
Soit $\pi : \R \to \T$ la projection naturelle. Alors $\pi|_{]y_0-1/4, y_0+1/4[}$ r\'ealise un hom\'emorphisme sur son image et on pose
$$
F(y') = (\pi|_{]y_0-1/4, y_0+1/4[})^{-1}(f([y'])), \quad y' \in ~�]y_0-\varepsilon, y_0 + \varepsilon[.
$$
On a donc construit $F$ sur $]y_0- \varepsilon, y_0 + \varepsilon[$ telle que $[F(y')] = f([y'])$ pour tous $y' \in ]y_0- \varepsilon, y_0 + \varepsilon[$. On it\`ere alors cette construction en rempla�ant $y_0$ par $y_0 \pm \varepsilon / 2$ pour \'etendre $F$ \`a $]y_0 - 3 \varepsilon / 2, y_0 + 3 \varepsilon /2[.$ En it\'erant ce processus, on obtient bien $F : \R \to \R$ telle que $\pi \circ F = f \circ \pi$. De plus $F$ est continue car $\pi$ est un hom\'eomorphisme local.

Si $F'$ est un autre relev\'e de $f$, on a $[F(y)] = [F'(y)]$ pour tout $y \in \R$, et donc $F-F'$ prends ses valeurs dans $\Z$. Comme elle est continue, elle est constante et $F' = F + k$ pour un $k \in \Z$.

\item Si $F$ est un relev\'e de $f$, alors $x \mapsto F(x+1)$ est aussi un relev\'e de $f$. Par la question pr\'ec\'edente, il existe $p$ tel que $F(x+1) = F(x) + p$ pour tout $x \in \R$. Soit $F'$ est un autre relev\'e et $k \in \Z$ tel que $F' = F + k$. Alors pour tout $x \in \R,$
$$F'(x+1) = F(x+1) + k = F(x) + p + k = F'(x) + p,$$
ce qui conclut.

\item \label{q:3} Supposons $p \geq 1$. Soit $F$ un relev\'e de $f$ et $G : x \mapsto F(x) - x$. On a que $G(1) = F(1) - 1 = F(0) + p - 1 = G(0) + p -1$. Ainsi $[G(0), G(0) + p-1[ \subset G([0,1[)$ par le th\'eor\`eme des valeurs interm\'ediaires, et donc 
$$
\#(\Z \cap G([0, 1[)) = p - 1.
$$
Ainsi on peut trouver $0 \leq x_1<\dots <x_{p-1}<1$ tels que $G(x_j) \in \Z$ pour tout $j$. Ceci implique $f([x_j]) = [x_j]$ pour tout $j$, et donc $\# \mathrm{Fix}(f) \geq p - 1.$

\item On a que $\deg(f^n) = p^n.$ En effet, si $F$ est un relev\'e de $f$ alors $F^n$ est un relev\'e de $f^n$, puisque
$$
[F^n(x)] = [F(F^{n-1})(x)] = f([F^{n-1}(x)]) = \dots = f^n([x]).
$$
De plus, pour $n \geq 2,$
$$
\begin{aligned}
F^{n}(x+1) &= F^{n-1}(F(x+1))�\\ &= F^{n-1}(F(x) + p) \\ &= F^{n-1}(F(x) + p - 1) + \deg(f^{n-1})  \\ &= \cdots \\ &= F^{n}(x) + p \deg f^{n-1}.
\end{aligned}
$$
Par suite $\deg(f^n) = p \deg(f^{n-1})$ et donc $\deg(f^n) = p^n.$
Ainsi $\#\mathrm{Fix}(f^n) \geq \deg(f^n) - 1 = p^n - 1$ par la question pr\'ec\'edente. Par suite,
$$
\liminf_n \frac{\log \#\mathrm{Fix}(f^n)}{n} \geq  \liminf_n \frac{\log(p^n-1)}{n} = \log p.
$$

\item Soit $F$ un relev\'e de $f$. On a $F' > 1$ et donc l'application $G : x \mapsto F(x)-x$ est strictement croissante. Ainsi (voir question \ref{q:3}), on a que $G|_{[0, 1[}$ r\'ealise un hom\'eomorphisme de $[0, 1[$ sur $[G(0), G(0) + p -1[$. Pour tout $x \in [0,1[$ on a 
$$
f([x]) = [x]�\quad \iff \quad G(x) \in \Z.
$$
Puisque $\#([G(0), G(0) + p -1[ ~\cap ~\Z) = p - 1$, on a que $\# \mathrm{Fix}(f) = p -1.$ 

Enfin, puisque $\deg(f^n) = \deg(f)^n,$ on a $\#\mathrm{Fix}(f^n) = p^n-1.$

\item Pour tout $H \in \mathcal{E}$, on a que $\Phi(H)$ est continue et pour tout $x \in \R$, $\Phi(H)(x+1) = \frac{1}{p} H(F(x+1)) = \frac{1}{p}H(F(x) + p) = \frac{1}{p}(H(F(x)) + p) = \Phi(H)(x) + 1$. Donc $\Phi$ pr\'eserve $\mathcal{E}$.

On a que $\mathcal{E}$ est un sous-ensemble ferm\'e de l'espace $C(\R, \R)$ des fonctions continues $\R \to \R$ muni de la norme infinie. Ainsi $\mathcal{E}$ est complet pour $d$. On a de plus
$$
d(\Phi(G), \Phi(H)) = \frac{1}{p}\sup_\R |G \circ F - H \circ F| = \frac{1}{p} d(G,H).
$$
Puisque $f'>1$, on a $F'>1$ et donc $p > 1$. Ainsi $\Phi$ est strictement contractante sur $(\mathcal{E},d)$. Le th\'eor\`eme du point fixe affirme alors qu'il existe un unique point fixe $H_0$, qui v\'erifie 
$$
H_0(x) = \frac{1}{p} H_0(F(x)), \quad x \in \R.
$$

\item On pose $h_0([x]) = [H_0(x)]$ pour tout $x \in \R$ (c'est bien d\'efini puisque $H_0(x+p) = H_0(x) + p$), et on a que $H_0$ rel\`eve $h_0$. De plus $H_0(x+1) = H_0(x) + 1$ pour tout $x$ implique que $\deg(h_0) = 1$.

De plus, on a que $E_p(h_0([x])) = [p H_0(x)] = [H_0(F(x))] = h_0([F(x)]) = h_0(f([x]))$. 

\item On peut d\'ej\`a remarquer que 
$$
F^{-1}(y + p) = F^{-1}(y) + 1, \quad y \in \R.
$$
Ceci implique 
$$
F^{-1}(H(p(x+1))) = F^{-1}(H(px) + p) = F^{-1}(H(px)) + 1, \quad x \in \R.
$$
Ainsi $\Psi$ pr\'eserve $\mathcal{E}$.
De plus puisque $F' > 1$ et $F(x+1) = F(x) + p$ on a  $(F^{-1})'(y) \leq \nu$ pour tout $y\in \R$, avec $0 <\nu < 1$. Par suite, pour tout $x \in \R$ et tous $H,G \in \mathcal{E}$,
$$
|F^{-1}(H(px)) - F^{-1}(G(px))|�\leq \nu |H(px) - G(px)|�\leq \nu \|H-G\|_{\infty}.
$$
Par suite $\Psi$ est strictement contractante sur $(\mathcal{E},d)$ et donc elle admet un unique point fixe $H_1$, qui v\'erifie
$$
H_1(y) = F^{-1}(H_1(py)), \quad x \in \R.
$$
Soit $G = H_0 \circ H_1.$ Alors
$$
G(px) = H_0(F(H_1(x))) = p G(x).
$$
Ceci implique que $G(0) = 0$ ; de plus pour tout $k \in \Z$, $G(k) = k$. Pour tout $m \in \N$, il s'en suit que
$$
G(k/p^m) = \frac{p^m}{p^m}G(k/p^m) = \frac{1}{p^m}G(k) = k/p^m.
$$
Puisque $\{k/p^m,~k \in \Z,~m \in \N\}$ est dense dans $\R$, il vient que $G = \mathrm{Id}_\R$ par continuit\'e. Par suite $H_1$ est bijective (car surjective, car $H(\cdot + 1) = H(\cdot) + 1$), et d'inverse $H_1^{-1} = H_0$ continu. Soit $h_1 : \T^2 \to \T^2$ l'application induite par $H_1$. Alors $h_1$ est bijective, d'inverse $h_0$, ce qui conclut.
\end{enumerate}



\end{document} 
