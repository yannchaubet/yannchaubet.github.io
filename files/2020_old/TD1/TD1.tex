\documentclass[a4paper,10pt,openany]{article}
\usepackage{fancyhdr}
\usepackage[T1]{fontenc}
\usepackage[margin=1.8cm]{geometry}
\usepackage[applemac]{inputenc}
\usepackage{lmodern}
\usepackage{enumitem}
\usepackage{microtype}
\usepackage{hyperref}
\usepackage{enumitem}
\usepackage{dsfont}
\usepackage{amsmath,amssymb,amsthm}
\usepackage{mathenv}
\usepackage{amsthm}
\usepackage{graphicx}
\usepackage[all]{xy}
\usepackage{lipsum}       % for sample text
\usepackage{changepage}
\theoremstyle{plain}
\newtheorem{thm}{Theorem}[section]
\newtheorem*{thm*}{Th\'eor\`eme}
\newtheorem{prop}[thm]{Proposition}
\newtheorem{cor}[thm]{Corollary}
\newtheorem{lem}[thm]{Lemma}
\newtheorem{propr}[thm]{Propri\'et\'e}
\theoremstyle{definition}
\newtheorem{deff}[thm]{Definition}
\newtheorem{rqq}[thm]{Remark}
\newtheorem{ex}[thm]{Exercice}
\newcommand{\e}{\mathrm{e}}
\newcommand{\prodscal}[2]{\left\langle#1,#2\right\rangle}
\newcommand{\devp}[3]{\frac{\partial^{#1} #2}{\partial {#3}^{#1}}}
\newcommand{\w}{\omega}
\newcommand{\dd}{\mathrm{d}}
\newcommand{\x}{\times}
\newcommand{\ra}{\rightarrow}
\newcommand{\pa}{\partial}
\newcommand{\vol}{\operatorname{vol}}
\newcommand{\dive}{\operatorname{div}}
\newcommand{\T}{\mathbf{T}}
\newcommand{\R}{\mathbf{R}}
\newcommand{\Z}{\mathbf{Z}}
\newcommand{\N}{\mathbf{N}}
\newcommand{\F}{\mathcal{F}}
\newcommand{\Homeo}{\mathrm{Homeo}}
\newcommand{\Matn}{\mathrm{Mat}_{n \times n}}
\DeclareMathOperator{\tr}{tr}
\newcommand{\id}{\mathrm{id}}


\title{\textsc{Syst\`emes dynamiques} \\ Feuille d'exercices 1}
\date{}
\author{}

\begin{document}

{\noindent \'Ecole Normale Sup\'erieure  \hfill Yann Chaubet } \\
{2020/2021 \hfill \texttt{chaubet@dma.ens.fr}}

{\let\newpage\relax\maketitle}
\maketitle
\noindent {\large \textbf{Exercice 1.} \textit{Points pr\'ep\'eriodiques}} \\ 

\noindent Soit $f : X \to X$ une application d\'efinie sur une ensemble $X$. On dit qu'un point $x \in X$ est \textit{pr\'ep\'eriodique} si $x$ n'est pas p\'eriodique et s'il existe $n \in \N$ tel que $f^n(x)$ est p\'eriodique.
{\begin{enumerate}
\item Donner un exemple d'application ayant un point pr\'ep\'eriodique. Peut-on trouver un exemple
bijectif ?
\item Montrer que si $X$ est fini, alors tout point est p\'eriodique ou pr\'ep\'eriodique. Montrer \'egalement qu'il existe des points p\'eriodiques.
\end{enumerate}
}
\hfill \break 
%% \\

\noindent {\large \textbf{Exercice 2.} \textit{Lemme de prolongation}} \\ 

\noindent Soit $V : \R^n \to \R^n$ un champ de vecteurs continu et
$$
\dot{x} = V(x)
$$
l'\'equation diff\'erentielle associ\'ee. Soit $x : (a,b) \to \R^n$ une solution de cette \'equation. On suppose qu'il existe un compact $K$ tel que $x(t) \in K$ pour tout $t \in (a,b)$. Montrer que la limite $\lim_{t \to b} x(t)$ existe.
\hfill \break \\

\noindent {\large \textbf{Exercice 3.} \textit{Automorphismes lin\'eaires du tore de dimension 2}} \\ 

\noindent On note $\T^2 = \R^2/\Z^2$ le tore de dimension $2$. On appellera \textit{feuilletage} de $\T^2$ une partition $\T^2 = \bigsqcup_{F \in \F} F$ o\`u pour tout $F \in \mathcal{F}$, il existe une immersion $\R \to \T^2$ (i.e. une application $\mathcal{C}^\infty$ de diff\'erentielle partout non nulle) d'image $F$. 
\begin{enumerate}
\item Donner une condition n\'ecessaire et suffisante pour que l'endomorphisme $f_A : \T^2 \to \T^2$ associ\'e \`a une matrice $A \in \mathrm{Mat}_{2\times 2}(\Z)$ soit un automorphisme.
\end{enumerate}
Dans toute la suite, $A$ d\'esigne une matrice de $\mathrm{SL}(2,\Z)$.
\begin{enumerate}[resume]
\item On suppose que $|{\tr  A}| \in \{0,1\}$. Montrer qu'il existe $n \in \N$ tel que $(f_A)^n = \id_{\T^2}.$
\item On suppose que $|{\tr A}| = 2.$ Montrer qu'il existe un feuilletage en cercles de $\T^2$, pr\'eserv\'e par $f_A$ et que $f_A$ (resp. $f_A^2$) agit par rotation sur chacun des cercles si $\tr(A) = 2$ (resp. $\tr(A) = -2$). On dit que $f_A$ est un \textit{twist de Dehn}.
\item On suppose que $|{\tr A}| > 2$.
\begin{enumerate}
\item Montrer que $A$ admet deux valeurs propres r\'eelles distinctes $\lambda, \lambda^{-1}$ avec $|\lambda| > 1$ et que les vecteurs propres associ\'es ont des pentes irrationnelles.
\item Montrer qu'il existe deux feuilletages $\F^s$ et $\F^u$ de $\T^2$, globalement pr\'eserv\'es par $f_A$, tels que chaque feuille est dense dans $\T^2$, et tels que la diff\'erentielle de $f_A$ multiplie par $\left|\lambda^{-1}\right|$ la norme des vecteurs tangents aux feuilles de $\F^s$ et par $|\lambda|$ celle des vecteurs tangents aux feuilles de $\F^u$. 

\end{enumerate}
\end{enumerate}
\hfill \break 

%%%
\noindent {\large \textbf{Exercice 4.} \textit{Persistance des orbites p\'eriodiques non d\'eg\'en\'er\'ees pour les flots}} \\ 

\noindent On consid\`ere un champ de vecteur $V : \R^n \to \R^n$ lisse (i.e. de classe $\mathcal{C}^\infty$) et complet (i.e. son flot existe en tout temps). On note $\varphi : \R \times \R^n \to \R^n$ son flot,
$$
\frac{\dd}{\dd t} \varphi(t,x) = V(\varphi(t,x)), \quad t \in \R, \quad x \in \R^n.
$$
On suppose qu'il existe $x_0 \in \R^n$ et $\tau_0 > 0$ tels que $\varphi(\tau_0, x_0) = x_0$ et $V(\varphi(t,x_0)) \neq 0$ pour tout $t \in \R$. 
\begin{enumerate}
\item Montrer que pour tout hyperplan affine $\Sigma$ contenant $x_0$ et transverse \`a $V(x_0)$ il existe un voisinage $U$ de $x_0$ dans $\R^n$ et une application lisse $\tau : U \to \R$ telle que $\tau(x_0) = \tau_0$ et
$$
\varphi(\tau(x), x) \in \Sigma, \quad x \in U.
$$ 
\end{enumerate}
En d'autres termes, $\Sigma$ est une section de Poincar\'e locale pour le flot $\varphi$. On note
$$
\begin{matrix}
P_\Sigma : &\Sigma \cap U &\to &\Sigma \\
&x &\mapsto &\varphi(\tau(x), x)
\end{matrix}
$$
l'application de retour associ\'ee \`a $\tau$. 
\begin{enumerate}[resume]
\item On suppose dans la suite que $x_0$ est une orbite p\'eriodique \textit{non d\'eg\'en\'er\'ee} du flot $\varphi$, c'est-\`a-dire que $1$ n'est pas valeur propre de $(\dd P_\Sigma)_{x_0}.$ Montrer que cette condition est intrins\`eque, i.e. qu'elle ne d\'epend pas de l'hypersurface affine $\Sigma$ choisie. 
\item On consid\`ere maintenant une famille $(V_s)_{s \in [-1,1]}$ de champs de vecteurs telle que l'application $(s, x) \mapsto V_s(x)$ est lisse et $V_0 = V$. Montrer qu'il existe $\varepsilon > 0$ tel que pour tout $s \in (-\varepsilon, \varepsilon),$ il existe $x_s \in \Sigma$ et $\tau_s > 0$ tels que 
$$
\varphi_s(\tau_s, x_s) = x_s,
$$
o\`u $\varphi_s$ est le flot associ\'e \`a $V_s$. Montrer de plus que $x_s$ et $\tau_s$ d\'ependent de mani\`ere lisse du param\`etre $s \in (-\varepsilon, \varepsilon).$
\end{enumerate}

\hfill \break 

\noindent {\large \textbf{Exercice 5.} \textit{Classes de conjugaison des applications expansives du cercle}} \\ 

\noindent On note $\T = \R / \Z$ le tore de dimension $1$ et on note $[y] = y \mod 1$ pour tout $y \in \R$. On dit qu'une application continue $F : \R \to \R$ \textit{rel\`eve} une application continue $f : \T \to \T$ si $\left[F(x)\right] = f\left([x]\right)$ pour tout $x \in \R$.
\begin{enumerate}
\item Montrer que pour tout $f : \T \to \T$ continue, il existe un rel\`evement de $f$, et que tous les rel\`evements diff\`erent d'un entier.
\end{enumerate}
Dans toute la suite, on fixe une application continue $f : \T \to \T$.
\begin{enumerate}[resume]
\item Montrer qu'il existe $p \in \Z$ tel que $F(x+1) = F(x) + p$ pour tout rel\`evement $F$ de $f$ et tout $x \in \R$. Cet entier s'appelle le \textit{degr\'e} de $f$.
\end{enumerate}
On suppose dans la suite que $p \geq 1$.
\begin{enumerate}[resume]
\item Montrer que $f$ a au moins $p-1$ points fixes.
\item En d\'eduire que 
$$
\liminf_{n \to \infty} \frac{\log \# \mathrm{Fix}(f^n)}{n} \geq \log p,
$$
o\`u $\mathrm{Fix}(f^n)$ est l'ensemble des points fixes de $f^n$.
\item Calculer $\# \mathrm{Fix}(f^n).$
\end{enumerate}
On suppose dans la suite que $p>1$ et on fixe un relev\'e $F$ de $f$. On note $\mathcal{E}$ l'ensemble des fonctions continues $H: \R \to \R$ telles que $H(x+1) = H(x)+1$ pour tout $x \in \R$. Pour tout $H \in \mathcal{E}$ on d\'efinit une application $\Phi(H) : \R \to \R$ par
$$
\Phi(H)(x) = \frac{1}{p}H(F(x)), \quad x \in \R.
$$
\begin{enumerate}[resume]
\item Montrer que $\Phi$ pr\'eserve $\mathcal{E}$ et que $\Phi : \mathcal{E} \to \mathcal{E}$ a un unique point fixe $H_0$. \\
\textit{Indication : on pourra montrer que $\mathcal{E}$ muni de la distance $d(H,G) = \sup_{\R} |H-G|$ est un espace m\'etrique complet.}
\item Montrer que $H_0$ rel\`eve une application continue $h_0 : \T  \to \T$ de degr\'e $1$, et que 
$$
h_0 \circ f = E_p \circ h_0
$$
o\`u $E_p : \T \to \T$ est l'application $[x] \mapsto [px]$.
\item On suppose de plus que $F$ est $\mathcal{C}^1$ et que $F'>1$. En consid\'erant l'application $\Psi : \mathcal{E} \to \mathcal{E} $ d\'efinie par $\Psi(H)(x) = F^{-1}(H(px))$, montrer que $h_0$ est un hom\'eomorphisme de $\T^2$.



\end{enumerate}





\end{document} 
