\documentclass[a4paper,12pt,openany]{article}
\usepackage{fancyhdr}
\usepackage[T1]{fontenc}
\usepackage[margin=1.8cm]{geometry}
\usepackage[applemac]{inputenc}
\usepackage{lmodern}
\usepackage{enumitem}
\usepackage{microtype}
\usepackage{hyperref}
\usepackage{enumitem}
\usepackage{dsfont}
\usepackage{amsmath,amssymb,amsthm}
\usepackage{mathenv}
\usepackage{mathrsfs}
\usepackage{amsthm}
\usepackage{graphicx}
\usepackage[all]{xy}
\usepackage{lipsum}       % for sample text
\usepackage{changepage}
\theoremstyle{plain}
\newtheorem{thm}{Theorem}[section]
\newtheorem*{thm*}{Th\'eor\`eme}
\newtheorem{prop}[thm]{Proposition}
\newtheorem{cor}[thm]{Corollary}
\newtheorem{lem}{Lemme}
\newtheorem{propr}[thm]{Propri\'et\'e}
\theoremstyle{definition}
\newtheorem{deff}[thm]{Definition}
\newtheorem{rqq}[thm]{Remark}
\newtheorem{ex}[thm]{Exercice}
\newcommand{\e}{\mathrm{e}}
\newcommand{\prodscal}[2]{\left\langle#1,#2\right\rangle}
\newcommand{\devp}[3]{\frac{\partial^{#1} #2}{\partial {#3}^{#1}}}
\newcommand{\w}{\omega}
\newcommand{\dd}{\mathrm{d}}
\newcommand{\x}{\times}
\newcommand{\ra}{\rightarrow}
\newcommand{\pa}{\partial}
\newcommand{\vol}{\operatorname{vol}}
\newcommand{\dive}{\operatorname{div}}
\newcommand{\T}{\mathbf{T}}
\newcommand{\R}{\mathbf{R}}
\newcommand{\Q}{\mathbf{Q}}
\newcommand{\Z}{\mathbf{Z}}
\newcommand{\N}{\mathbf{N}}
\newcommand{\C}{\mathbf{C}}
\newcommand{\F}{\mathcal{F}}
\newcommand{\Homeo}{\mathrm{Homeo}}
\renewcommand{\x}{\mathbf{x}}
\newcommand{\Matn}{\mathrm{Mat}_{n \times n}}
\DeclareMathOperator{\tr}{tr}
\newcommand{\id}{\mathrm{id}}
\newcommand{\htop}{h_\mathrm{top}}


\title{\textsc{Syst\`emes dynamiques} \\  Feuille d'exercices 10}
\date{}
\author{}

\begin{document}

{\noindent \'Ecole Normale Sup\'erieure  \hfill Yann Chaubet } \\
{2020/2021 \hfill  \texttt{chaubet@dma.ens.fr}}

{\let\newpage\relax\maketitle}
\maketitle
 
 
\vspace{1cm}
\noindent {\large \textbf{Exercice 1.} \textit{Moyennes de Birkhoff pour les permutations}} \vspace{1.5mm} 

On a pour tout $x \in X$ tel que $\sigma^d(x) = x$ avec $d > 0$ minimal, en notant $n = d \ell_n + r_n \text{ avec }r_n < d$
$$
\begin{aligned}
\frac{1}{n}\sum_{k=1}^n f(\sigma^k(x)) &= \frac{1}{n} \sum_{j=1}^{\ell_n} \left(\sum_{y \in \mathcal{O}(x)} f(y)\right) + \frac{1}{n} \sum_{k=d\ell_n + 1}^{n} f(\sigma^k(x)) \\
&= \frac{\ell_n}{n} \left(\sum_{y \in \mathcal{O}(x)} f(y)\right) + o(1/n)\\
&= \frac{1}{d} \left(\sum_{y \in \mathcal{O}(x)} f(y)\right) + o(1).
\end{aligned}
$$
\vspace{0.6cm}

\noindent {\large \textbf{Exercice 2.} \textit{Th\'eor\`eme ergodique et isom\'etries}} \vspace{1.5mm} 

\noindent
On sait que l'on a $S_n \varphi \to \psi$, $\mu$-presque partout pour une certaine fonction $\psi \in L^1(\mu)$. On note $G$ l'ensemble des points de $X$ telle que $\lim_n S_n \varphi(x)$ existe. 
 
\noindent
Si $x,x' \in X$ on a, puisque $\varphi$ est uniform\'ement continue,
$$
|S_n \varphi(x) - S_n\varphi(x')| \leqslant \frac{1}{n} \sum_{k=1}^n |\varphi(f^k(x))-\varphi(f^k(x'))| \leqslant C \varepsilon\left(\dd(x,x')\right) $$
o\`u $\varepsilon(t) \to 0$ quand $t \to 0$.  Comme $G$ est dense dans $X$ (car $\mu(U) > 0$ pour tout ouvert non vide $U$), on peut d\'efinir $\psi$ partout par $\psi(x) = \lim_{k} \psi(x_k)$ o\`u $x_k \in G$ et $x_k \to x.$  

\noindent
Ainsi $S_n \varphi \to \psi$ partout, avec $\psi$ continue.  
\noindent
Montrons que la convergence est uniforme. 

\noindent Soit $\varepsilon > 0$ et $x_1, \dots, x_N \in X$ tels que $\inf_{i=1, \dots, N} \dd(x,x_i) < \delta$ pour tout $x \in X,$ o\`u $\delta > 0$ est choisi de sorte que
$$
\forall x, x' \in X, \quad \dd(x,x') < \delta \implies |\varphi(x)-\varphi(x')|�\leqslant \varepsilon.
$$
Soit $n_0$ assez grand tel que $|S_n\varphi(x_i) - \psi(x_i)| < \varepsilon$ pour tout $n > n_0$ et tout $i = 1,\dots,N.$  

\noindent
Soit $x \in X$ et $i$ tel que $\dd(x,x_i) < \delta.$ Alors
$$
\begin{aligned}
|S_n\varphi(x) - \psi(x)|�&\leqslant |S_n\varphi(x) - S_n\varphi(x_i)| + |S_n\varphi(x_i) - \psi(x_i)|� \\
& \hspace{4.6cm} \quad \quad + |\psi(x_i) - \psi(x)| \\
& \leqslant 3 \varepsilon.
\end{aligned}
$$

\vspace{0.6cm}

\newpage
\noindent {\large \textbf{Exercice 3.} \textit{Th\'eor\`eme ergodique sur les espaces m\'etriques compacts}} \vspace{1.5mm} 

\noindent
Soit $(\varphi_j)$ une suite de $C^0(X)$ qui est dense dans $C^0(X)$.  Pour tout $j$, il existe $G_j \subset X$ de mesure totale telle que la limite $\lim_n S_n \varphi_j(x)$ existe pour tout $x\in G_j$.   
On d\'efinit $G$ l'ensemble de mesure totale par
$$
G = \bigcap_j G_j.
$$
Soit maintenant $\varphi \in C^0(X)$, et $x \in G.$ Soit $\varepsilon > 0.$ On prend $j$ tel que $\|\varphi_j - \varphi\|_\infty < \varepsilon.$  Alors pour tous $m,n$,
$$
\begin{aligned}
|S_n \varphi(x) - S_m \varphi(x)| &\leqslant |S_n�\varphi(x) - S_n \varphi_j(x)|�+ |S_n\varphi_j(x) - S_m \varphi_j(x)|�\\
& \hspace{4.6cm} \quad + |S_m \varphi_j(x) - S_m\varphi(x)|  \\
& \leqslant 2\varepsilon  + |S_n\varphi_j(x) - S_m\varphi_j(x)|.
\end{aligned}
$$
Si $m,n$ sont assez grands alors $|S_n\varphi_j(x) - S_m\varphi_j(x)| \leqslant \varepsilon$ puisque $(S_n\varphi_j(x))_n$ converge car $x \in G \subset G_j.$  
Ainsi, $S_n \varphi(x)$ converge quand $n \to +\infty$ par le crit\`ere de Cauchy.


\vspace{0.6cm} 

\noindent {\large \textbf{Exercice 4.} \textit{Unique ergodicit\'e et densit\'e des orbites}\vspace{1.5mm}}

\noindent
Soit $x \in X.$ On consid\`ere 
$$
\mu_n = \frac{1}{n} \sum_{k=1}^n \delta_{f^k(x)}, \quad n \geqslant 1.
$$
Alors par le TD 9, il existe une suite $n_k$ et une mesure de probabilit\'es $f$-invariante $\mu'$ telle que 
$$
\mu_{n_k}(\varphi) \to \mu'(\varphi), \quad \varphi \in C^0(X).
$$
Alors $\mu' = \mu$ par unicique ergodicit� de $f$, et $\mu'$ est de support total. Par cons\'equent, on a pour tout ouvert non vide $A \subset X$ 
$$
\#\{n \in \N,~f^n(x) \in A\} = +\infty.
$$


\vspace{0.6cm}

\noindent {\large \textbf{Exercice 5.} \textit{Le th\'eor\`eme de Von Neumann via le th\'eor\`eme de Birkhoff}} \vspace{1.5mm} 

\begin{enumerate}
\item Pour tout $n$ on a 
$$
\int_X |S_n \varphi|^2 \dd \mu = \int_X |\varphi|^2 \dd \mu.
$$

Par cons\'equent on a $\int_X |\bar \varphi|^2 \dd \mu \leqslant \int_X |\varphi|^2 \dd \mu$ par le lemme de Fatou, et donc $\bar \varphi \in L^2(\mu).$ 
\item Si $|\varphi| \in L^\infty(\mu)$ on a $\int_X |S_n \varphi - \bar \varphi|^2 \dd \mu \to 0$ par le th\'eor\`eme de convergence domin\'ee.

Posons $\varphi_k = \varphi \cdot1_{\{|\varphi|\leqslant k\}}.$ Alors 
$$
\int_X |\varphi|^2 \dd \mu \geqslant \int_X |\varphi - \varphi_k|^2 \dd \mu \geqslant k^2 \mu(\{|\varphi| > k),
$$

de sorte que 
$$
\mu(\{|\varphi|> k\}) \leqslant \frac{\|\varphi\|_{L^2(\mu)}^2}{k^2}, \quad k > 0.
$$
 Il suit que $\varphi_k \to \varphi$ $\mu$-presque partout et donc $\varphi_k \to \varphi$ dans $L^2(\mu)$ par convergence domin\'ee.



Soit $\varepsilon > 0$ et $k$ assez grand de sorte que $\|\varphi-\varphi_k\|_{L^2(\mu)} < \varepsilon.$  On a 
$$
\|S_n\varphi - S_m\varphi\|_{2} \leqslant\|S_n\varphi - S_n\varphi_k\|_2 + \|S_n \varphi_k - S_m \varphi_k\|_2 + \|S_m \varphi_k - S_m \varphi\|_2.
$$
On a pour tout $\ell$
$$
\|S_\ell \varphi - S_\ell \varphi_k\|_2 \leqslant \frac{1}{\ell} \sum_{j=1}^\ell \|(\varphi - \varphi_k) \circ f^j\|_2 \leqslant \|\varphi - \varphi_k\| < \varepsilon.
$$

D'autre part, comme $\varphi_k$ est born\'ee on sait que $S_n \varphi_k$ converge dans $L^2(\mu)$ ; on obtient que si $m,n$ sont assez grands, 
$$
\|S_n \varphi - S_m \varphi\|_2 < 3 \varepsilon.
$$

Ainsi $(S_n \varphi)$ converge dans $L^2(\mu)$, vers $\bar \varphi.$
\end{enumerate}
\vspace{0.6cm}

\noindent {\large \textbf{Exercice 6.} \textit{Explosion des sommes de Birkhoff et positivit\'e de la moyenne}} \vspace{1.5mm} 

\begin{enumerate}
\item Soit $\bar \varphi \in L^1(\mu)$ la fonction limite de $\displaystyle{\frac{1}{n}\sum_{k=0}^{n-1} \varphi \circ f^k}$ donn\'ee par le th\'eor\`eme de Birkhoff. Alors on a 
$$
\int \varphi \dd \mu = \int \bar \varphi \dd \mu \geqslant 0.
$$
\item On raisonne par r\'ecurrence sur $n$.  Pour $n = 1$ on a $T_n \varphi(x) = \varphi(x) \geqslant \varepsilon \chi_{A_\varepsilon}(x)$.  

On suppose que $\displaystyle{T_n \varphi(x) \geqslant \varepsilon \sum_{k=0}^{n-1} \chi_{A_\varepsilon}(f^k(x)).}$ On a $\varphi(f^n(x)) \geqslant \varepsilon \chi_{A_\varepsilon}(f^n(x))$ par ce qui pr\'ec\`ede.  Ainsi
$$
\begin{aligned}
T_{n+1}\varphi(x) &= T_n \varphi(x) + \varphi(f^n(x))  \\
& \geqslant \varepsilon \sum_{k=0}^n \chi_{A_\varepsilon}(f^k(x)).
\end{aligned}
$$

\item La question pr\'ec\'edente donne
$$
\bar \varphi \geqslant \varepsilon \bar \chi_{A_\varepsilon} \quad\mu\text{-presque partout sur }A_\varepsilon
$$
o\`u $\bar \varphi$ et $\bar \chi_{A_\varepsilon}$ sont les fonctions associ\'ees \`a $\varphi$ et $\chi_{A_\varepsilon}$ donn\'ees par le th\'eor\`eme ergodique.  

Puisque $\bar \varphi \circ f = \bar \varphi$ $\mu-$presque partout, l'in\'egalit\'e pr\'ec\'edente est vraie $\mu-$pp sur $B_\varepsilon.$



Par d\'efinition pour tout $x \in \complement B_\varepsilon$, on a $\chi_{A_\varepsilon}(f^k(x)) = 0$ pour tout $k \in \N.$ Il suit que $\chi_{A_\varepsilon} = 0$ sur $\complement B_\varepsilon$.  

D'autre part on a $\bar \varphi$ par hypoth\`ese puisque $T_n \varphi(x) \to +\infty$ pour $\mu-$presque tout $x$.  Ainsi
$$
\begin{aligned}
\mu(A_\varepsilon) &= \int_X \chi_{A_\varepsilon} \dd \mu  \\
& = \int_X \bar \chi_{A_\varepsilon} \dd \mu  \\ 
&= \int_{B_\varepsilon} \bar \chi_{A_\varepsilon}  \\
& \leqslant \frac{1}{\varepsilon} \int_{B_\varepsilon} \bar \varphi~\dd \mu  \\
& \leqslant \frac{1}{\varepsilon} \int_X \bar \varphi~ \dd \mu  \\
& \leqslant \frac{1}{\varepsilon} \int_X \varphi~ \dd \mu  \\
&\leqslant 0.
\end{aligned}
$$
 
On obtient $\mu(A_\varepsilon) = 0$ et donc $\mu(B_\varepsilon) = \mu \left(\bigcup_{k}f^{-k}(A_\varepsilon)\right) = 0$ puisque $f$ pr\'eserve $\mu$. 
\item On a le 
\begin{lem}
Soit $(a_n)$ une suite r\'eelle telle que $\sum_{n=0}^N a_n \to +\infty$ quand $N \to +\infty.$ Alors il existe $\varepsilon > 0$ et $N_0 > 0$ tels que 
$$
\sum_{n=N_0}^N a_n \geqslant \varepsilon, \quad N \geqslant N_0.
$$
\end{lem}
Admettant le lemme, on obtient que pour tout $x \in X$ tel que $T_N \varphi(x) \to +\infty$ quand $N \to +\infty$, il existe $\varepsilon(x), N_0(x) > 0$ tels que 
$$
\sum_{n=N_0(x)}^{N} \varphi(f^n(x)) \geqslant \varepsilon, \quad N \geqslant N_0(x).
$$
 Autrement dit, on a $x \in f^{-N_0(x)}(A_\varepsilon(x)) \subset B_{\varepsilon(x)}.$  





Ceci implique que l'ensemble 
$
\displaystyle{
\bigcup_{k > 0} B_{1/k}
}
$
est de mesure totale, puisque presque tout $x$ v\'erifie $T_N\varphi(x) \to +\infty$.  

Ainsi, par la question \textbf{2.}, il existe $k$ tel que $B_{1/k}$ est de mesure strictement positive, et donc $\int_\varphi \dd \mu > 0.$  

Il reste \`a montrer le lemme ; soit $(a_n)$ une suite r\'eelle telle que 
$$
\sum_{n=0}^N a_n \to +\infty, \quad N \to +\infty.
$$

On raisonne par l'absurde et on suppose que pour tout $\varepsilon > 0$ et tout $N \geqslant 0$, il existe $N' \geqslant N$ tel que 
$$
\sum_{n=N}^{N'} a_n < \varepsilon.
$$
 Posons $N_0 = 0$ et $\varepsilon_0 = 1.$ Alors il existe $N' \geqslant 0$ tel que, 
$$
\sum_{n=N_0}^{N'} a_n < 1.
$$

On pose $N_1 = N' + 1$. Alors il existe $N' \geqslant N_1$ tel que 
$$
\sum_{n=N_1 + 1}^{N'} a_n < \frac{1}{4}.
$$
 En it\'erant ce processus, on construit une suite $N_0 < N_1 < \cdots$ telle que
$$
\sum_{n=N_k}^{N_{k+1}-1} a_n < \frac{1}{(k+1)^2}, \quad k \geqslant 0.
$$

On obtient que 
$$
\limsup_k \sum_{n=0}^{N_k-1} a_n < +\infty,
$$
ce qui est absurde.
\end{enumerate}
\vspace{0.6cm}

\end{document}









\noindent {\large \textbf{Exercice 1.} \textit{Th\'eor\`eme ergodique et isom\'etries}} \vspace{1.5mm} 

\noindent Soit $(X, \dd)$ un espace m\'etrique compact, $f : X \to X$ une isom\'etrie (i.e. $\dd(f(x), f(y)) = \dd(x,y)$ pour tous $x,y \in X$) et $\mu$ une mesure bor\'elienne de probabilit\'es invariante par $f$ telle que $\mu(U) > 0$ pour tout ouvert $U$ non vide. Soit $\varphi : X \to \C$ une fonction continue et 
$$
S_n \varphi = \frac{1}{n}  \sum_{k=0}^{n-1} \varphi \circ f^k.
$$ 
\begin{enumerate}
\item Montrer que pour tout $\delta > 0$, il existe $J \in \N$ et $x_1, \dots, x_J \in X$ tels que $\displaystyle{X = \bigcup_{j = 1}^J B(x_j, \delta)}$ et pour tout $j = 1, \dots J$, la suite
$
S_n \varphi (x_j) 
$
converge quand $n \to +\infty$.
\item Montrer que $S_n\varphi$
converge uniform\'ement sur $X$ vers une fonction continue.
\end{enumerate}


\vspace{0.6cm}

\noindent {\large \textbf{Exercice 2.} \textit{\'Equidistribution des rotations irrationnelles du cercle}} \vspace{1.5mm} 

\noindent Soit $\alpha \in \T^d = \R^d/ \Z^d$ et $R_\alpha : \T^d \to \T^d$ donn\'ee par $R_\alpha(x) = x + \alpha.$
\begin{enumerate}
\item Montrer que $R_\alpha$ pr\'eserve la mesure de Haar $\mu$ sur $\T^d$.
\item Soit $\varphi \in L^2(\T^d)$. Montrer qu'il existe $\bar{\varphi} \in L^2(\T^d)$ telle que 
$$
\frac{1}{n} \sum_{k=0}^{n-1} \varphi \circ R_\alpha^k \to \bar \varphi,
$$
la convergence \'etant presque sžre, mais aussi ayant lieu dans $L^2(\T^d)$.
\end{enumerate}
On suppose que la famille $(1, \alpha_1, \dots, \alpha_d)$ est lin\'eairement ind\'ependante sur $\Q$, o\`u $\alpha = (\alpha_1, \dots, \alpha_d).$
\begin{enumerate}[resume]
\item Montrer que $\bar{\varphi}$ est presque partout \'egale \`a $\displaystyle{\int_{\T^d} \varphi ~\dd \mu}$.
\item Soit $C \subset \T^d$ un produit d'intervalles. Montrer que pour $\mu$ presque tout $x \in \T^d$, 
$$
\frac{1}{n} \#\Bigl\{k \in \{1, \dots, n\},~x + k\alpha \in C \Bigr\} \underset{n \to +\infty}{\longrightarrow} \mu(C).
$$
\item Montrer que la propri\'et\'e pr\'ec\'edente est vraie pour tout $x \in \T^d$. \\
\textit{Indication : on pourra approximer la fonction indicatrice de $C$ par des fonctions continues et utiliser l'exercice pr\'ec\'edent.}
\item On consid\`ere la suite des premiers chiffres des puissances de $2$ : $1, 2, 4, 8, 1 \dots$ Montrer que la fr\'equence d'apparition du chiffre $7$ dans cette suite est environ \'egale \`a $5.8\%$.
\end{enumerate}
\vspace{0.6cm}

\noindent {\large \textbf{Exercice 3.} \textit{Applications dilatantes du cercle}} \vspace{1.5mm} 

\noindent Pour tout $m \in \N_{\geq 2}$ on note $E_m$ la multiplication par $m$ sur $\R/\Z$. 

\begin{enumerate}
\item Soit $\varphi \in L^2(\R/\Z)$ et $m \in \N_{\geq 2}$. Montrer que $\displaystyle{S_n \varphi = \frac{1}{n}\sum_{k=0}^{n-1}{\varphi \circ (E_m)^k}}$ converge vers $\displaystyle{\int_{\R/\Z} \varphi ~\dd \mu}$ dans $L^2(\mu)$, o\`u $\mu$ est la mesure de Haar.
\end{enumerate}
On dira qu'un nombre $x \in [0,1)$ est normal si pour tout $m \geq 2$, son d\'eveloppement en base $m$
$$
x = 0,a_1a_2\dots,
$$
(qui est unique si on demande que pour tout $k$, il existe $k' \geq k$ tel que $a_{k'} \neq m-1$) satisfait
$$
\frac{1}{n} \#\Bigl\{k \in \{1, \dots, n\},~a_k = j\Bigr\} \underset{n \to +\infty}{\longrightarrow} \frac{1}{m}, \quad j = 0, \dots, m-1.
$$
\begin{enumerate}[resume]
\item Montrer que presque tout $x \in [0,1)$ est normal.
\end{enumerate}

\vspace{0.6cm}

\noindent {\large \textbf{Exercice 4.} \textit{Explosion des sommes de Birkhoff et positivit\'e de la moyenne}} \vspace{1.5mm} 

\noindent Soit $(X,\mathscr{A},\mu)$ un espace de probabilit\'es et $f : X \to X$ une application mesurable pr\'eservant $\mu$. Soit $\varphi \in L^1(\mu)$. On suppose que pour $\mu$ presque tout $x$,
$$
\lim_{n\to + \infty}\sum_{j=0}^n \varphi\left(f^k(x)\right) = + \infty.
$$
On cherche \`a montrer que $\displaystyle{\int_X \varphi ~\dd \mu }> 0$.

\noindent On note $T_n \varphi = \displaystyle{\sum_{k=0}^{n-1} \varphi \circ f^k}$ et pour tout $\varepsilon > 0$,
$$
A_\varepsilon = \bigcap_{n \geq 1} \Bigl\{T_n\varphi \geq \varepsilon\Bigr\}, \quad B_\varepsilon = \bigcup_{k \geq 0} f^{-k}(A_\varepsilon).
$$
\begin{enumerate}
\item Soit $x \in A_\varepsilon$. Montrer que pour tout $n \geq 1$,
$$
T_n\varphi(x) \geq \varepsilon \sum_{k=0}^{n-1} \chi_{A_\varepsilon}\left(f^k(x)\right).
$$
\item Montrer que si $\int_X \varphi~\dd \mu = 0$ alors $\mu(B_\varepsilon) = 0$.
\item Conclure.
\end{enumerate}
\vspace{0.6cm}


 
\noindent {\large \textbf{Exercice 5.} \textit{Moyenne temporelle des temps de retour}} \vspace{1.5mm} 

\noindent Soit $(X, \mu)$ un espace de probabilit\'es et $f : X \to X$ une application mesurable pr\'eservant $\mu$. Soit $A \subset X$ un ensemble mesurable de mesure non nulle, pour tout $x \in A$ on note $\displaystyle{\tau(x) = \inf \{ n \geq 1,  f^n(x) \in A \}}$ le temps de premier retour dans $A$, et $g(x) = f^{\tau(x)}(x)$ l'application de retour associ\'ee (d\'efinie presque partout sur $A$). On suppose que pour tout $\varphi \in L^1(\mu),$ les moyennes de Birkhoff associ\'ees \`a $\varphi$ convergent presque sžrement vers une constante. Montrer que pour $\mu$ presque tout $x$ de $A$,
$$
\lim_{n \to + \infty}\frac{1}{n} \sum_{k=0}^{n-1} \tau\left(g^k(x)\right) = \frac{1}{\mu(A)}.
$$

\vspace{0.6cm}
