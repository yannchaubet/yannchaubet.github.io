\documentclass[a4paper,10pt,openany]{article}
\usepackage{fancyhdr}
\usepackage[T1]{fontenc}
\usepackage[margin=1.8cm]{geometry}
\usepackage[applemac]{inputenc}
\usepackage{lmodern}
\usepackage{enumitem}
\usepackage{microtype}
\usepackage{hyperref}
\usepackage{enumitem}
\usepackage{dsfont}
\usepackage{amsmath,amssymb,amsthm}
\usepackage{mathenv}
\usepackage{amsthm}
\usepackage{graphicx}
\usepackage[all]{xy}
\usepackage{lipsum}       % for sample text
\usepackage{changepage}
\theoremstyle{plain}
\newtheorem{thm}{Theorem}[section]
\newtheorem*{thm*}{Th\'eor\`eme}
\newtheorem{prop}[thm]{Proposition}
\newtheorem{cor}[thm]{Corollary}
\newtheorem{lem}[thm]{Lemma}
\newtheorem{propr}[thm]{Propri\'et\'e}
\theoremstyle{definition}
\newtheorem{deff}[thm]{Definition}
\newtheorem{rqq}[thm]{Remark}
\newtheorem{ex}[thm]{Exercice}
\newcommand{\e}{\mathrm{e}}
\newcommand{\prodscal}[2]{\left\langle#1,#2\right\rangle}
\newcommand{\devp}[3]{\frac{\partial^{#1} #2}{\partial {#3}^{#1}}}
\newcommand{\w}{\omega}
\newcommand{\dd}{\mathrm{d}}
\newcommand{\x}{\times}
\newcommand{\ra}{\rightarrow}
\newcommand{\pa}{\partial}
\newcommand{\vol}{\operatorname{vol}}
\newcommand{\dive}{\operatorname{div}}
\newcommand{\T}{\mathbf{T}}
\newcommand{\R}{\mathbf{R}}
\newcommand{\Z}{\mathbf{Z}}
\newcommand{\N}{\mathbf{N}}
\newcommand{\C}{\mathbf{C}}
\newcommand{\F}{\mathcal{F}}
\newcommand{\Homeo}{\mathrm{Homeo}}
\newcommand{\Matn}{\mathrm{Mat}_{n \times n}}
\DeclareMathOperator{\tr}{tr}
\newcommand{\id}{\mathrm{id}}
\newcommand{\htop}{h_\mathrm{top}}


\title{\textsc{Syst\`emes dynamiques} \\ Feuille d'exercices 5}
\date{}
\author{}

\begin{document}

{\noindent \'Ecole Normale Sup\'erieure  \hfill \texttt{chaubet@dma.ens.fr} } \\
{2020/2021 \hfill }

{\let\newpage\relax\maketitle}
\maketitle

\noindent {\large \textbf{Exercice 1.}}  \vspace{1.5mm} 


\begin{enumerate}
\item Soient deux flots continus $(\varphi_t)_{t \in \R}$ et $(\phi_t)_{t\in \R}$ sur $\R^d$ qui sont topologiquement conjugu\'es : il existe un hom\'eomorphisme $h : \R^d \to \R^d$ tel que $h \circ \varphi_t = \phi_t \circ h$ pour tout $t \in \R.$
\begin{enumerate}
\item Montrer que $h$ transporte les points p\'eriodiques.
\item Montrer que pour tout $x \in \R^d$, l'orbite 
$\mathcal{O}_{\varphi}(x) = \{\varphi_t(x),~t\in \R\}$
est ferm\'ee si et seulement si l'orbite $\mathcal{O}_\phi(h(x))$ (d\'efinie identiquement) est ferm\'ee.
\item Montrer que $h$ transporte aussi les ensembles $\omega$-limites.
\end{enumerate}
\item Soient $A,B$ les matrices d\'efinies par
$$
A=\left(\begin{array}{ll}1 & 0 \\ 0 & 1\end{array}\right), \quad B=\left(\begin{array}{cc}1 & 1 \\ -1 & 1\end{array}\right).
$$
\begin{enumerate}
\item D\'eterminer les flots associ\'es \`a $A$ et $B$.
\item Montrer que pour tout $x \in \R^n$, il existe un unique $t \in \R$ tel que $\|\e^{tA}x\| = 1$.
\item Construire une conjugaison entre les flots associ\'es \`a $A$ et $B$.
\end{enumerate}

\item Soit 
$
C = \begin{pmatrix} 0 & -1 \\ 1 & 0 \end{pmatrix}.
$
Montrer que le flot associ\'e \`a $C$ n'est pas conjugu\'e \`a celui associ\'e \`a $B$. Montrer que $\e^{C}$ n'est pas conjugu\'ee (au sens topologique) \`a $\e^{B}$.
\end{enumerate}
\vspace{0.6cm}

\noindent {\large \textbf{Exercice 2.}} \vspace{1.5mm} 

\noindent On note $\mathcal{H}(\R^n)$ (resp. $\mathrm{GL}(\R^n)$ et $\mathcal{L}(\R^n)$) l'espace des endomorphismes hyperboliques (resp. endomorphismes inversibles et endomorphismes) de $\R^n$. Montrer que $\mathcal{H}(\R^n)$ (resp. $\mathrm{GL}(\R^n)$) est un ouvert dense de $\mathrm{GL}(\R^n)$ (resp. $\mathcal{L}(\R^n)$).

\vspace{0.6cm}

\noindent {\large \textbf{Exercice 3.}} \vspace{1.5mm} 

\noindent Soit $A \in \mathcal{H}(\R^n)$. Montrer qu'il existe $\delta > 0$ tel que pour tout $B \in \mathcal{L}(\R^n)$ v\'erifiant $\|A-B\| \leq \delta$, alors $A$ et $B$ sont topologiquement conjugu\'ees.

\vspace{0.6cm}

\noindent {\large \textbf{Exercice 4.}} \vspace{1.5mm} 

\noindent Soit $A \in \mathcal{H}(\R^n)$. Montrer que pour tout $\varepsilon>0$ il existe une norme $\|\cdot\|$ sur $\R^n$ telle que la norme d'op\'erateur associ\'ee de $A$ soit strictement inf\'erieure \`a $\rho(A) + \varepsilon.$
\vspace{0.6cm}

\noindent {\large \textbf{Exercice 5.}} \vspace{1.5mm} 

\noindent Soit $f : \R^n \to \R^n$ un $\mathcal{C}^1$ diff\'eomorphisme. Soit $x \in \R^n$ un point fixe hyperbolique de $f$. Montrer que la p\'eriode de tout point p\'eriodique assez proche de $x$ (et diff\'erent de $x$) a une p\'eriode strictement strictement plus grande que $n$.
\vspace{0.6cm}

\noindent {\large \textbf{Exercice 6.}} \vspace{1.5mm} 

\noindent Soit $E$ un espace vectoriel r\'eel de dimension finie et $A \in \mathcal{H}(E)$. On note $E = E^s \oplus E^u$ la d\'ecomposition en espaces stable et instable de $A$, et $\pi_s, \pi_u$ les projecteurs associ\'es. Pour tout $\gamma > 0$ on d\'efinit les c\^ones
$$
C_{\gamma}^{s}=\left\{x \in E:\left\|\pi_{u}(x)\right\| \leq \gamma\left\|\pi_{s}(x)\right\|\right\}, \quad C_{\gamma}^{u}=\left\{x \in E:\left\|\pi_{s}(x)\right\| \leq \gamma\left\|\pi_{u}(x)\right\|\right\}.
$$
\begin{enumerate}
\item Montrer que 
$$
E^s = \bigcup_{\gamma > 0} \bigcap_{n \geq 0} A^{-n}(C^s_\gamma), \quad E^u = \bigcup_{\gamma > 0} \bigcap_{n \geq 0} A^{n}(C^u_\gamma).
$$
\item Montrer que 
$$
E^s = \left\{x \in \R^n,~\sup_{n \geqslant 0} \|A^nx\| < +\infty \right\}, \quad E^u = \left\{x \in \R^n,~ \sup_{n \geqslant 0} \|A^{-n}x\| < +\infty\right\}.
$$
\end{enumerate}
\vspace{0.6cm}

\noindent {\large \textbf{Exercice 7.}} \vspace{1.5mm} 

\noindent Soit $\rho : \R \to \R$ la fonction d\'efinie par $\rho(x) = 0$ si $x \leq 0$ et 
$$
\rho(x) = \exp\left(-\frac{1}{x^2}\right), \quad x > 0.
$$

\begin{enumerate}
\item Montrer que $\rho$ est lisse.
\end{enumerate}
On d\'efinit le champ de vecteur $X$ sur $\R^2$ par 
$$
X(x,y) = \left(y + \rho(r^2)x,~-x + \rho(r^2)y\right), \quad r^2 = x^2 + y^2, \quad x,y \in \R.
$$
\begin{enumerate}[resume]
\item Montrer que $X$ est un champ de vecteurs lisse et calculer $\dd X(0).$
\end{enumerate}
Soit $K$ un compact de $\R_+$ contenant $0$.
\begin{enumerate}[resume]
\item Construire une fonction $\rho_K : \R_+ \to \R_+$ qui est lisse et telle que 
$$\rho(x) > 0 \quad \iff \quad x \notin K.$$
\end{enumerate}
On d\'efinit le champ de vecteur $X_K$ comme $X$ en rempla\c ant $\rho$ par $\rho_K$.
\begin{enumerate}[resume]
\item Montrer que pour tous $\varepsilon, r > 0$, on peut choisir $\rho_K$ tel que
$$
\|X_K(x,y) - (y,-x)\|< \varepsilon, \quad (x,y) \in \R^2, \quad \|(x,y)\|\leq r.
$$
\item Montrer que si $r^2 \in K$, alors le cercle de rayon $r$ centr\'e en $0$ est une orbite de $X_K$.
\item Soient $a<b$ tels que $a^2, b^2 \in K$ et tels que $]a^2, b^2[ \cap K = \emptyset.$ Que dire des trajectoires des points $(x,y)$ tels que $a^2 < x^2 + y^2 < b^2$ ?
\item Montrer que si $K$ et $K'$ ne sont pas hom\'eomorphes, alors les flots de $X_K$ et de $X_{K'}$ ne sont pas conjugu\'es.
\end{enumerate}


\end{document} 
