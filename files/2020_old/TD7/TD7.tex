\documentclass[a4paper,10pt,openany]{article}
\usepackage{fancyhdr}
\usepackage[T1]{fontenc}
\usepackage[margin=1.8cm]{geometry}
\usepackage[utf8]{inputenc}
\usepackage{lmodern}
\usepackage{enumitem}
\usepackage{microtype}
\usepackage{hyperref}
\usepackage{enumitem}
\usepackage{dsfont}
\usepackage{amsmath,amssymb,amsthm}
\usepackage{mathenv}
\usepackage{amsthm}
\usepackage{graphicx}
\usepackage[all]{xy}
\usepackage{lipsum}       % for sample text
\usepackage{changepage}
\theoremstyle{plain}
\newtheorem{thm}{Theorem}[section]
\newtheorem*{thm*}{Th\'eor\`eme}
\newtheorem{prop}[thm]{Proposition}
\newtheorem{cor}[thm]{Corollary}
\newtheorem{lem}[thm]{Lemma}
\newtheorem{propr}[thm]{Propri\'et\'e}
\theoremstyle{definition}
\newtheorem{deff}[thm]{Definition}
\newtheorem{rqq}[thm]{Remark}
\newtheorem{ex}[thm]{Exercice}
\newcommand{\e}{\mathrm{e}}
\newcommand{\prodscal}[2]{\left\langle#1,#2\right\rangle}
\newcommand{\devp}[3]{\frac{\partial^{#1} #2}{\partial {#3}^{#1}}}
\newcommand{\w}{\omega}
\newcommand{\dd}{\mathrm{d}}
\newcommand{\x}{\times}
\newcommand{\ra}{\rightarrow}
\newcommand{\pa}{\partial}
\newcommand{\vol}{\operatorname{vol}}
\newcommand{\dive}{\operatorname{div}}
\newcommand{\T}{\mathbf{T}}
\newcommand{\R}{\mathbf{R}}
\newcommand{\Z}{\mathbf{Z}}
\newcommand{\N}{\mathbf{N}}
\newcommand{\C}{\mathbf{C}}
\newcommand{\F}{\mathcal{F}}
\newcommand{\Homeo}{\mathrm{Homeo}}
\renewcommand{\x}{\mathbf{x}}
\newcommand{\Matn}{\mathrm{Mat}_{n \times n}}
\DeclareMathOperator{\tr}{tr}
\newcommand{\id}{\mathrm{id}}
\newcommand{\Id}{\mathrm{Id}}
\newcommand{\htop}{h_\mathrm{top}}


\title{\textsc{Syst\`emes dynamiques} \\ Feuille d'exercices 7}
\date{}
\author{}

\begin{document}

{\noindent \'Ecole Normale Sup\'erieure  \hfill Yann Chaubet } \\
{2020/2021 \hfill \texttt{chaubet@dma.ens.fr}}

{\let\newpage\relax\maketitle}
\maketitle

\noindent Soit $M$ une vari\'et\'e compacte et $f : M \to M$ un $\mathcal{C}^1$-diff\'eomorphisme de $M$. On rappelle qu'un ferm\'e $\Lambda \subset M$ invariant par $f$ est dit hyperbolique si tout $x \in \Lambda$, il existe une d\'ecomposition $T_xM = E^s(x) \oplus E^u(x)$ d\'ependant continument du param\`etre $x \in \Lambda$ et v\'erifiant les points suivants.

\begin{enumerate}
\item La d\'ecomposition est stable par $f$,
$$
\dd f_x(E^\bullet(x)) = E^\bullet(f(x)), \quad x \in M, \quad \bullet = s,u.
$$
\item \label{2} Il existe une norme lisse $\|\cdot\|$ sur $TM$ et $\lambda \in (0,1)$ telle que pour tout $x \in M$
$$
\begin{aligned}
\left \| \dd f_x v \right\| &\leq \lambda \| v \|, \quad v \in E^s(x), \\
\left \| \dd f^{-1}_x v \right\| &\leq \lambda \| v \|, \quad v \in E^u(x).
\end{aligned}
$$
\end{enumerate}
Si $\Lambda = M$, on dit que $f$ est un diff\'eomorphisme d'Anosov.

\vspace{0.6cm}

\noindent {\large \textbf{Exercice 1.} \textit{Normes adapt\'ees}} \vspace{1.5mm} 

\noindent Montrer que dans le cas d'un diff\'eomorphisme d'Anosov, on peut remplacer la condition \ref{2} de la d\'efinition ci-dessus par la condition suivante. Il existe une norme lisse $\|\cdot\|$ sur $TM$ et des constantes $C > 0$, $\lambda \in (0,1)$ telle que pour tout $x \in M$ et tout $n \in \N$
$$
\begin{aligned}
\left \| \dd (f^n)_x v \right\| &\leq C \lambda^n \| v \|, \quad v \in E^s(x), \\
\left \| \dd (f^{-n})_x v \right\| &\leq C\lambda^n \| v \|, \quad v \in E^u(x).
\end{aligned}
$$
\textit{Indication : on pourra commencer par construire une norme adapt\'ee continue, puis l'approcher par des normes lisses.}
\vspace{0.6cm}

\noindent {\large \textbf{Exercice 2.} \textit{Points p\'eriodiques des diff\'eomorphismes d'Anosov}} \vspace{1.5mm} 

\noindent Soit $M$ une vari\'et\'e compacte connexe
 et $f : M \to M$ un diff\'eomorphisme d'Anosov. 

\begin{enumerate}
\item Montrer que tout point p\'eriodique de $f$ est hyperbolique.
\item On veut montrer que $f$ est une application expansive de $(M,\dd)$, o\`u $\dd$ est la distance induite par n'importe quelle norme sur $TM$. Pour cela on raisonne par l'absurde et on suppose qu'il existe deux suites de $(x_k)$ et $(y_k)$ de $M$ telles que pour tout $k$ on a $x_k \neq y_k$ et
$
\dd(f^n(x_k), f^n(y_k)) < \frac{1}{k}
$
pour tout $n \in \Z.$
\begin{enumerate}
\item Montrer qu'on peut supposer que 
$
\displaystyle{\frac{\dd(f^n(x_k), f^n(y_k))}{\dd(x_k, y_k)}} \leqslant 2$ pour tout $n \in \Z$ et tout $k \in \N$.

\item Montrer que quitte \`a extraire on peut supposer que $x_k$ et $y_k$ sont contenus dans une carte autour d'un point $z \in M$ avec $x_k, y_k \to z \in M$ quand $k \to +\infty$, et que (dans ladite carte)
$$
\frac{y_k - x_k}{\|y_k - x_k\|} \to v \in S^{\dim(M)-1}.
$$
\item Montrer qu'on peut trouver $z^+,z^- \in M$ et des extractions $(n_j^+), (n_j^-)$ telles que $f^{\pm n_j^{\pm}}(z) \to z^{\pm}$ quand $j \to +\infty$.
\item En d\'eduire que $v$ v\'erifie $\left\|\dd \left(f^{n_j^\pm}\right)_z(v)\right\| \leqslant C$ pour tout $j$ assez grand et en d\'eduire une contradiction.
\end{enumerate}
\item Montrer que pour tout $\varepsilon > 0$, il existe une constante $C>0$ telle que 
$$
p_n(f) \overset{\mathrm{def}}{=} \mathrm{card}\bigl\{p \in M,~f^n(p) = p\bigr\} \leq C \e^{n(\htop(f) + \varepsilon)}, \quad n \geq 1.
$$
\end{enumerate}

\vspace{0.6cm}

\noindent {\large \textbf{Exercice 3.} \textit{Hyperbolicit\'e et transversalit\'e}} \vspace{1.5mm} 

\noindent Soit $f : M \to M$ un diff\'eomorphisme. On d\'efinit
$$
\mathrm{Gr}(f) = \{(f(x), x),~x \in M\}, \quad \Delta(M) = \{(x,x), ~ x \in M\}.
$$
Montrer que  $\mathrm{Gr}(f)$ et $\Delta(M)$ sont des sous-vari\'et\'es de $M\times M$. Montrer qu'un point fixe $p$ de $f$ est non d\'eg\'en\'er\'e (i.e. $1 \notin \mathrm{sp}(\dd f_p)$) si, et seulement si, $\mathrm{Gr}(f)$ et $\Delta(M)$ s'intersectent transversalement en $(p,p)$.

\vspace{0.6cm}

\noindent {\large \textbf{Exercice 4.} \textit{Pistage et stabilit\'e structurelle}} \vspace{1.5mm} 

\noindent Soit $f : \T^2 \to \T^2$ un diff\'eomorphisme d'Anosov.

\begin{enumerate}
\item Montrer qu'il existe une matrice $A \in M_2(\Z)$ telle que si $F : \R^2 \to \R^2$ rel\`eve $f$, alors
$$
F(x + k, y+\ell) = F(x,y) + A (k, \ell), \quad (x,y) \in \R^2, \quad (k,\ell) \in \Z^2.
$$
\end{enumerate}
On note $f_\star = f_A : \T^2 \to \T^2$.
\begin{enumerate}[resume]
\item Montrer que $\vert \mathrm{det} A \vert = 1.$
\item Montrer que les applications $f$ et $f_\star$ sont homotopes en tant qu'applications $\T^2 \to \T^2$.
\end{enumerate}
On suppose dans la suite que $|\mathrm{tr}(A)|>2$.
\begin{enumerate}[resume]
\item \label{item:shadow} Soit $r>0$. Montrer qu'il existe $\delta > 0$ tel que pour toute suite $(p_n)_{n \in \Z}$ de $\R^2$ v\'erifiant
$$
\|p_{n+1} - A p_n\| \leq r, \quad n \in \Z,
$$
il existe un unique $q \in \R^2$ tel que 
$$
\|A^nq- p_n\| \leq \delta, \quad n \in \Z.
$$
\textit{Indication : on pourra \'ecrire $p_n = a_n v + b_n w$ o\`u $Av = \lambda v$ et $Aw = \lambda^{-1}w$ avec $|\lambda| > 1$ et montrer que les suites $(\lambda^{-n} a_n)_{n \geqslant 0}$ et $(\lambda^nb_{-n})_{n \geqslant 0}$ sont de Cauchy.}
\item Montrer que pour toute application continue born\'ee $g : \R^2 \to \R^2$, l'application $\Id + g : \R^2 \to \R^2$ est surjective. \\
\textit{Indication : on pourra appliquer le th\'eor\`eme de Brouwer (toute application continue d'une boule ferm\'ee dans elle-m\^eme admet un point fixe).}
\item En d\'eduire qu'il existe une application continue surjective $h : \T^2 \to \T^2$ telle que $f_\star \circ h = h \circ f$.
\item Montrer que tout automorphisme hyperbolique de $\T^2$ est structurellement stable.
\end{enumerate}

\vspace{0.6cm}

\noindent {\large \textbf{Exercice 5.} \textit{Gradients de fonctions de Morse}} \vspace{1.5mm} 

\noindent Soit $M$ une vari\'et\'e compacte et $f : M \to \R$ une fonction lisse. On dit que $f$ est une fonction de Morse si pour tout point $p \in M$ tel que $\dd f_p = 0$, la matrice Hessienne de $f$ en $p$ (dans une carte locale) est non d\'eg\'en\'er\'ee.
\begin{enumerate}
\item Montrer que la condition pr\'ec\'edente ne d\'epend pas de la carte choisie.
\item Montrer que l'ensemble des fonctions de Morse est ouvert dans $\mathcal{C}^2(M, \R).$
\end{enumerate}
Soit $f : M \to M$ une fonction de Morse. On se donne une m\'etrique Riemannienne $g$ sur $M$ et on d\'efinit $\nabla^g f \in \mathcal{C}^\infty(M,TM)$ le $g$-gradient de $f$ par
$$
\dd f_p(v) = g_p(\nabla^g f, v), \quad p \in M, \quad v \in T_pM.
$$
On suppose que pour tout point critique $p \in \mathrm{Crit}(f)$, il existe des coordonn\'ees locales $(x^1, \dots, x^n)$ centr\'ees en $p$ telles que
$$
\begin{aligned}
g &= \sum_{i=1}^n (\dd x^i)^2, \\
f(x^1, \dots, x^n) &= f(p) \sum_{i=1}^r (x^i)^2 - \sum_{i=r+1}^n (x^i)^2.
\end{aligned}
$$
On note $\varphi_t : M \to M$ le flot de $X = -\nabla^gf$.
\begin{enumerate}[resume]
\item On suppose $\varphi_t(x) = x$. Montrer que $t = 0$ ou $\nabla^gf (x) = 0$.
\item Soit $x \in M$ un point non-errant. Montrer que $\nabla^g f(x) = 0$.
\item Soit $x \in M$. Montrer qu'il existe $p,q \in \mathrm{Crit}(f)$ tels que si $t \to +\infty$
$$
\varphi_t(x) \to p, \quad \varphi_{-t}(x) \to q.
$$
\end{enumerate}


\end{document}

\noindent {\large \textbf{Exercice 8.} \textit{Flots hamiltoniens}} \vspace{1.5mm} 

\noindent Soit $H : \R^{2n} \to \R$ une fonction lisse. Le champ hamiltonien $X$ associ\'e \`a $H$ est le champ de vecteurs sur $\R^{2n}$ d\'efini par

$$ X(x, \xi) = J \cdot \nabla H (x,\xi), \quad (x,\xi) \in \R^{2n},$$
o\`u $J = \begin{pmatrix} 0 & I_n \\ -I_n & 0 \end{pmatrix}$. On suppose qu'il existe une application lisse $A : \R^n \to S_n(\R)$ telle que $A(x)$ est d\'efinie positive pour tout $x$ et
$$
H(x,\xi) = \frac{1}{2} \bigl\langle A(x)\xi, \xi \bigr\rangle, \quad (x,\xi) \in \R^{2n}.
$$
\vspace{0.6cm}

 
