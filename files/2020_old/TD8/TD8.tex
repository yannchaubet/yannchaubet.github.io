\documentclass[a4paper,10pt,openany]{article}
\usepackage{fancyhdr}
\usepackage[T1]{fontenc}
\usepackage[margin=1.8cm]{geometry}
\usepackage[applemac]{inputenc}
\usepackage{lmodern}
\usepackage{enumitem}
\usepackage{microtype}
\usepackage{hyperref}
\usepackage{enumitem}
\usepackage{dsfont}
\usepackage{amsmath,amssymb,amsthm}
\usepackage{mathenv}
\usepackage{amsthm}
\usepackage{graphicx}
\usepackage[all]{xy}
\usepackage{lipsum}       % for sample text
\usepackage{changepage}
\theoremstyle{plain}
\newtheorem{thm}{Theorem}[section]
\newtheorem*{thm*}{Th\'eor \`eme}
\newtheorem{prop}[thm]{Proposition}
\newtheorem{cor}[thm]{Corollary}
\newtheorem{lem}[thm]{Lemma}
\newtheorem{propr}[thm]{Propri\'et\'e}
\theoremstyle{definition}
\newtheorem{deff}[thm]{Definition}
\newtheorem{rqq}[thm]{Remark}
\newtheorem{ex}[thm]{Exercice}
\newcommand{\e}{\mathrm{e}}
\newcommand{\prodscal}[2]{\left\langle#1,#2\right\rangle}
\newcommand{\devp}[3]{\frac{\partial^{#1} #2}{\partial {#3}^{#1}}}
\newcommand{\w}{\omega}
\newcommand{\dd}{\mathrm{d}}
\newcommand{\x}{\times}
\newcommand{\ra}{\rightarrow}
\newcommand{\pa}{\partial}
\newcommand{\vol}{\operatorname{vol}}
\newcommand{\dive}{\operatorname{div}}
\newcommand{\T}{\mathbf{T}}
\newcommand{\R}{\mathbf{R}}
\newcommand{\Z}{\mathbf{Z}}
\newcommand{\N}{\mathbf{N}}
\newcommand{\C}{\mathbf{C}}
\newcommand{\F}{\mathcal{F}}
\newcommand{\Tcal}{\mathcal{T}}
\newcommand{\Rcal}{\mathcal{R}}
\newcommand{\Zcal}{\mathcal{Z}}
\newcommand{\Ncal}{\mathcal{N}}
\newcommand{\Ccal}{\mathcal{C}}
\newcommand{\Acal}{\mathcal{A}}
\newcommand{\Fcal}{\mathcal{F}}
\newcommand{\Pcal}{\mathcal{P}}
\newcommand{\Scal}{\mathcal{S}}
\newcommand{\Homeo}{\mathrm{Homeo}}
\renewcommand{\x}{\mathbf{x}}
\newcommand{\Matn}{\mathrm{Mat}_{n \times n}}
\DeclareMathOperator{\tr}{tr}
\newcommand{\id}{\mathrm{id}}
\newcommand{\htop}{h_\mathrm{top}}


\title{\textsc{Syst \`emes dynamiques} \\ Feuille d'exercices 8}
\date{}
\author{}

\begin{document}

{\noindent \'Ecole Normale Sup\'erieure  \hfill  \texttt{chaubet@dma.ens.fr} } \\
{2020/2021 \hfill}

{\let\newpage\relax\maketitle}
\maketitle
\noindent {\large \textbf{Exercice 1.} \textit{Gradients de fonctions de Morse}} \vspace{1.5mm} 

\noindent Soit $M$ une vari\'et\'e compacte et $f : M \to \R$ une fonction lisse. On dit que $f$ est une fonction de Morse si pour tout point $p \in M$ tel que $\dd f_p = 0$, la matrice Hessienne de $f$ en $p$ (dans une carte locale) est non d\'eg\'en\'er\'ee.
\begin{enumerate}
\item Montrer que la condition pr\'ec\'edente ne d\'epend pas de la carte choisie.
\item Montrer que l'ensemble des fonctions de Morse est ouvert dans $\mathcal{C}^2(M, \R).$
\end{enumerate}
Soit $f : M \to M$ une fonction de Morse. On se donne une m\'etrique Riemannienne $g$ sur $M$ et on d\'efinit $\nabla^g f \in \mathcal{C}^\infty(M,TM)$ le $g$-gradient de $f$ par
$$
\dd f_p(v) = g_p(\nabla^g f, v), \quad p \in M, \quad v \in T_pM.
$$
On suppose que pour tout point critique $p \in \mathrm{Crit}(f)$, il existe des coordonn\'ees locales $(x^1, \dots, x^n)$ centr\'ees en $p$ telles que
$$
\begin{aligned}
g &= \sum_{i=1}^n (\dd x^i)^2, \\
f(x^1, \dots, x^n) &= f(p) + \sum_{i=1}^r (x^i)^2 - \sum_{i=r+1}^n (x^i)^2.
\end{aligned}
$$
On note $\varphi_t : M \to M$ le flot de $X = -\nabla^gf$.
\begin{enumerate}[resume]
\item On suppose $\varphi_t(x) = x$. Montrer que $t = 0$ ou $\nabla^gf (x) = 0$.
\item Soit $x \in M$ un point non-errant. Montrer que $\nabla^g f(x) = 0$.
\item Soit $x \in M$. Montrer qu'il existe $p,q \in \mathrm{Crit}(f)$ tels que si $t \to +\infty$
$$
\varphi_t(x) \to p, \quad \varphi_{-t}(x) \to q.
$$
\end{enumerate}

\vspace{0.6cm}

\noindent {\large \textbf{Exercice 2} \textit{Th\'eor \`emes d'extension : rappels}} \vspace{1.5mm} 
\begin{enumerate}
\item \'Enoncer le th\'eor \`eme d'extension de Carath\'eodory.
\item \'Enoncer le lemme de classe monotone.
\end{enumerate}
\vspace{0.6cm}

\noindent {\large \textbf{Exercice 3.} \textit{Tribu produit}} \vspace{1.5mm} 

Soit $(A,\mathcal{F})$ un espace mesurable. Soit $X = A^\N$ l'espace des suites sur $A$. Pour tout $n \in \N$ et $\mathbf{A} = (A_1, \dots, A_p) \in \mathcal{F}^p$, on note 
$$
C_{n,\mathbf{A}} = \Bigl\{ \mathbf{x} = (x_k) \in A^\N,~x_{n-1+j} \in A_j, ~j = 1, \dots, p\Bigr\}.
$$
On note aussi $C_{n, \mathbf{w}} = C_{n,\mathbf{A}}$ o \`u $\mathbf{A} = (\{\omega_0\}, \dots, \{\omega_{p-1}\})$ pour tout mot $\mathbf{w} = (\omega_0, \dots, \omega_{p-1}) \in A^p$ et tout $n \in \N$.


\begin{enumerate}
\item D\'efinir la tribu produit sur $A^\N$. On la note $\mathcal{F}^{\otimes \N}$.
\item Soit $E$ un ensemble. On se donne $\Scal \subset \Pcal(E)$ une semi-alg \`ebre, i.e. $\emptyset \in \Scal$ et 
$$
A \cap B \in \Scal', \quad \text{ et }\quad A \setminus B = \bigsqcup_{i=1}^q A_i, \quad A_i \in \Scal.
$$
Montrer que toute mesure sur $\Scal$ (i.e. une application $\sigma$-additive $\mu : \Scal \to [0, +\infty]$ telle que $\mu(\emptyset) = 0$) s'\'etend uniquement en une mesure sur 
$$
\Scal' = \{A_1 \cup \cdots \cup A_n,~A_i \in \Scal,~n \in \N\}.
$$
\item On se donne $P$ une mesure de probabilit\'e sur $A$ et on note $\Scal$ l'ensemble des cylindres. On d\'efinit $\mu : \Scal \to [0, +\infty]$ par $\mu(\emptyset) = 0$ et
\begin{equation}\label{eq:prod}
\mu\left(C_{n,\mathbf{A}}\right) = \prod_{j=1}^p P(A_j), \quad \mathbf{A} = (A_1, \dots, A_p) \in \mathcal{F}^p.
\end{equation}
On se donne des cylindres $S^n = S^n_0 \times S^n_1 \times \cdots \in \Scal$, ($n \in \N$) tels que $X= \bigcup_n S^n$ ; pour tout $k \in \N$ on consid \`ere
$
H_k : A^{k+1} \to [0,1]
$
d\'efinie par
$$
H_k(x_0, \dots, x_k) = \sum_{n \geq 0} \left( \prod_{j > k} P(S^n_j)\right)\left(\prod_{i=0}^k 1_{S^n_i}(x_i)\right).
$$
\begin{enumerate}
\item Montrer que l'on a pour tout $k \in \N$ et tous $x_0, \dots, x_k \in A$
$$
H_k(x_0, \dots, x_k) = \int_A H_{k+1}(x_0, \dots, x_k, x) \dd P(x).
$$
\item On suppose que $\sum_n \mu(S^n) < 1$. Construire par r\'ecurrence une suite $\mathbf{x} = (x_n) \in A^\N$ telle que 
$$
H_k(x_0, \dots, x_k) < 1, \quad k \in \N.
$$
\item En d\'eduire qu'il existe une unique mesure de probabilit\'es $\mu$ sur $(X, \F^{\otimes \N})$ invariante par le d\'ecalage (i.e. $\mu(\sigma^{-1}(C)) = C$ pour tout $C \in \Fcal^{\otimes \N}$ o \`u $\sigma$ est le d\'ecalage), telle que l'\'equation (\ref{eq:prod}) est satisfaite. On la note $P^{\otimes \N}.$
\end{enumerate}

\item Donner une preuve simple du fait pr\'ec\'edent dans le cas o \`u $A$ est fini et o \`u $\mathcal{F} = \mathcal{P}(A)$.
\end{enumerate}
On suppose dans la suite que $A = \{1, \dots, m\}$ et $\mathcal{F} = \mathcal{P}(A)$.
\begin{enumerate}[resume]
\item Soit $M = (m_{ij})$ est une matrice $m \times m$  \`a coefficients strictement positifs telle que
$$
\sum_{j=1}^m m_{ij} = 1, \quad i = 1, \dots, m.
$$
On suppose qu'il existe $v = (v_1, \dots, v_m) \in \R_+^m$ tel que $vM = v$ et $\sum_{i} v_i = 1$ (en fait il existe toujours un unique vecteur v\'erifiant cette propri\'et\'e : c'est le th\'eor \`eme de Perron-Frobenius). Montrer qu'il existe une unique mesure de probabilit\'es $P_M$ sur $X$ telle que pour tout $\mathbf{w} = (\omega_0, \dots, \omega_{p-1}) \in A^p$ et tout $n \in \N$,
$$
P_M\left(C_{n, \mathbf{w}}\right) = v_{\omega_0} \prod_{j=0}^{p-2} m_{\omega_j \omega_{j+1}}.
$$
Montrer que $\sigma$ pr\'eserve $P_M$.
\item Soit $P$ une probabilit\'e sur $A$ telle que $P(\{i\}) \neq 0$ pour tout $i$. Montrer que $P^{\otimes \N} = P_{M(P)},$ o \`u $M(P) \in \mathrm{Mat}_{m \times m}(\R_+^*)$ est la matrice de coefficients $M(P)_{ij} = P(\{j\})$ pour tous $i,j$.
\item On consid \`ere l'application 
$$
\begin{aligned}
H : \{1,\dots, m\}^\N &\to \T^1\\
\mathbf{x}=(x_k) &\mapsto \sum_{k=1}^\infty \frac{x_k-1}{m^k} + \Z.
\end{aligned}
$$
Soit $\mu_m$ la mesure de probabilit\'e \'equidistribu\'ee sur $\{1, \dots, m\}$. Montrer que
$$\mathrm{Leb}(H(C_{n,\mathbf{w}})) = \mu_m^{\otimes \N}(C_{n, \mathbf{w}}), \quad n \in \N, \quad \mathbf{w} \in \{1, \dots, m\}^p.$$
\item Montrer que le compl\'ementaire $Z \subset \T^1$ des points $m$-adiques est de mesure de Lebesgue totale.
\item Montrer que $H (\sigma(\mathbf{x})) = mH(\mathbf{x}),$ o \`u $\sigma$ est le d\'ecalage sur $X$, et que $H : H^{-1}(Z) \to Z$ est une bijection.
\end{enumerate}


\end{document}

\noindent {\large \textbf{Exercice 8.} \textit{Flots hamiltoniens}} \vspace{1.5mm} 

\noindent Soit $H : \R^{2n} \to \R$ une fonction lisse. Le champ hamiltonien $X$ associ\'e  \`a $H$ est le champ de vecteurs sur $\R^{2n}$ d\'efini par

$$ X(x, \xi) = J \cdot \nabla H (x,\xi), \quad (x,\xi) \in \R^{2n},$$
o \`u $J = \begin{pmatrix} 0 & I_n \\ -I_n & 0 \end{pmatrix}$. On suppose qu'il existe une application lisse $A : \R^n \to S_n(\R)$ telle que $A(x)$ est d\'efinie positive pour tout $x$ et
$$
H(x,\xi) = \frac{1}{2} \bigl\langle A(x)\xi, \xi \bigr\rangle, \quad (x,\xi) \in \R^{2n}.
$$


\vspace{0.6cm}




\end{document}
 
