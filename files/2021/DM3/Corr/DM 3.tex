\documentclass[french]{article}


\usepackage[applemac]{inputenc}
\usepackage[T1]{fontenc}
\usepackage[margin=2.8cm]{geometry}
\usepackage{babel}
\usepackage{array,epsfig}
\usepackage{amsmath}
\usepackage{amsfonts}
\usepackage{amssymb}
\usepackage{amsxtra}
\usepackage{amsthm}
\usepackage{latexsym}
\usepackage{dsfont}
\usepackage{mathrsfs}
\usepackage[mathscr]{eucal}
\usepackage{color}
\usepackage[all]{xy}
\usepackage{hyperref}
%\usepackage[notref,notcite]{showkeys}
\usepackage{graphicx}






\renewcommand{\labelenumi}{\textbf{\arabic{enumi}.}}
\renewcommand{\labelenumii}{(\alph{enumii})}
\renewcommand{\labelenumiii}{(\roman{enumiii})}
\renewcommand{\labelenumiv}{$\bullet$}





\theoremstyle{definition}
\newtheorem{defn}{D\'efinition}
\newtheorem{thm}{Th\'eor\`eme}
\newtheorem{cor}{Corollaire}
\newtheorem*{rmk}{Remarque}
\newtheorem{lem}{Lemme}
\newtheorem{ex}{Exercice}
\newtheorem*{soln}{Solution}
\newtheorem{prop}{Proposition}




\newcommand{\set}[1]{\left\{#1\right\}}
\newcommand{\tuple}[1]{\left(#1\right)}
\newcommand{\oin}[1]{\left(#1\right)}
\newcommand{\cin}[1]{\left[#1\right]}
\newcommand{\olin}[1]{\left(#1\right]}
\newcommand{\orin}[1]{\left[#1\right)}
\newcommand{\abs}[1]{\left|#1\right|}
\newcommand{\norm}[1]{\left\|#1\right\|}
\newcommand{\sprod}[1]{\left<#1\right>}
\newcommand{\floor}[1]{\left\lfloor#1\right\rfloor}
\newcommand{\ceil}[1]{\left\lceil#1\right\rceil}
\newcommand{\ol}[1]{\overline{#1}}
\newcommand{\wh}[1]{\widehat{#1}}
\newcommand{\wt}[1]{\widetilde{#1}}







\newcommand{\indi}{\mathds{1}}
\newcommand{\emb}{\hookrightarrow}
\newcommand{\proj}{\twoheadrightarrow}
\newcommand{\funct}{\rightsquigarrow}
\newcommand{\del}{\partial}
\newcommand{\Ra}{\Rightarrow}
\newcommand{\Lra}{\Leftrightarrow}





\renewcommand{\H}{\mathrm{H}}
\newcommand{\N}{\mathrm{N}}




\renewcommand{\Bbb}{\mathbb{B}}
\newcommand{\Cbb}{\mathbf{C}}
\newcommand{\Dbb}{\mathbb{D}}
\newcommand{\Ebb}{\mathbb{E}}
\newcommand{\Fbb}{\mathbb{F}}
\newcommand{\Hbb}{\mathbb{H}}
\newcommand{\Ibb}{\mathbb{I}}
\newcommand{\Kbb}{\mathbb{K}}
\newcommand{\kbb}{\mathbb{K}}
\newcommand{\Nbb}{\mathbf{N}}
\newcommand{\Obb}{\mathbb{O}}
\newcommand{\Pbb}{\mathbb{P}}
\newcommand{\Qbb}{\mathbf{Q}}
\newcommand{\Rbb}{\mathbf{R}}
\newcommand{\Sbb}{\mathbb{S}}
\newcommand{\Tbb}{\mathbf{T}}
\newcommand{\Zbb}{\mathbf{Z}}




\newcommand{\Acal}{\mathcal{A}}
\newcommand{\Bcal}{\mathcal{B}}
\newcommand{\Ccal}{\mathcal{C}}
\newcommand{\Dcal}{\mathcal{D}}
\newcommand{\Ecal}{\mathcal{E}}
\newcommand{\Fcal}{\mathcal{F}}
\newcommand{\Gcal}{\mathcal{G}}
\newcommand{\Ical}{\mathcal{I}}
\newcommand{\Jcal}{\mathcal{J}}
\newcommand{\Lcal}{\mathcal{L}}
\newcommand{\Mcal}{\mathcal{M}}
\newcommand{\Pcal}{\mathcal{P}}
\newcommand{\Ocal}{\mathcal{O}}
\newcommand{\Ucal}{\mathcal{U}}
\newcommand{\Vcal}{\mathcal{V}}
\newcommand{\Ncal}{\mathcal{N}}







\newcommand{\Fix}{\operatorname{Fix}}
\newcommand{\Homeo}{\operatorname{Homeo}}
\newcommand{\id}{\operatorname{id}}
\newcommand{\Var}{\operatorname{Var}}
\newcommand{\abf}{\mathbf{a}}










\newcommand{\ds}{\displaystyle}	

\title{\textsc{Syst\`emes dynamiques} \\Corrig\'e DM 3}
\date{}
\author{\Large Bas\'e sur la copie de Manh-Linh Nguyen}

\begin{document}
{\noindent \'Ecole Normale Sup\'erieure  \hfill Pour toute question :}
\\
{2021/2022 \hfill \hfill  \texttt{chaubet\at dma.ens.fr}}

{\let\newpage\relax\maketitle}
\maketitle

	
\vspace{0.2 cm}
\section*{\'Echauffement}
\begin{enumerate}
    \item \label{Partie1}
    \begin{enumerate}
        \item Si $f(q) \ge \frac{1}{2}$, le r\'esultat est \'evident. On suppose donc que $f(q) < \frac{1}{2}$. Pour $x \in [0,1]$, s'il existe $p \in \Nbb_{\ge 1}$ tel que 
        \begin{equation} \label{eq1}
            \abs{x - \frac{p}{q}} < \frac{f(p)}{q}
        \end{equation}
    alors $|qx - p| < f(q)$, d'o\`u $p < qx + f(q) < q + \frac{1}{2}$, i.e. $p \le q$. Ainsi
        $$A_q = \bigcup_{p = 1}^q A_{q,p}, \qquad A_{q,p}:=[0,1] \cap \oin{\tfrac{f(q) - p}{q}, \tfrac{f(q) + p}{q}}.$$
    Pour$p = 1,\ldots,q$, $\ell(A_{q,p}) \le \ell\tuple{\oin{\tfrac{f(q) - p}{q}, \tfrac{f(q) + p}{q}}} = \frac{2f(q)}{q}$. Donc
        \begin{equation} \label{eq2}
            \ell(A_q) \le \sum_{p = 1}^q \ell(A_{q,p}) \le \sum_{p=1}^q  = \sum_{p=1}^q \frac{2f(q)}{q} = 2f(q).
        \end{equation}
    \item Pour $x \in [0,1]$, \eqref{eq1} est vrai pour une infinit\'e de couples $(p,q)$ ssi pour tout $n \ge 1$, il existe $(p,q)$ avec $q \ge 1$ v\'erifiant \eqref{eq1}, i.e. $x \in A_q$. Il s'agit d'\'etudier la mesure de Lebesgue de l'ensemble $\bigcap_{n \ge 1}\bigcup_{q \ge n}A_q$. Sous l'hypoth\`ese du second point du {\bf Th\'eor\`eme}, la s\'erie $\sum_{n \ge 1} \ell(A_q)$ converge. Par le lemme de Borel-Cantelli
        $$\ell\tuple{\bigcap_{n \ge 1} \bigcup_{q \ge n}A_q} = 0,$$
    i.e. pour presque tout $x \in [0,1]$, \eqref{eq1} n'est vraie pour qu'un nombre fini de couples $(p,q)$.
    \end{enumerate}
\end{enumerate}


\section*{D\'eveloppement en fractions continues}
\begin{enumerate}
    \setcounter{enumi}{1}
    \item \label{Partie2} Pour $m = 1$, on a
        $$[a_1(x);T^1(x)] = \frac{1}{a(x) + T(x)} = \frac{1}{\floor{1/x} + \set{1/x}} = \frac{1}{1/x} = x.$$
    Pour $m \ge 2$, on a
        $$a_m(x) + T^m(x) = a(T^{m-1}(x)) + T^m(x) = \floor{\tfrac{1}{T^{m-1}(x)}} + \set{\tfrac{1}{T^{m-1}(x)}} = \frac{1}{T^{m-1}(x)}.$$
    Ainsi
        \begin{align*}
            [a_1(x),\ldots,a_m(x);T^m(x)] & = \frac{1}{a_1(x) + \dfrac{1}{\ddots + \dfrac{1}{a_{m-1}(x) + \dfrac{1}{a_m(x) + T^m(x)}}}} \\
            & = \frac{1}{a_1(x) + \dfrac{1}{\ddots + \dfrac{1}{a_{m-1}(x) + T^{m-1}(x)}}} \\
            & = [a_1(x),\ldots,a_{m-1}(x);T^{m-1}(x)],
        \end{align*}
    d'o\`u le r\'esultat.
    
    \item \label{Partie3} Le sens \og si \fg{} est clair. On d\'emontre le sens \og seulement si \fg{}. Pour $x \in I$ rationnel, on \'ecrit $x = \tfrac{a}{b}$ avec $(a,b) \in \Nbb_{\ge 1}^2$, $\gcd(a,b) = 1$ et $a < b$ (les cas o\`u $x = 0$ ou $x=1$ sont faciles). \'Ecrivons $b = qa + c$ $(0 \le c < a)$, alors 
        $$T(x) = \set{\frac{1}{x}} = \set{\frac{qa + c}{a}} = \frac{c}{a}.$$
    On voit que $T(x) \in \Qbb$ et que le d\'enominateur de $T(x)$ (sous forme r\'eduite) est strictement plus petit que celui de $x$. Ainsi, il existe $n \ge 1$ tel que $T^n(x)$ a d\'enominateur $1$, i.e. $T^n(x) \in \{0,1\}$. Donc $T^{n+1}(x) = 0$.
    
    \item \label{Partie4} 
    \begin{enumerate}
        \item \label{Partie4a}
             Quand $n = 0$, c'est clair.  Pour $n \ge 1$
                \begin{align*}
                    & p_{n-1}(x)q_n(x) - p_n(x)q_{n-1}(x) \\
                    = & \  p_{n-1}(x)(a_n(x) q_{n-1}(x) + q_{n-2}(x)) - (a_n(x) p_{n-1}(x) + p_{n-2}(x))q_{n-1}(x) \\
                    = & \  -(p_{n-2}(x)q_{n-1}(x) - p_{n-1}(x)q_{n-1}(x)),
                \end{align*}
            d'o\`u le r\'esultat suit (r\'ecurrence sur $n$).
            
        \item \label{Partie4b} Quand $n = 1$, on a
            $$\frac{p_1(x) + tp_0(x)}{q_1(x) + tq_0(x)} = \frac{1}{a_1(x) + t} = [a_1(x);t].$$
        Quand $n = 2$, on a
            \begin{align*}
                \frac{p_2(x) + tp_1(x)}{q_2(x) + tq_1(x)} & = \frac{a_2(x) + t}{a_1(x)a_2(x) + 1 + a_1(x)t} \\
                & = \frac{1}{a_1(x) + \dfrac{1}{a_2(x) + t}} \\
                & = [a_1(x),a_2(x);t].
            \end{align*}
        On consid\`ere $n \ge 3$. Si $a_n(x) = 1$ et $t = 0$, par r\'ecurrence
            \begin{align*}
                [a_1(x),\ldots,a_{n-1}(x),1;0] & = \frac{1}{a_1(x) + \dfrac{1}{\ddots + \dfrac{1}{a_{n-2}(x) + \dfrac{1}{a_{n-1}(x) + 1}}}} \\
                & = \cin{a_1(x),\ldots,a_{n-2}(x); \tfrac{1}{a_{n-1}(x) + 1}}\\
                & = \frac{p_{n-2}(x) + \tfrac{p_{n-3}(x)}{a_{n-1}(x) + 1}}{q_{n-2}(x) + \tfrac{q_{n-3}(x)}{a_{n-1}(x) + 1}} \\
                & = \frac{a_{n-1}(x)p_{n-2}(x) + p_{n-3}(x) + p_{n-2}(x)}{a_{n-1}(x)q_{n-2}(x) + q_{n-3}(x) + q_{n-2}(x)} \\
                & = \frac{p_{n-1}(x) + p_{n-2}(x)}{q_{n-1}(x) + q_{n-2}(x)}\\
                & = \frac{p_{n}(x)}{q_n(x)} & (a_n(x) = 1).
            \end{align*}
        Si $a_n(x) > 1$ ou $t > 0$, alors $\tfrac{1}{a_n(x) + t} < 1$. Par r\'ecurrence
            \begin{align*}
                [a_1(x),\ldots,a_{n}(x);t] & = \frac{1}{a_1(x) + \dfrac{1}{\ddots + \dfrac{1}{a_{n-1}(x) + \dfrac{1}{a_{n}(x) + t}}}} \\
                & = \cin{a_1(x),\ldots,a_{n-1}(x); \tfrac{1}{a_{n}(x) + t}}\\
                & = \frac{p_{n-1}(x) + \tfrac{p_{n-2}(x)}{a_{n}(x) + t}}{q_{n-1}(x) + \tfrac{q_{n-2}(x)}{a_{n}(x) + t}} \\
                & = \frac{a_{n}(x)p_{n-1}(x) + p_{n-2}(x) + tp_{n-1}(x)}{a_{n}(x)q_{n-1}(x) + q_{n-2}(x) + tq_{n-1}(x)} \\
                & = \frac{p_{n}(x) + tp_{n-1}(x)}{q_{n}(x) + tq_{n-1}(x)}.
            \end{align*}
        D'o\`u le r\'esultat.
        
        \item \label{Partie4c} De {\bf\ref{Partie1}.} et (\ref{Partie4b}), on a
            \begin{align*}
                \abs{x - \frac{p_n(x)}{q_n(x)}} & = \abs{\frac{p_n(x) + T^n(x)p_{n-1}(x)}{q_n(x) + T^n(x)q_{n-1}(x)} - \frac{p_n(x)}{q_n(x)}} \\
                & = \frac{T^n(x) \abs{p_{n-1}(x)q_n(x) - p_n(x)q_{n-1}(x)}}{q_n(x)(q_n(x) + T^n(x)q_{n-1}(x))} \\
                & = \frac{1}{q_n(x)\tuple{\frac{q_n(x)}{T^n(x)} + q_{n-1}(x)}} & (\ref{Partie4a}).
            \end{align*}
        Il suffit de montrer que
            $$q_n(x) + q_{n+1}(x) \ge \frac{q_n(x)}{T^n(x)} + q_{n-1}(x) \ge q_{n+1}(x).$$
        En effet, comme $a_{n+1}(x) = \floor{\frac{1}{T_n(x)}}$, on a
            $$a_{n+1}(x) \le \frac{1}{T_n(x)} < a_{n+1}(x) + 1.$$
        Il suit que
            $$a_{n+1}(x)q_n(x) + q_{n-1} \le \frac{q_n(x)}{T_n(x)} + q_{n-1}(x) < a_{n+1}(x)q_n(x) + q_n(x) + q_{n-1}(x),$$
        i.e.
            $$q_{n+1}(x) \le \frac{q_n(x)}{T_n(x)} + q_{n-1}(x) < q_{n+1}(x) + q_n(x),$$
        ce que nous voulions.
    \end{enumerate}
    
    \item Les suites $(p_n(x))_{n \ge 1}$ et $(q_n(x))_{n \ge 1}$ sont croissantes et positives. Pour $n \ge 3$, on a
        $$p_n(x) = a_n(x)p_{n-1}(x) + p_{n-2}(x) \ge p_{n-1}(x) + p_{n-2}(x) \ge 2\sqrt{p_{n-1}(x)p_{n-2}(x)}.$$
    Il suit que
        $$p_n(x) \cdots p_3(x) \ge 2\sqrt{p_{n-1}(x)p_{n-2}(x)} \cdots \sqrt{p_{2}(x)p_{1}(x)},$$
    i.e. $p_n(x)\sqrt{p_{n-1}(x)} \ge 2^{n-2}p_2(x)\sqrt{p_1(x)} \ge 2^{n-2}$. Donc $p_n(x)^2 \ge 2^{n-2}$ et on trouve que $p_n(x) \ge 2^{\frac{n-2}{2}}$. De m\^eme, $q_n(x) \ge 2^{\frac{n-2}{2}}$, donc 
        \begin{equation} \label{eq3}
            p_n(x)q_n(x) \ge 2^{n-2}.
        \end{equation}
    On consid\`ere deux cas.
    \begin{enumerate}
        \item $n$ est pair. De {\bf \ref{Partie2}.}, (\ref{Partie4a}) et (\ref{Partie4b}), on a
            \begin{align*}
                \frac{x}{p_n(x)/q_n(x)} & = \frac{q_n(x)(p_n(x) + T^n(x)p_{n-1}(x))}{p_n(x)(q_n(x) + T^n(x)q_{n-1}(x))} \\
                & = 1 + \frac{T^n(x)(p_{n-1}(x)q_n(x) - p_n(x)q_{n-1}(x))}{p_n(x)(q_n(x) + T^n(x)q_{n-1}(x))} \\
                & = 1 + \frac{T^n(x)}{p_n(x)(q_n(x) + T^n(x)q_{n-1}(x))}\\
                & = 1 + \frac{1}{p_n(x)\tuple{\frac{q_n(x)}{T^n(x)} + q_{n-1}(x)}} > 1.
            \end{align*}
        Donc
            \begin{align*}
                \abs{\log \frac{x}{p_n(x)/q_n(x)}} & = \log\tuple{1 + \frac{1}{p_n(x)\tuple{\frac{q_n(x)}{T^n(x)} + q_{n-1}(x)}}} \\
                & \le \frac{1}{p_n(x)\tuple{\frac{q_n(x)}{T^n(x)} + q_{n-1}(x)}}\\
                & \le \frac{1}{p_n(x)q_n(x)} & (T^n(x) \le 1) \\
                & \le \frac{1}{2^{n-2}} & (\text{d'apr\`es \eqref{eq3}}).
            \end{align*}
        
    \item $n$ est impair. Dans ce cas
        \begin{align*}
                \frac{p_n(x)/q_n(x)}{x} & = \frac{p_n(x)(q_n(x) + T^n(x)q_{n-1}(x))}{q_n(x)(p_n(x) + T^n(x)p_{n-1}(x))} \\
                & = 1 + \frac{T^n(x)(p_{n}(x)q_{n-1}(x) - p_{n-1}(x)q_{n}(x))}{q_n(x)(p_n(x) + T^n(x)p_{n-1}(x))} \\
                & = 1 + \frac{T^n(x)}{q_n(x)(p_n(x) + T^n(x)p_{n-1}(x))}\\
                & = 1 + \frac{1}{q_n(x)\tuple{\frac{p_n(x)}{T^n(x)} + p_{n-1}(x)}} > 1,
            \end{align*}
        et l'argument est similaire.
    \end{enumerate}
\end{enumerate}



\section*{La mesure de Gauss}
\begin{enumerate}
    \setcounter{enumi}{5}
    \item \label{Partie6}
        On \'ecrit $(0,1] = \bigsqcup_{n = 1}^\infty \olin{\frac{1}{n+1},\frac{1}{n}}$. Pour $n \in \Nbb_{\ge 1}$ et $x \in \olin{\frac{1}{n+1},\frac{1}{n}}$, on a $n \le \frac{1}{x} < n+1$, d'o\`u 
            $$T(x) = \set{\frac{1}{x}} = \frac{1}{x} - \floor{\frac{1}{x}} = \frac{1}{x} - n.$$
        Ainsi, pour toute fonction continue $f:I \to \Rbb_{\ge 0}$, on a
            \begin{align*}
                & \int_I f d(T_\ast \mu)\\
                = & \ 
                \int_{I} f \circ T \,d\mu \\
                = & \ \frac{1}{\log 2}\int_0^1 \frac{f(T(x))\,dx}{1 + x} \\
                = & \ \frac{1}{\log 2}\sum_{n = 1}^\infty \int_{\frac{1}{n+1}}^{\frac{1}{n}} \frac{f\tuple{\frac{1}{x} - n}dx}{1 + x} & \text{(convergence monotone)}\\
                = & \ \frac{1}{\log 2}\sum_{n = 1}^\infty \int_1^0 \frac{f(y)}{1 + \frac{1}{y+n}} \cdot \tuple{-\frac{dy}{(y+n)^2}} & (y = \frac{1}{x} - n) \\
                = & \ \frac{1}{\log 2}\sum_{n = 1}^\infty \int_0^1 \frac{f(y)dy}{(y+n)(y+n+1)} \\
                = & \ \frac{1}{\log 2}\sum_{n=1}^\infty \int_0^1 \tuple{\frac{1}{y+n} - \frac{1}{y+n+1}}f(y)dy \\
                = & \ \frac{1}{\log 2}\int_0^1 \frac{f(y)dy}{y+1} & \text{(convergence monotone)}\\
                = & \ \int_I f\, d\mu.
            \end{align*}
        Donc $T_\ast \mu = \mu$.
        
    \item \label{Partie7}
        \begin{enumerate}
            \item \label{Partie7a} Soit $t \in [0,1)$ et $x = \psi_{a_1,\ldots,a_m}(t) = [a_1,\ldots,a_m;t]$. On va montrer que $a_j(x) = a_j$ pour tout $j = 1,\ldots,m$ par r\'ecurrence sur $m$ (d'o\`u on a $x \in I_{a_1,\ldots,a_m}$). Quand $m = 1$, on a $x = \frac{1}{a_1 + t}$, donc $\frac{1}{a_1 + 1} < x \le \frac{1}{a_1}$, d'o\`u $a_1 \le \frac{1}{x} < a_1 + 1$. Mais alors $a_1(x) = a(x) = \cin{\frac{1}{x}} = a_1$. \\
            On consid\`ere $m \ge 2$. De {\bf\ref{Partie2}.}, on sait que
                $$x = [a_1(x),a_2(x),\ldots,a_m(x);T^m(x)] = \frac{1}{a_1(x) + [a_2(x),\ldots,a_m(x);t]}.$$
            Il suit que 
                \begin{equation} \label{eq4}
                    a_1 + [a_2,\ldots,a_m;t] = \frac{1}{x} = a_1(x) + [a_2(x),\ldots,a_m(x);T^m(x)].
                \end{equation}
            Si $m = 2$ et $t = 0$, alors $x = \frac{1}{a_1 + \frac{1}{a_2}} = [a_1;\frac{1}{a_2}]$. Il suit du cas o\`u $m=1$ que $a_1(x) = a_1$. Si $m \ge 3$ o\`u $t > 0$, on aura
                $$a_2 + \frac{1}{a_3 + \dfrac{1}{ \ddots + \dfrac{1}{a_m + t}}} > 1,$$
            donc $0 \le [a_2,\ldots,a_m;t] < 1$. En particulier $x \notin \Zbb$. De \eqref{eq4}, $[a_2(x),\ldots,a_m(x);T^m(x)] \notin \Zbb$, donc $[a_2(x),\ldots,a_m(x);T^m(x)] < 1$. Il suit que $a_1 = a_1(x)$. Dans tous cas, on a $a_1(x) = a_1$ et puis $[a_2,\ldots,a_m;t] = [a_2(x),\ldots,a_m(x);T^m(x)]$. Par l'hypoth\`ese de r\'ecurrence, $a_j(x) = a_j$ pour tout $2 \le j \le m$. \\
            Reciproquement, pour tout $x \in I_{a_1,\dots,a_m}$, on a $a_j(x) = a_j$ pour $1 \le j \le m$. De {\bf \ref{Partie2}.}
                $$x = [a_1,\ldots,a_m;T^m(x)] = \psi_{a_1,\ldots,a_m}(T^m(x)).$$
                
            \item \label{Partie7b}
                Soit $x = \psi_{a_1,\ldots,a_m}(t)$. Il suit du partie pr\'ec\'edent que $x \in I_{a_1,\ldots,a_m}$, i.e. $a_j(x) = a_j$ pour $j=1,\ldots,m$. Par r\'ecurrence, on a $p_j(x) = p_j$ et $q_j(x) = q_j$ pour tout $j = 1,\ldots,m$. En outre, de (\ref{Partie4b}), on a
                    $$\psi_{a_1,\ldots,a_m}(t) = [a_1(x),\ldots,a_m(x);t] = \frac{p_m(x) + tp_{m-1}(x)}{q_m(x) + tq_{m-1}(x)} = \frac{p_m + tp_{m-1}}{q_m + tq_{m-1}}.$$
            \item \label{Partie7c}
                Il suit de (\ref{Partie7b}) que la fonction $\psi_{a_1,\ldots,a_m}$ est continue est monotone. En particulier $I_{a_1,\ldots,a_m} = \psi_{a_1,\ldots,a_m}([0,1))$ est un intervalle dont les extr\'emit\'es sont $\frac{p_m + p_{m-1}}{q_m + q_{m_1}}$ et $\frac{p_m}{q_m}$. Ainsi
                    $$\ell(I_{a_1,\ldots,a_m}) = \abs{\frac{p_m + p_{m-1}}{q_m + q_{m-1}} - \frac{p_m}{q_m}} = \frac{\abs{p_{m-1}q_m - p_m q_{m-1}}}{q_m(q_m + q_{m-1})} = \frac{1}{q_m(q_m + q_{m-1})}$$
                (en effet, pour n'importe quel $x \in I_{a_1,\ldots,a_m}$, on a $p_{m-1}q_m - p_mq_{m-1} = p_{m-1}(x)q_m(x) - p_m(x)q_{m-1}(x) = (-1)^m$ par (\ref{Partie4b})).
                
            \item \label{Partie7d} On a vu dans (\ref{Partie7c}) que les $I_{a_1,\ldots,a_m}$ sont des intervalles, donc un sens est trivial. Pour le sens reciproque, il faut montrer que les bor\'eliens sont dans la tribu $\Fcal$ engendr\'ee par les intervalles de cette forme (et $I$). On divise la preuve en plusieurs \'etapes.
            \begin{enumerate}
                \item On a $$\forall n \in \Nbb_{n \ge 1}, \qquad I_n = \psi_n([0,1)) = \set{\tfrac{1}{n+t} | t \in [0,1)} = \olin{\tfrac{1}{n+1},\tfrac{1}{n}}.$$
                Ainsi les intervalles $\olin{\tfrac{1}{n+1},\tfrac{1}{n}}$ sont $\Fcal$-mesurables. De plus, pour tout $n \in \Nbb_{n \ge 1}$
                    $$\olin{0,\tfrac{1}{n}} = \bigcup_{k \ge n} \olin{\tfrac{1}{k+1},\tfrac{1}{k}} \in \Fcal.$$
                    
                \item Pour tous $n, k \in \Nbb_{n \ge 1}$, on a
                    $$I_{n,k} = \set{\tfrac{1}{n + \frac{1}{k+t}} | t \in [0,1)} = \orin{\tfrac{k}{nk+1}, \tfrac{k+1}{n(k+1)+1}}.$$
                Donc, puor tout $n \in \Nbb_{n \ge 1}$
                    $$\Fcal \owns \bigcup_{k \ge 1}  \orin{\tfrac{k}{nk+1}, \tfrac{k+1}{n(k+1)+1}} = \orin{\tfrac{1}{n+1},\tfrac{1}{n}}.$$
                Ainsi
                    $$\oin{0,\tfrac{1}{n}} = \bigcup_{k \ge n} \orin{\tfrac{1}{k+1},\tfrac{1}{k}} \in \Fcal.$$
                En particulier, le singletons $\set{\tfrac{1}{n}}$, $n \in \Nbb_{\ge 1}$ sont $\Fcal$-mesurables.
                
                \item {\it Pour tout $k \in \Nbb_{\ge 1}$, la fonction $\psi_k$ est $\Fcal$-mesurable}. En effet $\psi_k$ est \'evidemment injective d'image $I_k$. De plus, $\psi_k([0,1)) \cap \psi_{k'}([0,1)) = \varnothing$ si $k \neq k'$. 
                
                On consid\`ere un intervalle $I_{a_1,\ldots,a_m}$. Par r\'ecurrence triviale, on a $\psi_{a_1,\ldots,a_m} = \psi_{a_1} \circ \cdots \circ \psi_{a_m}$. En particulier 
                    $$I_{a_1,\ldots,a_m} = \psi_{a_1,\ldots,a_m}([0,1)) \subseteq \psi_{a_1}([0,1)) = I_{a_1}.$$
                Il suit que $\psi^{-1}_k(I_{a_1,\ldots,a_m}) = \varnothing$ quand $a_1 \neq k$.  Si $a_1 = k$, on aura
                    $$ \psi^{-1}_{k}(I_{a_1,\ldots,a_m}) = \psi^{-1}_k(\psi_{a_1} \circ \cdots \circ \psi_{a_m})([0,1)) = \begin{cases}
                       I_{a_2,\ldots,a_m} & \text{si } m \ge 2\\
                       [0,1) & \text{si } m = 1.
                    \end{cases}$$
                Dans tous cas, $\psi^{-1}_k(I_{a_1,\ldots,a_m}) \in \Fcal$ ($[0,1) \in \Fcal$) car $\{1\} \in \Fcal$). La collection $\{A \in \Fcal | \psi_k^{-1}(A) \in \Fcal\}$ est une tribu contenant les intervalles $I_{a_1,\ldots,a_m}$, donc \'egal \`a $\Fcal$.
                
                \item Les intervalles $\olin{0,\tfrac{k}{n}}$ et $\oin{0,\tfrac{k}{n}}$  (o\`u $1 \le k \le n$) sont $\Fcal$-mesurables. On le d\'emontre par r\'ecurrence sur $n$. Quand $n = k$ (en pariculier quand $n = 1$), c'est trait\'e. Pour $n \ge 2$ et $k < n$, \'ecrivons
                    $$n = km + r, \qquad m \ge 1, \qquad 0 \le r < k.$$
                Si $r = 0$, alors $\olin{0,\tfrac{k}{n}} = \olin{0,\tfrac{1}{m}} \in \Fcal$ (cas (i)) et $\oin{0,\tfrac{k}{n}} = \oin{0,\tfrac{1}{m}} \in \Fcal$  (cas (ii)). On suppose alors que $r > 0$. Sans peine, on voit que
                    $$\olin{0,\tfrac{k}{n}} = \psi^{-1}_m\tuple{\cin{\tfrac{r}{k},1}} \in \Fcal, \qquad \oin{0,\tfrac{k}{n}} = \psi^{-1}_m\tuple{\olin{\tfrac{r}{k},1}} \in \Fcal$$
                car $(0,1] \in \Fcal$ et $\oin{0,\tfrac{r}{k}},\olin{0,\tfrac{r}{k}}\in \Fcal$ par l'hypoth\`ese de r\'ecurrence.
                
                \item Il suit du cas pr\'ec\'edent que tous les intervalles $(u,v] = (0,v] \setminus (0,u]$ avec $u,v \in \Qbb$ sont $\Fcal$-mesurables. Ils engendrent la tribu bor\'elienne sur $I$, donc $\Fcal$ co\"incide avec cette tribu.
            \end{enumerate}
        \end{enumerate}
        
    \item \label{Partie8}
        C'est vrai pour n'importe quel bor\'elien $J$.\\
        Observons tout d'abord que pour tous $x \in [0,1)$ et $k \in \Nbb_{\ell \ge 1}$
            $$T(\psi_k(x)) = T\tuple{\tfrac{1}{x + k}} = \set{x + k} = x.$$
        Il suit que $T^m \circ \psi_{a_1,\ldots,a_m} = \id_{[0,1)}$. Or, par l'injectivit\'e de $\psi_{a_1,\ldots,a_m}$
            \begin{align*}
                \allowdisplaybreaks
                \ell(T^{-m}(J) \cap I_{a_1,\ldots,a_m}) & = \int_{I_{a_1,\ldots,a_m}} \indi_{T^{-m}(J)}\,d\ell \\
                & = \int_I (\indi_{T^{-m}(J)} \circ \psi_{a_1,\ldots,a_m}) \,d(\psi_{a_1,\ldots,a_m})_\ast \ell \\
                & = \int_I \indi_{J} \abs{\psi'_{a_1,\ldots,a_m}} \,d\ell \\
                & = \int_0^1 \indi_{J}(x) \frac{|p_{m-1}q_m - p_mq_{m-1}|}{(q_m + xq_{m-1})^2}\,dx\\
                & = \int_0^1 \indi_{J}(x) \frac{dx}{(q_m + xq_{m-1})^2}.
            \end{align*}
        Pour tout $x \in I$, on a $(q_m + xq_{m-1})^2 \le (q_m + q_{m-1})^2 \le 2q_m(q_m + q_{m-1})$ (car $(q_m)_{m \ge 1}$ est croissante).
        De m\^eme
            \begin{align*}
                (q_m + xq_{m-1})^2 \ge q_m^2 \ge q_m \cdot \frac{q_m + q_{m-1}}{2}.
            \end{align*}
        Ainsi, par (\ref{Partie7b})
            $$\int_0^1  \frac{\indi_{J}(x) dx}{(q_m + xq_{m-1})^2} \ge \frac{1}{2q_m(q_m + q_{m-1})} \int_0^1 \indi_{J}(x)\,dx = \frac{1}{2}\ell(I_{a_1,\ldots,a_m})\ell(J)$$
        et
            $$\int_0^1  \frac{\indi_{J}(x)dx}{(q_m + xq_{m-1})^2} \le \frac{2}{q_m(q_m + q_{m-1})} \int_0^1 \indi_{J}(x)\,dx = 2\ell(I_{a_1,\ldots,a_m})\ell(J).$$
        On conclut que  
            $$\frac{1}{2}\ell(I_{a_1,\ldots,a_m}) \le \frac{\ell(T^{-m}(J) \cap I_{a_1,\ldots,a_m})}{\ell(J)} \le 2\ell(I_{a_1,\ldots,a_m}).$$

    \item \label{Partie9} Soit $J \subseteq I$ un bor\'elien tel que $T^{-1}(J) = J$. En particulier, pour tout intervalle $I_{a_1,\ldots,a_m}$, on a (de {\bf \ref{Partie8}.})
        $$\ell(J) \ell( I_{a_1,\ldots,a_m}) \le 2\ell(J \cap I_{a_1,\ldots,a_m}).$$
    On va montrer que soit $\ell(J) = 0$ soit $\ell(J^c) = 0$ (d'o\`u soit $\mu(J) = 0$ soit $\mu(J) = 1$ car $\mu \ll \ell$). Fixons $\varepsilon > 0$. Par la construction de $\ell$ comme mesure ext\'erieure, il existe un bor\'elien $J' \supseteq J^c$, qui est une r\'eunion disjointe d'intervalles de la forme $I_{a_1,\ldots,a_m}$, tel que $0 \le \ell(J' \setminus J^c) < \varepsilon$. Ainsi, $\ell(J)\ell(J^c) \le \ell(J)\ell(J') \le 2\ell(J \cap J') = 2\ell(J' \setminus J^c) < 2\varepsilon$. C'est vrai pour tout $\varepsilon$, donc $\ell(J)\ell(J^c) = 0$, d'o\`u le r\'esultat.
\end{enumerate}




\section*{Applications aux approximations diophantiennes}
\begin{enumerate}
    \setcounter{enumi}{9}
    \item \label{Partie10} Montrons par r\'ecurrence sur $m = 1,\ldots,n$ que
        $$\prod_{k = 1}^m [a_k(x),\ldots,a_n(x)] =  \frac{1}{q_{m-2}(x) + \dfrac{q_{m-1}(x)}{[a_m(x),\ldots,a_n(x)]}}.$$
    Quand $m = 1$, les deux c\^ot\'es sont \'egales car $q_{-1}(x) = 0$ et $q_0(x) = 1$. Soit $m \ge 2$. Par l'hypoth\`ese de r\'ecurrence, on a
        \begin{align*}
            \prod_{k = 1}^m [a_k(x),\ldots,a_n(x)] & = \frac{[a_m(x),\ldots,a_n(x)]}{q_{m-3}(x) + \dfrac{q_{m-2}(x)}{[a_{m-1}(x),\ldots,a_n(x)]}} \\
            & = \frac{[a_m(x),\ldots,a_n(x)]}{q_{m-3}(x) + (a_{m-1}(x) + [a_m(x),\ldots,a_n(x)])q_{m-2}(x)} \\
            & = \frac{[a_m(x),\ldots,a_n(x)]}{q_{m-1}(x) + [a_m(x),\ldots,a_n(x)]q_{m-2}(x)}\\
            & = \frac{1}{q_{m-2}(x) + \dfrac{q_{m-1}(x)}{[a_m(x),\ldots,a_n(x)]}}.
        \end{align*}
    D'o\`u l'affirmation. En particulier, quand $n = 1$
        $$\prod_{k = 1}^n [a_k(x),\ldots,a_n(x)] = \frac{1}{q_{n-2}(x) + a_n(x)q_{n-1}(x)} = \frac{1}{q_n(x)}.$$
        
    \item \label{Partie11}
        Pour tous $k > k' \in \Nbb_{\ge 1}$, on a 
            $$a_{k} = a \circ T^{k-1} = (a \circ T^{k-k'-1}) \circ T^{k'} = a_{k - k'} \circ T^\ell.$$
        Donc, de (\ref{Partie4b}), on a
            $$[a_k(x),\ldots,a_n(x)] = [a_1(T^{k-1}(x)),\ldots,a_{n-k+1}(T^{k-1}(x))] = \frac{p_{n-k+1}(T^{k-1}(x))}{q_{n-k+1}(T^{k-1}(x))}.$$
        En outre, il suit de (\ref{Partie4c}) que
            $$|\log T^{k-1}(x) - [a_k(x),\ldots,a_n(x)]| \le \frac{1}{2^{n-k+1}}.$$
        En sommant par rapport \`a $k = 1,\ldots,n$, on obtient
            $$\abs{\sum_{k=1}^n \log T^{k-1}(x) - \log\prod_{k=1}^n [a_k(x),\ldots,a_n(x)]} \le \sum_{k=1}^n \frac{1}{2^{n-k+1}}$$
        i.e. (par {\bf \ref{Partie10}.})
            $$\abs{\sum_{k=1}^n \log T^{k-1}(x) - \log \frac{1}{q_n(x)}} \le 1 - \frac{1}{2^{n}} < 1.$$
        Il suit que
            $$\abs{\frac{1}{n}\log \frac{1}{q_n(x)} - \frac{1}{n}\sum_{k=1}^n \log T^{k-1}(x)} = O\tuple{\frac{1}{n}}, \qquad n \to \infty.$$
    \item \label{Partie12}
        Puisque $\mu$ est ergodique pour $T$, il suit du th\'eor\`eme ergodique de Birkhoff et {\bf \ref{Partie11}.} que pour presque tout $x \in I \setminus \Qbb$ ($\mu(\Qbb) = 0$), 
            \begin{align*}
                \lim_{n \to \infty}\frac{1}{n}\log\frac{1}{q_n(x)} & = \lim_{n \to \infty}\frac{1}{n}\sum_{k=1}^n \log T^{k-1}(x) = \int_I \log d\mu = \frac{1}{\log 2}\int_0^1 \frac{\log x\, dx}{1+x}.
            \end{align*}
        Pour tout $x \in (0,1)$, on a
            $$\frac{\log x}{1 + x} = \sum_{k = 0}^{\infty}(-1)^k x^k\log x.$$
        On a, pour tout $k \in \Nbb$
            $$\int_0^1 x^k \log x\,dx = \left. \frac{x^{k+1}\log x}{k+1} \right|_0^1 - \int_0^1 \frac{x^{k+1}}{k+1}\cdot \frac{dx}{x} = -\int_0^1 \frac{x^k\,dx}{k+1} = -\frac{1}{(k+1)^2}.$$
        De plus
            $$\sum_{k = 0}^{\infty}\int_0^1 |x^k \log x|\,dx = \sum_{k = 0}^{\infty} \int_0^1(-x^k \log x)\,dx = \sum_{k=0}^{\infty} \frac{1}{(k+1)^2} < \infty.$$
        Par convergence domin\'ee, on a
            \begin{align*}
                \int_0^1\frac{\log x}{1 + x} dx & = \sum_{k=0}^{\infty}(-1)^k \cdot \oin{-\frac{1}{(k+1)^2}} \\
                & = -\sum_{k=1}^{\infty} \frac{(-1)^k}{k^2}\\
                & = -\sum_{k=1}^{\infty}\frac{1}{k^2} + 2\cdot \sum_{k=1}^{\infty} \frac{1}{(2k)^2}\\
                & = -\frac{\pi^2}{6} + 2\cdot \frac{1}{4} \cdot \frac{\pi^2}{6}\\
                & = -\frac{\pi^2}{12}.
            \end{align*}
        Donc $\lim\limits_{n \to \infty}\dfrac{1}{n}\log\dfrac{1}{q_n(x)} = -\dfrac{\pi^2}{12 \log 2}$ pour presque tout $x \in I \setminus \Qbb$. Finalement, pourt tel $x$, il suit de (\ref{Partie4c}) que
            $$\frac{1}{2q_n(x)q_{n+1}(x)} \le \abs{x - \frac{p_n(x)}{q_n(x)}} \le \frac{1}{q_n(x)q_{n+1}(x)}.$$
        Ainsi
           \begin{align*}
                \frac{1}{n}\oin{\log\frac{1}{q_n(x)} + \log\frac{1}{q_{n+1}(x)} - \log 2} & \le \frac{1}{n}\log\abs{x - \frac{p_n(x)}{q_n(x)}} \\
                & \le  \frac{1}{n}\oin{\log\frac{1}{q_n(x)} + \log\frac{1}{q_{n+1}(x)}}.
           \end{align*}
        Il suit que $\dfrac{1}{n}\log\abs{x - \dfrac{p_n(x)}{q_n(x)}} \to -\dfrac{\pi^2}{6\log 2}$ quand $n \to \infty$.
            
    \item \label{Partie13}
        \begin{enumerate}
            \item \label{Partie13a} Pour tous $x \in I$ et $n,k \in \Nbb_{\ge 1}$, On a \'equivalence
                $$a_n(x) = k \Leftrightarrow a(T^{n-1}(x)) = k \Leftrightarrow k \le \tfrac{1}{T^{n-1}(x)} < k+1 \Leftrightarrow T^{n-1}(x) \in \olin{\tfrac{1}{k+1},\tfrac{1}{k}}.$$
            Donc
                \begin{align*}
                    \mu\{x \in I : a_n(x) = k \} & = \mu\tuple{T^{n-1}\tuple{\olin{\tfrac{1}{k+1},\tfrac{1}{k}}}} \\
                    & = \mu\tuple{\olin{\tfrac{1}{k+1},\tfrac{1}{k}}} & (T \text{ est } \mu \text{-invariant})\\
                    & = \int_{\tfrac{1}{k+1}}^{\tfrac{1}{k}} \frac{dx}{1+x}\\
                    & = \log(1+x) \Big|^{1/k}_{1/(k+1)}\\
                    & = \log \tuple{1 + \tfrac{1}{k}} - \log\tuple{1 + \tfrac{1}{k+1}}.
                \end{align*}
            Il suit que
                 $$\mu\{x \in I : a_n(x) \ge k\} = \sum_{\ell = k}^{\infty} \tuple{\log \tuple{1 + \tfrac{1}{\ell}} - \log\tuple{1 + \tfrac{1}{\ell+1}}} = \log\tuple{1 + \tfrac{1}{k}} \le \tfrac{1}{k}.$$
            Pour une suite $\abf = (a_n)_{n \ge 1}$ de r\'eels strictement positifs, on note
                $$E_n(\abf):=\{x \in I : a_n(x) > a_n\} = \{x \in I: a_n(x) \ge \floor{a_n} + 1\}$$
            Alors
                $$A(\abf)^c = \bigcap_{n \ge 1} \bigcup_{m \ge n} E_m(\abf).$$
            On a
                $$\sum_{n \ge 1}\mu(E_n(\abf)) = \sum_{n \ge 1}\frac{1}{\floor{a_n} + 1} < \sum_{n \ge 1}\frac{1}{a_n} < \infty$$
            Donc $\mu(A(\abf)^c) = 0$ (par le lemme de Borel-Cantelli), i.e. $\mu(A(\abf)) = 1$.
            
            \item \label{Partie13b} On a
                $$A(\abf) = \bigcup_{n \ge 1}\bigcap_{m \ge n}E_m(\abf)^c.$$
            Il faut donc d\'emontrer que pour tout $n \ge 1$
                $$\mu\tuple{\bigcap_{m \ge n} E_m(\abf)^c} = 0.$$
            Commen\c cons par le cas o\`u $n = 1$. Comme $\mu \ll \ell$, il suffira de d\'emontrer que $\ell\tuple{\bigcap_{m \ge 1} E_m(\abf)^c} = 0$. Pour tout $m \in \Nbb_{\ge 1}$, on a
                \begin{align*}
                    \bigcap_{k = 1}^m E_k(\abf)^c & = \{x \in I: \forall k = 1,\ldots,m, \ a_k(x) \le \floor{a_k}\} \\
                    & = \bigsqcup_{b_1 \le \floor{a_1}, \ldots,b_m \le \floor{a_m}} I_{b_1,\ldots,b_m}.
                \end{align*}
            Pour tous $m, b_1,\ldots,b_{m+1} \in \Nbb_{\ge 1}$ et $x \in I_{b_1,\ldots,b_m}$, on a \'equivalence
                $$T^m(x) \in I_{b_m+1} \Leftrightarrow a(T^m(x)) = b_{m+1} \Leftrightarrow a_{m+1}(x) = b_{m+1} \Leftrightarrow x \in I_{b_1,\ldots,b_{m+1}}$$
            i.e. $T^{-m}(I_{b_{m+1}}) \cap I_{b_1,\ldots,b_m} = I_{b_1,\ldots,b_{m+1}}$. En appliquant {\bf \ref{Partie8}.}, on a
                $$\ell\left(E_{m+1}(\abf) \cap I_{b_1,\dots,b_m}\right) \leq 2 \ell(I_{b_1, \dots, b_m}) \ell\left(E_{m+1}(\abf)\right),$$
                de sorte que
                $$
                \ell(E_{m+1}(\abf)^c \cap I_{b_1, \dots, b_m}) \leq \ell(I_{b_1, \dots, b_m})\left(1 - 2\ell(E_{m+1}(\abf)\right) \leq \ell(I_{b_1, \dots, b_m}) \left(1 - \frac{C}{a_{m+1} +1}\right),
                $$
                o\`u $C$ est une constante qui ne d\'epend de rien puisque $\ell(E_{m+1}(\abf))$ est de l'ordre de $1/(a_{m+1} +1)$ (en effet $\mu(\{x \in I,~a_{m+1}(x) = k) = \mu(\{x \in I,~a_{1}(x) = k\}) =  1/k - 1/({k} + 1)$ par \textbf{\ref{Partie6}.}, donc $\ell(E_{m+1}(\abf)) \geq C' \mu(E_{m+1}(\abf)) \geq \displaystyle{\frac{C}{a_{m+1} +1}}$). Par suite,
                $$
                \begin{aligned}
                \ell\tuple{\bigcap_{k = 1}^{m+1} E_k(\abf)^c} &=\ell\left( \bigsqcup_{b_1 \leq a_1, \dots, b_m \leq a_m} I_{b_1, \dots, b_m} \cap E_{m+1}(\abf)^{c}\right) \\
                &\leq \left(1 - \frac{C}{a_{m+1} +1}\right) \sum_{b_1 \leq a_1, \dots, b_m \leq a_m} \ell(I_{b_1,\dots,b_m}) \\
                &= \left(1 - \frac{C}{a_{m+1} +1}\right) \ell\tuple{\bigcap_{k = 1}^{m} E_k(\abf)^c}.
                \end{aligned}
                $$
           On obtient pour tout $m \in \Nbb_{\geq 1}$
                $$\log \ell\tuple{\bigcap_{k = 1}^{m+1} E_k(\abf)^c} \le \log\left(1 - \frac{C}{a_{m+1} + 1}\right) + \log \ell\tuple{\bigcap_{k = 1}^m E_k(\abf)^c}.$$
            Par continuit\'e des mesures, on obtient
                $$\log\ell\tuple{\bigcap_{m \ge 1} E_m(\abf)^c} \le \sum_{m \ge 1} \log \tuple{1 - \frac{C}{a_{m+1} + 1}} + \log \ell(E_1(\abf)^c).$$
            La s\'erie $\sum_{m \ge 1} \frac{1}{a_m + 1}$ diverge (si elle converge, on aura $\frac{1}{a_m + 1} \to 0$, donc $a_m \to \infty$; en particulier, $\frac{1}{a_m} \le \frac{2}{a_{m} + 1}$ pour $m$ assez grand, qui implique que $\sum_{m \ge 1} \frac{1}{a_m}$ converge, c'est absurde). Il suit que
                $$\sum_{m \ge 1}\log \tuple{1 - \frac{C}{a_{m+1} + 1}} \le -\sum_{m \ge 1}\frac{1}{a_{m+1} + 1} = -\infty,$$
            En cons\'equence, $\ell\tuple{\bigcap_{m \ge 1} E_m(\abf)^c} = 0$, d'o\`u $\mu\tuple{\bigcap_{m \ge 1} E_m(\abf)^c} = 0$. \\
            Consid\'erons maintenant $n \ge 1$ quelconque. Soit  $\abf':=(a'_{k})_{k \ge 1}$, o\`u $a'_k = a_{k+n-1}$. Pour tout $x \in I$ et tout $m \ge n$, on a \'equivalence
                \begin{align*}
                    x \in E_m(\abf) & \Leftrightarrow a_m(x) > a_m \Leftrightarrow a(T^{m-1}(x)) > a_m \Leftrightarrow a(T^{m-n}(T^{n-1}(x)) > a_m\\
                    & \Leftrightarrow a_{m-n+1}(T^{n-1}(x)) > a'_{m-n+1} \Leftrightarrow T^{n-1}(x) \in E_{m-n+1}(\abf').
                \end{align*}
            Donc $E_m(\abf) = T^{-n+1}(E_{m-n+1}(\abf'))$. Par la $\mu$-invariance de $T$, on a
                $$\mu\tuple{\bigcap_{m \ge n} E_m(\abf)^c} = \mu\tuple{\bigcap_{m \ge n} E_{m-n+1}(\abf')^c} = \mu\tuple{\bigcap_{m \ge 1} E_{m}(\abf')^c} = 0.$$
            La derni\`ere \'egalit\'e vient du fait que la s\'erie $\sum_{m \ge 1}\frac{1}{a_m'} = \sum_{m \ge n}\frac{1}{a_m}$ diverge, et qu'on a trait\'e le cas o\`u $n = 1$.
                
        \end{enumerate}
        
    \item \label{Partie14}
        \begin{enumerate}
            \item \label{Partie14a}
                Pour presque tout $x \in I$, on a $\lim\limits_{n \to \infty} \frac{1}{n}\log q_n(x) = \frac{\pi^2}{12 \log 2} < \log 4$ (Partie {\bf \ref{Partie12}.}), donc il existe $N(x) \in \Nbb_{n \ge 1}$ tel que
                    $$\forall n \ge N(x), \qquad q_n(x) < 4^n.$$
                La suite $(qf(q)_{q}$ est d\'ecroissante, donc 
                    $$\varphi(n) = 4^nf(4^n) \le q_n(x)f(q_n(x))$$
                pour tout $n \ge N(x)$.
                
            \item \label{Partie14b} On a n\'ecessairement $f(q) > 0$ pour tout $q$ (s'il existe $q_0$ tel que $f(q_0) = 0$, comme $(qf(q))_{q}$ est d\'ecroissante, on aura $f(q) = 0$ pour tout $q \ge q_0$, qui contradit le fait que $\sum_q f(q)$ diverge). Montrons que la s\'erie $\sum_n \varphi(n)$ diverge. En effet, la suite $(f(q))_q$ est d\'ecroissante car $f(q+1) \le \frac{qf(q)}{q+1} < f(q)$, donc
                $$\sum_{q \ge 1} f(q) = \sum_{n \ge 0} \sum_{q = 4^n}^{4^{n+1} - 1} f(q) < \sum_{n \ge 0} 3 \cdot 4^n f(4^{n}) = 3\sum_{n \ge 0}\varphi(n).$$
            Il suit de {\bf \ref{Partie12}.} que pour presque tout $x$, $a_{n+1}(x) > \frac{1}{\varphi(n)}$ infiniment souvent. Pour tel $x$ et $n$, par (\ref{Partie4c})
                $$\abs{x - \frac{p_n(x)}{q_n(x)}} \le \frac{1}{q_n(x)q_{n+1}(x)} \le \frac{1}{a_{n+1}(x)q_n(x)^2} < \frac{\varphi(n)}{q_n(x)^2}.$$
            De (\ref{Partie14a}), on a
                $$\abs{x - \frac{p_n(x)}{q_n(x)}} < \frac{f(q_n(x))}{q_n(x)}$$
            infiniment souvent. Mais la suite $(q_n(x))_{n \ge 1}$ est strictement croissante, d'o\`u la preuve de la premi\`ere partie du {\bf Th\'eor\`eme}.
        \end{enumerate}
\end{enumerate}

\end{document}


