\documentclass[a4paper,12pt,openany]{article}
\usepackage{fancyhdr}
\usepackage[T1]{fontenc}
\usepackage[margin=1.8cm]{geometry}
\usepackage[applemac]{inputenc}
\usepackage{lmodern}
\usepackage{enumitem}
\usepackage{microtype}
\usepackage{hyperref}
\usepackage{enumitem}
\usepackage{dsfont}
\usepackage{amsmath,amssymb,amsthm}
\usepackage{mathenv}
\usepackage{mathrsfs}
\usepackage{amsthm}
\usepackage{graphicx}
\usepackage[all]{xy}
\usepackage{lipsum}       % for sample text
\usepackage{changepage}
\theoremstyle{plain}
\newtheorem{thm}{Theorem}[section]
\newtheorem*{thm*}{Th\'eor\`eme}
\newtheorem{prop}[thm]{Proposition}
\newtheorem{cor}[thm]{Corollary}
\newtheorem{lem}[thm]{Lemma}
\newtheorem{propr}[thm]{Propri\'et\'e}
\theoremstyle{definition}
\newtheorem{deff}[thm]{Definition}
\newtheorem{rqq}[thm]{Remark}
\newtheorem{ex}[thm]{Exercice}
\newcommand{\e}{\mathrm{e}}
\newcommand{\prodscal}[2]{\left\langle#1,#2\right\rangle}
\newcommand{\devp}[3]{\frac{\partial^{#1} #2}{\partial {#3}^{#1}}}
\newcommand{\w}{\omega}
\newcommand{\dd}{\mathrm{d}}
\newcommand{\x}{\times}
\newcommand{\ra}{\rightarrow}
\newcommand{\pa}{\partial}
\newcommand{\vol}{\operatorname{vol}}
\newcommand{\dive}{\operatorname{div}}
\newcommand{\T}{\mathbf{T}}
\newcommand{\R}{\mathbf{R}}
\newcommand{\Q}{\mathbf{Q}}
\newcommand{\Z}{\mathbf{Z}}
\newcommand{\N}{\mathbf{N}}
\newcommand{\C}{\mathbf{C}}
\newcommand{\F}{\mathcal{F}}
\newcommand{\Homeo}{\mathrm{Homeo}}
\renewcommand{\x}{\mathbf{x}}
\newcommand{\Matn}{\mathrm{Mat}_{n \times n}}
\DeclareMathOperator{\tr}{tr}
\newcommand{\id}{\mathrm{id}}
\newcommand{\htop}{h_\mathrm{top}}


\title{\textsc{Syst\`emes dynamiques} \\ Corrig\'e 11}
\date{}
\author{}

\begin{document}

{\noindent \'Ecole Normale Sup\'erieure  \hfill Yann Chaubet } \\
{2021/2022 \hfill \texttt{chaubet@dma.ens.fr}}

{\let\newpage\relax\maketitle}
\maketitle

\noindent {\large \textbf{Exercice 1.} \textit{D\'ecroissance surexponentielle des corr\'elations pour les applications dilatantes du cercle}} \vspace{1.5mm} 

\begin{enumerate}
\item Soient $\varphi, \psi \in L^2(\mu)$, et \'ecrivons
$$
\varphi = \sum_{k} c_k \e_k, \quad \psi = \sum_k d_k \e_k,
$$
o\`u $\e_k(\theta) = \exp(2i\pi k \theta)$ pour tous $\theta \in \R/\Z$ et $k \in \Z.$
 

Alors
$$
\psi(m^n\theta) = \sum_k d_k \exp(2i\pi k m^n\theta), \quad \theta \in \R/\Z.
$$

Ainsi 
$$
\begin{aligned}
\int \varphi (\psi \circ E_m^n) \dd \mu &= \sum_k c_{-km^n} d_k \\
&= c_0 d_0 + \sum_{k \neq 0} c_{km^n} d_k, 
\end{aligned}
$$
et on a 
$$
\left| \sum_{k \neq 0} c_{km^n} d_k\right| \leqslant \left(\sum_{|j|\geqslant m^n}|c_j|^2 \right) \left(\sum_{j\neq0} |d_j|^2\right) \underset{n \to \infty}{\longrightarrow} 0.
$$

\item 
\begin{enumerate}
\item On a 
$$
\begin{aligned}
\int_0^{2\pi} \psi(\theta) \e^{-2ik\pi \theta} \dd \theta  &= \left[\psi(\theta) \frac{\e^{-2i\pi k\theta}}{-2i\pi k}\right]_0^{2\pi} - \frac{1}{-2i\pi k} \int_{0}^{2\pi} \psi'(\theta) \e^{-2i\pi k \theta}\dd \theta  \\
&= \frac{1}{2i \pi k } \int_0^{2\pi} \psi'(\theta) \e^{-2i\pi k \theta} \dd \theta
\end{aligned}
$$

Une r\'ecurrence imm\'ediate donne pour tout $N \geqslant 0$
$$
\int_0^{2\pi} \psi(\theta)\e^{-2i \pi k \theta} \dd \theta = \frac{1}{(2i k \pi)^N} \int_0^{2\pi} \psi^{(N)}(\theta)\e^{-2ik\pi\theta}\dd \theta.
$$

En particulier
$$
\left|\int_0^{2\pi} \psi(\theta)\e^{-2i \pi k \theta} \dd \theta\right| \leqslant \frac{\left\|\psi^{(N)}\right\|_\infty}{(2\pi)^{N-1}} \frac{1}{|k|^N}.
$$

\item On a, avec les notations de la question \textbf{1.},
$$
\begin{aligned}
\sum_{j\neq0} |c_{-jm^n}|^2 &\leqslant C \sum_{j\neq 0} \frac{1}{(|j|m^n)^N}  \\
&\leqslant \frac{C}{m^{nN}} \sum_{j \neq 0} \frac{1}{|j|^N}  \\
&\leqslant \tilde C \e^{-rn},
\end{aligned}
$$

o\`u $r = N \log m.$
\end{enumerate}
\end{enumerate}

\vspace{0.6cm}

\noindent {\large \textbf{Exercice 2.} \textit{Presque tous les nombres r\'eels sont normaux}} \vspace{1.5mm} 

\noindent
Soit $m \geqslant 2$. On pose
$$
\chi_{k,m} = \mathbf{1}_{\left[\frac{k}{m}, \frac{k+1}{m}\right[}, \quad k = 0, \dots, m-1.
$$
Alors pour tout $x \in [0,1)$ on a 
$$
a_j(x) = k \iff \chi_{k,m}(\{m^jx\}) = 1, \quad \{m^jx\} = m^jx - [m^jx].
$$
L'application $E_m : \T \to \T$ est m\'elangeante, donc ergodique pour $\mu$.  Par le th\'eor\`eme ergodique, il existe $A_m \subset \T$ avec $\mathrm{\mu}(A_m) = 1$ tel que pour tout $x \in A_m$ on a
$$
\begin{aligned}
\frac{1}{n} \#\{k,~a_k(x) = j\} &= \frac{1}{n} \sum_{k=1}^n [\chi_{j,m} \circ (E_m)^k] (x)  \underset{k \to +\infty}{\longrightarrow} \int_\T \chi_{j,m} \dd \mu = \frac{1}{m}
\end{aligned}
$$
Ainsi si on pose $A = \bigcap_m A_m$, on a $\mathrm{Leb}(A) = 1$ et tout $x \in A$ est normal.
\vspace{0.6cm}


\noindent {\large \textbf{Exercice 3.} \textit{\'Equidistribution des rotations irrationnelles du cercle}} \vspace{1.5mm} 


\begin{enumerate}
\item Soit $f \in L^1(\mu)$ telle que $f \circ R_\alpha = f$ $\mu$-presque partout.  Notons $f(\theta) = \sum_{k} c_k \e^{2i\pi k \cdot \theta}$.  Alors l'invariance de $f$ donne, pour $\mu$-presque tout $\theta \in \T^d$,
$$
\sum_k c_k \e^{2i\pi k \cdot \theta} \e^{2i \pi k \cdot \alpha} = \sum_{k} c_k \e^{2i\pi k \cdot \theta}.
$$

Ceci implique que 
$$c_k \e^{2i\pi k \cdot \alpha} = c_k, \quad k \in \Z^d.$$
 Puisque $(1, \alpha_1, \dots, \alpha_d)$ est libre sur $\Q$, on obtient
$$
c_k = 0, \quad k \neq 0,
$$

c'est-\`a-dire que $f$ est $\mu$-presque partout \'egale \`a une constante.  La transformation $R_\alpha$ n'est pas m\'elangeante : si $C = [-\varepsilon, \varepsilon]^d$, on a 
$$R_\alpha^{-j}(C) = \prod_{\ell = 1}^d \bigl[-\varepsilon - j\alpha_\ell, \varepsilon-j\alpha_\ell\bigr] \quad \mod \Z^d.$$



Ainsi si $\varepsilon > 0$ est assez petit on a $R_{\alpha}^{-j}(C) \cap C = \emptyset$, ce qui conclut. 
\item C'est le th\'eor\`eme de Birkhoff appliqu\'e \`a $\textbf{1}_C \in L^1(\mu)$.
\item On applique l'\textbf{Exercice 2} du TD 10 : $R_\alpha$ est une isom\'etrie et $\mu(U) > 0$ pour tout ouvert non vide $U$, donc $S_n\varphi$ converge uniform\'ement vers une fonction continue $\psi$.  Comme $R_\alpha$ est ergodique pour $\mu$, alors $\psi$ est constante \'egale \`a $\int_{\T^d} \varphi \dd \mu.$

\item Soit $\varepsilon > 0$ et $C$ un produit d'intervalles. On se donne deux fonctions lisses $\chi_\varepsilon^\pm : \T^d \to \R$ telles que 
$$
\chi_\varepsilon^{-} \leqslant \textbf{1}_C \leqslant \chi_\varepsilon^+ \quad \text{et} \quad \int_{\T^d} |\chi_\varepsilon^\pm - \textbf{1}_C|\dd \mu < \varepsilon.
$$

Alors pour tout $x \in \T^d$ on a 
$$
S_n \chi_\varepsilon^{-}(x) \leqslant S_n \textbf{1}_C(x) \leqslant S_n \chi_\varepsilon(x),
$$
 et donc
$$
\int \chi_\varepsilon^- \dd \mu \leqslant \liminf_n S_n \textbf{1}_C(x) \leqslant \limsup_n S_n \textbf{1}_C(x) \leqslant \int \chi_\varepsilon^+ \dd \mu.
$$




On obtient donc
$$
\mu(C) - \varepsilon \leqslant \liminf_n S_n \textbf{1}_C(x) \leqslant \limsup_n S_n \textbf{1}_C(x) \leqslant \mu(C) + \varepsilon.
$$

Le nombre $\varepsilon$ \'etant arbitraire, on a $S_n \varphi(x)\to \mu(C)$ quand $n \to +\infty.$
\item Le premier chiffre de $2^n$ est $j \neq 0$ si et seulement si, pour un certain $k \in \N$, on a
$$
10^kj \leqslant 2^n < 10^k(j+1),
$$
 ce qui \'equivaut \`a 
$$
\log_{10}(j) + k \leqslant n \log_{10}(2) < \log_{10}(j+1) + k.
$$
 
Ainsi, en notant $\alpha = \log_{10}(2) \notin \Q$, on a que le premier chiffre de $2^n$ est $j$ si et seulement si
$$
R_\alpha^n(0) \in [\log_{10}(j), \log_{10}(j+1)[ \quad \mod \Z.
$$

La fr\'equence d'apparition asymptotique de $7$ est donc donn\'ee par
$$
\lim_n \frac{1}{n} \sum \textbf{1}_{[\log_{10}(j), \log_{10}(j+1)[}(R_\alpha^n(0)) = \log_{10}(8/7) \approx 5,8\%.
$$


\end{enumerate}

\vspace{0.6cm}

\noindent {\large \textbf{Exercice 4.} \textit{M\'elange pour une famille dense de $L^2$}} \vspace{1.5mm}

\noindent
On pose 
$$
\displaystyle{
E = \left\{ \sum_{|k|\leqslant K} c_k e_k,~c_k \in \C,~K\in \N\right\}.
}
$$
Alors $E$ est dense dans $L^2(\mu)$ ; soient $\varphi, \psi \in L^2(\mu)$ et $\varepsilon > 0$, et $\varphi', \psi' \in E$ tels que $\|\varphi - \varphi'\|_{L^2(\mu)} < \varepsilon$ et $\|\psi - \psi'\|_{L^2(\mu)} < \varepsilon$.  
Pour tous $\varphi_1, \varphi_2 \in L^2(\mu)$ on notera 
$$C_n(\varphi_1, \varphi_2) = \int (\varphi_1 \circ f^n) \varphi_2 \dd \mu.$$  
Alors 
$$
\begin{aligned}
\left|C_n(\varphi, \psi) - \mu(\varphi) \mu(\psi)\right| & \leqslant C_n(|\varphi|, |\psi - \psi'|) + C_n(|\varphi - \varphi'|, |\psi'|) \\
& \quad \quad + \left|C_n(\varphi', \psi') - \mu(\varphi') \mu(\psi')\right| + \left|\mu(\varphi') \mu(\psi' - \psi)\right| \\
& \quad \quad + \left|\mu(\varphi'-\varphi)\mu(\psi)\right|  \\
&\leqslant \|\varphi\|_{L^2(\mu)} \|\psi - \psi'\|_{L^2(\mu)} + \|\psi'\|_{L^2(\mu)} \|\varphi - \varphi'\|_{L^2(\mu)} \\
& \quad \quad + |C_n(\varphi', \psi') - \mu(\varphi') \mu(\psi')| \\
&\quad \quad + \varepsilon \left(\|\varphi'\|_{L^2(\mu)} + \|\psi\|_{L^2(\mu)}\right)  \\
& \leqslant C \varepsilon + \left|C_n(\varphi', \psi') - \mu(\varphi') \mu(\psi')\right|.
\end{aligned}
$$



On obtient donc, avec une constante $C$ d\'ependant uniquement de $\varphi, \psi$, que pour tout $n \gg 1$,
$$
|C_n(\varphi, \psi) - \mu(\varphi) \mu(\psi)| \leqslant (C + 1) \varepsilon
$$

Cela conclut.

\vspace{0.6cm}

\noindent {\large \textbf{Exercice 5.} \textit{Automorphismes ergodiques du tore}} \vspace{1.5mm} 

\noindent
\underline{$(ii) \implies (i)$} Vu en cours.  
\vspace{0.3cm}

\noindent
\underline{$(i) \implies (iii)$} On suppose $1 \in \mathrm{sp}(A^r)$. Alors il existe $k \in \Z^d$ tel que 
$$\left({A^{\top}}\right)^r(k) = k.$$ 
 On d\'efinit
$$
\varphi(\theta) = \sum_{j=0}^{r-1} \e^{2i\pi k \cdot A^j\theta}, \quad \theta \in (\R/\Z)^d.
$$
 Alors
$$
\begin{aligned}
\varphi(A\theta) &= \sum_{j=0}^{r-1}\e^{2i \pi k \cdot A^{j+1}\theta}  \\
&= \varphi(\theta),
\end{aligned}
$$
o\`u on a utilis\'e $k \cdot A^r \theta = \left(A^\top\right)^r k \cdot \theta = k \cdot \theta.$

\vspace{0.3cm}
\noindent
\underline{$(iii) \implies (ii)$} On suppose que $1 \notin \mathrm{sp}(A^r)$ pour tout $r$.   Soient $k, \ell \in \Z^d$ ; on a 
$$
\int (\e_k \circ A^n) \e_\ell \dd \mu = 0
$$
si $n$ est assez grand.  En effet on a
$$(\e_k \circ A^n)(\theta) = \exp(2i\pi k \cdot A^n\theta) = \e_{\left(A^\top\right)^{n} k}(\theta),$$
et le fait que $1 \notin \mathrm{sp}(A^r)$ pour tout $r$ implique que l'application
$$
\Z \ni n \longmapsto \left(A^\top\right)^n k
$$
est injective.  En particulier $\left(A^\top\right)^nk \neq \ell$ pour tout $|n|$ assez grand.  
On peut alors appliquer l'\textbf{Exercice 4} pour conclure, puisque $(\e_k)_{k \in \Z}$ est dense dans $L^2(\T^d).$

\vspace{0.6cm}

\noindent {\large \textbf{Exercice 6.} \textit{Moyenne temporelle des temps de retour}} \vspace{1.5mm} 


\begin{enumerate}
\item On applique le th\'eor\`eme de Kac :
$$
\int_A \tau \dd \mu = \mu(X) - \mu(A_0^*), \quad A_0^* = \bigcap_{n \geqslant 0} f^{-n}(\complement A).
$$
 Or $A_0^*$ est invariant par $f$ : en effet, on a
$$
f^{-1}(A_0^*) = \bigcap_{n \geqslant 1} f^{-n}(\complement A),
$$
 ce qui donne $A_0^* \subset f^{-1}(A_0^*)$. D'autre part on a
$$
f^{-1}(A_0^*) \setminus A_0^* = \{x \in A,~f^n(x) \notin A,~n \geqslant 1\}.
$$
 Par le th\'eor\`eme de R\'ecurrence de Poincar\'e, on a donc $\mu(f^{-1}(A_0^*) \setminus A_0^*) = 0$ et donc $A_0^*$ est invariant.  

Par ergodicit\'e de $f$, on obtient $\mu(A_0^*) = 0$ ou $1$.  

Mais $\mu(A_0^*) = 1$ implique en particulier que $\mu(\complement A) = 1$ ce qui est impossible car $\mu(A) > 0.$  

On a bien $\int_A \tau \dd \mu = \mu(X) = 1.$
\item Il s'agit de montrer que $g$ est ergodique pour $\mu_A = \mu(A \cap \cdot) / \mu(A).$  

Soit $B \subset A$ un ensemble $g$-invariant de mesure non nulle. On note $\tau' : B \to \N_{\geqslant 1}$ le temps de retour associ\'e \`a $B$, qui est d\'efini $\mu$-presque partout sur $B$, et $g'$ l'application de premier retour.  

Puisque $B$ est $g$-invariant, on a $g(x) \in B$ pour presque tout $x \in B$, ce qui donne 
$$
g|_B = g' \quad \quad \mu-\text{presque partout sur }B.
$$ 

En utilisant $\tau' \geqslant \tau$, on obtient que
$$
\tau|_B = \tau' \quad \quad \mu-\text{presque partout sur }B.
$$
 
Par la question \textbf{1.}, on a donc 
$$
1 = \int_B \tau' \dd \mu = \int_B \tau \dd \mu = \underset{=1}{\underbrace{\int_A \tau \dd \mu}} - \int_{A \setminus B} \tau \dd \mu,
$$
 ce qui donne 
$$
0 = \int_{A\setminus B}\tau \dd \mu \geqslant \mu(A \setminus B) \quad \implies \quad \mu(B) = \mu(A).
$$



Ainsi $g$ est ergodique pour $\mu_A$, et donc, pour $\mu$-presque tout $x$ de $A$,
$$
\lim \frac{1}{n} \sum_{k=1}^n \tau\left(g^k(x)\right) = \frac{1}{\mu(A)}.
$$

\end{enumerate}

\noindent {\large \textbf{Exercice 7.} \textit{Un crit\`ere pour le m\'elange faible}} \vspace{1.5mm}


\begin{enumerate}
\item Supposons que $\lim_n \frac{1}{n}\sum_{0}^{n-1} |a_k| = 0.$ Pour tout $J \subset \N$ et $n \in \N$ on note
$$
\alpha_J(n) = \# \bigl(\{0, \dots, n-1\} \cap J \bigr).
$$
 Pour tout $j \geqslant 1$, l'ensemble
$$
I_j = \left\{n \geqslant 0,~|a_n| \geqslant \frac{1}{j}\right\}
$$
est de densit\'e $0$.  En effet, on a
$$
\frac{1}{n}\sum_{k=0}^{n-1} |a_k| \geqslant  \frac{1}{j} \frac{\alpha_{I_j}(n)}{n}.
$$

Soit $n_j > 0$ tel que pour tout $n \geqslant n_j$ on a 
$$
\frac{\alpha_{I_j}(n)}{n} \leqslant \frac{1}{j}.
$$

On peut supposer $(n_j)$ strictement croissante.



On pose
$$
E = \bigcup_{j \geqslant 1} \bigl(I_{j} \cap [n_j, n_{j+1}[\bigr).
$$
 C'est un ensemble de densit\'e $0$ car si $n \in [n_j, n_{j+1}[$ on a $E \cap [0, n[ \subset I_{j} \cap [0,n[$ et donc 
$$
\frac{\alpha_{E}(n)}{n} \leqslant \frac{\alpha_{I_j}(n)}{n} \leqslant \frac{1}{j}.
$$
 D'autre part, si $n \notin E$ avec $n \in [n_j, n_{j+1}[$, on a 
$\displaystyle{|a_n|< \frac{1}{j}}$,  et donc 
\begin{equation}\label{eq:mean}
\lim_{\substack{n \to +\infty \\n \notin E}} |a_n| = 0.
\end{equation}

R\'eciproquement on suppose que (\ref{eq:mean}) est satisfaite.  Soit $\varepsilon > 0$. Il existe $n_0 \geqslant$ tel que 
$$\alpha_E(n) \leqslant \varepsilon n, \quad  n \geqslant n_0,$$

et 
$$
\quad |a_n| \leqslant \varepsilon, \quad n \geqslant n_0, \quad n \notin E.
$$



On obtient donc, si $K = \sup_n |a_n|$,
$$
\begin{aligned}
\frac{1}{n} \sum_{k=0}^{n-1} |a_k| &= \frac{1}{n} \left( \sum_{\substack{k < n \\ k \notin E}} |a_k| + \sum_{\substack{k < n_0 \\ k \notin E}} |a_k| + \sum_{\substack{k > n_0 \\ k \notin E}} |a_k| \right)  \\
&\leqslant \frac{K \alpha_{E}(n)}{n} + \frac{Kn_0}{n} + \varepsilon,
\end{aligned}
$$
 et donc pour tout $n$ assez grand,
$$
\frac{1}{n} \sum_{k=0}^{n-1} |a_k| < (2K + 1) \varepsilon,
$$
ce qui conclut.
\item C'est direct par la question pr\'ec\'edente.

\end{enumerate}




\end{document}
 
 \noindent {\large \textbf{Exercice 1.} \textit{M\'elange et ergodicit\'e}} \vspace{1.5mm} 

\noindent Soit $(X, \mathscr{A}, \mu)$ un espace probabilis\'e et $f : X \to X$ une transformation mesurable. 
\begin{enumerate}
\item On suppose que $f$ est m\'elangeante pour $\mu$, i.e. pour tous $A,B \subset X$ mesurables,
$$
\lim _{n \rightarrow \infty} \mu(A \cap f^{-n}(B)) = \mu(A) \mu(B).
$$
Montrer que $f$ est ergodique pour $\mu$.
\item On suppose que $f$ est ergodique pour $\mu$. Montrer que 
$$
\lim _{n \rightarrow \infty} \frac{1}{n} \sum_{k=0}^{n-1} \mu\left(A \cap f^{-k}(B)\right)=\mu(A) \mu(B)
$$
\end{enumerate}
\vspace{0.6cm}


\noindent {\large \textbf{Exercice 2.} \textit{\'Equidistribution des rotations irrationnelles du cercle}} \vspace{1.5mm} 

\noindent Soit $\alpha \in \T^d = \R^d/ \Z^d$ et $R_\alpha : \T^d \to \T^d$ donn\'ee par $R_\alpha(x) = x + \alpha.$
\begin{enumerate}
\item Montrer que $R_\alpha$ pr\'eserve la mesure de Haar $\mu$ sur $\T^d$.
\item Soit $\varphi \in L^2(\T^d)$. Montrer qu'il existe $\bar{\varphi} \in L^2(\T^d)$ telle que 
$$
\frac{1}{n} \sum_{k=0}^{n-1} \varphi \circ R_\alpha^k \to \bar \varphi,
$$
la convergence \'etant presque sžre, mais aussi ayant lieu dans $L^2(\T^d)$.
\end{enumerate}
On suppose que la famille $(1, \alpha_1, \dots, \alpha_d)$ est lin\'eairement ind\'ependante sur $\Q$, o\`u $\alpha = (\alpha_1, \dots, \alpha_d).$
\begin{enumerate}[resume]
\item Montrer que $\bar{\varphi}$ est presque partout \'egale \`a $\displaystyle{\int_{\T^d} \varphi ~\dd \mu}$.
\item Soit $C \subset \T^d$ un produit d'intervalles. Montrer que pour $\mu$ presque tout $x \in \T^d$, 
$$
\frac{1}{n} \#\Bigl\{k \in \{1, \dots, n\},~x + k\alpha \in C \Bigr\} \underset{n \to +\infty}{\longrightarrow} \mu(C).
$$
\item Montrer que la propri\'et\'e pr\'ec\'edente est vraie pour tout $x \in \T^d$. \\
\textit{Indication : on pourra approximer la fonction indicatrice de $C$ par des fonctions continues et utiliser l'exercice pr\'ec\'edent.}
\item On consid\`ere la suite des premiers chiffres des puissances de $2$ : $1, 2, 4, 8, 1 \dots$ Montrer que la fr\'equence d'apparition du chiffre $7$ dans cette suite est environ \'egale \`a $5.8\%$.
\end{enumerate}

\vspace{0.6cm}


 
 
 
\noindent {\large \textbf{Exercice 2.} \textit{Ergodicit\'e du d\'ecalage}} \vspace{1.5mm} 

\noindent Soit $q \geq 1$ et $X = \{0, \dots, q-1\}^\N$, muni de la tribu produit et de la probabilit\'e $\mu$ induite par la probabilit\'e uniforme sur $\{0, \dots, q-1\}$. On note $\sigma : X \to X$ le d\'ecalage. 

\begin{enumerate}
\item Montrer que $\sigma$ est m\'elangeante.
\end{enumerate}
On dira qu'un r\'eel $x \in [0,1]$ est normal si, pour tout $q \geq 1$ et tout $j \in \{0, \dots, q-1\}$,
$$
\lim_{n \to + \infty} \frac{1}{n}\# \Bigl \{k \in \{1, \dots, n\},~a_k = j \Bigr\} = \frac{1}{q},
$$
o\`u $x = 0,a_{1}a_{2}\dots$ est le d\'eveloppement en base $q$ de $x$.
\begin{enumerate}[resume]
\item Montrer que presque tout $x$ de $[0,1]$ est normal.
\end{enumerate}

