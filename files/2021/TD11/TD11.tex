\documentclass[a4paper,10pt,openany]{article}
\usepackage{fancyhdr}
\usepackage[T1]{fontenc}
\usepackage[margin=1.8cm]{geometry}
\usepackage[applemac]{inputenc}
\usepackage{lmodern}
\usepackage{enumitem}
\usepackage{microtype}
\usepackage{hyperref}
\usepackage{enumitem}
\usepackage{dsfont}
\usepackage{amsmath,amssymb,amsthm}
\usepackage{mathenv}
\usepackage{mathrsfs}
\usepackage{amsthm}
\usepackage{graphicx}
\usepackage[all]{xy}
\usepackage{lipsum}       % for sample text
\usepackage{changepage}
\theoremstyle{plain}
\newtheorem{thm}{Theorem}[section]
\newtheorem*{thm*}{Th\'eor\`eme}
\newtheorem{prop}[thm]{Proposition}
\newtheorem{cor}[thm]{Corollary}
\newtheorem{lem}[thm]{Lemma}
\newtheorem{propr}[thm]{Propri\'et\'e}
\theoremstyle{definition}
\newtheorem{deff}[thm]{Definition}
\newtheorem{rqq}[thm]{Remark}
\newtheorem{ex}[thm]{Exercice}
\newcommand{\e}{\mathrm{e}}
\newcommand{\prodscal}[2]{\left\langle#1,#2\right\rangle}
\newcommand{\devp}[3]{\frac{\partial^{#1} #2}{\partial {#3}^{#1}}}
\newcommand{\w}{\omega}
\newcommand{\dd}{\mathrm{d}}
\newcommand{\x}{\times}
\newcommand{\ra}{\rightarrow}
\newcommand{\pa}{\partial}
\newcommand{\vol}{\operatorname{vol}}
\newcommand{\dive}{\operatorname{div}}
\newcommand{\T}{\mathbf{T}}
\newcommand{\R}{\mathbf{R}}
\newcommand{\Q}{\mathbf{Q}}
\newcommand{\Z}{\mathbf{Z}}
\newcommand{\N}{\mathbf{N}}
\newcommand{\C}{\mathbf{C}}
\newcommand{\F}{\mathcal{F}}
\newcommand{\Homeo}{\mathrm{Homeo}}
\renewcommand{\x}{\mathbf{x}}
\newcommand{\Matn}{\mathrm{Mat}_{n \times n}}
\DeclareMathOperator{\tr}{tr}
\newcommand{\id}{\mathrm{id}}
\newcommand{\htop}{h_\mathrm{top}}


\title{\textsc{Syst\`emes dynamiques} \\ Feuille d'exercices 11}
\date{}
\author{}

\begin{document}

{\noindent \'Ecole Normale Sup\'erieure  \hfill Yann Chaubet } \\
{2021/2022 \hfill \texttt{chaubet@dma.ens.fr}}

{\let\newpage\relax\maketitle}
\maketitle

\iffalse
\noindent {\large \textbf{Exercice 1.} \textit{M\'elange des d\'ecalages de Bernouilli}} \vspace{1.5mm} 

\noindent Soit $\Sigma_\ell = \{1, \dots, \ell\}$ o\`u $\ell \in \N_{\geq 1}$ et $\sigma$ le d\'ecalage sur $X = (\Sigma_\ell)^\Z$. On se donne $p_1, \dots, p_\ell \in [0,1]$ tels que $p_1 + \dots + p_\ell = 1$ et on consid\`ere l'unique probabilit\'e $\mu$ sur $(X, \mathcal{P}(\Sigma_\ell)^{\otimes \Z})$ telle que 
$$
\mu \Bigl(\bigl\{x = (x_n)_{n \in \Z} \in X,~x_j = k_j,~|j| \leq N \bigr\} \Bigr) = \prod_{j=-n}^n p_{k_j}, \quad N \in \N, \quad k_j \in \Sigma_\ell.
$$
Montrer que $\sigma$ est m\'elangeante sur $(X,\mu)$.


\vspace{0.6cm}
\fi


\noindent {\large \textbf{Exercice 1.} \textit{D\'ecroissance surexponentielle des corr\'elations pour les applications dilatantes du cercle}} \vspace{1.5mm} 

\noindent Pour tout $m \in \N_{\geq 2}$ on note $E_m$ la multiplication par $m$ sur $\R/\Z$. 

\begin{enumerate}
\item Montrer que $E_m$ est fortement m\'elangeante pour la mesure de Haar $\mu$ sur $\R/\Z$.
\item Soient $\varphi, \psi \in \mathcal{C}^{\infty}(\R/\Z)$. 
\begin{enumerate}
\item Montrer que pour tout $N > 0$, il existe $C > 0$ telle que
$$
\left| \int_{\R/\Z} \psi(\theta) \e^{-2i\pi k \theta} \dd \mu (\theta)\right| \leq \frac{C}{|k|^{N}}, \quad k \in \Z \setminus \{0\}.
$$
\item En d\'eduire que pour tout $r> 0$, il existe $C > 0$ telle que 
$$
\left|\int_{\R/\Z} \varphi \left(\psi \circ E_m^j\right) \dd \mu - \int_{\R/\Z} \varphi \dd \mu \int_{\R/\Z} \psi \dd \mu \right| \leq C \e^{-rj}, \quad j \in \N.
$$
\end{enumerate}
\end{enumerate}

\vspace{0.6cm}

\noindent {\large \textbf{Exercice 2.} \textit{Presque tous les nombres r\'eels sont normaux}} \vspace{1.5mm} 

\noindent On dit qu'un nombre $x \in [0,1)$ est normal si pour tout $m \geq 2$, son d\'eveloppement en base $m$
$$
x = 0,a_1a_2\dots,
$$
(qui est unique si on demande que pour tout $k$, il existe $k' \geq k$ tel que $a_{k'} \neq m-1$) satisfait
$$
\frac{1}{n} \#\Bigl\{k \in \{1, \dots, n\},~a_k = j\Bigr\} \underset{n \to +\infty}{\longrightarrow} \frac{1}{m}, \quad j = 0, \dots, m-1.
$$
Montrer que presque tout $x \in [0,1)$ est normal pour la mesure de Lebesgue.

\vspace{0.6cm}


\noindent {\large \textbf{Exercice 3.} \textit{\'Equidistribution des rotations irrationnelles du cercle}} \vspace{1.5mm} 

\noindent Soit $\alpha \in \T^d = \R^d/ \Z^d$ et $R_\alpha : \T^d \to \T^d$ donn\'ee par $R_\alpha(x) = x + \alpha.$ On suppose que la famille $(1, \alpha_1, \dots, \alpha_d)$ est lin\'eairement ind\'ependante sur $\Q$, o\`u $\alpha = (\alpha_1, \dots, \alpha_d).$
\begin{enumerate}
\item Montrer que $R_\alpha$ est ergodique pour la mesure de Haar sur $\T^d$. Est-elle m\'elangeante ?
\item Soit $C \subset \T^d$ un produit d'intervalles. Montrer que pour $\mu$ presque tout $x \in \T^d$, 
\begin{equation}\label{eq:equirepartition}
\frac{1}{n} \#\Bigl\{k \in \{1, \dots, n\},~x + k\alpha \in C \Bigr\} \underset{n \to +\infty}{\longrightarrow} \mu(C).
\end{equation}
\item Montrer que pour toute fonction $\varphi \in \mathcal{C}(\T^d, \C)$, la suite de fonctions $S_n \varphi = \displaystyle{\frac{1}{n}\sum_{k=0}^{n-1}} \varphi \circ R_\alpha^k$ converge uniform\'ement vers $\int_{\T^d} \varphi ~\dd \mu.$
\item Montrer que la convergence (\ref{eq:equirepartition}) a lieu pour tout $x \in \T^d$.
\item On consid\`ere la suite des premiers chiffres des puissances de $2$ : $1, 2, 4, 8, 1 \dots$ Montrer que la fr\'equence d'apparition du chiffre $7$ dans cette suite est \`a peu pr\`es $5.8\%$.
\end{enumerate}

\vspace{0.6cm}

\noindent {\large \textbf{Exercice 4.} \textit{M\'elange pour une famille dense de $L^2$}} \vspace{1.5mm}

\noindent Soit $(X, \mu)$ un espace de probabilit\'es et $f : X \to X$ une transformation mesurable. On suppose qu'il existe une base $(e_k)_{k \in \Z}$ de $L^2(X,\mu)$ telle que
$$
\lim_n \int_X e_k \left(e_\ell \circ f^n\right) \dd \mu = \int_X e_k\dd \mu \int_X e_\ell \dd \mu, \quad k, \ell \in \Z.
$$
Montrer que $f$ est fortement m\'elangeante.
\vspace{0.6cm}

\noindent {\large \textbf{Exercice 5.} \textit{Automorphismes ergodiques du tore}} \vspace{1.5mm} 

\noindent Soit $A \in \mathrm{GL}(d,\Z)$ et $f_A : \T^d \to \T^d$ l'automorphisme de $\T^d$ associ\'e. Montrer que les conditions suivantes sont \'equivalentes.
\begin{enumerate}[label = \textit{(\roman*)}]
\item La transformation $f_A$ est ergodique pour la mesure de Haar.
\item La transformation $f_A$ est m\'elangeante pour la mesure de Haar.
\item Aucune valeur propre de $A$ n'est racine de l'unit\'e.
\end{enumerate}

\vspace{0.6cm}

\noindent {\large \textbf{Exercice 6.} \textit{Moyenne temporelle des temps de retour}} \vspace{1.5mm} 

\noindent Soit $(X, \mu)$ un espace de probabilit\'es et $f : X \to X$ une transformation ergodique pour $\mu$. Soit $A \subset X$ un ensemble mesurable de mesure non nulle, $\tau : A \to \N_{\geq 1} \cup \{+\infty\}$ le temps de premier retour dans $A$, et $g : x \mapsto f^{\tau(x)}(x)$ l'application de premier retour associ\'ee (d\'efinie presque partout sur $A$). 
\begin{enumerate}
\item Montrer que $\displaystyle{\int_A \tau \dd \mu = 1}.$
\item En d\'eduire que pour $\mu$ presque tout $x$ de $A$,
$$
\lim_{n \to + \infty}\frac{1}{n} \sum_{k=0}^{n-1} \tau\left(g^k(x)\right) = \frac{1}{\mu(A)}.
$$
\end{enumerate}

\noindent {\large \textbf{Exercice 7.} \textit{Un crit\`ere pour le m\'elange faible}} \vspace{1.5mm}

\noindent On dira qu'un ensemble $E \subset \N$ est de densit\'e nulle si 
$$
\lim_{n} \frac{1}{n} \# \Bigl(\bigl\{1, \dots, n\}\cap E\Bigr) = 0.
$$
On dira qu'une suite strictement croissante d'entiers $(n_j)$ est de densit\'e $1$ si $\complement \{n_j,~j \in \N\}$ est de densit\'e nulle.
\begin{enumerate}
\item Soit $(a_n)$ une suite de nombre complexes born\'es. Montrer que 
$\displaystyle{
\lim_n \frac{1}{n} \sum_{k=0}^{n-1} |a_n| = 0
}
$
si et seulement si, il existe un ensemble de densit\'e nulle $E$ tel que 
$\displaystyle{\lim_{\substack{n \to +\infty \\ n\notin E}}} a_n = 0.$
\end{enumerate}
Soit $f$ une transformation mesurable d'un espace probabilis\'e $(X,\mathscr{A}, \mu)$. On suppose que $(\mathscr{A}, \mu)$ admet une base d\'enombrable. On dit que $f$ est faiblement m\'elangeante si pour tous $A, B \in \mathscr{A}$, on a 
$$
\lim_n \frac{1}{n} \sum_{k=0}^{n-1} \left|\mu(f^{-n}(A) \cap B) - \mu(A) \mu(B)\right| = 0.
$$
\begin{enumerate}[resume]
\item Montrer que $f$ est faiblement m\'elangeante si et seulement si il existe une suite d'entiers $(n_j)$ de densit\'e $1$ telle que
$$
\lim_j \mu\bigl(f^{-n_j}(A) \cap B\bigr) = \mu(A)\mu(B), \quad A,B \in \mathscr{A},
$$
ou encore si et seulement si
$$
\lim_n \frac{1}{n}\sum_{k=0}^{n-1}\Bigl(\mu(f^{-k}(A)\cap B)-\mu(A)\mu(B)\Bigr)^2  =0, \quad A,B \in \mathscr{A}.
$$
\end{enumerate}




\end{document}
 
 \noindent {\large \textbf{Exercice 1.} \textit{M\'elange et ergodicit\'e}} \vspace{1.5mm} 

\noindent Soit $(X, \mathscr{A}, \mu)$ un espace probabilis\'e et $f : X \to X$ une transformation mesurable. 
\begin{enumerate}
\item On suppose que $f$ est m\'elangeante pour $\mu$, i.e. pour tous $A,B \subset X$ mesurables,
$$
\lim _{n \rightarrow \infty} \mu(A \cap f^{-n}(B)) = \mu(A) \mu(B).
$$
Montrer que $f$ est ergodique pour $\mu$.
\item On suppose que $f$ est ergodique pour $\mu$. Montrer que 
$$
\lim _{n \rightarrow \infty} \frac{1}{n} \sum_{k=0}^{n-1} \mu\left(A \cap f^{-k}(B)\right)=\mu(A) \mu(B)
$$
\end{enumerate}
\vspace{0.6cm}


\noindent {\large \textbf{Exercice 2.} \textit{\'Equidistribution des rotations irrationnelles du cercle}} \vspace{1.5mm} 

\noindent Soit $\alpha \in \T^d = \R^d/ \Z^d$ et $R_\alpha : \T^d \to \T^d$ donn\'ee par $R_\alpha(x) = x + \alpha.$
\begin{enumerate}
\item Montrer que $R_\alpha$ pr\'eserve la mesure de Haar $\mu$ sur $\T^d$.
\item Soit $\varphi \in L^2(\T^d)$. Montrer qu'il existe $\bar{\varphi} \in L^2(\T^d)$ telle que 
$$
\frac{1}{n} \sum_{k=0}^{n-1} \varphi \circ R_\alpha^k \to \bar \varphi,
$$
la convergence \'etant presque sžre, mais aussi ayant lieu dans $L^2(\T^d)$.
\end{enumerate}
On suppose que la famille $(1, \alpha_1, \dots, \alpha_d)$ est lin\'eairement ind\'ependante sur $\Q$, o\`u $\alpha = (\alpha_1, \dots, \alpha_d).$
\begin{enumerate}[resume]
\item Montrer que $\bar{\varphi}$ est presque partout \'egale \`a $\displaystyle{\int_{\T^d} \varphi ~\dd \mu}$.
\item Soit $C \subset \T^d$ un produit d'intervalles. Montrer que pour $\mu$ presque tout $x \in \T^d$, 
$$
\frac{1}{n} \#\Bigl\{k \in \{1, \dots, n\},~x + k\alpha \in C \Bigr\} \underset{n \to +\infty}{\longrightarrow} \mu(C).
$$
\item Montrer que la propri\'et\'e pr\'ec\'edente est vraie pour tout $x \in \T^d$. \\
\textit{Indication : on pourra approximer la fonction indicatrice de $C$ par des fonctions continues et utiliser l'exercice pr\'ec\'edent.}
\item On consid\`ere la suite des premiers chiffres des puissances de $2$ : $1, 2, 4, 8, 1 \dots$ Montrer que la fr\'equence d'apparition du chiffre $7$ dans cette suite est environ \'egale \`a $5.8\%$.
\end{enumerate}

\vspace{0.6cm}


 
 
 
\noindent {\large \textbf{Exercice 2.} \textit{Ergodicit\'e du d\'ecalage}} \vspace{1.5mm} 

\noindent Soit $q \geq 1$ et $X = \{0, \dots, q-1\}^\N$, muni de la tribu produit et de la probabilit\'e $\mu$ induite par la probabilit\'e uniforme sur $\{0, \dots, q-1\}$. On note $\sigma : X \to X$ le d\'ecalage. 

\begin{enumerate}
\item Montrer que $\sigma$ est m\'elangeante.
\end{enumerate}
On dira qu'un r\'eel $x \in [0,1]$ est normal si, pour tout $q \geq 1$ et tout $j \in \{0, \dots, q-1\}$,
$$
\lim_{n \to + \infty} \frac{1}{n}\# \Bigl \{k \in \{1, \dots, n\},~a_k = j \Bigr\} = \frac{1}{q},
$$
o\`u $x = 0,a_{1}a_{2}\dots$ est le d\'eveloppement en base $q$ de $x$.
\begin{enumerate}[resume]
\item Montrer que presque tout $x$ de $[0,1]$ est normal.
\end{enumerate}

