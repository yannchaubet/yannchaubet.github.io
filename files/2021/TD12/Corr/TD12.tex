\documentclass[a4paper,12pt,openany]{article}
\usepackage{fancyhdr}
\usepackage[T1]{fontenc}
\usepackage[margin=1.8cm]{geometry}
\usepackage[applemac]{inputenc}
\usepackage{lmodern}
\usepackage{enumitem}
\usepackage{microtype}
\usepackage{hyperref}
\usepackage{enumitem}
\usepackage{dsfont}
\usepackage{amsmath,amssymb,amsthm}
\usepackage{mathenv}
\usepackage{mathrsfs}
\usepackage{amsthm}
\usepackage{graphicx}
\usepackage[all]{xy}
\usepackage{lipsum}       % for sample text
\usepackage{changepage}
\theoremstyle{plain}
\newtheorem{thm}{Theorem}[section]
\newtheorem*{thm*}{Th\'eor\`eme}
\newtheorem{prop}[thm]{Proposition}
\newtheorem{cor}[thm]{Corollary}
\newtheorem{lem}{Lemme}
\newtheorem{propr}[thm]{Propri\'et\'e}
\theoremstyle{definition}
\newtheorem{deff}[thm]{Definition}
\newtheorem{rqq}[thm]{Remark}
\newtheorem{ex}[thm]{Exercice}
\newcommand{\e}{\mathrm{e}}
\newcommand{\prodscal}[2]{\left\langle#1,#2\right\rangle}
\newcommand{\devp}[3]{\frac{\partial^{#1} #2}{\partial {#3}^{#1}}}
\newcommand{\w}{\omega}
\newcommand{\dd}{\mathrm{d}}
\newcommand{\x}{\times}
\newcommand{\ra}{\rightarrow}
\newcommand{\pa}{\partial}
\newcommand{\vol}{\operatorname{vol}}
\newcommand{\dive}{\operatorname{div}}
\newcommand{\T}{\mathbf{T}}
\newcommand{\R}{\mathbf{R}}
\newcommand{\Q}{\mathbf{Q}}
\newcommand{\Z}{\mathbf{Z}}
\newcommand{\N}{\mathbf{N}}
\newcommand{\C}{\mathbf{C}}
\newcommand{\Pcal}{\mathcal{P}}
\newcommand{\Fcal}{\mathcal{F}}
\newcommand{\Qcal}{\mathcal{Q}}
\newcommand{\Acal}{\mathcal{A}}
\newcommand{\Ccal}{\mathcal{C}}
\newcommand{\Bcal}{\mathcal{C}}
\newcommand{\Rcal}{\mathcal{R}}
\newcommand{\F}{\mathcal{B}}
\newcommand{\diam}{\mathrm{diam}}
\newcommand{\dist}{\mathrm{dist}}
\newcommand{\Homeo}{\mathrm{Homeo}}
\renewcommand{\x}{\mathbf{x}}
\newcommand{\Matn}{\mathrm{Mat}_{n \times n}}
\DeclareMathOperator{\tr}{tr}
\newcommand{\id}{\mathrm{id}}
\newcommand{\htop}{h_\mathrm{top}}


\title{\textsc{Syst\`emes dynamiques} \\ Feuille d'exercices 12}
\date{}
\author{}

\begin{document}

{\noindent \'Ecole Normale Sup\'erieure  \hfill Yann Chaubet } \\
{2021/2022 \hfill \texttt{chaubet@dma.ens.fr}}

{\let\newpage\relax\maketitle}
\maketitle


Dans la suite, si $(X, \Fcal, \mu)$ est un espace mesur\'e, on fera l'identification 
$$
\textrm{alg\`ebres finies de }\Fcal \quad \longleftrightarrow \quad \textrm{partitions finies } \Fcal-\text{mesurables de } X,
$$
en identifiant une alg\`ebre avec l'ensemble de ses atomes. On notera alors, si $\Pcal$ et $\Qcal$ sont deux partitions finies,
$$
\Pcal \vee \Qcal = \{P \cap Q,~P \in \Pcal,~Q\in \Qcal\},
$$
et cette notation co\"incide avec l'op\'eration $\vee$ sur les alg\`ebres via l'identification donn\'ee ci dessus.

\vspace{1cm}


\noindent {\large \textbf{Exercice 1.} \textit{Quelques propri\'et\'es de l'entropie d'une partition}} \vspace{1.5mm} 

\begin{enumerate}
\item On pose
$$
\phi(x) = -x \log(x), \quad x \in [0,1].
$$

Alors si $\Pcal = \{P_1, \dots, P_k\}$, on a, avec $p_i = \mu(P_i),$
$$
H_\mu(\Pcal) = -\sum_{i} p_i \log p_i =  \sum_i \phi(p_i).
$$

Par concavit\'e de $\phi$ on a
$$
\begin{aligned}
\frac{1}{k}ÊH_\mu(\Pcal) &= \frac{1}{k} \sum_i \phi(p_i)  \\
&\leqslant \phi \left(\frac{1}{k} \sum_i p_iÊ\right)  \\
&= \phi \left(\frac{1}{k}\right)  \\
&= -\frac{1}{k} \log \frac{1}{k},
\end{aligned}
$$

et donc $H_\mu(\Pcal) \leqslant \log k = \log \mathrm{Card}(\Pcal).$
\item On note $\Qcal = \{Q_1, \dots, Q_\ell\}$ et $q_j = \mu(Q_j).$  Alors
$$
H_\mu(\Pcal |\Qcal) = \sum_j q_j \sum_i \phi(\mu(P_i|Q_j)),
$$
 et donc 
$$
\begin{aligned}
H_\mu(\Pcal |\Qcal) = 0 &\iff \phi(\mu(P_i|Q_j)) = 0 \quad  \forall i,j  \\
&\iff \mu(P_iÊ|Q_j) = 0 \text{ ou }1 \quad \forall i,j  \\
&\iff P_i \cap Q_j = \emptyset \text{ ou } Q_j \subset P_i \mod 0 \quad \forall i,j  \\
&\iff \Pcal \leqslant \Qcal \mod 0.
\end{aligned}
$$

\item On rappelle le
\begin{lem}[Jensen]
Si $\phi : [0,1] \to \R_+$ est strictement concave on a
$$
\phi\left(\sum_{i=1}^na_i x_i\right) \geqslant \sum_{i=1}^n a_i \phi(x_i), \quad x_i, a_i \in [0,1], \quad \sum_i a_i = 1,
$$
avec \'egalit\'e si et seulement si $x_i = x_j$ d\`es que $a_i, a_j \neq 0.$
\end{lem}



On obtient
$$
\begin{aligned}
H_\mu(\Pcal |\Qcal) &= \sum_j q_j \sum_i \phi(\mu(P_i|Q_j))  \\
&= \sum_{ij} q_j \phi(\mu(P_i|Q_j))  \\
&\leqslant \sum_i \phi\left(\sum_j q_j\mu(P_i|Q_j)\right)  \\
&= \sum_i \phi\left(\sum_j \mu(P_i \cap Q_j)\right)  \\ 
&= \sum_i \phi(\mu(P_i))  \\
&= H(\Pcal),
\end{aligned}
$$
 avec \'egalit\'e si et seulement si
$$
\mu(P_i|Q_j) = \mu(P_i|Q_{j'}), \quad j,j' = 1, \dots, \ell, \quad i = 1, \dots, k.
$$
 On note $c_i = \mu(P_i |Q_j)$ pour n'importe quel $j$.




Alors $\sum_j q_j c_i = \sum_j \mu(P_i|Q_j) = \mu(P_i)$ de sorte que $c_i = p_i.$  On obtient donc
$$
\mu(P_i|Q_j) = \mu(P_i) \quad \forall i,j,
$$
 de sorte que
$$
\mu(P_i \cap Q_j) = \mu(P_i) \mu(Q_j) \quad \forall i,j.
$$



\item On a, en notant $\tilde p_i = \nu(P_i)$,
$$
\begin{aligned}
tH_\mu(\Pcal) + (1-t) H_\nu(\Pcal) &= t \sum_i \phi(p_i) + (1-t) \sum_i \phi(\tilde p_i)  \\
&\leqslant \phi\left(tp_i + (1-t)\tilde p_i\right)  \\
&= H_{t\mu + (1-t)\nu}(\Pcal).
\end{aligned}
$$\end{enumerate}
\vspace{0.6cm}

\noindent {\large \textbf{Exercice 2.} \textit{Quelques propri\'et\'es de l'entropie m\'etrique}} \vspace{1.5mm} 


\begin{enumerate}
\item On a montr\'e que pour toute partition $\Pcal$ on a 
$$
tH_\mu(\Pcal) + (1-t) H_\nu(\Pcal) \leqslant H_{t\mu + (1-t)\nu}(\Pcal).
$$

On a donc pour toutes partitions $\Pcal, \Qcal$,
$$
\begin{aligned}
tH_\mu(\Pcal) + (1-t) H_\nu(\Qcal) &\leqslant tH_\mu(\Pcal \vee \Qcal) + (1-t) H_\nu(\Pcal \vee \Qcal)  \\
&\leqslant  H_{t\mu + (1-t)\nu}(\Pcal \vee \Qcal).
\end{aligned}
$$
 Puisque $\Pcal^n_f \vee \Qcal^n_f = (\Pcal \vee \Qcal)^n_f$ on obtient
$$
th_\mu(f, \Pcal) + (1-t)h_\mu(f,\Qcal) \leqslant h_{t\mu + (1-t)\nu}(f, \Pcal \vee \Qcal) \leqslant h_{t\mu + (1-t)\nu}(f).
$$ 
 Cela conclut.  
\item Soit $\Pcal$ une partition. On note $\Pcal^n_f = \bigvee_{j=0}^n f^{-j}(\Pcal)$.  Alors
$$
\begin{aligned}
\frac{1}{n} H_\mu(\Pcal^{nk}_f) &= \frac{1}{n}ÊH_\mu\left(\bigvee_{i=0}^{nk-1}f^{-i}(\Pcal)\right)  \\
&= \frac{1}{n} H_\mu\left(\bigvee_{j=0}^{n-1}f^{-kj}\left(\bigvee_{i=0}^{k-1}f^{-i}(\Pcal)\right)\right).
\end{aligned}
$$


Par cons\'equent, on obtient 
$$
k h_\mu(f, \Pcal) = h_\mu(f^k, \Pcal^k_f).
$$
 Mais puisque $\Pcal \subset \Pcal^k_f$ on a
$$
h_\mu(f^k, \Pcal) \leqslant h_\mu(f^k, \Pcal^k_f).
$$
 Il vient donc
$$
h_\mu(f^k) = \sup_\Pcal h_\mu(f^k, \Pcal) \leqslant \sup_\Pcal h_\mu(f^k, \Pcal^k_f).
$$
 D'autre part on a \'evidemment
$$
\sup_\Pcal h_\mu(f^k, \Pcal^k_f) \leqslant \sup_\Pcal h_\mu(f^k, \Pcal).
$$
 Finalement 
$$kh_\mu(f) = \sup_\Pcal h_\mu(f,\Pcal) = \sup_{\Pcal} h_\mu(f^k, \Pcal^k_f) = h_\mu(f^k).$$




\item On pose $\Qcal = \{A, X \setminus A\}.$ Alors par le cours
$$
H_\mu(\Pcal^n_f \vee \Qcal) = H_\mu(\Qcal) + H_\mu(\Pcal^n_f |\Qcal).
$$
 On a par d\'efinition
$$
\begin{aligned}
H_\mu(\Pcal^n_f |\Qcal) &= \mu(A) \sum_{P \in \Pcal^n_f} \phi(\mu_A(P)) + \mu(\complement A) \sum_{P \in \Pcal^n_f} \phi(\mu_{\complement A}(P))  \\
&= \mu(A) H_{\mu_A}(\Pcal^n_f) + \mu(\complement A) H_{\mu_{\complement A}}(\Pcal^n_f).
\end{aligned}
$$
 Puisque $(\Pcal \vee \Qcal)^n_f = \Pcal^n_f \vee \Qcal$ (car $A$ est invariant), on obtient
$$
h_\mu(f, \Pcal \vee \Qcal) = \mu(A) h_{\mu_A}(f,\Pcal) + \mu(\complement A) h_{\mu_{\complement A}}(f, \Pcal).
$$
 D\`es lors, puisque $\Pcal \subset \Pcal \vee \Qcal$ on a 
$$
\begin{aligned}
h_\mu(f) &= \sup_\Pcal h_\mu(f, \Pcal)  \\
&\leqslant \sup_\Pcal h_\mu(\Pcal \vee \Qcal)  \\
&= \sup_\Pcal\left( \mu(A) h_{\mu_A}(f,\Pcal) + \mu(\complement A) h_{\mu_{\complement A}}(f, \Pcal)\right).
\end{aligned}
$$



Cela implique
$$
h_\mu(f) \leqslant \mu(A) h_{\mu_A}(f) + \mu(\complement A) h_{\mu_{\complement A}}(f).
$$
 Mais par la question \textbf{1.} on a 
$$
\begin{aligned}
h_\mu(f) &= h_{\mu(A) \mu_A + \mu(\complement A) \mu_{\complement A}}  \\
&\geqslant \mu(A) h_{\mu_A}(f) + \mu(\complement A) h_{\mu_{\complement A}}(f).
\end{aligned}
$$


\item Puisque $(\Pcal^n_f)^{k}_f = \Pcal^{n+k}_f$ on a 
$$
h_\mu(f, \Pcal^n_f) = \lim_k \frac{1}{k} H_\mu((\Pcal^n_f)^k_f) = \lim_k \frac{n+k}{k} \frac{1}{n+k} H_\mu(\Pcal^{n+k}_f) = h_\mu(f, \Pcal).
$$

\end{enumerate}

\vspace{0.6cm}

\noindent {\large \textbf{Exercice 3.} \textit{Une autre version du th\'eor\`eme de Kolmogorov-Sinai}} \vspace{1.5mm} 

\noindent
Il s'agit de montrer que pour tout $\Pcal$ on a 
$$
h_\mu(f, \Pcal) \leqslant \sup_n h_\mu(f, \Pcal_n).
$$
 On a par le cours
$$
h_\mu(f, \Pcal) \leqslant h_\mu(f, \Pcal_n) + H_\mu(\Pcal|\Pcal_n).
$$
Il suffit donc de montrer que $H_\mu(\Pcal|\Pcal_n) \to 0$ quand $n \to +\infty.$  

\noindent
Soit $\varepsilon > 0.$ On a  $\sigma\left(\bigcup_n \Acal_n\right) = \Fcal$ donc par un th\'eor\`eme du cours,il existe $\Ccal$ une alg\`ebre finie de $\bigcup_n \Acal_{n}$ telle que 
$$
H_\mu(\Pcal|\Ccal) < \varepsilon.
$$
 Il existe donc $n_0$ tel que $\Ccal \subset \Acal_{n_0}$, et donc pour tout $n \geqslant n_0$ on a
$$
H_\mu(\Pcal|\Pcal_n) \leqslant H_\mu(\Pcal|\Pcal_{n_0}) \leqslant H_\mu(\Pcal|\Ccal) < \varepsilon.
$$
 Ainsi $\lim_n H_\mu(\Pcal|\Pcal_n) = 0$, ce qui conclut.
\vspace{0.6cm}


\noindent {\large \textbf{Exercice 4.} \textit{Entropie m\'etrique et mesures bor\'eliennes}} \vspace{1.5mm} 

\begin{enumerate}
\item Par l'exercice pr\'ec\'edent il suffit de montrer que
$$
\sigma\left(\bigcup_n \Pcal_n \right) = \Bcal.
$$
 Soit $U$ un ouvert et $x\in U$. Alors il existe $n(x) \in \N$ tel que $\Pcal_{n(x)}(x) \subset U$.  Ainsi,
$$
U = \bigcup_{x \in U} \Pcal_{n(x)}(x).
$$
 Cette union est en fait d\'enombrable, puisque $\bigcup_n \Pcal_n$ est d\'enombrable.  

Ainsi $\bigcup_n \Pcal_n$ engendre tous les ouverts, et donc la tribu engendr\'ee par $\bigcup_n \Pcal_n$ est la tribu des Bor\'eliens.  
\item On d\'ecoupe le cercle en une union d'intervalles $S^1 = I_1 \sqcup \cdots \sqcup I_n$ avec $\mathrm{diam}(I_j) < 1/n$.  

On note alors $\Pcal_n = \{I_1, \dots, I_n\}$, et $x_1, \dots x_n$ les extr\'emit\'es des intervalles.




Alors $f^{-k}(\Pcal_n)$ est une partition compos\'ee d'intervalles d'extr\'emit\'es $f^{-k}(x_j)$, de sorte que 
$$
\mathrm{Card}\left(\bigvee_{k=0}^{\ell-1}f^{-k}(\Pcal_n)\right) \leqslant n\ell.
$$
 On a donc par l'\textbf{Exercice 1.},
$$
\begin{aligned}
h_\mu(f, \Pcal_n) &= \lim_\ell \frac{1}{\ell} H_\mu \left(\bigvee_{k=0}^{\ell-1}f^{-k}(\Pcal_n)\right)  \\
&\leqslant \lim_\ell \frac{\log (n\ell)}{\ell} = 0.
\end{aligned}
$$
 Par la question \textbf{1.}, on a puisque $\mathrm{diam}\Pcal_n(x) \to 0$ pour tout $x$, 
$$
h_\mu(f) = \lim h_\mu(f,\Pcal_n) = 0.
$$


\end{enumerate}

\vspace{0.6cm}


\noindent {\large \textbf{Exercice 5.} \textit{Entropie m\'etrique pour les applications expansives}} \vspace{1.5mm} 

\noindent
Il suffit de montrer que $\diam \Pcal^n_f \to 0$ quand $n \to +\infty$, o\`u
\begin{equation}\label{eq:1}
\diam \Pcal^n_f = \max_{P \in \Pcal^n_f} \diam P.
\end{equation}
 En effet, cela impliquerait par l'exercice pr\'ec\'edent que
$$
h_\mu(f) = \lim_n h_\mu(f, \Pcal^n_f).
$$
Or pour tout $n \geqslant 1$ on a 
$$
h_\mu(f, \Pcal^n_f) = h_\mu(f, \Pcal)
$$
par l'\textbf{Exercice 2}. Ainsi (\ref{eq:1}) implique $h_\mu(f) = h_\mu(f, \Pcal)$. Puisque $\Pcal^k_f \leqslant \Pcal^\ell_f$ pour tous $k \leqslant \ell$, on a que $(\diam \Pcal^n_f)_n$ d\'ecro\^it. 

\noindent 
Raisonnons par contraposition et supposons que $\lim_n \diam \Pcal^n_f = \varepsilon > 0$, de sorte que pour tout $n$ on a $\diam \Pcal^n_f \geqslant \varepsilon$.  Notons $\Pcal = \{P_1, \dots, P_r\}$. Alors les \'el\'ements de $\Pcal^n_f$ sont de la forme
$$
P_{i_0} \cap f^{-1}(P_{i_1}) \cap \cdots \cap f^{-n+1}(P_{i_{n-1}}), \quad i_j \in \{1,\dots, r\}.
$$
Puisque $\diam \Pcal^n_f \geqslant \varepsilon$, on peut trouver $x_n, y_n \in X$ et $P\in \Pcal^n_f$ tels que $\dist(x_n, y_n) \geqslant \varepsilon / 2$ et $x_n, y_n \in P$.  Ceci donne $N_{0,n}, N_{1,n}, \dots, N_{n-1,n} \in \{1, \dots, r\}$ tels que 
$$
f^j(x_n), f^j(y_n) \in P_{N_{j,n}}, \quad j = 0, \dots, n-1, \quad  n \geqslant 1.
$$
En particulier on a
 $$
\dist(f^j(x_n), f^j(y_n)) \leqslant \diam \Pcal, \quad j=0, \dots, n-1.
$$
Quitte a extraire, on peut supposer $x_n \to x$ et $y_n \to y$. Alors $\dist(x,y) \geqslant \varepsilon / 2$ et donc $x \neq y$.  

\noindent
D'autre part, on a pour tout $j \in \N$,
$$
\dist(f^j(x),f^j(y)) = \lim_n \dist(f^j(x_n), f^j(y_n)) \leqslant \diam\Pcal.
$$
Or $x\neq y$ donc par expansivit\'e on obtient $\diam \Pcal \geqslant \delta$, ce qui conclut.
\vspace{0.6cm}

\noindent {\large \textbf{Exercice 6.} \textit{In\'egalit\'e de Rokhlin}} \vspace{1.5mm}

\begin{enumerate}
\item On a $H_\mu(\Pcal \vee \Qcal) = H_\mu(\Qcal) + H_\mu(\Pcal|\Qcal)$ pour tous $\Pcal, \Qcal$.  Donc 
$$
\begin{aligned}
H_\mu(\Pcal|\Qcal) + H_\mu(\Qcal|\Rcal) &= H_\mu(\Pcal \vee \Qcal) + H_\mu(\Qcal \vee \Rcal) - H_\mu(\Qcal) - H_\mu(\Rcal)  \\
&= H_\mu(\Rcal|\Qcal) + H_\mu(\Pcal \vee \Qcal) - H_\mu(\Rcal)  \\
&= H_\mu(\Pcal \vee \Qcal \vee \Rcal) - H_\mu(\Rcal|\Pcal \vee \Qcal) \\ 
&\quad \quad  + H_\mu(\Rcal|\Qcal) - H_\mu(\Rcal)  \\
&\geqslant H_\mu(\Pcal \vee \Qcal \vee \Rcal) - H_\mu(\Rcal)  \\
&\geqslant H_\mu(\Pcal \vee \Rcal) - H_\mu(\Rcal)  \\
&= H_\mu(\Pcal|\Rcal).
\end{aligned}
$$
 
Ainsi on a obtenu
$$
D(\Pcal, \Rcal) \leqslant D(\Pcal, \Qcal) + D(\Qcal, \Rcal).
$$

D'autre part par l'exercice \textbf{1.} on a
$$
\begin{aligned}
D(\Pcal, \Qcal) = 0 \quad &\iff \quad H_\mu(\Pcal|\Qcal) = 0 = H_\mu(\Qcal|\Pcal)  \\
&\iff \quad \Pcal = \Qcal \mod 0.
\end{aligned}
$$





Enfin $D(\Pcal, \Qcal) = D(\Qcal, \Pcal)$ et donc $D$ est une distance.
 \item On a par le cours
$$
h_\mu(f, \Pcal) \leqslant h_\mu(f, \Qcal) + H(\Pcal|\Qcal).
$$

Ainsi
$$
\begin{aligned}
\left|h_\mu(f, \Pcal)-h_\mu(f,\Qcal)\right|Ê&\leqslant  \max(H_\mu(\Pcal|\Qcal), H_\mu(\Qcal|\Pcal)) \\
&\leqslant D(\Pcal, \Qcal).
\end{aligned}
$$

\end{enumerate}
\vspace{0.6cm}

\end{document}
 
 
 

