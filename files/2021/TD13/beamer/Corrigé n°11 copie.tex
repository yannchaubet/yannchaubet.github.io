\documentclass[a4paper,10pt]{beamer}
\usetheme{}
\usepackage[french]{babel}
\usepackage[T1]{fontenc}
%\usetheme{Boadilla}
%\usetheme{AnnArbor}
\usecolortheme{crane}
\setbeamertemplate{navigation symbols}{}
\setbeamertemplate{footline}[frame number]
\usefonttheme{serif}
\usepackage[applemac]{inputenc}
\usepackage{lmodern}
\usepackage{microtype}
\usepackage{hyperref}
\usepackage{dsfont}
\usepackage{amsmath}
\usepackage{mathenv}
\usepackage{amsthm}
\usepackage{graphicx}
\newcommand{\e}{\mathrm{e}}
\newcommand{\w}{\omega}
\newcommand{\dd}{\mathrm{d}}
\newcommand{\id}{\mathrm{id}}
\newcommand{\Id}{\mathrm{Id}}
%\newcommand{\x}{\times}
%\renewcommand{\x}{\mathbf{x}}
\newcommand{\Z}{\mathbb{Z}}
\newcommand{\Q}{\mathbb{Q}}
\newcommand{\R}{\mathbb{R}}
\newcommand{\C}{\mathbb{C}}
\newcommand{\N}{\mathbb{N}}
\newcommand{\T}{\mathbb{T}}
\newcommand{\Acal}{\mathcal{A}}
\newcommand{\Ccal}{\mathcal{C}}
\newcommand{\Pcal}{\mathcal{P}}
\newcommand{\Fcal}{\mathcal{F}}
\newcommand{\Scal}{\mathcal{S}}
\newcommand{\Lcal}{\mathcal{L}}
\newcommand{\Bcal}{\mathcal{B}}
\newcommand{\Mcal}{\mathcal{M}}
\newcommand{\Rcal}{\mathcal{R}}
\newcommand{\Qcal}{\mathcal{Q}}
\newcommand{\vol}{\mathrm{vol}}
\newcommand{\diam}{\operatorname{diam}}
\newcommand{\dist}{\operatorname{dist}}
\newcommand{}{}
\newcommand{\ra}{\rightarrow}
\renewcommand{\P}{\mathbf{P}}
\newcommand{\Repac}{\mathrm{Rep}_{\mathrm{ac}}}


\newtheorem{lemme}[theorem]{Lemme}
\newtheorem{thm}[theorem]{Th\'eor\`eme}

\theoremstyle{plain}
\newenvironment{remark}



%\AtBeginSection[]{
  %{Summary}
  %\small \tableofcontents[currentsection, hideothersubsections]
  % 
%}

\title{\textbf{Syst\`emes dynamiques}}
\subtitle{TD n�12}
\date{8 d\'ecembre 2020}
\author[Yann Chaubet]{Yann Chaubet}
%\institute[Universit\'e Paris-Sud]{\inst{1} Universit\'e Paris-Sud}

\begin{document}

{Exercice 1}
On pose $\varphi_n = \log \|A^{n}\|$. Il suffit de montrer que $(\varphi_n)$ est sous-additive.  

On a, si $\|\cdot\|$ est une norme d'op\'erateur (i.e. $\|AB\| \leqslant \|A\|\|B\|$),
$$
\begin{aligned}
\varphi_{n+m}(x) &= \log \|A^{n+m}(x)\|� \\
&= \log \left\|�A(f^{n+m-1}(x)) \cdots A(f^{m-1}(x)) \cdots A(x)\right\|  \\
&= \log\left\|�A(f^{n+m-1}(x)) \cdots A(f^{m-1}(x))\right\|�\\
&\quad \quad \quad \quad  + \log \left\|A(f^{m-1}(x)) \cdots A(x) \right\|  \\
&= \varphi_m(x) + \varphi_n(f^m(x)).
\end{aligned}
$$

Le r\'esultat pour la norme $\|\cdot\|$ est alors une cons\'equence du th\'eor\`eme ergodif sous-additif de Kingman.  

Si $\|\cdot\|'$ est une autre norme, on a pour un $C > 0$
$$
\frac{\log (1/C)}{n} + \frac{1}{n}\log \|A^n\| \leqslant \frac{1}{n} \log \|A^n\|'�\leqslant \frac{\log C}{n} +  \frac{1}{n} \log \|A^n\|,
$$

ce qui conclut.



{Exercice 2}
\textbf{1.} La transformation $R_\alpha$ est ergodique, donc $\lambda_\pm$ est constante.   \\

\textbf{2.} On a 
$$
\left(
\begin{matrix}
\displaystyle{\frac{1+\e^{2i\theta}}{2}} & \displaystyle{\frac{1-\e^{-2i\theta}}{2}} \\ \displaystyle{\frac{\e^{2i\theta} - 1}{2i}} & \displaystyle{\frac{1 + \e^{2i\theta}}{2}}.
\end{matrix}
\right) 
= 
\e^{i\theta} \begin{pmatrix} \cos \theta & -\sin \theta \\ \sin \theta & \cos \theta \end{pmatrix}
$$


\textbf{3.} Rappel : $f : U \to \R$ est sous-harmonique si pour tout $z \in U$ et $r > 0$ tel que $B(z,r) \subset U$, on a 
$$
f(z) \leqslant \frac{1}{2\pi} \int_0^{2\pi}f\left(z + r\e^{i\theta}\right) \dd \theta.
$$

On note que si $\varphi : U \to \C$ est holomorphe et ne s'annule pas, alors $z \mapsto \log|\varphi(z)|$ est harmonique. En effet,
$$
\partial_z\partial_{\bar z} \log |\varphi(z)| = \partial_z\partial_{\bar z} \left( \log \varphi(z) + \log \bar \varphi(z)\right) = 0.
$$



Chaque coefficient de $C_n(z)$ d\'epend de mani\`ere holomorphe de $z$, et donc $z \mapsto |C_n(z)_{ij}|$ est harmonique pour tous $ij$.  

En particulier si $\|A\| = \max_{ij}|a_{ij}|$ on a que $z \mapsto \log \|C_n(z)\|$ est sous-harmonique. 

\textbf{4.} On a 
$$
\lambda_+ = \lim_{n \to +\infty} \frac{1}{n} \int \log \|A^n(x)\|�\dd \mu(x).
$$
 Mais
$$
\begin{aligned}
C_n\left(\e^{2i\pi x}\right) &= A_\sigma \e^{2i\pi((n-1)\alpha + x)}R_{2\pi((n-1) \alpha + x)} \cdots  A_\sigma \e^{2i\pi x}R_{2\pi x} \\
&= \e^{2i\pi \tau(x)} A^n(x),
\end{aligned}
$$
 o\`u
$$
\tau(x) = nx + \sum_{k=0}^{n-1} k \alpha = nx + \frac{n(n-1)}{2}\alpha.
$$
 Par la question \textbf{3.} on a 
$$
\lim_n \frac{1}{n} \int_0^1 \log \left\|C_n\left(\e^{2i\pi x}\right)\right\|�\dd x \geqslant \lim_n \frac{1}{n}�\log \|C_n(0)\|.
$$



Or on a 
$$
\begin{aligned}
C_n(0) &= \left( A_\sigma \begin{pmatrix} {\frac{1}{2}}�&  {\frac{1}{2i}} \\  {\frac{-1}{2i}} &  {\frac{1}{2}}�\end{pmatrix}  \right)^n�� \\
&= \frac{1}{2^n} \left(A_\sigma \begin{pmatrix} 1 & -i \\ i & 1 \end{pmatrix} \right)^n  \\
&= \frac{1}{2^n} \underset{B}{\underbrace{ \begin{pmatrix} \sigma & -i\sigma \\ i\sigma^{-1} & \sigma^{-1} \end{pmatrix}}}^n.
\end{aligned}
$$
 On a $\mathrm{sp}(B) = \{0, \sigma + \sigma^{-1}\}$.  Par suite
$$
\frac{1}{n}\log\|C_n(0)\| = \frac{1}{n} \log \left\|\left(\frac{B}{2}\right)^n\right\|  \underset{n \to +\infty}{\longrightarrow} \log \rho(B/2)  = \log \frac{\sigma + \sigma^{-1}}{2}.
$$


{Exercice 3}
On note $f = \sigma$ le d\'ecalage sur $\Sigma = \{1, \dots, m\}^{\N}$, et $\mu = (p_1, \dots, p_m)^{\otimes \N}$.  On d\'efinit alors $A : \Sigma \to \mathrm{GL}(d, \R)$ par  
$$
A(x) = A_{x_0}, \quad x = (x_0, x_1, \dots) \in \Sigma.
$$
 Alors, avec les notations de l'Exercice \textbf{1.}, on a 
$$
A^n(x) = A_{x_{n-1}} \cdots A_{x_0}, \quad x = (x_k) \in \Sigma.
$$
 Alors par l'Exercice \textbf{1.} on a 
$$
\frac{1}{n} \log \|A^{n}(x)\| \to \lambda \in \R\cup \{-\infty\}
$$
avec $\lambda$ une constante (car $f$ est ergodique).  

De plus $\lambda > - \infty$. En effet on a $\nu > 1$ tel que 
$$
\nu^{-n}\|v\|\leqslant A^n(x) \leqslant \nu^n\|v\|, \quad v \in \R^d, \quad x \in \Sigma,
$$

ce qui donne $\lambda \geqslant - \log \nu > -\infty.$


{Exercice 4}
\textbf{1.} Si $A \in \mathrm{SL}(2,\R)$ on a $\|A^{-1}\|^{-1} = \|A\|^{-1}$,  et donc
$$
-\lambda_+(x) = \lim \frac{1}{n} \log \|A^n(x)\|^{-1} = \lim \frac{1}{n} \log \|A^{-n}(x)\|^{-1} = \lambda_-(x).
$$


\textbf{2.} On a 
$$
\begin{aligned}
\|A^n(x)\|^{-1}\|v\| &= \|A^n(x)^{-1}\|^{-1}\|v\|  \\
&\leqslant \|A^n(x)v\|� \\
&\leqslant \|A^n(x)\|\|v\|.
\end{aligned}
$$

Donc 
$$
\frac{1}{n} \log \left(\|A^n(x)^{-1}\|^{-1}\|v\|\right)�\geqslant \frac{1}{n} \log \|A^n(x)v\| \leqslant \frac{1}{n} \log \|A^n(x)\|�\|v\|.
$$

Cela donne
$$
\lambda_-(x) \leqslant \liminf_n  \frac{1}{n} \log \|A^n(x)v\| \leqslant \limsup_n  \frac{1}{n} \log \|A^n(x)v\| \leqslant \lambda_+(x)
$$



\textbf{3.} Fait : pour tout $B \in \mathrm{SL}(2)$, il existe $u,v \in \R^2$ tels que $u\perp v$ et
$$
\|u\| = \|v\| = 1, \quad \|Bv\| = \|B\|^{-1}, \quad \|Bu\| = \|B\|, \quad \langle Bu, Bv \rangle =0.
$$
 
En effet, on prend $(u,v)$ qui diagonalise $B^\top B$ avec $B^{\top} B u = \lambda u$ et $B^\top B v = \lambda^{-1}$ avec $\lambda \geqslant \lambda^{-1}$. On a alors $\lambda = \|B\|$, ce qui conclut.   \\

\textbf{4.} $\alpha_n$ est d\'efini par 
$$
s_n(x) = \sin(\alpha_n) u_{n+1}(x) + \cos(\alpha_n) s_{n+1}(x).
$$
 On a 
$$
\begin{aligned}
\|A^{n+1}(x)s_n(x)\| &\geqslant \|\sin(\alpha_n) A^{n+1}(x) u_{n+1}(x)\|  \\
&= |\sin(\alpha_n)| \|A^{n+1}(x)\|.
\end{aligned}
$$

D'autre part
$$
\begin{aligned}
\|A^{n+1}(x)s_n(x)\| &\leqslant \|A(f^n(x))\|\|A^n(x) s_n(x)\|  \\
&= \|A(f^n(x))\|\|A^n(x)\|^{-1}
\end{aligned}
$$

Il suit que 
$$
|\sin(\alpha_n)|�\leqslant \frac{\|A(f^n(x))\|}{\|A^{n+1}(x)\|\|A^n(x)\|}.
$$



On a 
$$
\begin{aligned}
\frac{1}{n}\log \|A(f^n(x))\| &= \frac{1}{n}\log \|A(x)\| + \frac{1}{n}�\sum_{k=0}^{n-1} \log \frac{\|A(f^{k+1}(x))\|}{\|A(f^k(x))\|}  \\
&= \frac{1}{n} \log \|A(x)\| + \frac{1}{n} \sum_{k=0}^{n-1} \psi(f^k(x)),
\end{aligned}
$$
o\`u $\psi(x) = \log\|A(f(x))\|�- \log \|A(x)\|$.  Alors $\psi \in L^1(\mu)$ et donc le th\'eor\`eme ergodique de Birkhoff implique que la limite 
$$
\lim_n \frac{1}{n} \log \|A(f^n(x))\|
$$
existe pour $\mu-$presque tout $x$.  

D'autre part puisque $\log \|A\| \in L^1$ on a, pour tout $\varepsilon > 0$,
$$
\mu\left(\left\{x~:~\frac{1}{n} \log \|A(f^n(x))\| > \varepsilon \right\}\right) \to 0, \quad n \to +\infty.
$$
 En cons\'equence $\frac{1}{n} \log \|A(f^n(\cdot))\| \to 0$ en probabilit\'es quand $n \to +\infty.$



Comme $\frac{1}{n} \log \|A(f^n(\cdot))\|$ converge aussi $\mu$--pp quand $n \to +\infty$, on a que $\frac{1}{n} \log \|A(f^n(\cdot))\| \to 0$ $\mu$--pp.  

Par cons\'equent
$$
\frac{1}{n} \log |\sin(\alpha_n)| \leqslant \frac{1}{n} \log \|A(f^n(x))\| - \frac{1}{n} \log \|A^{n+1}(x)\| - \frac{1}{n} \log \|A^{n}(x)\|,
$$
 et donc 
$$
\limsup_n \frac{1}{n} |\sin(\alpha_n)|�\leqslant -2\lambda_+(x).
$$


\textbf{5.} Soit $\varepsilon > 0$ tel que $\beta = 2 \lambda_+(x) - \varepsilon > 0.$ On a pour tout $n$ assez grand
$$
|\sin(\alpha_n)|�\leqslant \exp\left(-\beta n\right).
$$
 Ceci implique pour tous $m\geqslant n \geqslant 0$
$$
\mathrm{dist}_{\R P^1}(s_n(x), s_m(x)) \leqslant C \sum_{k=n}^{m-1} \e^{-\beta k} \leqslant \frac{C\e^{-\beta n}}{1 - \e^{-\beta}},
$$
ce qui conclut.  Ici on a utilis\'e $\dist_{\R P^1}(u,v) \leqslant C\sin(\mathrm{angle}(u,v))$.




\end{document}




