\documentclass[a4paper,10pt,openany]{article}
\usepackage{fancyhdr}
\usepackage[T1]{fontenc}
\usepackage[margin=1.8cm]{geometry}
\usepackage[applemac]{inputenc}
\usepackage{lmodern}
\usepackage{enumitem}
\usepackage{microtype}
\usepackage{hyperref}
\usepackage{enumitem}
\usepackage{dsfont}
\usepackage{amsmath,amssymb,amsthm}
\usepackage{mathenv}
\usepackage{amsthm}
\usepackage{graphicx}
\usepackage[all]{xy}
\usepackage{lipsum}       % for sample text
\usepackage{changepage}
\theoremstyle{plain}
\newtheorem{thm}{Theorem}[section]
\newtheorem*{thm*}{Th\'eor\`eme}
\newtheorem{prop}[thm]{Proposition}
\newtheorem{cor}[thm]{Corollary}
\newtheorem{lem}[thm]{Lemma}
\newtheorem{propr}[thm]{Propri\'et\'e}
\theoremstyle{definition}
\newtheorem{deff}[thm]{Definition}
\newtheorem{rqq}[thm]{Remark}
\newtheorem{ex}[thm]{Exercice}
\newcommand{\e}{\mathrm{e}}
\newcommand{\prodscal}[2]{\left\langle#1,#2\right\rangle}
\newcommand{\devp}[3]{\frac{\partial^{#1} #2}{\partial {#3}^{#1}}}
\newcommand{\w}{\omega}
\newcommand{\dd}{\mathrm{d}}
\newcommand{\x}{\times}
\newcommand{\ra}{\rightarrow}
\newcommand{\pa}{\partial}
\newcommand{\vol}{\operatorname{vol}}
\newcommand{\dive}{\operatorname{div}}
\newcommand{\T}{\mathbf{T}}
\newcommand{\R}{\mathbf{R}}
\newcommand{\Z}{\mathbf{Z}}
\newcommand{\N}{\mathbf{N}}
\newcommand{\C}{\mathbf{C}}
\newcommand{\F}{\mathcal{F}}
\newcommand{\Homeo}{\mathrm{Homeo}}
\newcommand{\Matn}{\mathrm{Mat}_{n \times n}}
\DeclareMathOperator{\tr}{tr}
\newcommand{\id}{\mathrm{id}}
\newcommand{\htop}{h_\mathrm{top}}


\title{\textsc{Syst\`emes dynamiques} \\ Feuille d'exercices 6}
\date{}
\author{}

\begin{document}

{\noindent \'Ecole Normale Sup\'erieure  \hfill Yann Chaubet } \\
{2021/2022 \hfill \texttt{chaubet@dma.ens.fr}}
{\let\newpage\relax\maketitle}
\maketitle
\noindent Dans toute la suite, si $p$ est un point fixe hyperbolique d'un diff\'eomorphisme $f$ d'une vari\'et\'e $M$, on note $W^u(f,p)$ et $W^s(f,p)$ (resp. $W^u_\mathrm{loc}(f,p)$ et $W^s_\mathrm{loc}(f,p)$) les vari\'et\'es instables et stables globales (resp. locales) de $p$.

\vspace{0.6cm}

\noindent {\large \textbf{Exercice 1.} \textit{Vari\'et\'e stable locale}} \vspace{1.5mm} 

\noindent Soit $M$ une vari\'et\'e compacte, $f : M \to M$ un $\mathcal{C}^1$-diff\'eomorphisme et $p \in M$ un point fixe hyperbolique de $f$. 
\begin{enumerate}
\item Rappeler le th\'eor\`eme de la vari\'et\'e stable (la version perturbation Lipschitz d'un isomorphisme hyperbolique).
\item Montrer que l'espace tangent \`a la vari\'et\'e stable de $p$ est l'espace stable associ\'e \`a $\dd f_p$.
\item Montrer qu'il existe un voisinage $U$ de $p$ et des coordonn\'ees locales $\varphi = (x_1, \dots, x_n) : U \to V \subset \R^n$ centr\'ees en $p$ telles que, si $\tilde{f} = \varphi \circ f \circ \varphi^{-1}$, on a (pr\`es de $0$)
$$
W^s_\mathrm{loc}(\tilde{f}, 0) = \R^r \oplus \{0\}, \quad W^u_\mathrm{loc}(\tilde{f}, 0) = \{0\}\oplus \R^{n-r}.
$$
\item Montrer qu'il existe $c>0$ et $\delta \in (0,1)$ tels pour tous $\tilde{x}_s, \tilde{y}_s \in \R^r \oplus \{0\}$ assez proche de $0$, 
$$
\left\|\tilde{f}^n(\tilde{x}_s) - \tilde{f}^n(\tilde{y}_s)\right\| \leq c \delta^n \|\tilde{x}_s - \tilde{y}_s\|.
$$
\end{enumerate}

\vspace{0.6cm}

\noindent {\large \textbf{Exercice 2.} \textit{Int\'erieur de la vari\'et\'e stable}} \vspace{1.5mm} 

\noindent Soit $M$ une vari\'et\'e compacte, $f : M \to M$ un $\mathcal{C}^1$-diff\'eomorphisme et $p \in M$ un point fixe hyperbolique de $f$. On suppose que la vari\'et\'e stable $W^s(p)$ de $p$ v\'erifie $\dim W^s(p) < \dim M$. Montrer que $W^s(p)$ est d'int\'erieur vide.

\vspace{0.6cm}


\noindent {\large \textbf{Exercice 3.} \textit{Points p\'eriodiques hyperboliques}} \vspace{1.5mm} 

\noindent Soit $M$ une vari\'et\'e compacte, $f : M \to M$ un $\mathcal{C}^1$-diff\'eomorphisme. On suppose que tous les points p\'eriodiques de $f$ sont hyperboliques. Montrer que pour tout $n \in \N$, 
$$
\# \{x \in M,~f^n(x) = x\} < \infty.
$$
\vspace{0.6cm}



\noindent {\large \textbf{Exercice 4.} \textit{Calculs de vari\'et\'es stables}} \vspace{1.5mm} 

\noindent Montrer, dans les cas suivants, que $0 \in \R^n$ est un point fixe hyperbolique du syst\`eme $\dot{x} = f(x)$ et calculer ses vari\'et\'es stables et instables.
\begin{enumerate}
\item $n=2$ et $f(x_1, x_2) = (-x_1, x_2 - 5 \varepsilon x_1^2)$ o\`u $\varepsilon > 0$ est assez petit.
\item $n=3$ et $f(x_1, x_2, x_3) = (-x_1, -x_2 + x_1^2, x_3 + x_1^2)$.
%\item $n$ est quelconque et
%$
%f(x) = Ax + \bigl( \langle S_1x,x\rangle, \dots, \langle S_nx,x\rangle\bigr)
%$
%o\`u $A$ est une matrice r\'eelle dont les valeurs propres ont partie r\'eelle non nulle, et $S_j$ est une matrice sym\'etrique r\'eelle pour tout $j = 1, \dots, n$.
\end{enumerate}


\vspace{0.6cm}

\noindent {\large \textbf{Exercice 5.} \textit{Vari\'et\'e stable de l'application du chat}} \vspace{1.5mm} 

\noindent On consid\`ere $f_L : \T^2 \to \T^2$ l'application associ\'ee \`a la matrice hyperbolique
$
L =
\begin{pmatrix}
2 & 1 \\ 1 & 1
\end{pmatrix}.
$
Montrer que $[0] \in \T^2$ est un point fixe hyperbolique de $f_L$ et que sa vari\'et\'e stable est dense dans $\T^2$.

\vspace{2cm}

\noindent {\large \textbf{Exercice 6.} \textit{Le lemme de Morse}} \vspace{1.5mm} 

\noindent Soit $f \in \mathcal{C}^\infty(\R^n, \R)$. On suppose dans la suite que $f(0) = 0$, $\dd f_0 = 0$ et que la Hessienne de $f$ en $0$, 
$$\mathrm{Hess}_f(0) = \left(\frac{\partial^2 f}{\partial x_i \partial x_j} (0) \right)_{1\leq i,j \leq n},$$
est non d\'eg\'en\'er\'ee. On note $S_n(\R)$ l'espace des matrices sym\'etriques r\'eelles.

\begin{enumerate}

\item Montrer que pour toute matrice $S_0 \in S_n(\R) \cap \mathrm{GL}(n,\R)$, il existe un voisinage $\mathcal{U}$ de $S_0$ dans $S_n(\R)$ et une application lisse $\varphi : \mathcal{U} \to \mathrm{GL}(n,\R)$ telle que
$$
S = \varphi(S)^\top S_0\varphi(S), \quad S \in \mathcal{U}.
$$

\item Montrer que $$f(x) = x^\top Q(x) x, \quad x \in \R^n,$$ o\`u $x \mapsto Q(x)$ est une application lisse $\R^n \to S_n(\R)$.

\item En d\'eduire qu'il existe $r \in \{0, \dots, n\}$ et des voisinages $U,V$ de $0$ dans $\R^n$ et un diff\'eomorphisme $\nu : U \to V$ lisse tel que 
$$
(f \circ \nu) (y_1, \dots, y_n) = \sum_{j=1}^r y_j^2 - \sum_{j=r+1}^n y_j^2.
$$

\item On pose $g = f \circ \nu$. Montrer que $0$ est un point fixe hyperbolique du syst\`eme $\dot{y} = \nabla g(y)$ et calculer ses vari\'et\'es stables et instables locales.

\end{enumerate}

\vspace{0.6cm}


\noindent {\large \textbf{Exercice 7.} \textit{Lin\'earisation du pendule}} \vspace{1.5mm} 

\noindent On consid\`ere la fonction $H : \R^2 \to \R$ d\'efinie par $\displaystyle{H(\theta, \omega) = \frac{1}{2}\omega^2 + 1 - \cos(\theta)}$ et on consid\`ere le syst\`eme diff\'erentiel
$$
\begin{aligned}
\dot{\theta} &= \frac{\partial H}{\partial \omega}(\theta,\omega), \\
\dot{\omega} &= - \frac{\partial H}{\partial \theta} (\theta,\omega). 
\end{aligned}
$$
Montrer que toute trajectoire associ\'ee \`a des petites conditions initiales est l'image d'un cercle par un diff\'eomorphisme local $\varphi$ d\'efini au voisinage de $0$, tel que $\varphi(0) = 0$ et $\dd \varphi_0 = \id.$

\noindent Le syst\`eme du pendule est-il localement conjugu\'e au syst\`eme lin\'earis\'e, donn\'e par $\dot \theta = \omega$ et $\dot \omega = -\dot \theta$ ?

\vspace{0.6cm}

\noindent {\large \textbf{Exercice 8.} \textit{Lin\'earisation des s\'eries formelles}} \vspace{1.5mm} 

\noindent Soit $\lambda \in \C$ tel que $\lambda^n \neq 1$ pour tout $n \neq 0$ et $\displaystyle{f(z) = \lambda z + \sum_{n=2}^{+\infty} a_n z^n}$. 
\begin{enumerate}
\item Montrer qu'il existe une s\'erie formelle $h(z)$ telle que $(h \circ f)(z) = \lambda h(z)$.
\item On suppose que $f$ a un rayon de convergence non nul. Montrer que si $|\lambda|$ est assez grand alors $h$ a un rayon de convergence non nul.
\end{enumerate}

\noindent

\end{document}

\noindent {\large \textbf{Exercice 8.} \textit{Flots hamiltoniens}} \vspace{1.5mm} 

\noindent Soit $H : \R^{2n} \to \R$ une fonction lisse. Le champ hamiltonien $X$ associ\'e \`a $H$ est le champ de vecteurs sur $\R^{2n}$ d\'efini par

$$ X(x, \xi) = J \cdot \nabla H (x,\xi), \quad (x,\xi) \in \R^{2n},$$
o\`u $J = \begin{pmatrix} 0 & I_n \\ -I_n & 0 \end{pmatrix}$. On suppose qu'il existe une application lisse $A : \R^n \to S_n(\R)$ telle que $A(x)$ est d\'efinie positive pour tout $x$ et
$$
H(x,\xi) = \frac{1}{2} \bigl\langle A(x)\xi, \xi \bigr\rangle, \quad (x,\xi) \in \R^{2n}.
$$
\vspace{0.6cm}

 
