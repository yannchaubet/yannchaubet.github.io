\documentclass[a4paper,12pt]{moderncv}
\usepackage[utf8]{inputenc}
\moderncvstyle{classic}            
\moderncvcolor{new} 
\usepackage[french]{babel}


\usepackage{lmodern}
\usepackage{enumitem}
\usepackage{microtype} 
\usepackage[scale=0.775]{geometry}

%\photo[64pt][0pt]{photo.jpg}
\homepage {www.imo.universite-paris-saclay.fr/\textasciitilde chaubet/}
\email{yann.chaubet[at]dpmms.cam.ac.uk}
\extrainfo{né le 13 décembre 1995}
\firstname{Yann}
\familyname{Chaubet}
%\title{PhD student at Universit\'e Paris-Saclay}
%\phone[mobile]{+33698963707}
\address{Centre for Mathematical Sciences \\
Wilberforce Road \\ Cambridge,
CB3 0WB\\ United Kingdom \\ 
Bureau : E.013 \\ \\}

\definecolor{ultramarine}{rgb}{0.07, 0.4, 0.56} 

\begin{document}
\makecvtitle

\nopagenumbers


\section{Scolarité, parcours professionnel}
\cventry{Nov 2022---}{Herchel Smith Research Fellow}{}{Université de Cambridge}{}{}
\cventry{2019---2022}{Thèse de mathématiques}{}{sous la direction de Colin Guillarmou, intitulée \og Sur quelques applications géométriques de la théorie spectrale des flots hyperboliques \fg}{soutenue le 12 septembre 2022 à l'Université Paris-Saclay}{}
\cventry{2014--2019}{Élève normalien}{}{École Normale Sup\'erieure, Paris}{}{}
\cventry{2017--2018}{Master 2}{}{Mathématiques fondamentales, Sorbonne Université}{\textit{mention très bien}}{}
\cventry{}{Mémoire de Master 2}{\textit{The Ruelle zeta function for Anosov flows}}{sous la direction de Colin Guillarmou}{}{}
\cventry{2015--2016}{Master 1}{}{Mathématiques, ENS Paris}{\textit{mention très bien}}{}
\cventry{2014--2015}{Licence}{}{Mathématiques, ENS Paris}{\textit{mention très bien}}{}
\cventry{}{Mémoire de licence}{\textit{Le théorème de masse positive}}{sous la supervision de Cécile Huneau}{}{}
\cventry{2012--2014}{Classes pr\'eparatoires aux grandes écoles}{}{Mathématiques, Lyc\'ee Joffre, Montpellier}{}{}
%\cventry{2012}{Baccalaur\'eat}{}{Lyc\'ee Georges Pompidou, Castelnau-le-Lez}{\textit{mention très bien}}{}

\section{Publications}
{\subsection{\textcolor{ultramarine}{{Prépublications}}}
\cventry{}{Combinatorial zeta functions counting triangles}{}{avec Léo Bénard, Viet Dang et Thomas Schick}{arXiv preprint 2303.11226. 2023}{}
\cventry{}{Geodesic Levy flights and expected stopping time for random searches}{}{avec Yannick Guedes Bonthonneau, Thibault Lefeuvre et Léo Tzou}{arXiv preprint 2211.13973. 2022. Soumis}{}
\cventry{}{Resolvent of vector fields and Lefschetz numbers}{}{avec Yannick Guedes Bonthonneau}{arXiv preprint arXiv:2211.08809. 2022. Soumis}{}
\cventry{}{Dynamical zeta functions for billiards}{}{avec Vesselin Petkov}{arXiv preprint arXiv:2201.00683. 2022. Soumis}{}
%Dans cet article, nous démontrons la conjecture de Lax-Phillips modifiée pour des obstacles analytiques.
%Dans cet article, nous introduisons un objet, la torsion dynamique, défini à l'aide la fonction zêta de Ruelle d'un flot d'Anosov de contact  tordue par une représentation acyclique du groupe fondamental. Nous montrons que la torsion dynamique est intimement liée à un invariant topologique, la torsion de Turaev, qui est définie de manière purement combinatoire.
%\newpage
{\subsection{\textcolor{ultramarine}{{Articles acceptés pour publication}}}
\cventry{2023}{Dynamical torsion for contact Anosov flows}{}{avec Nguyen Viet Dang}{\textit{À paraître dans Analysis \& PDE}}{}
\cventry{2022}{Closed billiard trajectories with prescribed bounces}{Annales Henri Poincar\'e}{}{pages 1–25. Springer, 2022}{}
\cventry{}{Poincar\'e series for surfaces with boundary}{Nonlinearity}{35(12):5993, 2022}{}{}
%Dans cet article, nous calculons la valeur en zéro de séries de Poincaré qui comptent les orthogéodésiques d'une surface à bord totalement géodésique.
\cventry{}{Closed geodesics with prescribed intersection numbers}{à paraître dans Geometry \& Topology}{}{}{}
%Dans cet article, nous explicitons la croissance asymptotique des géodésiques fermées d'une surface qui sont sujettes à certaines contraintes d'intersection.}
%\newpage

\section{Enseignement}
\cventry{2022--2023}{Supervision sessions}{}{Cambridge University}{Differential geometry course}{}
\cventry{2019--2022}{Moniteur}{}{ENS Paris. Chargé de TD du cours de systèmes dynamiques et organisation de journées d'accueil pour les classes préparatoires}{}{}
\cventry{2018-2019}{Examinations orales (colles)}{}{PC*, Lyc\'ee Saint-Louis}{}{}



\section{Expériences}
\cventry{2017}{Recherche et développement en analyse de la mobilité}{}{Paris}{}{Stage de cinq mois en apprentissage machine chez Geo4Cast (ex it4pme).}{}
\cventry{2016}{Stage de Master 1}{}{sous la direction de Enrique Pujals, IMPA, Rio de Janeiro}{}{Mémoire intitulé ``A $C^\infty$-closing lemma and the Lefschetz formula''.}{}



\section{Exposés}
{\subsection{\textcolor{ultramarine}{Conférences internationales}}
\cventry{Juin 2023}{Analytic techniques in Dynamics and Geometry}{}{Les Diablerets}{}{}
\cventry{Mar 2023}{Paris-London Analysis Seminar}{}{London}{}{}
\cventry{Mar 2023}{Rencontre aléatoire, dynamique et spectre}{}{Grenoble}{}{}
\cventry{Mar 2023}{École de printemps géométrie et dynamique}{}{Lille}{}{}
\cventry{Nov 2022}{Geometrical Inverse Problems}{}{Linz}{}{}
\cventry{Mar 2022}{Rigidity problems in geometry}{}{Roscoff}{}{}
\cventry{Fev 2022}{Spectra and Dynamics on (Locally) Symmetric Spaces}{}{Paderborn}{}{}
\cventry{Dec 2021}{Spectral Geometry in the clouds}{}{Online}{}{}
\cventry{Nov 2021}{ANR "Al\'eatoire, dynamique et spectre" meeting}{}{Nantes}{}{}
\cventry{Sep 2021}{Hyperbolic dynamical systems and resonances}{}{Porquerolles}{}{}
\cventry{Avr 2021}{Ruelle-Pollicott Resonances in dynamics and semiclassical analysis}{}{Lausanne (online)}{}{}
\cventry{Avr 2020}{Spectral problems, hyperbolic dynamical systems and quantum chaos}{}{Online}{}{}
\cventry{Oct 2018}{NxSE Tech Forum}{}{Paris}{}{}

\subsection{\textcolor{ultramarine}{Séminaires}}
\cventry{Mar 2023}{Séminaire de Géométrie}{}{Laboratoire de mathématiques Jean Leray, Nantes}{}{}
\cventry{Fev 2023}{Informal seminar of geometry and dynamics}{}{UCL, London}{}{}
\cventry{Fev 2023}{Working group on inverse problem}{}{Cambridge University}{two talks}{}
\cventry{Fev 2023}{London Analysis and Probability Seminar}{}{Imperial College, London}{}{}
\cventry{Fev 2023}{Ergodic Theory and Dynamical Systems Seminar}{}{Warwick Unversity}{}{}
%\cventry{Nov 2022}{Semiclassical limits of modes and quasi modes}{}{Rambouillet}{}{}
\cventry{Dec 2022}{Séminaire Géométrie et Topologie}{}{Institut de Mathématiques de Marseille}{}{}
\cventry{Nov 2022}{Séminaire Géométrie et Topologie}{}{Institut de Mathématiques de Jussieu}{}{}
\cventry{Mai 2022}{S\'eminaire Gaston Darboux}{}{Institut Montpellierain Alexander Grothendieck}{}{}
\cventry{Avr 2022}{Oberseminar Dynamische Systeme}{}{Ruhr-Universität, Bochum}{}{}
\cventry{Avr 2022}{Seminari di Sistemi Dinamici}{}{Centro di Ricerca Matematica Ennio De Giorgi}{Pisa}{}
\cventry{Fev 2022}{S\'eminaire de Physique Math\'ematique}{}{Institut de Math\'ematiques de Bordeaux}{}{}
\cventry{Juin 2021}{S\'eminaire de g\'eom\'etrie}{}{Institut de Math\'ematiques d'Orsay}{}{}
\cventry{Avr 2021}{Geometrische Analysis und Zahlentheorie}{}{Universit\"at Paderborn}{}{}
\cventry{Mar 2021}{S\'eminaire de vulgarisation des doctorants}{}{Institut de math\'ematiques d'Orsay}{}{}
\cventry{Nov 2020}{S\'eminaire d'analyse des doctorants}{}{Institut de math\'ematiques d'Orsay}{}{}
\cventry{Fev 2020}{S\'eminaire d'Analyse Harmonique}{}{Institut math\'ematique d'Orsay}{}{}
\cventry{Jan 2020}{S\'eminaire Teich}{}{Institut math\'ematique de Marseille}{}{}
\cventry{Fev 2019}{Working group : b-calculus}{}{Institut math\'ematique d'Orsay}{}{}


\end{document}

\section{Interests}
\cvitem{Languages}{French (mother tongue), English (fluent), Portuguese, Spanish (basics).}
\cvitem{Programming skills}{Python, Matlab}
\cvitem{Sports}{Freestyle skiing (winner of a Da Camp contest in 2011), surfing, football, gymnastics.}
\cvitem{Hobbies}{Piano, cinema, literature, chess.}



\section{Participation to conferences}
\cventry{Oct 2019}{Auditor}{Recent developments in microlocal analysis}{MSRI}{}{}
\cventry{Apr 2019}{Auditor}{Summer school : From Quantum to Classical}{CIRM}{}{}
\cventry{Mar 2019}{Auditor}{Workshop : Dynamics of geodesic flows and applications to PDE}{Ecole Polytechnique}{}{}
\cventry{Dec 2018}{Auditor}{Workshop : Analytic study of flows}{Orsay}{}{}
\cventry{20-22 Jun 2018}{Auditor}{Journ\'ees de physique math\'ematique : Quantum Chaos}{Lyon}{}{}
\cventry{3-9 Jun 2018}{Auditor}{Workshop : Analytic study of flows}{Peyresq}{}{}
\cventry{Fev 2018}{Auditor}{Workshop : Inverse problems and analytic study of flows}{Orsay}{}{}
\cventry{4-8 Jul 2016}{Auditor}{International Conference on Dynamical Systems}{Buzios}{}{}




\section{Classes taken at UPMC}
\cvitem{2017-2018}{Differential and Riemannian geometry}
\cvitem{}{Pseudodifferential operators}
\cvitem{}{Riemann surfaces}
\cvitem{}{Algebraic topology}
\cvitem{}{Complex geometry and Hodge theory}
\cvitem{}{An introduction to dynamical systems}
\cvitem{}{Dynamical systems in geometry and topology}
\cvitem{}{Hamiltonian manifolds and geometric quantification}
\cvitem{}{Monodromy of algebraic families of curves}
\cvitem{}{Stability of Minkowski spacetime}

\section{Classes taken at IMPA}
\cvitem{2016}{Topics in hyperbolic dynamics}
\cvitem{}{Topology of manifolds}


\section{Classes taken at ENS}
\subsection{Mathematics}
\cvitem{2015-2016}{Dynamical Systems}
\cvitem{}{Analysis of partial differential equations}
\cvitem{}{Stochastic processes}
\cvitem{}{Workgroup, \textit{Boundary rigidity problems in Riemannian geometry}, under the supervision of Colin Guillarmou}
\cvitem{2014-2015}{Differential geometry}
\cvitem{}{Functional analysis}
\cvitem{}{Complex analysis}
\cvitem{}{Topology and differential calculus}
\cvitem{}{Measure, integration and probability}
\cvitem{}{Algebra 1}
\cvitem{}{Workgroup, \textit{Modelisation in biological mathematics}, under the supervision of Bertrand Maury}

\subsection{Physics}
\cvitem{}{Quantum physics}
\cvitem{}{Relativity and electromagnetism}
\cvitem{}{Statistical physics}

\section{Skills}
\cvitem{Languages}{French (mother tongue), English (fluent), Portuguese, Spanish.}
\cvitem{Programming skills}{Python, Matlab}

\section{Interests}
\cvitem{Sports}{I'm very fond of board sports, especially freestyle skiing (I won a Da Camp contest in 2011) and surfing. I also practice football, gymnastics and trampoline.}
\cvitem{Hobbies}{I like cinema and movies in general. I play the piano. I'm fond of literature, and traveling. I like to play chess.}

\section{Travels}
I took a year off after my Master 1, in order to discover the world, to surf, and to learn other languages. Before doing an internship in data sciences, I travelled around New Zealand, Southeast Asia (Malaysia, Singapore, Indonesia), and Central America (Costa-Rica and Nicaragua).
