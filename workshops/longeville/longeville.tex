\documentclass[12pt,a4paper]{amsart}
\usepackage[utf8]{inputenc}
\usepackage{geometry}
\usepackage[frenchb]{babel}
\geometry{letterpaper}
\usepackage[titletoc,toc,title]{appendix}
\usepackage{graphicx}
\usepackage{amssymb}
\usepackage{epstopdf}
\usepackage{mathrsfs}
\usepackage{amsmath}
\usepackage{amsthm}
\usepackage[most]{tcolorbox}
\usepackage{hyperref}
\usepackage{enumitem}
\newcommand{\ds}{\displaystyle}

\usepackage{pgfplots}
\pgfplotsset{compat=1.15}
\usepackage{mathrsfs}
\usetikzlibrary{arrows}
\pagestyle{empty}

\newcommand{\enorm}[1]{\left |\!\left |\!\left |#1\right |\!\right |\!\right |_{k}}

\newcommand{\R}{\mathbb{R}}
\newcommand{\N}{\mathbb{N}}
\newcommand{\C}{\mathbb{C}}
\newcommand{\bv}{\mathbf{v}}
\newcommand{\bx}{\mathrm{x}}
\newcommand{\cG}{c_{\rm g}}
\newcommand{\cC}{c_{\rm c}}
\newcommand{\cS}{c_{\rm s}}
\newcommand{\cN}{\LC_N}

\begin{document}
\newtheorem{theoreme}{Theorem}
\newtheorem{ex}{Example}
\newtheorem{definition}{Definition}
\newtheorem{lem}{Lemma}
\newtheorem{remarque}{Remark}
\newtheorem{exemple}{Example}
\newtheorem{proposition}{Proposition}
\newtheorem*{theo}{Theorem}
\newtheorem{corolaire}{Corollary}
\newtheorem{hyp}{Hypothesis}
\newtheorem*{rec}{Recurrence Hypothesis}

\renewcommand\Im{\mathrm{Im}~}
\renewcommand\Re{\mathrm{Re}~}
\newcommand\T{\mathbb{T}}
\newcommand\cR{\mathcal{R}}
\newcommand\cL{\mathcal{L}}
\newcommand\cX{\mathcal{X}}
\newcommand\cH{\mathcal{H}}
\newcommand\cK{\mathcal{K}}
\newcommand\Sp{\mathbb{S}}
\newcommand\Vol{\mathrm{Vol}}
\newcommand\spt{\mathrm{spt}}
\newcommand\Tr{\mathrm{Tr}}
\newcommand\WF{\mathrm{WF}}
\newcommand\Res{\mathrm{Res}}
\newcommand\n{\boldsymbol{n}}
\newcommand\dd{\mathrm{d}}
\newcommand\cD{\mathcal{D}}
\newcommand\wit{\widetilde}
\newcommand\red{\textcolor{red}}
\newcommand\hp{\mathcal{T}^{d-1}X}

\title{Three days workshop on geometry, analysis and dynamics}
\date{}
\author{Organized by Yann Chaubet and Leo Tzou}
\maketitle


\section*{Program}

\begin{center}
\begin{tabular}{|c|c|c|c|c|c|}
\hline
&Sunday 14/09 & Monday 15/09 & Tuesday 16/09 & Wednesday 17/09 \\
\hline
 $8:00-9:30$ & & Breakfast & Breakfast  & Breakfast \\
 $9:30 - 10:30$ &  & \textbf{Jézéquel} & \textbf{Beaufort} &  \\
  $10:30 - 11:00$ &  & Coffee & Coffee & \textbf{Guillarmou}  \\
  $11:00 - 12:00$ &  & \textbf{Ceki\'c} & \textbf{Faure} & \\
  $12:30-13:30$ & &  Lunch & Lunch & Lunch\\
 $13:30- 16:00$ & & Ocean & Ocean & Departure \\
   $16:00-16:30$ & & Coffee & Coffe & \\
   $16:30-17:30$ & & \textbf{Doll} & \textbf{Bonthonneau} & \\
   $17:30-18:30$ & & \textbf{Wolf} & \textbf{Mazzucchelli} & \\
   $20:00$ & Arrival and dinner & Dinner& Dinner &   \\ 

 \hline 
\end{tabular}
\end{center}


\end{document}

\section*{Historique de reclassement}
J'ai pris mes fonctions le 1\ier~ septembre 2023 en tant que maître de conférences à Nantes Université. Un arrêté de reclassement m'a été proposé en avril 2024. Après vérification, il comportait deux problèmes~:
\begin{enumerate}[label=(\roman*)]
\item mes années de service à l'école normale supérieure n'avaient pas été prises en compte, alors qu'il m'a été confirmé par les ressources humaines de l'ENS qu'elles sont prises en compte par les administrations --- en tant qu'élève de l'ENS, on est fonctionnaire de catégorie A ;
\item un classement au mauvais échelon : sans compter les années de service à l'ENS, mon ancienneté s'élevait donc à 47 mois, soit 4 ans et 11 mois en ajoutant l'année de bonification pour la préparation du doctorat. En comptant l'ancienneté septembre--avril, on arrivait donc presque à l'échelon 4, mais on m'avait proposé l'échelon 2. 
\end{enumerate}

J'ai signifié que je contestais cet arrêt, et j'ai envoyé plusieurs mails de relance, en mai et juillet 2024, puis en février 2025. 

J'ai enfin reçu une réponse en mars 2025. Le service de gestion des personnels (SGP) m'expliquait alors que l'article 3 du décret du 23 avril 2009 ne pouvait être utilisé pour comptabiliser mes années de service à l'ENS. J'ai répondu en envoyant l'arrêté de reclassement de Armand Koening, MCF depuis 2023 à l'Université de Bordeaux, qui a vu ses années d'ENS comptabilisées au titre de l'article 10 (et non pas l'article 3) du même décret. Je joins son arrêté à ce document.

 Cependant, le SGP m'a répondu qu'ils n'avaient pas la même interprétation des textes, et m'a envoyé un nouvel arrêté de reclassement, qui ne prend toujours pas en compte mes années de service à l'ENS (mais qui rétablit le bon calcul et règle le second problème (ii) sus-mentionné). Je joins l'arrêté à ce document. 
 
 Je précise que j'ai aussi échangé avec Quentin Gazda, MCF à Sorbonne Université depuis 2024, pour qui les années d'ENS ont été comptabilisées. 

J'ai envoyé une lettre de réclamation (que je joins aussi à ce mail) au ministère, restée sans réponse.


\section*{Ancienneté}
\begin{itemize}[label = ---]
\item 4 années de service à l'école normale supérieure en tant qu'élève normalien, les 12 derniers mois de service étant intégralement consacrés à la préparation du doctorat (année de pré-doctorat) ;
\item 4 mois de stage dans l'entreprise IT4PME en tant qu'ingénieur (pendant une année de césure à l'ENS)
\item 37 mois de contrat doctoral à l'université Paris--Saclay avec monitorat à l'ENS (64h d'enseignement x 3 ans) ;
\item 10 mois de contrat post-doctoral à l'Université de Cambridge ;
\item 17 mois d'ancienneté (en comptant ce mois-ci) en tant que maître de conférences à l'Université de Nantes.

\end{itemize}

\section*{Salaires perçus entre septembre 2023 et décembre 2024}

\begin{center}

\medskip
\begin{tabular}{|c|c|c|c|}
\hline
Mois & Salaire brut &  $2\, \times$ Smic brut \\
\hline
 Sep. 2023 & 2356,72 & 3494,4  \\
 Oct. 2023 & 2972,57 & 3494,4 \\
 Nov. 2023 & 2681,20 & 3494,4 \\
 Déc. 2023 &  2681,20 & 3494,4 \\
 Jan. 2024 & 2681,20 & 3533.84 \\
 Fév. 2024 & 2797,86 & 3533.84 \\
 Mar. 2024 & 2814,14 & 3533.84 \\
 Avr. 2024 & 2764,40 & 3533.84 \\
 Mai 2024 & 6177,00 & 3533.84 \\
 Juin 2024 & 2764,40 & 3533.84 \\
 Juil. 2024 & 2764,40 & 3533.84 \\
 Août 2024 & 2764,40 & 3533.84 \\
 Sep. 2024 & 2764,40 & 3533.84 \\
 Oct. 2024 & 2764,40 & 3533.84 \\
 Nov. 2024 & 2731,59 & 3603.6 \\
 Déc. 2024 &  3865,19 & 3603.6 \\
 \hline 
 Total & 48344.47 & 56523.2 \\
 \hline
\end{tabular}
\end{center}



%%%% AVEC NET %%%%

\begin{center}
\begin{tabular}{|c|c|c|c|}
\hline
Mois & Salaire brut & Salaire net &  $2\, \times$ Smic brut \\
\hline
 Sep. 2023 & 2356,72 & 1878,58 & 3494,4  \\
 Oct. 2023 & 2972,57 & 2 336,95 & 3494,4 \\
 Nov. 2023 & 2681,20 & 2126,34 & 3494,4 \\
 Déc. 2023 &  2681,20 & 2126,34 & 3494,4 \\
 Jan. 2024 & 2681,20 & 2126,34 & 3533.84 \\
 Fév. 2024 & 2797,86 & 2226,03 & 3533.84 \\
 Mar. 2024 & 2814,14 & 2235,47 & 3533.84 \\
 Avr. 2024 & 2764,40 & 2195,94 & 3533.84 \\
 Mai 2024 & 6177,00 & 5250,70 & 3533.84 \\
 Juin 2024 & 2764,40 & 2189,42 & 3533.84 \\
 Juil. 2024 & 2764,40 & 2189,42 & 3533.84 \\
 Août 2024 & 2764,40 & 2189,42 & 3533.84 \\
 Sep. 2024 & 2764,40 & 2189,42 & 3533.84 \\
 Oct. 2024 & 2764,40 & 2189,42 & 3533.84 \\
 Nov. 2024 & 2731,59 & 2156,61 & 3603.6 \\
 Déc. 2024 &  3865,19  & 3046,35 & 3603.6 \\
 \hline 
 Total & 48344.47 & 40779.70 & 56523.2 \\
 \hline
\end{tabular}
\end{center}

